% This work is licensed under the Creative Commons
% Attribution-NonCommercial-ShareAlike 4.0 International License. To view a copy
% of this license, visit http://creativecommons.org/licenses/by-nc-sa/4.0/ or
% send a letter to Creative Commons, PO Box 1866, Mountain View, CA 94042, USA.

\chapter{Die Finite-Elemente-Methode}
\setcounter{section}{-1}
\section{Einleitung}
In diesem Dokument werden folgende Notationen verwendet:
\begin{itemize}
\item $u_x:=\frac{\partial u(x)}{\partial x}$ für die Ableitung einer diffbaren Funktion $u, x\mapsto u(x)$
\item $\frac{\partial u}{\partial n}:=\nabla u\cdot n$ wobei $n$ die Normale an $u$ ist
\end{itemize}

\subsection{Durchbiegung einer Membran}
\begin{itemize}
\item Gegeben ist eine Membran als Graph der Funktion $u:\Omega\subseteq\R^2\to\R$.
\item Aus der Physik ist bekannt, dass die Deformationsarbeit proportional zur Flächenänderung ist. Die Flächenänderung ist
\begin{align*}
\frac{1}{2}\cdot\int\limits_\Omega\left(u^2_x+u_y^2\right)\d x\d y
\end{align*}
\item Die Energie des Systems ist 
\begin{align*}
\frac{1}{2}\cdot\int\limits_\Omega\left(u^2_x+u_y^2\right)\d x\d y-\int\limits_\Omega f\cdot u\d x\d y
\end{align*}
wobei $f$ eine von außen einwirkende Kraft ist
\item Es wirkt das \textbf{physikalische Minimierungsprinzip}, d.h. das System strebt stets einen Zustand minimaler Gesamtenergie an. Gesucht ist also eine Funktion $u$ derart, dass 
\begin{align*}
	E(u)&\leq E(v)&\forall v\in\tilde{V}\\
	\gdw ~ E(u)&\leq E(u+t\cdot v)\forall t\in\R,~&\forall v\in V
\end{align*}
Dabei ist $V$ ein Funktionenraum, dessen Funktionen auf dem Rand verschwinden.
\item Setze $\varphi(v,t):=E(u+t\cdot v)$, wobei $v$ als Parameter und $t$ als Variable aufgefasst wird. Somit lautet die notwendige Bedingung an das Energieminimum
\begin{align*}
\frac{\d\varphi}{\d t}(v,0)=0
\end{align*}
\end{itemize}
Nachrechnen:
\begin{align*}
\frac{\d\varphi}{\d t}(v,t)
&=\frac{\d}{\d t}E(u+t\cdot v)\\
&=\frac{\d}{\d t}\int\limits_\Omega\left(\frac{1}{2}\cdot\left((u+t\cdot v)_x^2+(u+t\cdot v)_y^2\right)-f(u+t\cdot v)\right)\d x\d y\\
&=\int\limits_\Omega\left((u+t\cdot v)_x\cdot v_x+(u+t\cdot v)_y\cdot v_y-f\cdot v\right)\d x\d y
\end{align*}
Setze nun $t:=0$. Dann folgt aus der notwendigen Bedingung
\begin{align*}
0=\int\limits_\Omega\left(u_x\cdot v_x+u_y\cdot v_y-f\cdot v\right)\d x\d y
\end{align*}
Es entsteht die Variationsaufgabe: Finde $u(t)$ so, dass 
\begin{align*}
\int\limits_\Omega u_x\cdot v_x+u_y\cdot v_y\d x\d y=\int\limits_\Omega f\cdot v\d x\d y~~~\forall v\in V.
\end{align*}

Durch partielle Integration (siehe Anhang für nähere Erklärung) erhält man aus dem linken Integral
\begin{align*}
\int\limits_\Omega u_x\cdot v_x+u_y\cdot v_y\d x\d y
=-\int\limits_\Omega\left(u_{xx}+u_{yy}\right)\cdot v
+\int\limits_{\partial\Omega}\underbrace{u_x\cdot v\cdot n_x+u_y\cdot v\cdot n_y}_{=(\nabla u\cdot n)\cdot v=0\text{, da $v=0$ auf }\partial\Omega}\d\gamma
\end{align*}
Somit folgt:
\begin{align*}
-\int\limits_\Omega\big(\underbrace{u_{xx}+u_{yy}}_{=\laplace u}\big)\cdot v\d x\d y=\int\limits_\Omega f\cdot v\d x\d y~\forall v\in V\\
\Longrightarrow-\laplace u\equiv f\text{ auf }\Omega\Longrightarrow\textbf{``Poisson-Gleichung''}
\end{align*}

\section{Sobolev-Räume}
Bezeichnungen für dieses Kapitel:

\begin{itemize}
\item $d\geq1$ sei die Raumdimension
\item $\Omega\subseteq\R^d$ sei offen und beschränkt
\item $p\in[1,\infty)$ reelle Zahl
\item $q\in(1,\infty]\mit\frac{1}{p}+\frac{1}{q}=1$ \textbf{konjungierter / dualer Exponent}
\item $\alpha=(\alpha_1,\ldots,\alpha_d)\in\N_0^d$ Multiindex mit
\begin{align*}
|\alpha|:=\alpha_1+\ldots+\alpha_d\\
D^\alpha\varphi:=\frac{\partial^{|\alpha|}\varphi}{\partial x_1^{\alpha_1}\hdots\partial x_d^{\alpha_d}}
\end{align*}
\item $L^p(\Omega):=\left\lbrace f:\Omega\to\R:f\text{ messbar und }\int\limits_\Omega |f(x)|^p\d\mathcal{L}(x)<\infty\right\rbrace$ Lebesgue-Räume
\end{itemize}

\textbf{Bemerkungen.}
\begin{enumerate}
\item Da $\Omega$ beschränkt ist, gilt $L^p(\Omega)\subseteq L^1(\Omega)$ und die kanonische Injektion ist stetig.
\item Es gilt die Gauß-Formel:
\begin{align}\label{GaussFormel}
\int\limits_\Omega\varphi\cdot D^\alpha\psi\d x=(-1)^{|\alpha|}\cdot\int\limits_\Omega\psi\cdot D^\alpha\varphi\d x~~~\forall \varphi,\psi\in C_0^\infty(\Omega)
\end{align}
\end{enumerate}

\begin{definition}[schwache Ableitung]
Seien $\varphi,\psi\in L^1(\Omega)$ und sei $\alpha\in\N_0^\alpha$ ein Multiindex. Dann heißt $\psi$ die \textbf{$\alpha$-te schwache Ableitung} von $\varphi:\gdw$
\begin{align*}
\forall v\in C_0^\infty(\Omega):\int\limits_\Omega\varphi\cdot D^\alpha v\d x=(-1)^{|\alpha|}\cdot\int\limits_\Omega\psi\cdot v\d x
\end{align*}
Kurzschreibweise: $\psi=D^\alpha\varphi$
\end{definition}

\textbf{Bemerkungen.}
\begin{enumerate}
\item Die $\alpha$-te schwache Ableitung ist eindeutig bestimmt im Sinne des $L^1$ (also bis auf Lebesgue-Nullmengen).
\item Ist $\varphi\in C^{|\alpha|}(\Omega)$, dann existiert die schwache $\alpha$-te 	Ableitung, die mit der klassischen Ableitung übereinstimmt.
\end{enumerate}

\begin{beisp}
$d=1$, $\Omega=(-1,1)$, $\varphi(x):=|x|$\\
Behauptung: $\varphi'(x)=\left\lbrace\begin{array}{cl}
-1, & \falls -1<x<0\\
1, & \falls 0\leq x<1
\end{array}\right.$\\
Die schwache Ableitung existiert also und der Wert an der Stelle 0 ist nicht relevant.
\begin{proof}
	Sei $v \in C_0^\infty(\Omega)$ beliebig. Dann
	\begin{align*}
		\int_{-1}^1 \varphi v' \d x &= \int_{-1}^0 \varphi v' \d x + \int_{0}^1 \varphi v' \d x \\
															&\stackeq{\text{part}} ~~~ \big[\varphi v\big]_{-1}^0 - \int_{-1}^0 (-1) v \d x + \big[\varphi v\big]_{0}^1 - \int_{0}^1 (1) v \d x \\
															&\stackeq{v = 0 \text{ auf Rand}} ~~~ - \int_{-1}^0 (-1) v \d x - \int_{0}^1 (1) v \d x \\
															&\stackeq{\text{Def}} - \int_{-1}^1 \varphi'(x)v\d x
	\end{align*}
\end{proof}
\end{beisp}

\begin{definition}[Sobolev-Räume]
Für $k\in\N_0$, $p\in[1,\infty)$ definieren wir den \textbf{Sobolev-Raum} wie folgt;
\begin{align}
W^{k,p}(\Omega):=\Big\lbrace\varphi\in L^p(\Omega):D^\alpha\varphi\text{ (schwache Ableitung) existiert und erfüllt }D^\alpha\varphi\in L^p(\Omega)~\forall|\alpha|\leq k\Big\rbrace
\end{align}
Als Norm vereinbaren wir
\begin{align*}
\Vert\varphi\Vert_{k,p,\Omega}:=\left(\sum\limits_{|\alpha|\leq k}\left\Vert D^\alpha\varphi\right\Vert^p_{L^p}\right)^{\frac{1}{p}}
=\left(\sum\limits_{|\alpha|\leq k}\int\limits_\Omega\left| D^\alpha\varphi(x)\right|^p\d x\right)^{\frac{1}{p}}
\end{align*}
Durch
\begin{align*}
|\varphi|_{k,p,\Omega}:=\left(\sum\limits_{|\alpha|= k}\left\Vert D^\alpha\varphi\right\Vert^p_{L^p}\right)^{\frac{1}{p}}
\end{align*}
wird eine Halbnorm definiert.\\
Für $p=2$ schreiben wir $H^k(\Omega):=W^{k,2}(\Omega)$.
\end{definition}

\begin{satz}[Eigenschaften der Sobolev-Räume]
Es gilt:
\begin{enumerate}
\item $\left(W^{k,p}(\Omega),\Vert\cdot\Vert_{k,p,\Omega}\right)$ ist ein Banachraum.
\item $C^\infty(\overline{\Omega})$ liegt dicht in $W^{k,p}(\Omega)$.
\item $H^k(\Omega)$ ist ein Hilbertraum mit dem Skalarprodukt
\begin{align*}
\langle \varphi,\psi\rangle_k:=\sum\limits_{|\alpha|\leq k}\int\limits_\Omega D^\alpha\varphi\cdot D^\alpha\psi\d x.
\end{align*}
\end{enumerate}
\end{satz}

\begin{satz}[Glätte von stückweise glatten Funktionen]\enter
Seien $\Omega_1,\Omega_2\subseteq\Omega$ zwei nichtleere, offene, beschränkte und disjunkte Teilmengen mit stückweise glattem Rand. Gelte
$\overline{\Omega}=\overline{\Omega_1}\cup\overline{\Omega_2}$.
Weiterhin sei $\varphi\in L^p(\Omega)$ so, dass 
\begin{align*}
\exists k\geq 1:\forall i\in\lbrace1,2\rbrace:\varphi|_{\Omega_i}\in C^k(\Omega_i).
\end{align*}
Dann gilt:
\begin{align*}
\varphi\in W^{k,p}(\Omega)\Longleftrightarrow\varphi\in C^{k-1}(\Omega)
\end{align*}
\end{satz}

\begin{definition}
Die Vervollständigung des $C_0^\infty(\Omega)$ bzgl. der Norm $\Vert\cdot\Vert_{k,p,\Omega}$ wird mit $W_0^{k,p}(\Omega)$ bezeichnet. Außerdem setze $H_0^k(\Omega):=W_0^{k,2}(\Omega)$.
\end{definition}

\begin{definition}[Lipschitz-Rand]\enter
$\Omega$ hat einen \textbf{Lipschitz-Rand} $:\gdw\exists N\in\N,\exists U_1,\ldots,U_N\subseteq\R^d$ offen, sodass
\begin{enumerate}
\item $
\begin{aligned}
\partial\Omega\subseteq\bigcup\limits^N_{i=1} U_i
\end{aligned}$
\item $
\begin{aligned}
\forall i\in\lbrace1,\ldots,N\rbrace:\partial\Omega\cap U_i
\end{aligned}$
 ist darstellbar als Graph einer Lipschitz-stetigen Funktion
\end{enumerate}
Das Gebiet $\Omega$ wird dann \textbf{Lipschitz-Gebiet} genannt.
\end{definition}

\begin{bemerkung}
Für Lipschitz-Gebiete existiert fast überall auf $\partial\Omega$ der äußere Normalenvektor.
\end{bemerkung}

\begin{satz}[Spursatz]\enter
Seien $\Omega$ eine Lipschitz-Gebiet, $k\in\N$, $l\in\lbrace 0,\ldots,k-1\rbrace$.\\
Dann gibt es eine stetige lineare Abbildung
\begin{align*}
\gamma_l:W^{k,p}(\Omega)\rightarrow L^p(\partial\Omega)
\end{align*}
mit der Eigenschaft
\begin{align*}
\gamma_l(\varphi)=\frac{\partial^l}{\partial u^l}\varphi|_{\partial\Omega}\qquad\forall\varphi\in C^k(\overline{\Omega}).
\end{align*}
\end{satz}

\begin{bemerkung}\
\begin{itemize}
\item $\gamma_0$ bildet die Funktion $\varphi$, die auf $\Omega$ lebt, auf ihre Randdaten ab.
\item  $\gamma_1$ gibt die Normalenableitung zurück.
\item Dass Bild von $W^{k,p}(\Omega)$ unter $\gamma_l$ ist ein abgeschlossener Unterraum von $L^p(\partial\Omega)$, genauer:
\begin{align*}
\gamma_l\left(W^{k,p}(\Omega)\right)=W^{k-l-\frac{1}{p},p}(\partial\Omega)
\end{align*}
\end{itemize}
\end{bemerkung}

\begin{satz}
Sei $\Omega$ ein Lipschitz-Gebiet. Dann gilt:
\begin{align*}
W_0^{k,p}(\Omega)=\left\lbrace\varphi\in W^{k,p}(\Omega):
\forall l\in\lbrace0,\ldots,k-1\rbrace:\gamma_l(\varphi)=0\right\rbrace
\end{align*}
\end{satz}

\begin{definition}
Seien $(X,\Vert\cdot\Vert_X)$ und $(Y,\Vert\cdot\Vert_Y)$ zwei normierte Vektorräume.
\begin{enumerate}
\item Eine lineare Abbildung $A:X\to Y$ heißt \textbf{kompakt} $:\gdw$ das Bild der abgeschlossenen Einheitskugel in $X$ im Raum $Y$ abgeschlossen ist.
\item $X$ heißt \textbf{stetig eingebettet} in $Y$, in Zeichen $X\hookrightarrow Y:\gdw X\subseteq Y$ und die kanonische Injektion $I:X\to Y,~\varphi\mapsto\varphi$ stetig ist.
\item $X$ heißt \textbf{kompakt eingebettet} in $Y$, in Zeichen $X\stackrel{C}{\hookrightarrow} Y:\gdw X\subseteq Y$ und die kanonische Injektion $I:X\to Y,~\varphi\mapsto\varphi$ kompakt ist.
\end{enumerate}
\end{definition}

\begin{bemerkung}\
\begin{enumerate}
\item $X\hookrightarrow Y\Longrightarrow\exists c>0:\forall\varphi\in X:\Vert\varphi\Vert_Y\leq c\cdot\Vert\varphi\Vert_X$
\item $X\stackrel{C}{\hookrightarrow} Y\Longrightarrow X\hookrightarrow Y$
\item Ist $\stackrel{C}{\hookrightarrow} Y$ und $(\varphi_n)_{n\in\N}\subseteq X$ eine beschränkte Folge in $X$, so besitzt $(\varphi_n)_{n\in\N}\subseteq Y$ eine konvergente Teilfolge.
\end{enumerate}
\end{bemerkung}

\begin{satz}[Einbettung]\
\begin{enumerate}
\item Sei $p<d$. Dann gilt:
\begin{align*}
W^{k,p}(\Omega)\hookrightarrow W^{k-1,p}(\Omega)\qquad\forall q\in\left[1,\frac{p\cdot d}{d-p}\right]\\
W^{k,p}(\Omega)\stackrel{C}{\hookrightarrow} W^{k-1,p}(\Omega)\qquad\forall q\in\left[1,\frac{p\cdot d}{d-p}\right)
\end{align*}
\item Sei $p=d$. Dann gilt:
\begin{align*}
W^{k,p}(\Omega)\hookrightarrow W^{k-1,p}(\Omega)\qquad\forall q\in\left[1,\infty\right)
\end{align*}
\item Sei $k>\frac{d}{p}$. Dann gilt:
\begin{align*}
W^{k,p}(\Omega)\stackrel{C}{\hookrightarrow} C^l(\overline{\Omega})\qquad\forall 0\leq l\leq k-\frac{d}{p}
\end{align*}
\end{enumerate}
\end{satz}

\begin{bemerkung}\
\begin{itemize}
\item Es gilt stets
\begin{align*}
W^{k,p}(\Omega)\stackrel{C}{\hookrightarrow} W^{k-1,p}(\Omega)
\end{align*}
\item Falls $p=2$, $d\in\lbrace2,3\rbrace$: 
\begin{align*}
	H^2(\Omega)\hookrightarrow C(\overline{\Omega})
\end{align*}
Für $H^1(\Omega)$-Funktionen sind Punkte weiter nicht sinnvoll.
\item Falls $p=2,~d=2$: 
\begin{align*}
H^1(\Omega\stackrel{C}{\hookrightarrow} L^q(\Omega)\qquad\forall q\in[1,\infty)
\end{align*}
\item Falls $p=2,~d=3$: 
\begin{align*}
&H^1(\Omega)\stackrel{C}{\hookrightarrow} L^q(\Omega)\qquad\forall q\in[1,6)\\
&H^1(\Omega)\stackrel{C}{\hookrightarrow} L^6(\Omega)
\end{align*}
\end{itemize}
\end{bemerkung}

\begin{beispiel}
	Für $d=2$ gibt es $H^1(\Omega)$-Funktionen, die nicht stetig sind:
	\begin{itemize}
		\item Sei $\Omega=$ Einheitskreis
		\item $u(x,y):=\ln\left(\ln\left(\frac{4}{\sqrt{x^2+y^2}}\right)\right)$
	\end{itemize}
	$u$ hat einen Pol im Ursprung, d.h. $u\not\in C(\overline{\Omega})$, aber:\enter
	\ul{Behauptung:} 
	\[\|u\|^2_{1,2,\Omega}\leq 8\pi + \frac{2\pi}{\ln(4)}< \infty\]
	\[\implies u \in H^1(\Omega)\]
	\begin{proof}
	\[\|u\|^2_{1,2,\Omega} = \underbrace{\int_\Omega |u|^2 \d x \d y}_{:=\text{ I}} +
	\underbrace{\int_\Omega |u_x|^2 + |u_y|^2 \d x \d y}_{:=\text{ II}}\]
	Es gilt für alle $z\geq 4$:
	\[\ln(\ln(z))\leq \sqrt{z}\]
	Mit der Koflächenformel erhalten wir:
	\begin{align*}
		\text{I} &= \int_\Omega |u|^2 \d x \d y \\
			&\leq \int_\Omega \left(\frac{4}{\sqrt{x^2 + y^2}}\right)^2 \d x \d y \\
			&\stackeq{\text{Koflä.}} \int_0^{2\pi}\int_0^1\left(\frac{4}{r}\right)^2 r \d r \d \varphi \\
			&= 8 \pi
	\end{align*}
	Als nächstes betrachten wir II. Dafür schauen wir uns zunächst die partiellen Ableitungen von $u$ an:
	\begin{align*}
		u_x &= \frac{1}{\ln\left(\frac{4}{r}\right)}\frac{r}{4}\left(-\frac{4}{r^2}\right)\frac{x}{r} = - \frac{x}{r^2\ln\left(\frac{4}{r}\right)} \\
		u_y &= - \frac{y}{r^2\ln\left(\frac{4}{r}\right)} \\
		\implies u_x^2 + u_y^2 &= \frac{x^2+y^2}{\left(r^2\ln\left(\frac{4}{r}\right)\right)^2} \\
		&= \frac{r^2}{\left(r^2\ln\left(\frac{4}{r}\right)\right)^2} \\
		&= \frac{1}{r^2\left(\ln\left(\frac{4}{r}\right)\right)^2} \\
	\end{align*}
	Also schlussendlich:
	\begin{align*}
		\text{II} ~~ &\stackeq{\text{vgl. I}} \int_0^{2\pi}\int_0^1\frac{r}{r^2\left(\ln\left(\frac{4}{r}\right)\right)^2} \d r \d \varphi \\
			 &= 2\pi \int_0^1 \frac{1}{r^2\left(\ln\left(\frac{4}{r}\right)\right)^2} \d r \\
			 &=\left[2\pi\left(\frac{1}{\ln\left(\frac{4}{r}\right)}\right)\right]_{r=0}^1 \\
			 &= \frac{2\pi}{\ln(4)}
	\end{align*}
\end{proof}
\end{beispiel}
