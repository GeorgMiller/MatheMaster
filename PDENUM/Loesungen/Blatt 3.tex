% This work is licensed under the Creative Commons
% Attribution-NonCommercial-ShareAlike 4.0 International License. To view a copy
% of this license, visit http://creativecommons.org/licenses/by-nc-sa/4.0/ or
% send a letter to Creative Commons, PO Box 1866, Mountain View, CA 94042, USA.

\documentclass[12pt,a4paper]{article} 

% This work is licensed under the Creative Commons
% Attribution-NonCommercial-ShareAlike 4.0 International License. To view a copy
% of this license, visit http://creativecommons.org/licenses/by-nc-sa/4.0/ or
% send a letter to Creative Commons, PO Box 1866, Mountain View, CA 94042, USA.

% PACKAGES
\usepackage[english, ngerman]{babel}	% Paket für Sprachselektion, in diesem Fall für deutsches Datum etc
\usepackage[utf8]{inputenc}	% Paket für Umlaute; verwende utf8 Kodierung in TexWorks 
\usepackage[T1]{fontenc} % ö,ü,ä werden richtig kodiert
\usepackage{amsmath} % wichtig für align-Umgebung
\usepackage{amssymb} % wichtig für \mathbb{} usw.
\usepackage{amsthm} % damit kann man eigene Theorem-Umgebungen definieren, proof-Umgebungen, etc.
\usepackage{mathrsfs} % für \mathscr
\usepackage[backref]{hyperref} % Inhaltsverzeichnis und \ref-Befehle werden in der PDF-klickbar
\usepackage[english, ngerman, capitalise]{cleveref}
\usepackage{graphicx}
\usepackage{grffile}
\usepackage{setspace} % wichtig für Lesbarkeit. Schöne Zeilenabstände

\usepackage{enumitem} % für custom Liste mit default Buchstaben
\usepackage{ulem} % für bessere Unterstreichung
\usepackage{contour} % für bessere Unterstreichung
\usepackage{epigraph} % für das coole Zitat

\usepackage{tikz}

% This work is licensed under the Creative Commons
% Attribution-NonCommercial-ShareAlike 4.0 International License. To view a copy
% of this license, visit http://creativecommons.org/licenses/by-nc-sa/4.0/ or
% send a letter to Creative Commons, PO Box 1866, Mountain View, CA 94042, USA.

% THEOREM-ENVIRONMENTS

\newtheoremstyle{mystyle}
  {20pt}   % ABOVESPACE \topsep is default, 20pt looks nice
  {20pt}   % BELOWSPACE \topsep is default, 20pt looks nice
  {\normalfont} % BODYFONT
  {0pt}       % INDENT (empty value is the same as 0pt)
  {\bfseries} % HEADFONT
  {}          % HEADPUNCT (if needed)
  {5pt plus 1pt minus 1pt} % HEADSPACE
	{}          % CUSTOM-HEAD-SPEC
\theoremstyle{mystyle}

% Definitionen der Satz, Lemma... - Umgebungen. Der Zähler von "satz" ist dem "section"-Zähler untergeordnet, alle weiteren Umgebungen bedienen sich des satz-Zählers.
\newtheorem{satz}{Satz}[section]
\newtheorem{lemma}[satz]{Lemma}
\newtheorem{korollar}[satz]{Korollar}
\newtheorem{proposition}[satz]{Proposition}
\newtheorem{beispiel}[satz]{Beispiel}
\newtheorem{definition}[satz]{Definition}
\newtheorem{bemerkungnr}[satz]{Bemerkung}
\newtheorem{theorem}[satz]{Theorem}

% Bemerkungen, Erinnerungen und Notationshinweise werden ohne Numerierungen dargestellt.
\newtheorem*{bemerkung}{Bemerkung.}
\newtheorem*{erinnerung}{Erinnerung.}
\newtheorem*{notation}{Notation.}
\newtheorem*{aufgabe}{Aufgabe.}
\newtheorem*{lösung}{Lösung.}
\newtheorem*{beisp}{Beispiel.} %Beispiel ohne Nummerierung
\newtheorem*{defi}{Definition.} %Definition ohne Nummerierung
\newtheorem*{lem}{Lemma.} %Lemma ohne Nummerierung


% SHORTCUTS
\newcommand{\R}{\mathbb{R}}				 % reelle Zahlen
\newcommand{\Rn}{\R^n}						 % der R^n
\newcommand{\N}{\mathbb{N}}				 % natürliche Zahlen
\newcommand{\Z}{\mathbb{Z}}				 % ganze Zahlen
\newcommand{\C}{\mathbb{C}}			   % komplexe Zahlen
\newcommand{\gdw}{\Leftrightarrow} % Genau dann, wenn
\newcommand{\with}{\text{ mit }}   % mit
\newcommand{\falls}{\text{falls }} % falls
\newcommand{\dd}{\text{ d}}        % Differential d

% ETWAS SPEZIELLERE ZEICHEN
%disjoint union
\newcommand{\bigcupdot}{
	\mathop{\vphantom{\bigcup}\mathpalette\setbigcupdot\cdot}\displaylimits
}
\newcommand{\setbigcupdot}[2]{\ooalign{\hfil$#1\bigcup$\hfil\cr\hfil$#2$\hfil\cr\cr}}
%big times
\newcommand*{\bigtimes}{\mathop{\raisebox{-.5ex}{\hbox{\huge{$\times$}}}}} 

% WHITESPACE COMMANDS
%non-restrict newline command
\newcommand{\enter}{$ $\newline} 
%praktischer Tabulator
\newcommand\tab[1][1cm]{\hspace*{#1}}

% TEXT ÜBER ZEICHEN
%das ist ein Gleichheitszeichen mit Text darüber, Beispiel: $a\stackeq{Def} b$
\newcommand{\stackeq}[1]{
	\mathrel{\stackrel{\makebox[0pt]{\mbox{\normalfont\tiny #1}}}{=}}
} 
%das ist ein beliebiges Zeichen mit Text darüber, z. B.  $a\stackrel{Def}{\Rightarrow} b$
\newcommand{\stacksymbol}[2]{
	\mathrel{\stackrel{\makebox[0pt]{\mbox{\normalfont\tiny #1}}}{#2}}
} 

% UNDERLINE
% besseres underline 
\renewcommand{\ULdepth}{1pt}
\contourlength{0.5pt}
\newcommand{\ul}[1]{
	\uline{\phantom{#1}}\llap{\contour{white}{#1}}
}


% hier noch ein paar Commands die nur ich nutze, weil ich sie mir im Laufe der Jahre angewöhnt habe und sie mir jetzt nicht abgewöhnen will:

\newcommand{\gdw}{\Leftrightarrow}   % genau dann, wenn



% This work is licensed under the Creative Commons
% Attribution-NonCommercial-ShareAlike 4.0 International License. To view a copy
% of this license, visit http://creativecommons.org/licenses/by-nc-sa/4.0/ or
% send a letter to Creative Commons, PO Box 1866, Mountain View, CA 94042, USA.

\renewcommand{\div}{\text{ div}}      % Divergenz
\newcommand{\laplace}{\triangle}   % Laplace Operator
\newcommand{\Vertiii}[1]{{\left\vert\kern-0.25ex\left\vert\kern-0.25ex\left\vert #1 
    \right\vert\kern-0.25ex\right\vert\kern-0.25ex\right\vert}}
\newcommand{\T}{\mathcal{T}} %Triangulierung
\newcommand{\meas}{\text{meas}} % Das Maß einer Menge, meist Lebesguemaß


\author{Willi Sontopski}

\parindent0cm %Ist wichtig, um führende Leerzeichen zu entfernen

\usepackage{scrpage2}
\pagestyle{scrheadings}
\clearscrheadfoot

\ihead{Willi Sontopski}
\chead{PDENM WiSe 18 19}
\ohead{}
\ifoot{Blatt 3}
\cfoot{Version: \today}
\ofoot{Seite \pagemark}

\begin{document}
%\setcounter{section}{1}

\section*{Aufgabe 3.1}
Gegeben sei die Differentialgleichung 
\begin{align*}
\frac{\d^4 u(x)}{\d x^4}=f(x)\qquad\forall x\in(0,1)
\end{align*}
mit den Randbedingungen
\begin{enumerate}[label=(\alph*)]
\item $\begin{aligned}
u(0)=u'(0)=u(1)=u'(1)=0
\end{aligned}$
\item $\begin{aligned}
u(0)=u''(0)=u(1)=u''(1)=0
\end{aligned}$
\item $\begin{aligned}
u(0)=u''(0)=u'(1)=u'''(1)=0
\end{aligned}$
\end{enumerate}
Geben Sie jeweils eine schwache Formulierung an.

\begin{lösung}
\underline{Zu (a):}\\

\underline{Zu (b):}\\

\underline{Zu (c):}\\

\end{lösung}

\section*{Aufgabe 3.2}
Sei $\Omega=(0,1)^2$. Wie sieht eine schwache Formulierung des Problems
\begin{align*}
-5\cdot\frac{\partial^2 u(x,y)}{\partial x^2}+6\cdot\frac{\partial^2 u(x,y)}{\partial x\partial y}-8\cdot\frac{\partial^2 u(x)}{\partial y^2}+4\cdot\frac{\partial u(x,y)}{\partial y}+u(x,y)&=f(x,y) \text{ in }\Omega\\
u&\equiv 0\text{ auf }\partial\Omega
\end{align*}
aus?

\begin{lösung}
Sei $A=(a_{i,j})_{i,j\in\lbrace1,2\rbrace}\in\R^{2\times 2}$. Dann gilt:
\begin{align*}
A\cdot\nabla u&=\begin{pmatrix}
 a_{1,1} & a_{1,2}\\
 a_{2,1} & a_{2,2}
\end{pmatrix}\cdot\begin{pmatrix}
\frac{\partial u}{\partial x}\\
\frac{\partial u}{\partial y}
\end{pmatrix}=\begin{pmatrix}
a_{1,1}\cdot\frac{\partial u}{\partial x}+a_{1,2}\cdot\frac{\partial u}{\partial y}\\
a_{2,1}\cdot\frac{\partial u}{\partial x}+a_{2,2}\cdot\frac{\partial u}{\partial y}
\end{pmatrix}\\
\implies
\div(A\cdot\nabla u)&=\frac{\partial}{\partial x}\left(a_{1,1}\cdot\frac{\partial u}{\partial x}+a_{1,2}\cdot\frac{\partial u}{\partial y}\right)+\frac{\partial}{\partial y}\left(a_{2,1}\cdot\frac{\partial u}{\partial x}+a_{2,2}\cdot\frac{\partial u}{\partial y}\right)\\
&=a_{1,1}\cdot\frac{\partial^2 u}{\partial x^2}+\big(a_{1,2}+a_{2,1}\big)\cdot\frac{\partial^2 u}{\partial x\partial y}+a_{2,2}\cdot\frac{\partial^2 u}{\partial y^2}
\end{align*}
Setze also 
\begin{align*}
A:=\begin{pmatrix}
5 & -3\\
-3 & 8
\end{pmatrix}.
\end{align*}
Dann lässt sich die Differentialgleichung des gegebenen Problems umschreiben zu 
\begin{align*}
-\div\big(A\cdot\nabla u(x,y)\big)+4\cdot\frac{\partial u(x,y)}{\partial y}+u(x,y)=f(x,y)
\end{align*}
\end{lösung}

\section*{Aufgabe 3.3}
Das Problem 
\begin{align*}
-\Delta u=1\text{ in }\Omega,\qquad\frac{\partial u}{\partial n}=0\text{ auf }\partial\Omega
\end{align*}
hat keine Lösung. Welche Bedingung müsste erfüllt sein, damit eine Lösung existiert?

\begin{proof}

\end{proof}

\section*{Aufgabe 3.4}
Wie lautet eine geeignete schache Formulierung der Randwertaufgabe
\begin{align*}
-\Delta u+a\cdot\nabla u+\alpha\cdot u\equiv f\text{ in }\Omega,\qquad\frac{\partial u}{\partial n}+\beta\cdot u\equiv g\text{ auf } \partial\Omega
\end{align*}
Geben Sie eine hinreichende Bedingung an, unter denen die eindeutige Lösbarkeit der schwachen Formulierung gesichert ist und beweisen Sie, dass Ihre Annahmen genügen.

\begin{lösung}

\end{lösung}

\section*{Aufgabe 3.5}
Das Differentialgleichungssystem
\begin{align*}
-\Delta u_1-c\cdot u_2&\equiv f_1\text{ in }\Omega\\
-\Delta u_2+c\cdot u_1&\equiv f_2\text{ in }\Omega
\end{align*}
mit $u_1|_{\partial\Omega}=u_2|_{\partial\Omega}=0,c\in L^\infty(\Omega)$ sowie $f_1,f_2\in L^2(\Omega)$ besitzt eine eindeutige schwache Lösung.

\begin{proof}

\end{proof}


\end{document}
