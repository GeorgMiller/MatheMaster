% This work is licensed under the Creative Commons
% Attribution-NonCommercial-ShareAlike 4.0 International License. To view a copy
% of this license, visit http://creativecommons.org/licenses/by-nc-sa/4.0/ or
% send a letter to Creative Commons, PO Box 1866, Mountain View, CA 94042, USA.

\documentclass[12pt,a4paper]{article} 

% This work is licensed under the Creative Commons
% Attribution-NonCommercial-ShareAlike 4.0 International License. To view a copy
% of this license, visit http://creativecommons.org/licenses/by-nc-sa/4.0/ or
% send a letter to Creative Commons, PO Box 1866, Mountain View, CA 94042, USA.

% PACKAGES
\usepackage[english, ngerman]{babel}	% Paket für Sprachselektion, in diesem Fall für deutsches Datum etc
\usepackage[utf8]{inputenc}	% Paket für Umlaute; verwende utf8 Kodierung in TexWorks 
\usepackage[T1]{fontenc} % ö,ü,ä werden richtig kodiert
\usepackage{amsmath} % wichtig für align-Umgebung
\usepackage{amssymb} % wichtig für \mathbb{} usw.
\usepackage{amsthm} % damit kann man eigene Theorem-Umgebungen definieren, proof-Umgebungen, etc.
\usepackage{mathrsfs} % für \mathscr
\usepackage[backref]{hyperref} % Inhaltsverzeichnis und \ref-Befehle werden in der PDF-klickbar
\usepackage[english, ngerman, capitalise]{cleveref}
\usepackage{graphicx}
\usepackage{grffile}
\usepackage{setspace} % wichtig für Lesbarkeit. Schöne Zeilenabstände

\usepackage{enumitem} % für custom Liste mit default Buchstaben
\usepackage{ulem} % für bessere Unterstreichung
\usepackage{contour} % für bessere Unterstreichung
\usepackage{epigraph} % für das coole Zitat

\usepackage{tikz}

% This work is licensed under the Creative Commons
% Attribution-NonCommercial-ShareAlike 4.0 International License. To view a copy
% of this license, visit http://creativecommons.org/licenses/by-nc-sa/4.0/ or
% send a letter to Creative Commons, PO Box 1866, Mountain View, CA 94042, USA.

% THEOREM-ENVIRONMENTS

\newtheoremstyle{mystyle}
  {20pt}   % ABOVESPACE \topsep is default, 20pt looks nice
  {20pt}   % BELOWSPACE \topsep is default, 20pt looks nice
  {\normalfont} % BODYFONT
  {0pt}       % INDENT (empty value is the same as 0pt)
  {\bfseries} % HEADFONT
  {}          % HEADPUNCT (if needed)
  {5pt plus 1pt minus 1pt} % HEADSPACE
	{}          % CUSTOM-HEAD-SPEC
\theoremstyle{mystyle}

% Definitionen der Satz, Lemma... - Umgebungen. Der Zähler von "satz" ist dem "section"-Zähler untergeordnet, alle weiteren Umgebungen bedienen sich des satz-Zählers.
\newtheorem{satz}{Satz}[section]
\newtheorem{lemma}[satz]{Lemma}
\newtheorem{korollar}[satz]{Korollar}
\newtheorem{proposition}[satz]{Proposition}
\newtheorem{beispiel}[satz]{Beispiel}
\newtheorem{definition}[satz]{Definition}
\newtheorem{bemerkungnr}[satz]{Bemerkung}
\newtheorem{theorem}[satz]{Theorem}

% Bemerkungen, Erinnerungen und Notationshinweise werden ohne Numerierungen dargestellt.
\newtheorem*{bemerkung}{Bemerkung.}
\newtheorem*{erinnerung}{Erinnerung.}
\newtheorem*{notation}{Notation.}
\newtheorem*{aufgabe}{Aufgabe.}
\newtheorem*{lösung}{Lösung.}
\newtheorem*{beisp}{Beispiel.} %Beispiel ohne Nummerierung
\newtheorem*{defi}{Definition.} %Definition ohne Nummerierung
\newtheorem*{lem}{Lemma.} %Lemma ohne Nummerierung


% SHORTCUTS
\newcommand{\R}{\mathbb{R}}				 % reelle Zahlen
\newcommand{\Rn}{\R^n}						 % der R^n
\newcommand{\N}{\mathbb{N}}				 % natürliche Zahlen
\newcommand{\Z}{\mathbb{Z}}				 % ganze Zahlen
\newcommand{\C}{\mathbb{C}}			   % komplexe Zahlen
\newcommand{\gdw}{\Leftrightarrow} % Genau dann, wenn
\newcommand{\with}{\text{ mit }}   % mit
\newcommand{\falls}{\text{falls }} % falls
\newcommand{\dd}{\text{ d}}        % Differential d

% ETWAS SPEZIELLERE ZEICHEN
%disjoint union
\newcommand{\bigcupdot}{
	\mathop{\vphantom{\bigcup}\mathpalette\setbigcupdot\cdot}\displaylimits
}
\newcommand{\setbigcupdot}[2]{\ooalign{\hfil$#1\bigcup$\hfil\cr\hfil$#2$\hfil\cr\cr}}
%big times
\newcommand*{\bigtimes}{\mathop{\raisebox{-.5ex}{\hbox{\huge{$\times$}}}}} 

% WHITESPACE COMMANDS
%non-restrict newline command
\newcommand{\enter}{$ $\newline} 
%praktischer Tabulator
\newcommand\tab[1][1cm]{\hspace*{#1}}

% TEXT ÜBER ZEICHEN
%das ist ein Gleichheitszeichen mit Text darüber, Beispiel: $a\stackeq{Def} b$
\newcommand{\stackeq}[1]{
	\mathrel{\stackrel{\makebox[0pt]{\mbox{\normalfont\tiny #1}}}{=}}
} 
%das ist ein beliebiges Zeichen mit Text darüber, z. B.  $a\stackrel{Def}{\Rightarrow} b$
\newcommand{\stacksymbol}[2]{
	\mathrel{\stackrel{\makebox[0pt]{\mbox{\normalfont\tiny #1}}}{#2}}
} 

% UNDERLINE
% besseres underline 
\renewcommand{\ULdepth}{1pt}
\contourlength{0.5pt}
\newcommand{\ul}[1]{
	\uline{\phantom{#1}}\llap{\contour{white}{#1}}
}


% hier noch ein paar Commands die nur ich nutze, weil ich sie mir im Laufe der Jahre angewöhnt habe und sie mir jetzt nicht abgewöhnen will:

\newcommand{\gdw}{\Leftrightarrow}   % genau dann, wenn



% This work is licensed under the Creative Commons
% Attribution-NonCommercial-ShareAlike 4.0 International License. To view a copy
% of this license, visit http://creativecommons.org/licenses/by-nc-sa/4.0/ or
% send a letter to Creative Commons, PO Box 1866, Mountain View, CA 94042, USA.

\renewcommand{\div}{\text{ div}}      % Divergenz
\newcommand{\laplace}{\triangle}   % Laplace Operator
\newcommand{\Vertiii}[1]{{\left\vert\kern-0.25ex\left\vert\kern-0.25ex\left\vert #1 
    \right\vert\kern-0.25ex\right\vert\kern-0.25ex\right\vert}}
\newcommand{\T}{\mathcal{T}} %Triangulierung
\newcommand{\meas}{\text{meas}} % Das Maß einer Menge, meist Lebesguemaß


\author{Willi Sontopski}

\parindent0cm %Ist wichtig, um führende Leerzeichen zu entfernen

\usepackage{color}

\usepackage{scrpage2}
\pagestyle{scrheadings}
\clearscrheadfoot

\ihead{Willi Sontopski}
\chead{PDENM WiSe 18 19}
\ohead{}
\ifoot{Blatt 1}
\cfoot{Version: \today}
\ofoot{Seite \pagemark}

\begin{document}
%\setcounter{section}{1}

\section*{Aufgabe 1.1}
Die Funktion
\begin{align*}
H:(-1,1)\to\R,\qquad H(x):=\left\lbrace\begin{array}{cl}
0, & \falls x<0\\
1, & \falls x\geq0
\end{array}\right.
\end{align*}
besitzt keine schwache Ableitung $f\in L^1((-1,1))$.
\begin{proof}
Offenbar ist die Funktion
\begin{align*}
\tilde{H}:\equiv H|_{(-1,0)\cup(0,1)}.
\end{align*}
beliebig oft stetig differenzierbar mit $\tilde{H}'\equiv0$. Somit ist die schwache Ableitung von $\tilde{H}$ bereits die Nullfunktion und es gilt nach Definition ($\alpha=(0)$) für alle $v\in C_0^\infty((-1,1))$:
\begin{align*}
\int\limits_{(-1,0)\cup(0,1)}\tilde{H}(x)\cdot v'(x)\d x=-1\cdot\int\limits_{(-1,0)\cup(0,1)} 0\cdot v\d x=0
\end{align*}

\textcolor{red}{Das kann doch nicht stimmen. Dann würde hier analog zu unten folgen, dass $v(0)=0$ gelten muss, was ja nicht gilt, oder?}\\

Nehmen wir nun an, dass $H$ eine schwache Ableitung $f\in L^1((-1,1))$ besitzt. Dann gilt per Definition ($\alpha:=(0)$) für alle $v\in C_0^\infty((-1,1))$:
\begin{align*}
-1\cdot\int\limits_{(-1,1)} f(x)\cdot v(x)\d x
&\stackeq{\text{Def}}
\int\limits_{(-1,1)} H(x)\cdot v'(x)\d x\\
&\stackeq{\text{Def }H}
\int\limits_{[0,1)} v'(x)\d x\\
&\stackeq{\text{HS}}
\lim\limits_{k\uparrow 1} v(k)-v(0)\\
&\stackeq{\text{komp. T}}
-v(0)
\end{align*}
\end{proof}

\section*{Aufgabe 1.2}
Sei $\Omega:=(0,2)$ und 
\begin{align*}
v(x)&:=\left\lbrace\begin{array}{cl}
x, & \falls x\in[0,1]\\
2-x, &\falls x\in(1,2]
\end{array}\right.\\
v_n(x)&:=\left\lbrace\begin{array}{cl}
-n\cdot\frac{(x-1)^2}{2}+1-\frac{1}{2\cdot n}, & \falls x\in\left[1-\frac{1}{n},1+\frac{1}{n}\right]\\
v(x),&\sonst
\end{array}\right.\\
\end{align*}
Dann gilt $\limn v_n=v$ in $H^2((0,2))$.
\begin{proof}
Die Differenz
\begin{align*}
v(x)-v_n(x)=\left\lbrace\begin{array}{cl}
x+n\cdot\frac{(x-1)^2}{2}-1+\frac{1}{2\cdot n}, & \falls x\in\left[1-\frac{1}{n},1\right]\\
1-x+n\cdot\frac{(x-1)^2}{2}+\frac{1}{2\cdot n}, & \falls x\in\left(1,1+\frac{1}{n}\right]\\
0,&\sonst
\end{array}\right.\stackrel{\text{}}{\geq}0\\
\end{align*}
hat die schwache Ableitung
\begin{align*}
\frac{\d}{\d x}(v(x)-v_n(x))=\left\lbrace\begin{array}{cl}
1+n\cdot(x-1), & \falls x\in\left[1-\frac{1}{n},1\right]\\
-1+n\cdot(x-1), & \falls x\in\left(1,1+\frac{1}{n}\right]\\
0,&\sonst
\end{array}\right.
\end{align*}

\begin{align*}
&\Vert v-v_n\Vert_{1,2,\Omega}
\stackeq{\text{Def}}
\Bigg(\underbrace{\int\limits_0^2|v -v_n|^2\d x}_{=:I}+\underbrace{\int\limits_0^2\left|\frac{\d}{\d x} \big(v(x)-v_n(x)\big)\right|^2\d x}_{=:II}\Bigg)^{\frac{1}{2}}\\
I&=\int\limits_{1-\frac{1}{n}}^1 x+n\cdot\frac{(x-1)^2}{2}-1+\frac{1}{2\cdot n}\d x+
\int\limits_1^{1+\frac{1}{n}}1-x+n\cdot\frac{(x-1)^2}{2}+\frac{1}{2\cdot n}\d x\\
&=\left[\frac{x^2}{2}+\frac{n}{2}\cdot\frac{(x-1)^3}{3}-x+\frac{x}{2\cdot n}\right]_{x=1-\frac{1}{n}}^1
+\left[x-\frac{x^2}{2}+\frac{n}{2}\cdot\frac{(x-1)^3}{3}+\frac{x}{2\cdot n}\right]_{x=1}^{1+\frac{1}{n}}\\
&=\frac{1}{2}-1+\frac{1}{2\cdot n}-\frac{(1-\frac{1}{n})^2}{2}-\frac{n}{2}\cdot\frac{1}{-3\cdot n^3}+1-\frac{1}{n}-\frac{1-\frac{1}{n}}{2\cdot n}\\
&~~~+1+\frac{1}{n}-\frac{(1+\frac{1}{n})^2}{n}+\frac{n}{2}\cdot\frac{1}{3\cdot n^3}+\frac{1+\frac{1}{n}}{2\cdot n}-1+\frac{1}{2}-\frac{1}{2\cdot n}\\
&=\frac{1}{2}-\frac{(1-\frac{1}{n})^2}{2}-\frac{n}{2}\cdot\frac{1}{-3\cdot n^3}-\frac{1-\frac{1}{n}}{2\cdot n}
-\frac{(1+\frac{1}{n})^2}{2}+\frac{n}{2}\cdot\frac{1}{3\cdot n^3}+\frac{1+\frac{1}{n}}{2\cdot n}+\frac{1}{2}\\
&=1-\frac{n^2+1}{n^2}+\frac{1}{3\cdot n^2}+\frac{1}{n^2}\\
&=\frac{1}{3\cdot n^2}
\end{align*}
\begin{align*}
II&=\int\limits_{1-\frac{1}{n}}^1 |\underbrace{1+n\cdot(x-1)}_{\geq0}|\d x+
\int\limits_1^{1+\frac{1}{n}}|\underbrace{-1+n\cdot(x-1)}_{\leq0}|\d x\\
&=\int\limits_{1-\frac{1}{n}}^1 1+n\cdot(x-1)\d x+
\int\limits_1^{1+\frac{1}{n}}1-n\cdot(x-1)\d x\\
&=\left[x+n\cdot\left(\frac{x^2}{2}-x\right)\right]_{x=1-\frac{1}{n}}^1+\left[x-n\cdot\left(\frac{x^2}{2}-x\right)\right]_{x=1}^{1+\frac{1}{n}}\\
&=1+n\cdot\left(\frac{1}{2}-1\right)-1+\frac{1}{n}-n\cdot\left(\frac{\left(1-\frac{1}{n}\right)^2}{2}-1+\frac{1}{n}\right)\\
&~~~+1+\frac{1}{n}-n\cdot\left(\frac{\left(1+\frac{1}{n}\right)^2}{2}-1-\frac{1}{n}\right)-1+n\cdot\left(\frac{1}{2}-1\right)\\
%&=-n+\frac{2}{n}-n\cdot\left(\frac{\left(1-\frac{1}{n}\right)^2}{2}-1+\frac{1}{n}\right)
%-n\cdot\left(\frac{\left(1+\frac{1}{n}\right)^2}{2}-\frac{1}{n}\right)\\
&\stackeq{\text{CAS}}
\frac{1}{n}\\
&\implies\Vert v-v_n\Vert_{1,2,\Omega}=\sqrt{\frac{1}{3\cdot n^2}+\frac{1}{n}}
\stackrel{n\to\infty}{\longrightarrow}0
\end{align*}

\end{proof}

\section*{Aufgabe 1.3}
Seien $\Omega_1,\Omega_2\subseteq\Omega$ zwei nichtleere, offene, beschränkte und disjunkte Teilmengen von $\Omega$ mit stückweise glattem Rand und $\overline{Omega}=\overline{\Omega_1}\cup\overline{\Omega_2}$. Weiter sei $\varphi\in L^p(\Omega)$ so, dass $\varphi|_{\Omega_i}\in C^1(\Omega_i),~i\in\lbrace1,2\rbrace$ gilt. Dann gilt:
\begin{align*}
\varphi\in W^{1,p}(\Omega)\Longleftrightarrow\varphi\in C(\Omega)
\end{align*}
\begin{proof}
\underline{Zeige ``$\Rightarrow$'':}\\

\underline{Zeige ``$\Leftarrow$'':}\\

\end{proof}

\section*{Aufgabe 1.4}
Sei $u_s:[0,1]\to\R\mit u_s(x):=x^\cdot(1-x)^s,~s>0$. Dann gilt:
\begin{enumerate}[label=(\alph*)]
\item $\forall s>0:u_s\in C^0([0,1])$
\item $u_s\in H_0^1([0,1])\Longleftrightarrow s>\frac{1}{2}$
\end{enumerate}
\begin{proof}
\underline{Zeige (a):}\\

\underline{Zeige (b):}\\


\end{proof}

\section*{Aufgabe 1.5}
Sei $v\in H^1((0,1))$ gegeben. Geben Sie je einen expliziten Spuroperator für $v(0)$ und $v(1)$ an, der mit Hilfe von Integralen über $v$ und $v'$ geschrieben wird.

\begin{proof}

\end{proof}

\section*{Aufgabe 1.6}
Sei $\Omega\subseteq Q\subseteq\R^d\mit Q:=I_1\times\ldots\times I_d$ mit minimaler Seitenlänge $l_{\min}$, d.h.
\begin{align*}
l_{\min}=\min\limits_{i\in\lbrace1,\ldots,d\rbrace}\int\limits_{I_i} 1\d t.
\end{align*}
Dann gilt:
\begin{align*}
\forall v\in W_0^{1,p}(\Omega):\Vert v\Vert_{0,p,\Omega}\leq l_{\min}\cdot|v|_{1,p,\Omega}
\end{align*}
\begin{proof}

\end{proof}

\end{document}