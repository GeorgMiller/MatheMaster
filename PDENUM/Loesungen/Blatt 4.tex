% This work is licensed under the Creative Commons
% Attribution-NonCommercial-ShareAlike 4.0 International License. To view a copy
% of this license, visit http://creativecommons.org/licenses/by-nc-sa/4.0/ or
% send a letter to Creative Commons, PO Box 1866, Mountain View, CA 94042, USA.

\documentclass[12pt,a4paper]{article} 

% This work is licensed under the Creative Commons
% Attribution-NonCommercial-ShareAlike 4.0 International License. To view a copy
% of this license, visit http://creativecommons.org/licenses/by-nc-sa/4.0/ or
% send a letter to Creative Commons, PO Box 1866, Mountain View, CA 94042, USA.

% PACKAGES
\usepackage[english, ngerman]{babel}	% Paket für Sprachselektion, in diesem Fall für deutsches Datum etc
\usepackage[utf8]{inputenc}	% Paket für Umlaute; verwende utf8 Kodierung in TexWorks 
\usepackage[T1]{fontenc} % ö,ü,ä werden richtig kodiert
\usepackage{amsmath} % wichtig für align-Umgebung
\usepackage{amssymb} % wichtig für \mathbb{} usw.
\usepackage{amsthm} % damit kann man eigene Theorem-Umgebungen definieren, proof-Umgebungen, etc.
\usepackage{mathrsfs} % für \mathscr
\usepackage[backref]{hyperref} % Inhaltsverzeichnis und \ref-Befehle werden in der PDF-klickbar
\usepackage[english, ngerman, capitalise]{cleveref}
\usepackage{graphicx}
\usepackage{grffile}
\usepackage{setspace} % wichtig für Lesbarkeit. Schöne Zeilenabstände

\usepackage{enumitem} % für custom Liste mit default Buchstaben
\usepackage{ulem} % für bessere Unterstreichung
\usepackage{contour} % für bessere Unterstreichung
\usepackage{epigraph} % für das coole Zitat

\usepackage{tikz}

% This work is licensed under the Creative Commons
% Attribution-NonCommercial-ShareAlike 4.0 International License. To view a copy
% of this license, visit http://creativecommons.org/licenses/by-nc-sa/4.0/ or
% send a letter to Creative Commons, PO Box 1866, Mountain View, CA 94042, USA.

% THEOREM-ENVIRONMENTS

\newtheoremstyle{mystyle}
  {20pt}   % ABOVESPACE \topsep is default, 20pt looks nice
  {20pt}   % BELOWSPACE \topsep is default, 20pt looks nice
  {\normalfont} % BODYFONT
  {0pt}       % INDENT (empty value is the same as 0pt)
  {\bfseries} % HEADFONT
  {}          % HEADPUNCT (if needed)
  {5pt plus 1pt minus 1pt} % HEADSPACE
	{}          % CUSTOM-HEAD-SPEC
\theoremstyle{mystyle}

% Definitionen der Satz, Lemma... - Umgebungen. Der Zähler von "satz" ist dem "section"-Zähler untergeordnet, alle weiteren Umgebungen bedienen sich des satz-Zählers.
\newtheorem{satz}{Satz}[section]
\newtheorem{lemma}[satz]{Lemma}
\newtheorem{korollar}[satz]{Korollar}
\newtheorem{proposition}[satz]{Proposition}
\newtheorem{beispiel}[satz]{Beispiel}
\newtheorem{definition}[satz]{Definition}
\newtheorem{bemerkungnr}[satz]{Bemerkung}
\newtheorem{theorem}[satz]{Theorem}

% Bemerkungen, Erinnerungen und Notationshinweise werden ohne Numerierungen dargestellt.
\newtheorem*{bemerkung}{Bemerkung.}
\newtheorem*{erinnerung}{Erinnerung.}
\newtheorem*{notation}{Notation.}
\newtheorem*{aufgabe}{Aufgabe.}
\newtheorem*{lösung}{Lösung.}
\newtheorem*{beisp}{Beispiel.} %Beispiel ohne Nummerierung
\newtheorem*{defi}{Definition.} %Definition ohne Nummerierung
\newtheorem*{lem}{Lemma.} %Lemma ohne Nummerierung


% SHORTCUTS
\newcommand{\R}{\mathbb{R}}				 % reelle Zahlen
\newcommand{\Rn}{\R^n}						 % der R^n
\newcommand{\N}{\mathbb{N}}				 % natürliche Zahlen
\newcommand{\Z}{\mathbb{Z}}				 % ganze Zahlen
\newcommand{\C}{\mathbb{C}}			   % komplexe Zahlen
\newcommand{\gdw}{\Leftrightarrow} % Genau dann, wenn
\newcommand{\with}{\text{ mit }}   % mit
\newcommand{\falls}{\text{falls }} % falls
\newcommand{\dd}{\text{ d}}        % Differential d

% ETWAS SPEZIELLERE ZEICHEN
%disjoint union
\newcommand{\bigcupdot}{
	\mathop{\vphantom{\bigcup}\mathpalette\setbigcupdot\cdot}\displaylimits
}
\newcommand{\setbigcupdot}[2]{\ooalign{\hfil$#1\bigcup$\hfil\cr\hfil$#2$\hfil\cr\cr}}
%big times
\newcommand*{\bigtimes}{\mathop{\raisebox{-.5ex}{\hbox{\huge{$\times$}}}}} 

% WHITESPACE COMMANDS
%non-restrict newline command
\newcommand{\enter}{$ $\newline} 
%praktischer Tabulator
\newcommand\tab[1][1cm]{\hspace*{#1}}

% TEXT ÜBER ZEICHEN
%das ist ein Gleichheitszeichen mit Text darüber, Beispiel: $a\stackeq{Def} b$
\newcommand{\stackeq}[1]{
	\mathrel{\stackrel{\makebox[0pt]{\mbox{\normalfont\tiny #1}}}{=}}
} 
%das ist ein beliebiges Zeichen mit Text darüber, z. B.  $a\stackrel{Def}{\Rightarrow} b$
\newcommand{\stacksymbol}[2]{
	\mathrel{\stackrel{\makebox[0pt]{\mbox{\normalfont\tiny #1}}}{#2}}
} 

% UNDERLINE
% besseres underline 
\renewcommand{\ULdepth}{1pt}
\contourlength{0.5pt}
\newcommand{\ul}[1]{
	\uline{\phantom{#1}}\llap{\contour{white}{#1}}
}


% hier noch ein paar Commands die nur ich nutze, weil ich sie mir im Laufe der Jahre angewöhnt habe und sie mir jetzt nicht abgewöhnen will:

\newcommand{\gdw}{\Leftrightarrow}   % genau dann, wenn



% This work is licensed under the Creative Commons
% Attribution-NonCommercial-ShareAlike 4.0 International License. To view a copy
% of this license, visit http://creativecommons.org/licenses/by-nc-sa/4.0/ or
% send a letter to Creative Commons, PO Box 1866, Mountain View, CA 94042, USA.

\renewcommand{\div}{\text{ div}}      % Divergenz
\newcommand{\laplace}{\triangle}   % Laplace Operator
\newcommand{\Vertiii}[1]{{\left\vert\kern-0.25ex\left\vert\kern-0.25ex\left\vert #1 
    \right\vert\kern-0.25ex\right\vert\kern-0.25ex\right\vert}}
\newcommand{\T}{\mathcal{T}} %Triangulierung
\newcommand{\meas}{\text{meas}} % Das Maß einer Menge, meist Lebesguemaß


\author{Willi Sontopski}

\parindent0cm %Ist wichtig, um führende Leerzeichen zu entfernen

\usepackage{scrpage2}
\pagestyle{scrheadings}
\clearscrheadfoot

\ihead{Willi Sontopski}
\chead{PDENM WiSe 18 19}
\ohead{}
\ifoot{Blatt 4}
\cfoot{Version: \today}
\ofoot{Seite \pagemark}

\newcommand{\G}{\mathcal{G}}

\begin{document}
%\setcounter{section}{1}

\section*{Aufgabe 4.1}
Gegeben sei die Randwertaufgabe
\begin{align*}
-u''\equiv f,\qquad u(0)=0,~u(1)=0
\end{align*}
Dann liefert die Finite-Elemente-Methode mit linearen Elementen in den Gitterpunkten die exakten Werte der Lösung.

\begin{proof}
Setze
\begin{align*}
\G_\xi:\R\to\R,\qquad
\G_\xi(x):=\left\lbrace\begin{array}{cl}
(1-\xi)\cdot x, &\falls x<\xi\\
\xi\cdot(1-x), &\falls x\geq\xi
\end{array}\right.\qquad\forall\xi\in\R.
\end{align*}
Sei $v\in H_0^1((0,1))$ beliebig. Dann gilt:
\begin{align*}
\int\limits_0^1\G_\xi'(x)\cdot v'(x)\d x
&=
\end{align*}
\end{proof}

\section*{Aufgabe 4.2}
Seien $\hat{K}:=(-1,1)^2$ das (große) Einheitsquadrat und $\hat{\Sigma}:=\left\lbrace\hat{N}_1,\hat{N}_2,\hat{N}_3,\hat{N}_4\right\rbrace$ mit den Knotenfunktionalen 
\begin{align*}
\hat{N}_1(\hat{v}):=\hat{v}(0,-1),\quad
\hat{N}_2(\hat{v}):=\hat{v}(1,0),\quad
\hat{N}_3(\hat{v}):=\hat{v}(0,1),\quad
\hat{N}_4(\hat{v}):=\hat{v}(-1,0)
\end{align*}
gegeben. Weiterhin seien 
\begin{align*}
\hat{V}_1:=\hat{Q}_1:=\spann\big(1,\hat{x},\hat{y},\hat{x}\cdot\hat{y}\big)
\end{align*}
der Raum der bilinearen Funktionen und
\begin{align*}
\hat{V}_2:=\hat{Q}_1^{\text{rot}}:=\spann\big(1,\hat{x},\hat{y},\hat{x}^2-\hat{y}^2\big)
\end{align*}
der Raum der rotiert bilinearen Funktionen.
\begin{enumerate}[label=(\alph*)]
\item $\hat{\Sigma}$ ist bzgl. $\hat{V}_1$ nicht unisolvent.
\item $\hat{\Sigma}$ ist bzgl. $\hat{V}_2$ unisolvent.
\item Bestimmen Sie die (nodalen) Basisfunktionen $\hat{\varphi}_i,i=1,\ldots,4$ für $\hat{\Sigma}$ und $\hat{V}_2$.
\end{enumerate}
\begin{proof}
\underline{Zeige (a):}\\

\underline{Zeige (b):}\\

\underline{Zeige (c):}\\

\end{proof}

\section*{Aufgabe 4.3}
Seien $\hat{K}:=(0,1)^2$ mit den Ecken 
\begin{align*}
\hat{P}_1=(0,0),\qquad
\hat{P}_2(1,0),\qquad
\hat{P}_3(1,1),\qquad
\hat{P}_4(0,1)
\end{align*}
das (kleine) Einheitsquadrat und $K$ mit den Ecken
\begin{align*}
P_1(x_1,y_1),\qquad
P_2(x_2,y_2),\qquad
P_1(x_3,y_3),\qquad
P_1(x_4,y_4)
\end{align*}
ein beliebiges Viereck.
\begin{enumerate}[label=(\alph*)]
\item Bestimmen Sie die Koeffizienten der bilinearen Abbildung $F:\hat{K}\to K$, die $\hat{P}_i$ auf $P_i$, $i=1,\ldots,4$ abbildet. Die Abbildung $F$ heißt \textbf{bilinear} $:\gdw F\in\left(\hat{Q}_1\right)^2$.
\item Zeigen Sie, dass die Determinante der Jacobi-Matrix $DF(\hat{x},\hat{y})$ der Abbildung $F$ eine affine Funktion ist.
\item Unter welchen Bedingungen an die Koeffizienten von $F$ stellt $F$ eine affine Abbildung dar? Deuten Sie diese Bedingungen geometrisch.
\end{enumerate}

\begin{lösung}
\underline{Zu (a):}\\

\underline{Zeige (b):}\\

\underline{Zu (c):}\\

\end{lösung}

\section*{Aufgabe 4.4}
Für ein achsenparalleles Rechteck $K$ sei $Q_2$ die Menge der Polynome auf $K$, die bezüglich $x$ und $y$ höchstens zweiten Grades sind. Es seien $(e_j)_{j=1}^4$ die vier Kanten und $(x_j)_{j=1}^4$ die vier Ecken von $K$. Untersuchen Sie, ob durch die neun Freiheitsgrade
\begin{enumerate}[label=\roman*]
\item Funktionswerte $f(x_j)$ in den Ecken,
\item $\int\limits_{e_j} f\d s$ und $\int\limits_K f\d\lambda$
\end{enumerate}
einer Funktion $f\in Q_2$ die Unisolvenz sichergestellt wird.

\begin{lösung}
\underline{Fall 1: Referenzquadrat}\\
Wir untersuchen die Aussage zunächst für das Referenzquadrat $\hat{K}:=(0,1)^2$.\\

\underline{Fall 2: Allgemeiner Fall}\\
Wir können den allgemeinen Fall auf den Fall des Referenzquadrates (Fall 1) zurückführen.

\end{lösung}

\section*{Aufgabe 4.5}
Seien $\hat{K}:=(0,1)$ und $K:=(a,a+h)$ mit $a\in\R$ und $h>0$ zwei offene Intervalle. Weiterhin sei $F:\hat{K}\to K$ die affine Abbildung, die $F(0)=1$ und $F(1)=a+h$ erfüllt.\\
Dann besteht für $\varphi\in W^{m,p}(K),m\in\N_0,p\in[1,\infty)$ und $\hat{\varphi}:=\varphi\circ F\in W^{m,p}(\hat{K})$ der Zusammenhang
\begin{align*}
\big|\hat{\varphi}\big|_{m,p,\hat{K}}=h^m\cdot \left(\frac{1}{h}\right)^{\frac{1}{p}}\cdot|\varphi|_{m,p,K}.
\end{align*}
\begin{proof}

\end{proof}


\end{document}