% This work is licensed under the Creative Commons
% Attribution-NonCommercial-ShareAlike 4.0 International License. To view a copy
% of this license, visit http://creativecommons.org/licenses/by-nc-sa/4.0/ or
% send a letter to Creative Commons, PO Box 1866, Mountain View, CA 94042, USA.

\documentclass[12pt,a4paper]{article} 

% This work is licensed under the Creative Commons
% Attribution-NonCommercial-ShareAlike 4.0 International License. To view a copy
% of this license, visit http://creativecommons.org/licenses/by-nc-sa/4.0/ or
% send a letter to Creative Commons, PO Box 1866, Mountain View, CA 94042, USA.

% PACKAGES
\usepackage[english, ngerman]{babel}	% Paket für Sprachselektion, in diesem Fall für deutsches Datum etc
\usepackage[utf8]{inputenc}	% Paket für Umlaute; verwende utf8 Kodierung in TexWorks 
\usepackage[T1]{fontenc} % ö,ü,ä werden richtig kodiert
\usepackage{amsmath} % wichtig für align-Umgebung
\usepackage{amssymb} % wichtig für \mathbb{} usw.
\usepackage{amsthm} % damit kann man eigene Theorem-Umgebungen definieren, proof-Umgebungen, etc.
\usepackage{mathrsfs} % für \mathscr
\usepackage[backref]{hyperref} % Inhaltsverzeichnis und \ref-Befehle werden in der PDF-klickbar
\usepackage[english, ngerman, capitalise]{cleveref}
\usepackage{graphicx}
\usepackage{grffile}
\usepackage{setspace} % wichtig für Lesbarkeit. Schöne Zeilenabstände

\usepackage{enumitem} % für custom Liste mit default Buchstaben
\usepackage{ulem} % für bessere Unterstreichung
\usepackage{contour} % für bessere Unterstreichung
\usepackage{epigraph} % für das coole Zitat

\usepackage{tikz}

% This work is licensed under the Creative Commons
% Attribution-NonCommercial-ShareAlike 4.0 International License. To view a copy
% of this license, visit http://creativecommons.org/licenses/by-nc-sa/4.0/ or
% send a letter to Creative Commons, PO Box 1866, Mountain View, CA 94042, USA.

% THEOREM-ENVIRONMENTS

\newtheoremstyle{mystyle}
  {20pt}   % ABOVESPACE \topsep is default, 20pt looks nice
  {20pt}   % BELOWSPACE \topsep is default, 20pt looks nice
  {\normalfont} % BODYFONT
  {0pt}       % INDENT (empty value is the same as 0pt)
  {\bfseries} % HEADFONT
  {}          % HEADPUNCT (if needed)
  {5pt plus 1pt minus 1pt} % HEADSPACE
	{}          % CUSTOM-HEAD-SPEC
\theoremstyle{mystyle}

% Definitionen der Satz, Lemma... - Umgebungen. Der Zähler von "satz" ist dem "section"-Zähler untergeordnet, alle weiteren Umgebungen bedienen sich des satz-Zählers.
\newtheorem{satz}{Satz}[section]
\newtheorem{lemma}[satz]{Lemma}
\newtheorem{korollar}[satz]{Korollar}
\newtheorem{proposition}[satz]{Proposition}
\newtheorem{beispiel}[satz]{Beispiel}
\newtheorem{definition}[satz]{Definition}
\newtheorem{bemerkungnr}[satz]{Bemerkung}
\newtheorem{theorem}[satz]{Theorem}

% Bemerkungen, Erinnerungen und Notationshinweise werden ohne Numerierungen dargestellt.
\newtheorem*{bemerkung}{Bemerkung.}
\newtheorem*{erinnerung}{Erinnerung.}
\newtheorem*{notation}{Notation.}
\newtheorem*{aufgabe}{Aufgabe.}
\newtheorem*{lösung}{Lösung.}
\newtheorem*{beisp}{Beispiel.} %Beispiel ohne Nummerierung
\newtheorem*{defi}{Definition.} %Definition ohne Nummerierung
\newtheorem*{lem}{Lemma.} %Lemma ohne Nummerierung


% SHORTCUTS
\newcommand{\R}{\mathbb{R}}				 % reelle Zahlen
\newcommand{\Rn}{\R^n}						 % der R^n
\newcommand{\N}{\mathbb{N}}				 % natürliche Zahlen
\newcommand{\Z}{\mathbb{Z}}				 % ganze Zahlen
\newcommand{\C}{\mathbb{C}}			   % komplexe Zahlen
\newcommand{\gdw}{\Leftrightarrow} % Genau dann, wenn
\newcommand{\with}{\text{ mit }}   % mit
\newcommand{\falls}{\text{falls }} % falls
\newcommand{\dd}{\text{ d}}        % Differential d

% ETWAS SPEZIELLERE ZEICHEN
%disjoint union
\newcommand{\bigcupdot}{
	\mathop{\vphantom{\bigcup}\mathpalette\setbigcupdot\cdot}\displaylimits
}
\newcommand{\setbigcupdot}[2]{\ooalign{\hfil$#1\bigcup$\hfil\cr\hfil$#2$\hfil\cr\cr}}
%big times
\newcommand*{\bigtimes}{\mathop{\raisebox{-.5ex}{\hbox{\huge{$\times$}}}}} 

% WHITESPACE COMMANDS
%non-restrict newline command
\newcommand{\enter}{$ $\newline} 
%praktischer Tabulator
\newcommand\tab[1][1cm]{\hspace*{#1}}

% TEXT ÜBER ZEICHEN
%das ist ein Gleichheitszeichen mit Text darüber, Beispiel: $a\stackeq{Def} b$
\newcommand{\stackeq}[1]{
	\mathrel{\stackrel{\makebox[0pt]{\mbox{\normalfont\tiny #1}}}{=}}
} 
%das ist ein beliebiges Zeichen mit Text darüber, z. B.  $a\stackrel{Def}{\Rightarrow} b$
\newcommand{\stacksymbol}[2]{
	\mathrel{\stackrel{\makebox[0pt]{\mbox{\normalfont\tiny #1}}}{#2}}
} 

% UNDERLINE
% besseres underline 
\renewcommand{\ULdepth}{1pt}
\contourlength{0.5pt}
\newcommand{\ul}[1]{
	\uline{\phantom{#1}}\llap{\contour{white}{#1}}
}


% hier noch ein paar Commands die nur ich nutze, weil ich sie mir im Laufe der Jahre angewöhnt habe und sie mir jetzt nicht abgewöhnen will:

\newcommand{\gdw}{\Leftrightarrow}   % genau dann, wenn



% This work is licensed under the Creative Commons
% Attribution-NonCommercial-ShareAlike 4.0 International License. To view a copy
% of this license, visit http://creativecommons.org/licenses/by-nc-sa/4.0/ or
% send a letter to Creative Commons, PO Box 1866, Mountain View, CA 94042, USA.

\renewcommand{\div}{\text{ div}}      % Divergenz
\newcommand{\laplace}{\triangle}   % Laplace Operator
\newcommand{\Vertiii}[1]{{\left\vert\kern-0.25ex\left\vert\kern-0.25ex\left\vert #1 
    \right\vert\kern-0.25ex\right\vert\kern-0.25ex\right\vert}}
\newcommand{\T}{\mathcal{T}} %Triangulierung
\newcommand{\meas}{\text{meas}} % Das Maß einer Menge, meist Lebesguemaß


\author{Willi Sontopski}

\parindent0cm %Ist wichtig, um führende Leerzeichen zu entfernen

\usepackage{scrpage2}
\pagestyle{scrheadings}
\clearscrheadfoot

\ihead{Willi Sontopski}
\chead{PDENM WiSe 18 19}
\ohead{}
\ifoot{Blatt 4}
\cfoot{Version: \today}
\ofoot{Seite \pagemark}

\newcommand{\G}{\mathcal{G}}

\begin{document}
%\setcounter{section}{1}

\section*{Aufgabe 4.1}
Gegeben sei die Randwertaufgabe
\begin{align*}
-u''\equiv f,\qquad u(0)=0,~u(1)=0
\end{align*}
Dann liefert die Finite-Elemente-Methode mit linearen Elementen in den Gitterpunkten die exakten Werte der Lösung.

\begin{proof}
Setze
\begin{align*}
\G_\xi:\R\to\R,\qquad
\G_\xi(x):=\left\lbrace\begin{array}{cl}
(1-\xi)\cdot x, &\falls x<\xi\\
\xi\cdot(1-x), &\falls x\geq\xi
\end{array}\right.\qquad\forall\xi\in\R.
\end{align*}
Sei $v\in H_0^1((0,1))$ beliebig. Dann gilt:
\begin{align*}
\G_\xi'(x)=\left\lbrace\begin{array}{cl}
(1-\xi), &\falls x<\xi\\
-\xi, &\falls x\geq\xi
\end{array}\right.\qquad\forall\xi\in\R.
\end{align*}
und damit
\begin{align*}
\int\limits_0^1\G_\xi'(x)\cdot v'(x)\d x
&=\int\limits_0^\xi\G_\xi'(x)\cdot v'(x)\d x+\int\limits_\xi^1\G_\xi'(x)\cdot v'(x)\d x\\
&=\int\limits_0^\xi (1-\xi)\cdot v'(x)\d x+\int\limits_\xi^1-\xi\cdot v'(x)\d x\\
&=(1-\xi)\cdot\int\limits_0^\xi v'(x)\d x-\xi\cdot\int\limits_\xi^1 v'(x)\d x\\
&\stackeq{\text{HS}}
(1-\xi)\cdot\big(v(\xi)-\underbrace{v(0)}_{=0}\big)-\xi\cdot\big(\underbrace{v(1)}_{=0}-v(\xi)\big)\\
&=v(\xi)
\end{align*}
Schwache Formulierung $u,v\in H_0^1\big((0,1)\big)=V$: Multiplikation mit $v$ und Integration liefert
\begin{align*}
-\int\limits_0^1 u''(x)\cdot v(x)\d x=\int\limits_0^1 f(x)\cdot v(x)\d x
\end{align*}
Setze (partielle Integration)
\begin{align*}
a(u,v)&:=\int\limits_0^1 u'\cdot v'\d x=\int\limits_0^1 f\cdot v=:f(v)\qquad\forall v\in V
\end{align*}
Nun nutzen wir den diskreten Raum $V_h\subseteq V$ anstelle von $V$, wobei $V_h$ der Raum der linearen Finiten-Elemente ist:
\begin{align*}
a(u_h,v_h)=l(v_2)\qquad\forall v_h\in V_h\subseteq V
\end{align*}
%TODO Hier könnte man eine Skizze einfügen (optional)
Nun nutzen wir die \ul{Galerkin-Orthogonalität}:
\begin{align*}
a(u-u_h,v_h)
&=l(v_h)-l(v_h)=0\\
\implies 0
&=a(u-u_h,v_h)
=\int\limits_0^1 v_h'\cdot(u-u_h)'\d x
\end{align*}
Betrachte die diskrete Testfunktion $v_h:=\G_{x_i}$.
\begin{align*}
v_h=\G_{x_i}\in V_h\qquad\forall x_i\\
\implies
0=a(u-u_h,v_h)
&=\int\limits_0^1\G_{x_i}\cdot(u-u_h)'\d x=(u-u_h)(x_i)
\end{align*}
\end{proof}

\section*{Aufgabe 4.2}
Seien $\hat{K}:=(-1,1)^2$ das (große) Einheitsquadrat und $\hat{\Sigma}:=\left\lbrace\hat{N}_1,\hat{N}_2,\hat{N}_3,\hat{N}_4\right\rbrace$ mit den Knotenfunktionalen 
\begin{align*}
\hat{N}_1(\hat{v}):=\hat{v}(0,-1),\quad
\hat{N}_2(\hat{v}):=\hat{v}(1,0),\quad
\hat{N}_3(\hat{v}):=\hat{v}(0,1),\quad
\hat{N}_4(\hat{v}):=\hat{v}(-1,0)
\end{align*}
gegeben. Weiterhin seien 
\begin{align*}
\hat{V}_1:=\hat{Q}_1:=\spann\big(1,\hat{x},\hat{y},\hat{x}\cdot\hat{y}\big)
\end{align*}
der Raum der bilinearen Funktionen und
\begin{align*}
\hat{V}_2:=\hat{Q}_1^{\text{rot}}:=\spann\big(1,\hat{x},\hat{y},\hat{x}^2-\hat{y}^2\big)
\end{align*}
der Raum der rotiert bilinearen Funktionen.
\begin{enumerate}[label=(\alph*)]
\item $\hat{\Sigma}$ ist bzgl. $\hat{V}_1$ nicht unisolvent.
\item $\hat{\Sigma}$ ist bzgl. $\hat{V}_2$ unisolvent.
\item Bestimmen Sie die (nodalen) Basisfunktionen $\hat{\varphi}_i,i=1,\ldots,4$ für $\hat{\Sigma}$ und $\hat{V}_2$.
\end{enumerate}
\begin{proof}
\underline{Zeige (a):}\\
\begin{align*}
N_1(\hat{x}\cdot\hat{y})&=(\hat{x}\cdot\hat{y})(0,-1)=0\\
\implies N_i(\hat{x}\cdot\hat{y})&=0\qquad\forall i\in\lbrace1,2,3,4\rbrace
\end{align*}
Also ist $\hat{Q}_1$ nicht unisolvent für $\hat{\Sigma}$.\\

\underline{Zeige (b):}\\
\begin{tabular}{c|cccc}
& 1 & $\hat{x}$ & $\hat{y}$ & $\hat{x}^2-\hat{y}^2$\\ \hline
$N_1$ & 1 & 0 & -1 & -1\\
$N_2$ & 1 & 1 & 0 & 1\\
$N_3$ & 1 & 0 & 1 & -1\\
$N_4$ & 1 & -1 & 0 & 1
\end{tabular}
Diese Matrix hat Determinante $\det(A)=8\neq0$.\\

\underline{Zeige (c):} 
Idee: $N_i(\hat{\varphi})=\delta_{i,j}$.
\begin{align*}
\hat{\varphi}_1&=\frac{1}{4}\cdot\left(1-\left(\hat{x}^2-\hat{y}^2\right)-2\cdot y\right)\\
\hat{\varphi}_2&=\frac{1}{4}\cdot\left(1+\left(\hat{x}^2-\hat{y}^2\right)+2\cdot x\right)\\
\end{align*}
\textbf{Alternative vom Tutor:}\\
Benutze erneut obige Matrix:
\begin{align*}
M:=\begin{pmatrix}
1 & 0 & -1 & -1\\
1 & 1 & 0 & 1\\
 1 & 0 & 1 & -1\\
 1 & -1 & 0 & 1
\end{pmatrix}^T
\end{align*}
Auch hier ist die Grundidee wieder 
\begin{align*}
\hat{N}_j(\hat{\varphi}_i)&=\delta_{i,j},\qquad
\hat{b}_k=\sum\limits_{i=1}^4 w_i^k\cdot\hat{\varphi}_i\\
\implies
\sum\limits_{j=1}^4\hat{N}_j(\hat{b}_k)\cdot\hat{\varphi}_j
&=\sum\limits_{j=1}^4\sum\limits_{i=1}^4 w_i^k\cdot\underbrace{\hat{N}_j(\hat{\varphi}_i)}_{=\delta_{i,j}}\cdot\hat{\varphi}_j\\
&=\sum\limits_{j=1}^4 w_j^k\cdot\hat{\varphi}_j\\
&=\hat{b}_k\\
\implies
M\cdot\begin{pmatrix}
\hat{\varphi}_1\\ \hat{\varphi}_2 \\ \hat{\varphi}_3 \\ \hat{\varphi}_4
\end{pmatrix}
&=\begin{pmatrix}
\hat{b}_1\\ \hat{b}_2 \\ \hat{b}_3 \\ \hat{b}_4
\end{pmatrix}
=\begin{pmatrix}
1\\ \hat{x} \\ \hat{y} \\\hat{x}^2-\hat{y}^2
\end{pmatrix}\\
M^{-1}&=\begin{pmatrix}
\frac{1}{4} & 0 & -\frac{1}{2} & -\frac{1}{4}\\
\frac{1}{4} & \frac{1}{2} & 0 & \frac{1}{4}\\
\frac{1}{4} & 0 & \frac{1}{2} & -\frac{1}{4}\\
\frac{1}{4} & -\frac{1}{2} & 0 & \frac{1}{4}
\end{pmatrix}\\
\implies\begin{pmatrix}
\hat{\varphi}_1\\ \hat{\varphi}_2 \\ \hat{\varphi}_3 \\ \hat{\varphi}_4
\end{pmatrix}&=M^{-1}\cdot\begin{pmatrix}
\hat{b}_1\\ \hat{b}_2 \\ \hat{b}_3 \\ \hat{b}_4
\end{pmatrix}
\end{align*}
	
\end{proof}

\section*{Aufgabe 4.3}
Seien $\hat{K}:=(0,1)^2$ mit den Ecken 
\begin{align*}
\hat{P}_1=(0,0),\qquad
\hat{P}_2(1,0),\qquad
\hat{P}_3(1,1),\qquad
\hat{P}_4(0,1)
\end{align*}
das (kleine) Einheitsquadrat und $K$ mit den Ecken
\begin{align*}
P_1(x_1,y_1),\qquad
P_2(x_2,y_2),\qquad
P_1(x_3,y_3),\qquad
P_1(x_4,y_4)
\end{align*}
ein beliebiges Viereck.
\begin{enumerate}[label=(\alph*)]
\item Bestimmen Sie die Koeffizienten der bilinearen Abbildung $F:\hat{K}\to K$, die $\hat{P}_i$ auf $P_i$, $i=1,\ldots,4$ abbildet. Die Abbildung $F$ heißt \textbf{bilinear} $:\gdw F\in\left(\hat{Q}_1\right)^2$.
\item Zeigen Sie, dass die Determinante der Jacobi-Matrix $DF(\hat{x},\hat{y})$ der Abbildung $F$ eine affine Funktion ist.
\item Unter welchen Bedingungen an die Koeffizienten von $F$ stellt $F$ eine affine Abbildung dar? Deuten Sie diese Bedingungen geometrisch.
\end{enumerate}

\begin{lösung}
%TODO Hier könnte man eine Skizze einfügen.
\underline{Zu (a):}\\
Bilinear bedeutet von folgender Form:
\begin{align*}
F(\hat{x},\hat{y})&\begin{pmatrix}
b_1\\ b_2
\end{pmatrix}
+\begin{pmatrix}
b_{1,1} & b_{1,2}\\
b_{2,1} & b_{2,2}
\end{pmatrix}\cdot\begin{pmatrix}
\hat{x}\\\hat{y}
\end{pmatrix}+\begin{pmatrix}
\delta_1\\\delta_2
\end{pmatrix}\cdot\hat{x}\cdot\hat{y}\\
\begin{pmatrix}
x_1\\ y_1
\end{pmatrix}
&=P_1=F(\hat{P}_1)=F(0,0)=\begin{pmatrix}
b_1\\ b_2
\end{pmatrix}\\
\implies
\begin{pmatrix}
b_1\\ b_2
\end{pmatrix}&=
\begin{pmatrix}
x_1\\ y_1
\end{pmatrix},\qquad\begin{pmatrix}
b_{1,1} & b_{1,2}\\
b_{2,1} & b_{2,2}
\end{pmatrix}=\Big ( P_2-P_1\quad P_4-P_1  \Big )
\end{align*}

\underline{Zeige (b):} Nachrechnen.\\

\underline{Zu (c):}
\begin{align*}
F\text{ affin}&\Longrightarrow\begin{pmatrix}
\delta_1\\\delta_2
\end{pmatrix}=0=\Big (P_1-P_2+P_3-P_4\Big )\\
&\Longleftrightarrow P_1-P_2=P_4-P_3\\
&\Longleftrightarrow P_1-P_4=P_2-P_3\\
&\implies K\text{ ist ein Parallelogramm}
\end{align*}
\end{lösung}

\section*{Aufgabe 4.4}
Für ein achsenparalleles Rechteck $K$ sei $Q_2$ die Menge der Polynome auf $K$, die bezüglich $x$ und $y$ höchstens zweiten Grades sind. Es seien $(e_j)_{j=1}^4$ die vier Kanten und $(x_j)_{j=1}^4$ die vier Ecken von $K$. Untersuchen Sie, ob durch die neun Freiheitsgrade
\begin{enumerate}[label=(\roman*)]
\item Funktionswerte $f(x_j)$ in den Ecken,
\item $\int\limits_{e_j} f\d s$ und $\int\limits_K f\d(x,y)$
\end{enumerate}
einer Funktion $f\in Q_2$ die Unisolvenz sichergestellt wird.

\begin{lösung}
\underline{Fall 1: Referenzquadrat}\\
Wir untersuchen die Aussage zunächst für das Referenzquadrat $\hat{K}:=(0,1)^2$.
\begin{align*}
Q_2=\spann\left\lbrace 1,x,y,x\cdot y,x^2,y^2,x\cdot y^2,x^2\cdot y,x^2\cdot y^2\right\rbrace
\end{align*}
%TODO Hier Skizze einfügen
Es genügt zu zeigen, dass\\
``Wenn $v\in Q_2$ für alle $\d\circ F$ verschwindet, folgt $v\equiv0$.``\\
Auf jeder Kante $e_j$, $v$ ist ein Polynom zweiten Grades, also
\begin{align*}
v=c_j\cdot(1-s)\cdot s
\end{align*}
da $v$ in den Ecken verschwindet. Es folgt
\begin{align*}
0\stackeq{!}\int\limits_{e_j} v=c_j\cdot\frac{1}{6}\implies c_j=0
\end{align*}
Also verschwindet $v$ auf den Kanten. Somit ist $v$ 0 auf dem Rand von $K$. Folglich können wir $v$ schreiben als
\begin{align*}
v(x,y)=c\cdot(1-x)\cdot x\cdot(1-y)\cdot y
\stackrel{0\stackeq{!}\int v\d (x,y)=0}{\implies}\cdot\frac{1}{36}\implies c=0
\end{align*}
da dies die einzige Funktion ist, die komplett auf dem Rand verschwindet. Damit ist die Unisolvenz gezeigt.\\

\underline{Fall 2: Allgemeiner Fall}\\
Wir können den allgemeinen Fall auf den Fall des Referenzquadrates (Fall 1) zurückführen.

\end{lösung}

\section*{Aufgabe 4.5}
Seien $\hat{K}:=(0,1)$ und $K:=(a,a+h)$ mit $a\in\R$ und $h>0$ zwei offene Intervalle. Weiterhin sei $F:\hat{K}\to K$ die affine Abbildung, die $F(0)=1$ und $F(1)=a+h$ erfüllt.\\
Dann besteht für $\varphi\in W^{m,p}(K),m\in\N_0,p\in[1,\infty)$ und\\ $\hat{\varphi}:=\varphi\circ F\in W^{m,p}(\hat{K})$ der Zusammenhang
\begin{align*}
\big|\hat{\varphi}\big|_{m,p,\hat{K}}=h^m\cdot \left(\frac{1}{h}\right)^{\frac{1}{p}}\cdot|\varphi|_{m,p,K}.
\end{align*}
\begin{proof}
%TODO Hier könnte man eine Skizze einfügen
Setze
\begin{align*}
F:\hat{K}&=(0,1)\to K=(a,a+h),\qquad F(\hat{x}):=a+\hat{x}\cdot h\\
\hat{\varphi}&=\varphi\big(F(\hat{x})\big)\\
\implies
\frac{\partial\hat{\varphi}(\hat{x})}{\partial\hat{x}}&\stackeq{\text{KR}}\frac{\partial\varphi\big(F(\hat{x})\big)}{\partial x}\cdot\underbrace{\frac{\partial F(\hat{x})}{\partial\hat{ x}}}_{=h}
\implies
\hat{\varphi}^{(m)}(\hat{x})=\varphi^{(m)}\big(F(\hat{x})\big)\cdot h^m\\
\big|\hat{\varphi}\big|_{m,p,\hat{K}}
&=\int\limits_{\hat{K}}\left|\hat{\varphi}^{(m)}(\hat{x})\right|^p\d x\\
&=\int\limits_{\hat{K}}\left|\varphi^{(m)}\big(F(\hat{x})\big)\right|^p\cdot h^{m\cdot p}\d\hat{x}\\
&\stackeq{(\ast)}
\int\limits_K\left|\varphi^{(m)}(x)\right|^p\cdot h^{m\cdot p}\cdot\frac{1}{h}\d x\\
&=h^{m\cdot p}\cdot\frac{1}{h}\cdot|\varphi|^p_{m,p,K}
\end{align*}
$(\ast)$: Integration durch Substitution: $\hat{x}=\frac{x-a}{h}$, $\d\hat{x}=\frac{1}{h}\d x$.\\
Das Ziehen der $p$-ten Wurzel liefert die Behauptung.
\end{proof}
\end{document}