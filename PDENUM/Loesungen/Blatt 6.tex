% This work is licensed under the Creative Commons
% Attribution-NonCommercial-ShareAlike 4.0 International License. To view a copy
% of this license, visit http://creativecommons.org/licenses/by-nc-sa/4.0/ or
% send a letter to Creative Commons, PO Box 1866, Mountain View, CA 94042, USA.

\documentclass[12pt,a4paper]{article} 

% This work is licensed under the Creative Commons
% Attribution-NonCommercial-ShareAlike 4.0 International License. To view a copy
% of this license, visit http://creativecommons.org/licenses/by-nc-sa/4.0/ or
% send a letter to Creative Commons, PO Box 1866, Mountain View, CA 94042, USA.

% PACKAGES
\usepackage[english, ngerman]{babel}	% Paket für Sprachselektion, in diesem Fall für deutsches Datum etc
\usepackage[utf8]{inputenc}	% Paket für Umlaute; verwende utf8 Kodierung in TexWorks 
\usepackage[T1]{fontenc} % ö,ü,ä werden richtig kodiert
\usepackage{amsmath} % wichtig für align-Umgebung
\usepackage{amssymb} % wichtig für \mathbb{} usw.
\usepackage{amsthm} % damit kann man eigene Theorem-Umgebungen definieren, proof-Umgebungen, etc.
\usepackage{mathrsfs} % für \mathscr
\usepackage[backref]{hyperref} % Inhaltsverzeichnis und \ref-Befehle werden in der PDF-klickbar
\usepackage[english, ngerman, capitalise]{cleveref}
\usepackage{graphicx}
\usepackage{grffile}
\usepackage{setspace} % wichtig für Lesbarkeit. Schöne Zeilenabstände

\usepackage{enumitem} % für custom Liste mit default Buchstaben
\usepackage{ulem} % für bessere Unterstreichung
\usepackage{contour} % für bessere Unterstreichung
\usepackage{epigraph} % für das coole Zitat

\usepackage{tikz}

% This work is licensed under the Creative Commons
% Attribution-NonCommercial-ShareAlike 4.0 International License. To view a copy
% of this license, visit http://creativecommons.org/licenses/by-nc-sa/4.0/ or
% send a letter to Creative Commons, PO Box 1866, Mountain View, CA 94042, USA.

% THEOREM-ENVIRONMENTS

\newtheoremstyle{mystyle}
  {20pt}   % ABOVESPACE \topsep is default, 20pt looks nice
  {20pt}   % BELOWSPACE \topsep is default, 20pt looks nice
  {\normalfont} % BODYFONT
  {0pt}       % INDENT (empty value is the same as 0pt)
  {\bfseries} % HEADFONT
  {}          % HEADPUNCT (if needed)
  {5pt plus 1pt minus 1pt} % HEADSPACE
	{}          % CUSTOM-HEAD-SPEC
\theoremstyle{mystyle}

% Definitionen der Satz, Lemma... - Umgebungen. Der Zähler von "satz" ist dem "section"-Zähler untergeordnet, alle weiteren Umgebungen bedienen sich des satz-Zählers.
\newtheorem{satz}{Satz}[section]
\newtheorem{lemma}[satz]{Lemma}
\newtheorem{korollar}[satz]{Korollar}
\newtheorem{proposition}[satz]{Proposition}
\newtheorem{beispiel}[satz]{Beispiel}
\newtheorem{definition}[satz]{Definition}
\newtheorem{bemerkungnr}[satz]{Bemerkung}
\newtheorem{theorem}[satz]{Theorem}

% Bemerkungen, Erinnerungen und Notationshinweise werden ohne Numerierungen dargestellt.
\newtheorem*{bemerkung}{Bemerkung.}
\newtheorem*{erinnerung}{Erinnerung.}
\newtheorem*{notation}{Notation.}
\newtheorem*{aufgabe}{Aufgabe.}
\newtheorem*{lösung}{Lösung.}
\newtheorem*{beisp}{Beispiel.} %Beispiel ohne Nummerierung
\newtheorem*{defi}{Definition.} %Definition ohne Nummerierung
\newtheorem*{lem}{Lemma.} %Lemma ohne Nummerierung


% SHORTCUTS
\newcommand{\R}{\mathbb{R}}				 % reelle Zahlen
\newcommand{\Rn}{\R^n}						 % der R^n
\newcommand{\N}{\mathbb{N}}				 % natürliche Zahlen
\newcommand{\Z}{\mathbb{Z}}				 % ganze Zahlen
\newcommand{\C}{\mathbb{C}}			   % komplexe Zahlen
\newcommand{\gdw}{\Leftrightarrow} % Genau dann, wenn
\newcommand{\with}{\text{ mit }}   % mit
\newcommand{\falls}{\text{falls }} % falls
\newcommand{\dd}{\text{ d}}        % Differential d

% ETWAS SPEZIELLERE ZEICHEN
%disjoint union
\newcommand{\bigcupdot}{
	\mathop{\vphantom{\bigcup}\mathpalette\setbigcupdot\cdot}\displaylimits
}
\newcommand{\setbigcupdot}[2]{\ooalign{\hfil$#1\bigcup$\hfil\cr\hfil$#2$\hfil\cr\cr}}
%big times
\newcommand*{\bigtimes}{\mathop{\raisebox{-.5ex}{\hbox{\huge{$\times$}}}}} 

% WHITESPACE COMMANDS
%non-restrict newline command
\newcommand{\enter}{$ $\newline} 
%praktischer Tabulator
\newcommand\tab[1][1cm]{\hspace*{#1}}

% TEXT ÜBER ZEICHEN
%das ist ein Gleichheitszeichen mit Text darüber, Beispiel: $a\stackeq{Def} b$
\newcommand{\stackeq}[1]{
	\mathrel{\stackrel{\makebox[0pt]{\mbox{\normalfont\tiny #1}}}{=}}
} 
%das ist ein beliebiges Zeichen mit Text darüber, z. B.  $a\stackrel{Def}{\Rightarrow} b$
\newcommand{\stacksymbol}[2]{
	\mathrel{\stackrel{\makebox[0pt]{\mbox{\normalfont\tiny #1}}}{#2}}
} 

% UNDERLINE
% besseres underline 
\renewcommand{\ULdepth}{1pt}
\contourlength{0.5pt}
\newcommand{\ul}[1]{
	\uline{\phantom{#1}}\llap{\contour{white}{#1}}
}


% hier noch ein paar Commands die nur ich nutze, weil ich sie mir im Laufe der Jahre angewöhnt habe und sie mir jetzt nicht abgewöhnen will:

\newcommand{\gdw}{\Leftrightarrow}   % genau dann, wenn



% This work is licensed under the Creative Commons
% Attribution-NonCommercial-ShareAlike 4.0 International License. To view a copy
% of this license, visit http://creativecommons.org/licenses/by-nc-sa/4.0/ or
% send a letter to Creative Commons, PO Box 1866, Mountain View, CA 94042, USA.

\renewcommand{\div}{\text{ div}}      % Divergenz
\newcommand{\laplace}{\triangle}   % Laplace Operator
\newcommand{\Vertiii}[1]{{\left\vert\kern-0.25ex\left\vert\kern-0.25ex\left\vert #1 
    \right\vert\kern-0.25ex\right\vert\kern-0.25ex\right\vert}}
\newcommand{\T}{\mathcal{T}} %Triangulierung
\newcommand{\meas}{\text{meas}} % Das Maß einer Menge, meist Lebesguemaß


\author{Willi Sontopski}

\parindent0cm %Ist wichtig, um führende Leerzeichen zu entfernen

\usepackage{scrpage2}
\pagestyle{scrheadings}
\clearscrheadfoot

\ihead{Willi Sontopski}
\chead{PDENM WiSe 18 19}
\ohead{}
\ifoot{Blatt 6}
\cfoot{Version: \today}
\ofoot{Seite \pagemark}


\begin{document}
%\setcounter{section}{1}

Hier nur die Lösungen. Aufgabenstellungen siehe Vorlesungswebsite.

\section*{Aufgabe 6.1}
Wozu brauchen wir Quasi-Interpolation?
%TODO Antwort ergänzen
\subsection*{Aufgabe 6.1 (a)}
%TODO Evtl. Skizze einfügen

\begin{align*}
	R_h v&=\sum\limits_{Z\in N_h}\Big(\pi_Z v)\varphi_Z\\
	\pi_Z:&=\frac{1}{|w_z|}\int\limits_{w_z} v
\end{align*}
%TODO hier fehlt evtl. was
\begin{align*}
	&=\underbrace{\pi_{(0,0)}v}_{=\frac{1}{|\Omega|}\int\limits_\Omega v}\underbrace{\varphi_{(0,0)}}_{=v}\\
	\overset{\text{Symmetrie}}&=
	\frac{1}{2\cdot 2}8\underbrace{\int\limits_{\tilde{T}}v\d(x,y)}_{=\int\limits_{x=0}^1\int\limits_{y=0}^x 1-x\d \d x=\frac{1}{6}}\\
	&=\frac{2}{6}v=\frac{1}{3}v
\end{align*}

Originale Clément-Intrepolation: "Summe über alle Knoten"
\begin{align*}
	\frac{1}{w_z^2}\int\limits_{w_z^2} v\d x
	=\frac{1}{1\cdot 1}2\cdot\underbrace{\int\limits_{\tilde{T}}v\d(x,y)}_{=\frac{1}{6}}\\
	\implies R_h v=\frac{1}{3}\underbrace{\sum\limits_{z\in l_2^1}\varphi_z}_{\equiv1}
\end{align*}

\subsection*{Aufgabe 6.1 (b)}
Scott-Zhang-Approximation
%TODO Hier könnte man eine Skizze einfügen
\begin{align*}
	\int\limits_{E_z}\psi_{z,j}(a)\varphi_{z,m}(s)\d s
	&=\delta_{j,k}\qquad\forall j,k\in\lbrace1,2\rbrace\\
	\int\limits_{E_z}\psi_z(s)\varphi_z(s)\d s
	&=\left\lbrace\begin{array}{cl}
		1, &\falls z=z'\\
		0, &\falls z'\in N_h,z'\neq z
	\end{array}\right.\\
	SZ_h v(x)
	\sum\limits_{Z\in N_h}\varphi_z(x)\int\limits_{E_z}\psi_z(s)v(s)\d s
\end{align*}
%$N_h$ ist die Knotenmenge
$v\in V_h$ (Finite-Elemente-Raum)
\begin{align*}
	v(x)=\sum\limits_{z\in N_h} v_i(z)\varphi_z(x)
\end{align*}
Also
\begin{align*}
	\int\limits_{E_z}\psi_z(s)v(s)\d s
	=\sum\limits_{z'\in N_h} v_i(z')\int\limits_{E_z}\psi_z(s)\varphi_{z'}(s)\d s
	=v(z)\\
	\implies
	SZ_h v(x)=v(x)
\end{align*}

$SZ_h v\in V_h\cap H_0^1(\Omega)$ falls $v\in H_0^1(\Omega)$\\
$SZ_h v=v$ falls $v\in V_h$

\begin{align*}
	\left.\begin{array}{l}
			\int\limits_0^1\overbrace{\psi_z(s)}^{a+bs}(s)\d s=0\\
		\int\limits_0^1\psi_z(s)(1-s)\d s=1
	\end{array}\right\rbrace\qquad
	\psi_z(s)=2(2-3s)
\end{align*}

\section*{Aufgabe 6.2}
\subsection*{Aufgabe 6.2 (a)}
\begin{align*}
	b_T&=\left\lbrace\begin{array}{cl}
		27, &\falls \lambda_{T,1}\lambda_{T,2}\lambda_{T,3},x\in T\\
		0, &\sonst
	\end{array}\right.\\
	\lambda_{T_i}&\geq 0\qquad x\in T\qquad 0\leq b_T\\
	\lambda_{T_i}&0\frac{1}{3}\qquad i=1,2,3\\
	b_T(x)&=27\cdot\left(\frac{1}{3}\right)^3=1\\
	x\in\partial T&\implies b_T(x)
\end{align*}
Zweite Funktion:
\begin{align*}
	b_E=\left\lbrace\begin{array}{cl}
		4, &\falls\lambda_{T,1}\lambda_{T,2},e\in\text{edge}\\
		0, &\sonst
	\end{array}\right.\\
	x\in\partial\omega\implies b_E=0\qquad 0\leq b_E\leq 1\\
	x=\frac{1}{2}n_1+\frac{1}{2}n_2\implies b_E(x)=1
\end{align*}

\subsection*{Aufgabe 6.2 (b)}
%TODO Ich habe keinen Bock mehr...
\subsubsection*{Aufgabe 6.2 (b) (i)}


\subsubsection*{Aufgabe 6.2 (b) (ii)}


\subsubsection*{Aufgabe 6.2 (b) (iii)}

\section*{Aufgabe 6.3}
\subsection*{Aufgabe 6.3 (a)}

\subsection*{Aufgabe 6.3 (b)}

\section*{Aufgabe (Zusatz)}
\subsection*{Aufgabe (Zusatz) (a)}

\subsection*{Aufgabe (Zusatz) (b)}

\end{document}