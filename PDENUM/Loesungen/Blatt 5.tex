% This work is licensed under the Creative Commons
% Attribution-NonCommercial-ShareAlike 4.0 International License. To view a copy
% of this license, visit http://creativecommons.org/licenses/by-nc-sa/4.0/ or
% send a letter to Creative Commons, PO Box 1866, Mountain View, CA 94042, USA.

\documentclass[12pt,a4paper]{article} 

% This work is licensed under the Creative Commons
% Attribution-NonCommercial-ShareAlike 4.0 International License. To view a copy
% of this license, visit http://creativecommons.org/licenses/by-nc-sa/4.0/ or
% send a letter to Creative Commons, PO Box 1866, Mountain View, CA 94042, USA.

% PACKAGES
\usepackage[english, ngerman]{babel}	% Paket für Sprachselektion, in diesem Fall für deutsches Datum etc
\usepackage[utf8]{inputenc}	% Paket für Umlaute; verwende utf8 Kodierung in TexWorks 
\usepackage[T1]{fontenc} % ö,ü,ä werden richtig kodiert
\usepackage{amsmath} % wichtig für align-Umgebung
\usepackage{amssymb} % wichtig für \mathbb{} usw.
\usepackage{amsthm} % damit kann man eigene Theorem-Umgebungen definieren, proof-Umgebungen, etc.
\usepackage{mathrsfs} % für \mathscr
\usepackage[backref]{hyperref} % Inhaltsverzeichnis und \ref-Befehle werden in der PDF-klickbar
\usepackage[english, ngerman, capitalise]{cleveref}
\usepackage{graphicx}
\usepackage{grffile}
\usepackage{setspace} % wichtig für Lesbarkeit. Schöne Zeilenabstände

\usepackage{enumitem} % für custom Liste mit default Buchstaben
\usepackage{ulem} % für bessere Unterstreichung
\usepackage{contour} % für bessere Unterstreichung
\usepackage{epigraph} % für das coole Zitat

\usepackage{tikz}

% This work is licensed under the Creative Commons
% Attribution-NonCommercial-ShareAlike 4.0 International License. To view a copy
% of this license, visit http://creativecommons.org/licenses/by-nc-sa/4.0/ or
% send a letter to Creative Commons, PO Box 1866, Mountain View, CA 94042, USA.

% THEOREM-ENVIRONMENTS

\newtheoremstyle{mystyle}
  {20pt}   % ABOVESPACE \topsep is default, 20pt looks nice
  {20pt}   % BELOWSPACE \topsep is default, 20pt looks nice
  {\normalfont} % BODYFONT
  {0pt}       % INDENT (empty value is the same as 0pt)
  {\bfseries} % HEADFONT
  {}          % HEADPUNCT (if needed)
  {5pt plus 1pt minus 1pt} % HEADSPACE
	{}          % CUSTOM-HEAD-SPEC
\theoremstyle{mystyle}

% Definitionen der Satz, Lemma... - Umgebungen. Der Zähler von "satz" ist dem "section"-Zähler untergeordnet, alle weiteren Umgebungen bedienen sich des satz-Zählers.
\newtheorem{satz}{Satz}[section]
\newtheorem{lemma}[satz]{Lemma}
\newtheorem{korollar}[satz]{Korollar}
\newtheorem{proposition}[satz]{Proposition}
\newtheorem{beispiel}[satz]{Beispiel}
\newtheorem{definition}[satz]{Definition}
\newtheorem{bemerkungnr}[satz]{Bemerkung}
\newtheorem{theorem}[satz]{Theorem}

% Bemerkungen, Erinnerungen und Notationshinweise werden ohne Numerierungen dargestellt.
\newtheorem*{bemerkung}{Bemerkung.}
\newtheorem*{erinnerung}{Erinnerung.}
\newtheorem*{notation}{Notation.}
\newtheorem*{aufgabe}{Aufgabe.}
\newtheorem*{lösung}{Lösung.}
\newtheorem*{beisp}{Beispiel.} %Beispiel ohne Nummerierung
\newtheorem*{defi}{Definition.} %Definition ohne Nummerierung
\newtheorem*{lem}{Lemma.} %Lemma ohne Nummerierung


% SHORTCUTS
\newcommand{\R}{\mathbb{R}}				 % reelle Zahlen
\newcommand{\Rn}{\R^n}						 % der R^n
\newcommand{\N}{\mathbb{N}}				 % natürliche Zahlen
\newcommand{\Z}{\mathbb{Z}}				 % ganze Zahlen
\newcommand{\C}{\mathbb{C}}			   % komplexe Zahlen
\newcommand{\gdw}{\Leftrightarrow} % Genau dann, wenn
\newcommand{\with}{\text{ mit }}   % mit
\newcommand{\falls}{\text{falls }} % falls
\newcommand{\dd}{\text{ d}}        % Differential d

% ETWAS SPEZIELLERE ZEICHEN
%disjoint union
\newcommand{\bigcupdot}{
	\mathop{\vphantom{\bigcup}\mathpalette\setbigcupdot\cdot}\displaylimits
}
\newcommand{\setbigcupdot}[2]{\ooalign{\hfil$#1\bigcup$\hfil\cr\hfil$#2$\hfil\cr\cr}}
%big times
\newcommand*{\bigtimes}{\mathop{\raisebox{-.5ex}{\hbox{\huge{$\times$}}}}} 

% WHITESPACE COMMANDS
%non-restrict newline command
\newcommand{\enter}{$ $\newline} 
%praktischer Tabulator
\newcommand\tab[1][1cm]{\hspace*{#1}}

% TEXT ÜBER ZEICHEN
%das ist ein Gleichheitszeichen mit Text darüber, Beispiel: $a\stackeq{Def} b$
\newcommand{\stackeq}[1]{
	\mathrel{\stackrel{\makebox[0pt]{\mbox{\normalfont\tiny #1}}}{=}}
} 
%das ist ein beliebiges Zeichen mit Text darüber, z. B.  $a\stackrel{Def}{\Rightarrow} b$
\newcommand{\stacksymbol}[2]{
	\mathrel{\stackrel{\makebox[0pt]{\mbox{\normalfont\tiny #1}}}{#2}}
} 

% UNDERLINE
% besseres underline 
\renewcommand{\ULdepth}{1pt}
\contourlength{0.5pt}
\newcommand{\ul}[1]{
	\uline{\phantom{#1}}\llap{\contour{white}{#1}}
}


% hier noch ein paar Commands die nur ich nutze, weil ich sie mir im Laufe der Jahre angewöhnt habe und sie mir jetzt nicht abgewöhnen will:

\newcommand{\gdw}{\Leftrightarrow}   % genau dann, wenn



% This work is licensed under the Creative Commons
% Attribution-NonCommercial-ShareAlike 4.0 International License. To view a copy
% of this license, visit http://creativecommons.org/licenses/by-nc-sa/4.0/ or
% send a letter to Creative Commons, PO Box 1866, Mountain View, CA 94042, USA.

\renewcommand{\div}{\text{ div}}      % Divergenz
\newcommand{\laplace}{\triangle}   % Laplace Operator
\newcommand{\Vertiii}[1]{{\left\vert\kern-0.25ex\left\vert\kern-0.25ex\left\vert #1 
    \right\vert\kern-0.25ex\right\vert\kern-0.25ex\right\vert}}
\newcommand{\T}{\mathcal{T}} %Triangulierung
\newcommand{\meas}{\text{meas}} % Das Maß einer Menge, meist Lebesguemaß


\author{Willi Sontopski}

\parindent0cm %Ist wichtig, um führende Leerzeichen zu entfernen

\usepackage{scrpage2}
\pagestyle{scrheadings}
\clearscrheadfoot

\ihead{Willi Sontopski}
\chead{PDENM WiSe 18 19}
\ohead{}
\ifoot{Blatt 5}
\cfoot{Version: \today}
\ofoot{Seite \pagemark}

\newcommand{\G}{\mathcal{G}}

\begin{document}
%\setcounter{section}{1}

\section*{Aufgabe 5.1}
Sei $K=(a,a+h)$ mit $a\in\R$ und $h>0$ ein offenes Intervall.

\subsection*{Aufgabe 5.1(a)}
Es gilt für alle $v\in P_k(V)$ und $0\leq m\leq l$ sowie $p,q\in[1,\infty]$ die inverse Ungleichung
\begin{align*}
|v|_{l,p,K}\leq c\cdot h^{m-l+\left(\frac{1}{p}-\frac{1}{q}\right)}\cdot|v|_{m,q,K}
\end{align*}
wobei $C$ eine Konstante ist, die nicht von $K$ abhängt (und damit insbesondere auch nicht von $h$).

\begin{proof}
Nutze Transformation auf Referenzelement $\hat{K}=(0,1)$, also
\begin{align*}
f:\R\to\R,\qquad x\mapsto h\cdot x+a
\end{align*}
%TODO
\end{proof}

\subsection*{Aufgabe 5.1 (b)}
Wie muss die rechte Seite angepasst werden, wenn die volle $W^{l,p}$-Norm in ähnlicher Weise abgeschätzt werden soll?

\begin{lösung}
%TODO
\end{lösung}

\section*{Aufgabe 5.2}
Im Rahmen nichtkonformer Finite Elemente soll die folgende Verallgemeinerung des Dualitätsarguments von Aubin-Nitsche untersucht werden.\\
Sei $H$ ein Hilbertraum mit Norm $\Vert\cdot\Vert_H$ und Skalarprodukt $(\cdot,\cdot)$. Ferner sei $V_h\subseteq H$ abgeschlossen mit Norm $\Vert\cdot\Vert_h$ und $V$ ein Hilbertraum, welcher stetig in $H$ eingebettet ist $(V\hookrightarrow H$). Die auf $V+V_h$ definierte und stetige Bilinearform $a_h$ stimme auf $V$ mit $a$ überein. Dann gilt:
\begin{align*}
\Vert u-u_h\Vert\leq\sup\limits_{g\in H\setminus\lbrace0\rbrace}\frac{1}{\Vert g\Vert_H}\cdot\Big( &M\cdot\Vert u-u_h\Vert_h\cdot\Vert\varphi_g-\varphi_{g,h}\Vert_h\\
&+\big|a_h(u-u_h,\varphi_g)-(u-u_h,g)\big|\\
&+\big|a_h(u,\varphi_g-\varphi_{g,h})-(f,\varphi_g-\varphi_{g,h}\big|\Big)
\end{align*}
wobei zu $g\in H$ die Funktionen $\varphi_g\in V$ und $\varphi_{g,h}\in V_h$ als Lösungen von
\begin{align*}
a(w,\varphi_g)=(w,g)\quad\forall w\in V\qquad\text{bzw.}\qquad a_h(w_h,\varphi_{g,h})=(w_h,g)\quad\forall w_h\in V_h
\end{align*}
definiert sind.

\begin{proof}
%TODO
\end{proof}

\section*{Aufgabe 5.3}
Sei $\Omega$ ein konvexes Polyeder. Beweisen Sie für die Diskretisierung der Poisson-Gleichung
\begin{align*}
\left\lbrace\begin{array}{rl}
-\Delta u=f &\text{ in }\Omega\\
u|_{\partial\Omega}=0
\end{array}\right.\mit f\in L^2(\Omega)
\end{align*}
mit Crouzeix-Raviart-Elementen über einer regulären Zerlegung von $\Omega$ die $L^2$-Fehlerabschätzung
\begin{align*}
\Vert u-u_h\Vert_0\leq c\cdot h^2\cdot\Vert u\Vert_2
\end{align*}

\begin{proof}
\underline{Teilschritt (a):}\\
Das Dualitätsargument von Aubin-Nitsche ist anwendbar, denn:\nl
%TODO
\underline{Teilschritt (b):}\\
Aus der Vorlesung ist bekannt:
\begin{align*}
\Vert u-u_h\Vert_h\leq c\cdot h\cdot\Vert u\Vert_2
\end{align*}
Bei der Abschätzung von $\Vert\varphi_g-\varphi_{g,h}\Vert$ hilft, dass $\varphi_g-\varphi_{g,h}$ als Diskretisierungsfehler des Problems $a(w,\varphi)=(w,g)$ angesehen werden kann.\nl
\underline{Teilschritt (c):}\\
Es ist bekannt:
\begin{align*}
\big|L_w(z)\big|:=\Big|a_h(w,z)-(\hat{f},z)\Big|\leq c\cdot h\cdot|w|_2\cdot\Vert z\Vert_h\qquad\forall z\in V_h+v,\forall w\in H^2
\end{align*}
%TODO
\underline{Teilschritt (d):}\\
%TODO
\end{proof}

\section*{Aufgabe 5.4}
Sei $\Omega$ ein konvexes Polyeder. Es erfülle $w\in H^2(\Omega)\cap H_0^1(\Omega)$ die Poisson-Gleichung $-\Delta w=\hat{f}$. Ferner sei für $z\in V_h+V$
\begin{align*}
L_w(z):=a_h(w,z)-(\hat{f},z):=\sum\limits_K\int\limits_K\nabla w\cdot\nabla z\d x-\int\limits_\Omega\hat{f}\cdot z\d x
\end{align*}
wobei $V_h$ den Raum der Crouzeix-Raviart-Elemente über einer regulären Zerlegung von $\Omega$ bezeichne und $V=H_0^1(\Omega)$ sein.

\subsection*{Aufgabe 5.4 (a)}
Es gilt
\begin{align*}
L_w(z)=\sum\limits_K\sum\limits_{E\in\delta K}\int\limits_E\left(\frac{\partial}{\partial n_K}-\frac{\partial w^I}{\partial n_K}\right)\cdot\left(z-\overline{z(E)}\right)\d s
\end{align*}
Dabei bezeichne $w^I\in V_h\cap C^0(\Omega)$ die stetige, stückweise lineare Interpolierende, die $w$ in den Eckpunkten der Dreiecke interpoliert. Auf einer Kante $E$ sei
\begin{align*}
\overline{z(E)}=\frac{1}{|E|}\cdot\int\limits_E z\d s
\end{align*}
der Integralmittelwert einer Funktion $z\in V_h+V$.

\subsection*{Aufgabe 5.4 (b)}
Es gilt die Abschätzung
\begin{align*}
\big|L_w(z)\big|\leq c\cdot h\cdot|w|_2\cdot\Vert z\Vert_h
\end{align*}
\begin{proof}
Wir nutzen in dem Beweis die Spurungleichung in Kombination mit dem Bramble-Hilbert-Lemma verwendet werden.
%TODO
\end{proof}
\end{document}