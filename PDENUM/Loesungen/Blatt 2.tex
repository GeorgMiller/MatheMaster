% This work is licensed under the Creative Commons
% Attribution-NonCommercial-ShareAlike 4.0 International License. To view a copy
% of this license, visit http://creativecommons.org/licenses/by-nc-sa/4.0/ or
% send a letter to Creative Commons, PO Box 1866, Mountain View, CA 94042, USA.

\documentclass[12pt,a4paper]{article} 

% This work is licensed under the Creative Commons
% Attribution-NonCommercial-ShareAlike 4.0 International License. To view a copy
% of this license, visit http://creativecommons.org/licenses/by-nc-sa/4.0/ or
% send a letter to Creative Commons, PO Box 1866, Mountain View, CA 94042, USA.

% PACKAGES
\usepackage[english, ngerman]{babel}	% Paket für Sprachselektion, in diesem Fall für deutsches Datum etc
\usepackage[utf8]{inputenc}	% Paket für Umlaute; verwende utf8 Kodierung in TexWorks 
\usepackage[T1]{fontenc} % ö,ü,ä werden richtig kodiert
\usepackage{amsmath} % wichtig für align-Umgebung
\usepackage{amssymb} % wichtig für \mathbb{} usw.
\usepackage{amsthm} % damit kann man eigene Theorem-Umgebungen definieren, proof-Umgebungen, etc.
\usepackage{mathrsfs} % für \mathscr
\usepackage[backref]{hyperref} % Inhaltsverzeichnis und \ref-Befehle werden in der PDF-klickbar
\usepackage[english, ngerman, capitalise]{cleveref}
\usepackage{graphicx}
\usepackage{grffile}
\usepackage{setspace} % wichtig für Lesbarkeit. Schöne Zeilenabstände

\usepackage{enumitem} % für custom Liste mit default Buchstaben
\usepackage{ulem} % für bessere Unterstreichung
\usepackage{contour} % für bessere Unterstreichung
\usepackage{epigraph} % für das coole Zitat

\usepackage{tikz}

% This work is licensed under the Creative Commons
% Attribution-NonCommercial-ShareAlike 4.0 International License. To view a copy
% of this license, visit http://creativecommons.org/licenses/by-nc-sa/4.0/ or
% send a letter to Creative Commons, PO Box 1866, Mountain View, CA 94042, USA.

% THEOREM-ENVIRONMENTS

\newtheoremstyle{mystyle}
  {20pt}   % ABOVESPACE \topsep is default, 20pt looks nice
  {20pt}   % BELOWSPACE \topsep is default, 20pt looks nice
  {\normalfont} % BODYFONT
  {0pt}       % INDENT (empty value is the same as 0pt)
  {\bfseries} % HEADFONT
  {}          % HEADPUNCT (if needed)
  {5pt plus 1pt minus 1pt} % HEADSPACE
	{}          % CUSTOM-HEAD-SPEC
\theoremstyle{mystyle}

% Definitionen der Satz, Lemma... - Umgebungen. Der Zähler von "satz" ist dem "section"-Zähler untergeordnet, alle weiteren Umgebungen bedienen sich des satz-Zählers.
\newtheorem{satz}{Satz}[section]
\newtheorem{lemma}[satz]{Lemma}
\newtheorem{korollar}[satz]{Korollar}
\newtheorem{proposition}[satz]{Proposition}
\newtheorem{beispiel}[satz]{Beispiel}
\newtheorem{definition}[satz]{Definition}
\newtheorem{bemerkungnr}[satz]{Bemerkung}
\newtheorem{theorem}[satz]{Theorem}

% Bemerkungen, Erinnerungen und Notationshinweise werden ohne Numerierungen dargestellt.
\newtheorem*{bemerkung}{Bemerkung.}
\newtheorem*{erinnerung}{Erinnerung.}
\newtheorem*{notation}{Notation.}
\newtheorem*{aufgabe}{Aufgabe.}
\newtheorem*{lösung}{Lösung.}
\newtheorem*{beisp}{Beispiel.} %Beispiel ohne Nummerierung
\newtheorem*{defi}{Definition.} %Definition ohne Nummerierung
\newtheorem*{lem}{Lemma.} %Lemma ohne Nummerierung


% SHORTCUTS
\newcommand{\R}{\mathbb{R}}				 % reelle Zahlen
\newcommand{\Rn}{\R^n}						 % der R^n
\newcommand{\N}{\mathbb{N}}				 % natürliche Zahlen
\newcommand{\Z}{\mathbb{Z}}				 % ganze Zahlen
\newcommand{\C}{\mathbb{C}}			   % komplexe Zahlen
\newcommand{\gdw}{\Leftrightarrow} % Genau dann, wenn
\newcommand{\with}{\text{ mit }}   % mit
\newcommand{\falls}{\text{falls }} % falls
\newcommand{\dd}{\text{ d}}        % Differential d

% ETWAS SPEZIELLERE ZEICHEN
%disjoint union
\newcommand{\bigcupdot}{
	\mathop{\vphantom{\bigcup}\mathpalette\setbigcupdot\cdot}\displaylimits
}
\newcommand{\setbigcupdot}[2]{\ooalign{\hfil$#1\bigcup$\hfil\cr\hfil$#2$\hfil\cr\cr}}
%big times
\newcommand*{\bigtimes}{\mathop{\raisebox{-.5ex}{\hbox{\huge{$\times$}}}}} 

% WHITESPACE COMMANDS
%non-restrict newline command
\newcommand{\enter}{$ $\newline} 
%praktischer Tabulator
\newcommand\tab[1][1cm]{\hspace*{#1}}

% TEXT ÜBER ZEICHEN
%das ist ein Gleichheitszeichen mit Text darüber, Beispiel: $a\stackeq{Def} b$
\newcommand{\stackeq}[1]{
	\mathrel{\stackrel{\makebox[0pt]{\mbox{\normalfont\tiny #1}}}{=}}
} 
%das ist ein beliebiges Zeichen mit Text darüber, z. B.  $a\stackrel{Def}{\Rightarrow} b$
\newcommand{\stacksymbol}[2]{
	\mathrel{\stackrel{\makebox[0pt]{\mbox{\normalfont\tiny #1}}}{#2}}
} 

% UNDERLINE
% besseres underline 
\renewcommand{\ULdepth}{1pt}
\contourlength{0.5pt}
\newcommand{\ul}[1]{
	\uline{\phantom{#1}}\llap{\contour{white}{#1}}
}


% hier noch ein paar Commands die nur ich nutze, weil ich sie mir im Laufe der Jahre angewöhnt habe und sie mir jetzt nicht abgewöhnen will:

\newcommand{\gdw}{\Leftrightarrow}   % genau dann, wenn



% This work is licensed under the Creative Commons
% Attribution-NonCommercial-ShareAlike 4.0 International License. To view a copy
% of this license, visit http://creativecommons.org/licenses/by-nc-sa/4.0/ or
% send a letter to Creative Commons, PO Box 1866, Mountain View, CA 94042, USA.

\renewcommand{\div}{\text{ div}}      % Divergenz
\newcommand{\laplace}{\triangle}   % Laplace Operator
\newcommand{\Vertiii}[1]{{\left\vert\kern-0.25ex\left\vert\kern-0.25ex\left\vert #1 
    \right\vert\kern-0.25ex\right\vert\kern-0.25ex\right\vert}}
\newcommand{\T}{\mathcal{T}} %Triangulierung
\newcommand{\meas}{\text{meas}} % Das Maß einer Menge, meist Lebesguemaß


\author{Willi Sontopski}

\parindent0cm %Ist wichtig, um führende Leerzeichen zu entfernen

\usepackage{scrpage2}
\pagestyle{scrheadings}
\clearscrheadfoot

\ihead{Willi Sontopski}
\chead{PDENM WiSe 18 19}
\ohead{}
\ifoot{Blatt 2}
\cfoot{Version: \today}
\ofoot{Seite \pagemark}

\begin{document}
%\setcounter{section}{1}

\section*{Aufgabe 2.1}
Seien $V$ ein normierter Raum, $a(\cdot,\cdot)$ eine symmetrische Bilinearform auf $V$ mit
\begin{align*}
a(v,v)>0\qquad\forall v\in V\setminus\lbrace0\rbrace
\end{align*}
und $f$ ein lineares Funktional auf $V$, Dann sind für $u\in V$ äquivalent:
\begin{enumerate}[label=(\roman*)]
\item $u$ minimiert das Funktional 
\begin{align}\label{Variationsproblem}\tag{Variationsproblem}
J(v):=\frac{1}{2}\cdot a(v,v)-f(v)
\end{align}
\item \begin{align}\label{Variationsgleichung}\tag{Variationsgleichung}
a(u,v)=f(v)\qquad\forall v\in V
\end{align}
\end{enumerate}

\begin{proof}
\underline{Zeige (i) $\implies$ (ii):}\\

\underline{Zeige (ii) $\implies$ (i):}\\

\end{proof}

\section*{Aufgabe 2.2}
Seien $V$ linearer Raum und $a:V\times V\to\R$ eine symmetrische Bilinearform mit
\begin{align*}
a(v,v)>0\qquad\forall v\in V\setminus\lbrace0\rbrace.
\end{align*}
Ferner seien $l:V\to\R$ ein lineares Funktional und $W\subseteq V$ eine konvexe Teilmenge. Dann sind äquivalent:
\begin{enumerate}[label=(\roman*)]
\item Die Größe
\begin{align}\label{eq1}
J(v):=\frac{1}{2}\cdot a(v,v)-f(v)
\end{align}
nimmt ihr Minimum in $W$ bei $u\in W$ an.
\item Es gilt:
\begin{align}\label{eq2}
a(u,w-u)\geq l(w-u)\qquad\forall w\in W
\end{align}
\end{enumerate}
Außerdem ist das Minimum $u\in W$ von \eqref{eq1} eindeutig.

\begin{proof}
\underline{Zeige (i) $\implies$ (ii):}\\

\underline{Zeige (ii) $\implies$ (i):}\\

\end{proof}

\section*{Aufgabe 2.3}
Seien $V$ ein Vektorraum und $a(\cdot,\cdot)$ eine Bilinearform auf $V$ mit
\begin{align*}
a(v,v)>0\qquad\forall v\in V\setminus\lbrace0\rbrace.
\end{align*}
Dann gilt: Sind $u_1,u_2$ Lösungen von
\begin{align}\label{eq3}
a(u,v)=g(v)\qquad\forall v\in V
\end{align}
so gilt $u_1=u_2$.

\begin{proof}
Seien also $u_1,u_2\in V$ Lösungen von \eqref{eq3}. 


\end{proof}


\section*{Aufgabe 2.4}
Seien $V$ ein Hilbertraum, $a(\cdot,\cdot)$ eine symmetrische, stetige, $V$-elliptische Bilinearform auf $V$ und $f$ ein lineares, stetiges Funktional auf $V$. Dann gilt:

\begin{enumerate}[label=(\alph*)]
\item $a(\cdot,\cdot)$ ist ein Skalarprodukt auf $V$ und daher
\begin{align*}
\Vert v\Vert_E:=\sqrt{a(v,v)}\qquad\forall v\in V
\end{align*}
eine Norm auf $V$.
\item Die Variationsgleichung
\begin{align*}
a(u,v)=g(v)\qquad\forall v\in V
\end{align*}
besitzt eine eindeutige Lösung.
\end{enumerate}
\begin{proof}
\underline{Zeige (a):}\\
$a$ ist nach Voraussetzung schon mal bilinear und symmetrisch. Fehlt nur noch positive Definitiheit. Aus der Eigenschaft, dass $a$ $V$-elliptisch ist, folgt
\begin{align*}
a(v,v)\stackrel{V\text{-ell}}{\geq}
\alpha\cdot\Vert v\Vert^2_V\geq0
\end{align*}
TODO: Das ist die falsche Definition von V-elliptisch. Und hier fehlt wohl eine Eigenschaft, denn so gilt das nicht

\underline{Zeige (b):}\\
Nutze Lax-Milgram...

\end{proof}

\section*{Aufgabe 2.5}
Seien $\Omega=(0,1)$ und $V:=\big(L^2(\Omega)\big)^2$. Als Norm für $v=(v_1,v_2)\in V$ verwenden wir
\begin{align*}
\Vert u\Vert_V:=\left(\Vert v_1\Vert_{0,2}^2+\Vert v_2\Vert^2_{0,2}\right)^{\frac{1}{2}}
\end{align*}
Dann ist die auf $V$ definierte Bilinearform
\begin{align*}
a(v,w):=\int\limits_0^1 v_1(x)\cdot w_1(x)+\Big(\big(v_2(x)-v_1(x)\big)\cdot\big(w_2(x)-w_1(x)\big)\Big)\d x
\end{align*}
koerziv. Bestimmen Sie auch die möglichst große Koerzivitätskonstante $\alpha$.

\begin{proof}
Betrachte die Abschätzung
\begin{align}\label{Hinweis2.5}
2\cdot a\cdot b\leq\gamma\cdot a^2+\frac{b^2}{\gamma}\qquad\forall a,b\in\R,\forall\gamma>0
\end{align}
als gegeben.


\end{proof}

\section*{Aufgabe 2.6}
Sei $a(\cdot,\cdot)$ eine stetige
\begin{align*}
a(v,w)\leq M\cdot\Vert v\Vert_V\cdot\Vert w\Vert_V,
\end{align*}
$V$-elliptische
\begin{align*}
\alpha\cdot\Vert v\Vert^2_V\leq a(v,v)
\end{align*}
und symmetrische Bilinearform.\\
Dann ist
\begin{align*}
\Vert e\Vert_E:=\sqrt{a(v,v)}\qquad\forall v\in V
\end{align*}
eine Norm. Seien außerdem $V^N\subseteq V$ ein endlichdimensionaler Teilraum von $V$ und $u\in V,u^N\in V^N$ Lösungen von
\begin{align*}
a(u,v)=f(v)\quad\forall v\in V,\qquad a(u^N,v^N)=f(v^N)\quad\forall v^N\in V^N\subseteq V
\end{align*}
Außerdem gilt für den Fehler des \textit{Ritz-Galerkin-Verfahrens}:
\begin{enumerate}[label=(\alph*)]
\item $\begin{aligned}
\vert u-u^N\Vert_E=\min\limits_{v^n\in V^n}\Vert u-v^N\Vert_E
\end{aligned}$
\item $\begin{aligned}
\Vert u-u^N\Vert-V\leq\left(\frac{M}{\alpha}\right)^{\frac{1}{2}}\cdot\min\limits_{v^N\in V^N}\Vert u-v^N\Vert_V
\end{aligned}$
\end{enumerate}
\begin{proof}
\underline{Zeige, dass $\Vert\cdot\Vert_E$ eine Norm auf $V$ ist:}\\

\underline{Zeige (a):}\\

\underline{Zeige (b):}\
\end{proof}

\end{document}
