\setcounter{chapter}{0}
\renewcommand{\thechapter}{\Alph{chapter}}
\chapter{Anhang}
\setcounter{equation}{1}
\section{partielle Integration mit Satz von Gauß}
Im Beispiel aus der Vorlesung erhalten wir folgende Gleichheit:
\begin{align}
	\int_{\Omega} u_xv_x + u_yv_y \d x\d y = -\int_{\Omega} (u_{xx} + u_{yy})v\d x\d y + \int_{\partial\Omega} u_x v n_x + u_y v n_y \d\gamma
\end{align}
Dies wird hier noch einmal im Detail erklärt.
\enter
Zunächst gehen wir zu einer kürzeren und allgemeineren Notation über. Wir schreiben die Summen der partiellen Ableitungen als Skalarprodukt von Gradienten auf, z.B. :
\[u_xv_x + u_yv_y = \nabla u \cdot \nabla v\]
Außerdem bedienen wir uns einer Indentität, die man durch Nachrechnen für passende Funktionen nachweisen kann:
\begin{align}\div(fg) = f \cdot \nabla g + g \div(f)\end{align}
Wir nutzen diese Gleichung spezifisch mit $f=\nabla u$ und $g=v$.
Nun können wir (A.2) beweisen:
\begin{proof}
\begin{align*}
	\int_{\Omega} u_xv_x + u_yv_y \d x\d y ~~ &\stackeq{\text{Notation}} ~~ \int_{\Omega} \nabla u \cdot \nabla v \d x\d y \\
																		 &\stackeq{\text{A.3}} \int_{\Omega}\div(v\nabla u) \d x\d y - \int_{\Omega} v \div(\nabla u) \d x\d y \\
																		 &\stackeq{\text{Gauß}} \int_{\partial\Omega}v \nabla u \cdot n \d\gamma - \int_{\Omega} v \div(\nabla u) \d x\d y \\
																		 &\stackeq{\div(\nabla) = \laplace} ~~~ \int_{\partial\Omega}v \nabla u \cdot n \d\gamma - \int_{\Omega} v \laplace u \d x\d y \\
																		 &\stackeq{\text{Def}} \int_{\partial\Omega} u_x v n_x + u_y v n_y \d\gamma - \int_{\Omega} (u_{xx} + u_{yy})v\d x\d y 
\end{align*}
\end{proof}
