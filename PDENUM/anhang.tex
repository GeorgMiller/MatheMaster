% This work is licensed under the Creative Commons
% Attribution-NonCommercial-ShareAlike 4.0 International License. To view a copy
% of this license, visit http://creativecommons.org/licenses/by-nc-sa/4.0/ or
% send a letter to Creative Commons, PO Box 1866, Mountain View, CA 94042, USA.

\setcounter{chapter}{0}
\renewcommand{\thechapter}{\Alph{chapter}}
\chapter{Anhang}
\setcounter{equation}{1}
\section{partielle Integration mit Satz von Gauß}
Im Beispiel aus der Vorlesung erhalten wir folgende Gleichheit:
\begin{align}
	\int_{\Omega} u_xv_x + u_yv_y \d x\d y = -\int_{\Omega} (u_{xx} + u_{yy})v\d x\d y + \int_{\partial\Omega} u_x v n_x + u_y v n_y \d\gamma
\end{align}
Dies wird hier noch einmal im Detail erklärt.
\enter
Zunächst gehen wir zu einer kürzeren und allgemeineren Notation über. Wir schreiben die Summen der partiellen Ableitungen als Skalarprodukt von Gradienten auf, z.B. :
\[u_xv_x + u_yv_y = \nabla u \cdot \nabla v\]
Außerdem bedienen wir uns einer Indentität, die man durch Nachrechnen für passende Funktionen nachweisen kann:
\begin{align}\div(fg) = f \cdot \nabla g + g \div(f)\end{align}
Wir nutzen diese Gleichung spezifisch mit $f=\nabla u$ und $g=v$.
Nun können wir (A.2) beweisen:
\begin{proof}
\begin{align*}
	\int_{\Omega} u_xv_x + u_yv_y \d x\d y ~~ &\stackeq{\text{Notation}} ~~ \int_{\Omega} \nabla u \cdot \nabla v \d x\d y \\
																		 &\stackeq{\text{A.3}} \int_{\Omega}\div(v\nabla u) \d x\d y - \int_{\Omega} v \div(\nabla u) \d x\d y \\
																		 &\stackeq{\text{Gauß}} \int_{\partial\Omega}v \nabla u \cdot n \d\gamma - \int_{\Omega} v \div(\nabla u) \d x\d y \\
																		 &\stackeq{\div(\nabla) = \laplace} ~~~ \int_{\partial\Omega}v \nabla u \cdot n \d\gamma - \int_{\Omega} v \laplace u \d x\d y \\
																		 &\stackeq{\text{Def}} \int_{\partial\Omega} u_x v n_x + u_y v n_y \d\gamma - \int_{\Omega} (u_{xx} + u_{yy})v\d x\d y 
\end{align*}
\end{proof}

\section{Koflächenformel}
\begin{satz}\enter
	Sei $f:B(0,R)\rightarrow \R$ integrierbar. Dann ist $f$ für fast alle $r>0$ über der Sphäre $S_r := \partial B(0,r)$ integrierbar und es gilt:
	\begin{align*}
		\int_{B(0,R)}f(x) \d\lambda(x) &= \int_0^R \left(\int_{S_r} f(x) \d\gamma(x)\right) \d r \\
																	 &= \int_0^R \left(\int_{S_1} f(rx) \d\gamma(x)\right)r^{n-1} \d r
	\end{align*}
\end{satz}
	\begin{proof}
		siehe Forster 2 (wenn nicht, bitte melden)
	\end{proof}
	
\section{Fixpunktsätze}
\begin{satz}[Banach'scher Fixpunktsatz]\label{BanachscherFixpunktsatz}\enter
Sei $W$ ein Banachraum, $Q:N\to N$ mit $N\subseteq W$ sei eine Kontraktion, d. h.
\begin{align*}
\exists\gamma<1:\forall u,v\in N:\Vert Q(u)-Q(v)\Vert_V\leq\gamma\cdot\Vert u-v\Vert_V.
\end{align*}
Dann hat $Q$ genau einen Fixpunkt.
\end{satz}
