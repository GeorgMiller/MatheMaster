\chapter{Satz von Riemann-Roch}
\section{Bewertungen}
Sei $F$ ein Körper und $R \subseteq F$ ein Ring.

\begin{definition}
    $R$ ist ein \textbf{Bewertungsring} von $F$
    $$:\iff \forall x  \in F^{\times}: x \in R \text{ oder } x^{-1} \in R.$$
\end{definition}

\begin{beispiel}
    Ist $R$ diskreter Bewertungsring von $F$, dann ist er ein Bewertungsring von $F$ (I.2.16).
\end{beispiel}

\begin{definition}
    Ein \textbf{angeordnete abelsche Gruppe} ist eine abelsche Gruppe $(\Gamma, +)$ mit einer
    Totalordnung, die mit der Addition verträglich ist.

    Eine \textbf{Bewertung} auf $F$ ist eine Abbildung
    $$ v: F \to \Gamma \cup \{\infty\}, $$
    wobei $\Gamma$ eine angeordnete abelsche Gruppe ist und folgendes gilt
    \begin{enumerate}[label=(\arabic*)]
        \item $v(x) = \infty \iff x = 0$
        \item $v(xy) = v(x) + v(y)$
        \item $v(x+y) \geq \min\{v(x), v(y)\}$.
    \end{enumerate}

    Der \textbf{Bewertungsring} von $v$ ist $\mathcal{O}_v := \{x \in F \mid v(x) \geq 0\}$,\\
    das \textbf{Bewertungsideal} von $v$ ist $m_v := \{x \in F \mid v(x) > 0\}$,\\
    der \textbf{Restklassenkörper} von $v$ ist $F_v := \sfrac{\mathcal{O}_v}{m_v}$ und\\
    die \textbf{Wertegruppe} von $v$ ist $\Gamma_v = v(F^\times)$.

    Zwei Bewertungen $v_1, v_2$ auf $F$ heißen \textbf{äquivalent}, wenn es einen ordnungserhaltenden
    Isomorphismus
    $$ \varphi : \Gamma_{v_1} \to \Gamma_{v_2}$$ mit $v_2 = \varphi \circ v_1$ gibt.
\end{definition}

\begin{bemerkungnr}
    \begin{enumerate}[label=\alph*)]
        \item $\mathcal{O}_v$ ist ein lokaler Ring mit maximalem Ideal $m_v$, demnach ist 
        $F_v$ wohldefiniert.

        \item Für $x,y \in F$ mit $v(x) \ne v(y)$ ist
        $$ v(x+y) = \min\{v(x),v(y)\}.$$
    \end{enumerate}
\end{bemerkungnr}

\begin{satz}
    Ist $v$ eine Bewertung auf $F$, so ist $\mathcal{O}_v$ ein Bewertungsring von $F$.
    Ist umgekehrt $\mathcal{O}$ ein Bewertungsring von $F$, so ist 
    $$ v_\mathcal{O}: F^\times \to \sfrac{F^\times}{\mathcal{O}^\times}$$
    mit
    $$ x \mathcal{O}^\times \leq y \mathcal{O}^\times \iff \frac{y}{x} \in \mathcal{O}$$
    eine Bewertung auf $F$ mit Bewertungsring $\mathcal{O}$, und
    $ v \mapsto \mathcal{O}_v$ liefert eine Bijektion zwischen Äquivalenzklassen auf von 
    Bewertungen auf $F$ und Bewertungsringen von $F$.
\end{satz}
\begin{proof}
    \begin{itemize}
        \item $\mathcal{O}_v$ Bewertungsring: nach (ii) ist $v(x^{-1}) = -v(x)$
        \item $\leq$ ist eine Totalordnung:
            $x \mathcal{O}^\times \leq y \mathcal{O}^\times \land y \mathcal{O}^\times \leq x \mathcal{O}^\times \implies \frac{x}{y} \in \mathcal{O} \land \frac{y}{x} \in \mathcal{O}$
            $$\implies \frac{x}{y} \in \mathcal{O}^\times \implies x \mathcal{O}^\times = y \mathcal{O}^\times$$
        \item $v_\mathcal{O}$ ist Bewertung: 
            $v_\mathcal{O}(x) \leq v_\mathcal{O}(y) \implies \frac{y}{x} \in \mathcal{O} 
            \implies 1 + \frac{y}{x} \in \mathcal{O}$
            $$\implies \frac{x+y}{x} \in \mathcal{O} \implies v_\mathcal{O}(x+y) \geq v_\mathcal{O}(x) = \min\{v_\mathcal{O}(x), v_\mathcal{O}(y)\}$$
        \item $\mathcal{O}_{v_\mathcal{O}} = \mathcal{O}$:
            $v_\mathcal{O}(x) \geq 0 \iff x \mathcal{O}^\times \geq \mathcal{O}^\times
            \iff \frac{x}{1} \in \mathcal{O}$
        \item $v_{\mathcal{O}_v} \sim v$: $v(x) = v(y) \iff \frac{y}{x} \in \mathcal{O}_v^\times \iff x \mathcal{O}_v^\times = y \mathcal{O}_v^\times$
        \begin{align*}
            \iff v_{\mathcal{O}_v}(x) = v_{\mathcal{O}_v}(y)\\
            \implies \varphi: \Gamma_v \to \Gamma_{v_{\mathcal{O}_v}}, v(x) \mapsto v_{\mathcal{O}_v}(x) = x \mathcal{O}_v^\times
        \end{align*}
        ist ein Isomorphismus mit $v_{\mathcal{O}_v} = \varphi \circ v$.
    \end{itemize}
\end{proof}

\begin{korollar}
    Ist $\mathcal{O}$ ein Bewertungsring von $F$, so ist $\mathcal{O}$ lokal, ganzabgeschlossen und $\Quot(\mathcal{O}) = F$.
\end{korollar}
\begin{proof}
    \begin{itemize}
        \item $\mathcal{O}$ lokal: $\mathcal{O} = \mathcal{O}_{v_\mathcal{O}}$ lokal nach 1.4
        \item $\Quot(\mathcal{O}) = F$: $\checkmark$
        \item $\mathcal{O}$ ganzabgeschlossen: $x \in F^\times, x^n + \sum\limits_{i=0}^{n-1}a_i x^i = 0, a_i \in \mathcal{O}$
        $$x \notin \mathcal{O} \implies x^{-1} \in \mathcal{O} \implies x = - \sum\limits_{i=0}^{n-1}a_i x^{i - n +1} \in \mathcal{O} 
        \text{ Widerspruch.}$$
    \end{itemize}
\end{proof}

\begin{bemerkungnr}
    Man nennt eine Bewertung \textbf{diskret}, wenn $\mathcal{O}_v$ ein diskreter Bewertungsring ist, also 
    $\Gamma_v \cong \Z$.

    Eine diskrete Bewertung heißt \textbf{normiert}, wenn $\Gamma_v = \Z$, und ein $f \in F$ mit $v(f) = 1$ heißt
    ein \textbf{Primelement} von $v$.
\end{bemerkungnr}

\begin{theorem}[Schwacher Approximationssatz, Artin-Whaples '45]
    Sind $v_1,\ldots,v_n$ paarweise verschiedene normierte diskrete Bewertungen auf $F$, so gibt es zu 
    $x_1,\ldots,x_n \in F$ und $k_1, \ldots,k_n \in \Z$ ein $x \in F$, sodass für alle $i = 1, \ldots,n$
    $$ v_i(x - x_i) > k_i$$
    gilt.  
\end{theorem}
\begin{proof}
    Wähle $k_0 > \max\{k_1, \ldots, k_n\} - \min\limits_{i,j}\{v_i(x_j)\}$.\\
    \underline{Beh. 1:} $\forall i \forall i\ne j \exists b_{ij}$ mit $v_i(b_{ij}) > 0$ und $v_j(b_{ij}) < 0$.

    \underline{Bew.:} $\mathcal{O}_{v_i} \ne \mathcal{O}_{v_j} \stackrel{I.2.15}{\implies} \mathcal{O}_{v_i} \not \subseteq \mathcal{O}_{v_j}
    \land \mathcal{O}_{v_j} \not \subseteq \mathcal{O}_{v_i}$ 
    \begin{align*}
        & \implies \text{es existiert } b \in \mathcal{O}_{v_i} \setminus \mathcal{O}_{v_j} \text{ und } b' \in \mathcal{O}_{v_j} \setminus \mathcal{O}_{v_i}\\
        & \implies v_i(\frac{b}{b'}) > 0 \land v_j(\frac{b}{b'}) < 0
    \end{align*}
    \underline{Beh. 2:} $\forall i \exists b_i: v_i(b_i) > k_0 \land v_j(b_i) < -k_0 \forall j \ne i$

    \underline{Bew.:} Wähle für jedes $i$ ein $l_i > k_0 \land l_i > \max\limits_{j}|v_{j}(b_{i,j})|$ und setze
    $$ b_i := \sum\limits_{j \ne i} b_{i,j}^{l_j}$$

\end{proof}

\section{Primstellen}
Sei $F | K$ ein Funktionenkörper.

\begin{definition}
    Dies bedeute, dass $F$ ein Funktionenkörper über $K$ ist mit Konstantenkörper $\tilde{K}=K$.
\end{definition}

\begin{satz}
    Ist $v$ eine nicht-triviale Bewertung auf $F$, die trivial auf $K$ ist,
    so ist $\Gamma_v \cong \Z$ und der Restklassenkörper $F_v$ ist eine endliche Erweiterung von $K$.
\end{satz}
\begin{proof}
    Wähle $t \in F$ transzendent über $K$ mit $[F:K(t)] < \infty$.
    Wie im Beweis von I.3.15 sieht man, dass $w:= v|_{K(t)}$ nicht trivial ist, denn $\mathcal{O}_v$ ist ganzabgeschlossen (1.6).
    Wie im Beweis von I.2.21 sieht man, dass $\mathcal{O}_w = \mathcal{O}_P$ für ein 
    $P \in K[t]$ normiert irreduzibel oder $\mathcal{O}_w = \mathcal{O}_{\infty}$.
    Somit ist $w$ diskret (d.h. $\Gamma_w \equiv \Z$) und $[K(t)_w:K] < \infty$.
    Mit 2.6 und 2.7 folgt, dass auch $v$ diskret und $$ [F_v:K] = [F_v:K(t)_w]\cdot [K(t)_w:K] < \infty.$$
\end{proof}

\begin{definition}
    Eine \textbf{Primstelle} von $F|K$ ist eine Äquivalenzklasse von nicht trivialen Bewertungen auf $F$, die trivial auf $K$ sind.

    Wir bezeichnen mit $v_P$ den Vertreter der Primstelle $P$ mit $\Gamma_{v_P} = \Z$. 
    Desweiteren verwenden wir folgende Notation:
    \begin{align*}
        \mathcal{O}_P &:= \mathcal{O}_{v_P}\\
        F_P &:= F_{v_P}\\
        f(P) &:= \begin{cases}
            \overline{f} \in F_P, & f \in \mathcal{O}_P\\
            \infty, & f \notin \mathcal{O}_P
        \end{cases}\\
        \deg P &:= [F_P:K] \text{ der Grad von }P\\
        S(F|K) &:= \{P \mid P \text{ ist Primstelle von } F|K\}\\
        S^d(F|K) &:= \{P \in S(F|K) \mid \deg P = d\}.
    \end{align*}

    Ist $E|F$ eine endliche Erweiterung mit Konstantenkörper $L$ und $ Q \in S(E|L)$, so sagt man, $Q$ liegt über $P \in S(F|K)$, falls
    $\mathcal{O}_Q \cap F = \mathcal{O}_P$, i.Z. $Q|_F = P$.
\end{definition}

\begin{beispiel}
    Die Primstellen von $K(T)|K$ sind
    \begin{align*}
        S(K(T)|K) = \{P \in K[T] \mid P \text{ normiert und irreduzibel}\} \cup \{\infty\}
    \end{align*}
    und $\deg P$ ist genau der Grad des Polynoms $P$, und $\deg(\infty) = 1.$
\end{beispiel}

\begin{lemma}
    Ist $E|F$ endlich und $P \in S(F|K)$, so gibt es mindestens ein aber nur endlich viele $Q \in S(E|L)$ mit $Q_F = P$.
\end{lemma}
\begin{proof}
    2.3 + 2.8. $K \subseteq \mathcal{O}_P$, $L|K$ ganz $\stackrel{\mathcal{O}_Q\text{ ganzabgeschlossen}}{\implies}$
    $L \subseteq \mathcal{O}_Q$.
\end{proof}

\begin{satz}
    $S(F|K)$ ist unendlich.
\end{satz}
\begin{proof}
    Wähle $t \in F$ transzendent $\implies$ $[F:K(t)] < \infty$. 
    $S(K(T)|K)$ unendlich (I.4.2) $\implies$ $S(F|K)$ unendlich.
\end{proof}

\begin{definition}
    Sei $x \in F, P \in S(F|K)$. $P$ ist \textbf{Nullstelle} von $x$ $: \iff x(P) = 0 \iff v_P(x) > 0$.

    $P$ ist \textbf{Polstelle} von $x$ $:\iff x(P) = \infty \iff v_P(x) < 0$..
\end{definition}

\begin{satz}
    Sei $x \in F$. Ist $x \in K$, so hat $x$ keine Null- oder Polstellen. Ist $x \notin K$, so hat $x$ sowohl Nullstellen als auch Polstellen,
    aber jeweils nur endlich viele.
\end{satz}
\begin{proof}
    Ist $x \notin K$, so ist $x$ transzendent über $K$ und $[F:K(x)] < \infty$.
    Dann ist 
    \begin{align*}
        &\{P \in S(F|K) \mid v_P(x) > 0 [\text{bzw. } < 0]\}\\
        =& \{P \in S(F|K) \mid v_P|_{K(x)} = v_x [\text{bzw. } v_{\infty}]\}
    \end{align*}
    nichtleer und endlich nach 3.5.
\end{proof}

\section{Divisoren}
Sei $F|K$ ein Funktionenkörper.

\begin{definition}
    Die Gruppe der \textbf{Divisoren} von $F|K$ ist die freie abelsche Gruppe $\Div(F|K)$ auf $S(F|K)$, d.h.
    $$ \Div(F|K) := \bigoplus\limits_{P \in S(F|K)} \Z = \{\sum\limits_{P \in S(F|K)} n_P P \mid n_p \in \Z, \text{ fast alle }0\}.$$
    Wir identifizieren Primstellen $P \in S(F|K)$ mit dem \textbf{Primdivisor} $P \in \Div(F|K)$.

    Für $A,B \in \Div(F|K), A = \sum\limits_P n_P P$ und $x \in F^{\times}$ sei 
    \begin{align*}
        \supp(A) &:= \{P \in S(F|K) \mid n_P \ne 0\}, \text{ der \textbf{Träger} von }A\\
        v_P(A) &:= n_P\\
        \deg(A) &:= \sum\limits_P \deg(P), \text{ der \textbf{Grad} von }A\\
        A \leq B &:\iff v_P(A) \leq v_P(B) \forall P \in S(F|K)\\
        A \text{ ist \textbf{effektiv}} &:\iff A \geq 0\\
        A_+ &:= \sum\limits_{v_P(A) > 0} v_P(A) P\\
        A_- &:= \sum\limits_{v_P(A) < 0} -v_P(A) P
    \end{align*}
\end{definition}

\begin{bemerkungnr}
    \begin{enumerate}
        \item Die Definitionen $\deg(P)$ für $P$ als Primstelle oder als Primdivisor stimmen überein.
        \item Für $A \in \Div(F|K)$ ist $A = A_+ - A_-$ mit $A_+,A_-$ effektiv.
        \item Die Abbildung
        $$ \deg: \Div(F|K) \to \Z$$
        ist ein Gruppenhomomorphismus.
    \end{enumerate}
\end{bemerkungnr}

\begin{beispiel}
    Für $x \in F^{\times}$ ist $\sum\limits_P v_P(x) P$ nach 3.8 ein Divisor.
\end{beispiel}

\begin{definition}
    Für $x \in F^{\times}$ ist
    $$ (x) := \sum\limits_P v_P(x) P \in \Div(F|K)$$
    der \textbf{Hauptdivisor} von $x$.

    Desweiteren ist
    \begin{align*}
        (x)_0 := (x)_+ = \sum\limits_{v_P(x) > 0} v_P(x) P, & \text{ der \textbf{Nullstellendivisor} von } x\\
        (x)_\infty := (x)_- = \sum\limits_{v_P(x) < 0} -v_P(x) P, & \text{ der \textbf{Polstellendivisor} von } x.\\
    \end{align*}
\end{definition}

\begin{bemerkungnr}
    \begin{enumerate}
        \item Für $x \in F^{\times}$ ist $v_P((x)) = v_P(x)$.
        \item Die Abbildung
        $$ (\cdot): F^{\times} \to \Div(F|K)$$
        ist ein Gruppenhomomorphismus (1.3(ii)). 
        Insbesondere ist $\mathscr{P}(F|K) \leq \Div(F|K)$.
    \end{enumerate}
\end{bemerkungnr}

\begin{beispiel}
    In $F=K(T)$ hat jeder Hauptdivisor Grad $0$ (\#7).

    Umgekehrt ist jeder Divisor $A \in \Div(K(T)|K)$ vom Grad $0$ ein Hauptdivisor, also
    $A = (x)$ für ein $x \in K(T)^{\times}$ (Übung).
\end{beispiel}

\begin{definition}
    $$ \mathscr{C}(F|K) := \sfrac{\Div(F|K)}{\mathscr{P}(F|K)}$$
    ist die \textbf{Divisorenklassengruppe} von $F|K$.
    Für $A,B \in \Div(F|K)$ ist
    \begin{align*}
        [A] := A + \mathscr{P}(F|K) \in \mathscr{C}\\
        A \sim B : \iff [A] = [B] & \text{("linear äquivalent")}.
    \end{align*}
\end{definition}

\begin{beispiel}
    Für $x \in F^{\times}$ ist $(x) \sim 0$, aber wenn $x \notin K$ ist $(x) \ne 0$ und sogar $(x) \not \geq 0$ und $(x) \not \leq 0$.
\end{beispiel}

\section{Der Riemann-Roch-Raum}
Sei $F|K$ ein Funktionenkörper, $A,B \in \Div(F|K)$

\begin{definition}
    Der \textbf{Riemann-Roch-Raum} zu $A$ ist 
    $$ \mathcal{L}(A) := \{x \in F^{\times} \mid (x) + A \geq 0\} \cap \{0\}.$$
\end{definition}

\begin{bemerkungnr}
    $\mathcal{L}(A)$ besteht aus den Funktionen, deren Pole durch $A_+$ beschränkt sind und die mindestens die
    durch $A_-$ vorgegebenen Nullstellen haben.
\end{bemerkungnr}

\begin{lemma}
    \begin{enumerate}
        \item $\mathcal{L}(A)$ ist ein $K$-Vektorraum.
        \item $\mathcal{L}(0) = K$
        \item $\mathcal{L}(A) \ne \{o\} \implies \text{ es existiert } A'\geq 0: A \sim A'$
        \item $ A < 0 \implies \mathcal{L}(A) = \{0\}$
        \item $A \leq B \implies \mathcal{L}(A) \subseteq \mathcal{L}(B)$
        \item $A \sim B \implies \mathcal{L}(A) \cong \mathcal{L}(B)$ als $K$-Vektorraum
    \end{enumerate}
\end{lemma}
\begin{proof}
    \underline{(a):} Seien $x,y \in \mathcal{L}(A), \lambda \in K, P \in S(F|K)$.
    \begin{align*}
        v_P(x + y) \geq \min\{v_P(x), v_P(y)\} \geq - v_P(A)\\
        v_P(\lambda x) = v_P(\lambda) + v_P(x) \geq - v_P(A)
    \end{align*}

    \underline{(b):} 4.8

    \underline{(c):} $0 \ne x \in \mathcal{L}(A) \implies A' := (x) + A \geq 0$

    \underline{(d):} $0 \ne x \in \mathcal{L}(A)$. 
    $$x \in K \implies (x) + A = A \not \geq 0$$
    $$ x \notin K \implies \text{ es existiert } P \text{ mit } v_P(x) < 0 \implies v_P((x) + A) \leq v_P(x) < 0,$$
    insbesondere $(x) + A \not \geq 0$.

    \underline{(e):} folgt direkt aus Definition

    \underline{(f):} $B=A+(x)$ $$ \varphi: \begin{cases}
        \mathcal{L}(A) \to \mathcal{L}(B)\\
        z \mapsto x^{-1}z
    \end{cases}$$
    ist $K$-linear mit Inversem
    $$ \varphi^{-1}: \begin{cases}
        \mathcal{L}(B) \to \mathcal{L}(A)\\
        z \mapsto xz
    \end{cases}$$
\end{proof}

\begin{lemma}
    $A \leq B \implies \dim_K \sfrac{\mathcal{L}(B)}{\mathcal{L}(A)} \leq \deg B - \deg A$
\end{lemma}
\begin{proof}
    o.E. $B=A+P$ mit $P \in S(F|K)$.
    Wähle $t \in F$ mit $v_P(t) = v_P(B) = v_P(A) +1$
    Definiere
    $$ \psi : \begin{cases}
        \mathcal{L}(B) \to F_P\\
        x \mapsto (tx)(P) := tx + m_{v_P} \in F_P
    \end{cases}$$
    \begin{itemize}
        \item $\psi$ ist wohldefiniert: $v_P(tx) = v_P(B) + v_P(x) \geq 0$
        \item $\psi$ ist $K$-linear: $\checkmark$
        \item $\ker \psi = \mathcal{L}(A)$: $x \in \ker \psi \iff v_P(tx) > 0$
        \begin{align*}
            & \iff v_P(x) + v_P(A) \geq 0\\
            & \iff x \in \mathcal{L}(A)\\
            & \implies \dim_K \sfrac{\mathcal{L}(B)}{\mathcal{L}(A)} = \dim_K \im (\psi) \leq \dim_K F_P\\
            & \qquad = \deg P = \deg B - \deg A
        \end{align*}
    \end{itemize}
\end{proof}

\begin{satz}
    $\dim_K \mathcal{L}(A) \leq \deg A_+ + 1$
\end{satz}
\begin{proof}
    \begin{align*}
        \dim_K \mathcal{L}(A) &\leq \dim_K \mathcal{L}(A_+)\\
        & = \dim_K \sfrac{\mathcal{L}(A_+)}{\mathcal{L}(0)} + \dim_K \mathcal{L}(0)\\
        & \leq \deg A_+ - \deg 0 + 1
    \end{align*}
\end{proof}

\begin{definition}
    Die \textbf{Dimension} des Divisors $A$ ist definiert durch
    $$ \dim A := \dim_K \mathcal{L}(A) \in \Z_{\geq 0}.$$
\end{definition}

\begin{satz}
    Für $x \in F \setminus K$ ist 
    $$ \deg (x)_0 = \deg (x)_\infty = [F:K(x)].$$
\end{satz}
\begin{proof}
    Sei $ A := (x)_\infty, n := [F:K(x)]$.\\
    \underline{Teil 1:} $\deg (x)_\infty \leq n$.

    Die Polstellen $P_1,\ldots, P_r$ von  $x$ sind genau die Fortsetzungen von $P_\infty \in S(K(x)|K)$
    auf $F$ mit Primelement $x^{-1}$.
    \begin{align*}
        \implies \deg A &= \deg \left(\sum\limits_{i=1}^r -v_{P_i}(x)P_i\right)\\
        &= \sum\limits_{i=1}^r v_{P_i}(x) \deg P_i\\
        &= \sum\limits_{i=1}^r e(P_i | P_\infty)v_{P_\infty}(x^{-1})\cdot f(P_i|P_\infty)\deg P_\infty\\
        &= \sum\limits_{i=1}^r e(P_i | P_\infty)f(P_i|P_\infty)\\
        \stackrel{2.7}{\leq} [F: K(x)] = n
    \end{align*}
    \underline{Teil 2:} $\deg (x)_\infty \geq n$

    Wähle Basis $b_1,\ldots,b_n$ von $F|K(x)$, $B \in \Div(F|K)$ mit $B \geq 0$ und $B \geq (b_j)_\infty \forall j$.
    Für $l \in \N$ ist $x^i b_j \in \mathcal{L}(lA+B)$ mit $i=0,\ldots,l$ und $j=1,\ldots,n$.
    Da $b_1,\ldots,b_n$ linear unabhängig über $K(x)$ sind und $x^0, \ldots, x^l$ linear unabhängig über $K$ sind,
    ist $$ \{x^i b_j \mid i=0, \ldots,l, j=1, \ldots,n\}$$ linear unabhängig über $K$.
    \begin{align*}
        &\implies \dim_K \mathcal{L}(lA + B) \geq (l + 1) \cdot n\\
        &\implies (l + 1) \cdot n \leq \dim (lA + B) \stackrel{5.5}{\leq} \deg (lA + B) + 1\\
        &\qquad = l \cdot \deg A + \deg B + 1\\
        & \implies l(\deg A - n) \geq n - \deg B - 1 \forall l\\
        &\implies \deg A - n \geq 0 \\
        &\implies \deg (x)_\infty \geq n
    \end{align*}
    \underline{Teil 3:} $\deg (x)_0 = \deg (x^{-1})_\infty \stackrel{\text{Teil 1 + 2}}{=} [F: K(x^{-1})] = [F:K(x)]$
\end{proof}

\begin{korollar}
    \begin{enumerate}[label=(\alph*)]
        \item Für $x \in F^\times$ ist $\deg ((x)) = 0$
        \item $A \sim A' \implies \dim A = \dim A'$ und $\deg A = \deg A'$
        \item $\deg A < 0 \implies \dim A = 0$
        \item Für $\deg (A) = 0$ gilt:
        $$ A \in \mathscr{P}(F|K) \iff \dim A \geq 1 \iff \dim A =1.$$
    \end{enumerate}
\end{korollar}
\begin{proof}
    \underline{(a):} $\deg (x) = \deg (x)_0 - \deg (x)_\infty \stackrel{5.7}{=} 0$\\
    \underline{(b):} $\dim A \stackrel{5.3(f)}{=} \dim A'$ 
    \begin{align*}
        A' = A + (x) \implies \deg A' = \deg A + \deg (x) \stackrel{(a)}{=} \deg A
    \end{align*}
    \underline{(c):} $\dim A > 0 \implies$ ex $A' \sim A$, $A' \geq 0$
    \begin{align*}
        & \implies \deg A' \geq \deg 0 = 0\\
        & \implies \deg A \stackrel{(b)}{=} \deg A' \geq 0
    \end{align*}
    \underline{(d):}$A=(x) \implies 0 \ne x^{-1} \in \mathcal{L}(A) \implies \dim A \geq 1$
    \begin{align*}
        & \implies A \sim A' \geq 0, \text{ da } \deg A' = \deg A =0, \text{ also } A'=0\\
        &\stackrel{(b)}{\implies} \dim A = \dim 0 = 1
    \end{align*}
    \begin{align*}
        \dim A = 1, 0 \ne x \in \mathcal{L}(A) \implies A + (x) \geq 0, \deg (A+ (x)) = 0 \\
        \implies A + (x) = 0, \text{ also } A = (x^{-1})
    \end{align*}
\end{proof}

\begin{bemerkungnr}
    (a) sagt, dass Funktionen immer gleich viele Null- und Polstellen haben (mit Vielfachheit gezählt).
    Wegen (b) induzieren $\dim$ und $\deg$ Abbildungen
    $$ \mathscr{C}(F|K) \to \Z,$$
    die wir wieder so bezeichnen.
\end{bemerkungnr}

\section{Das Geschlecht}
Sei $F|K$ ein Funktionenkörper.

\begin{bemerkungnr}
    Nach 5.5. $ \deg A - \dim A + 1 \geq 0$ für alle $A \geq 0$, nach 5.3(c) und 5.8 somit auch für alle
    $A$ mit $\dim A > 0$. Nach 5.4 ist 
    $$ \deg A - \dim A +1 $$
    zudem monoton in $A$.
\end{bemerkungnr}

\begin{satz}
    Es gibt ein $g \geq 0$ mit
    $$ \deg A - \dim A + 1 \leq g$$
    für alle $A \in \Div(F|K)$.
\end{satz}
\begin{proof}
    Wähle $x \in F|K$, setze $A = (x)_\infty$
    $$ \stackrel{5.7}{\implies} n := \deg (A) = [F : K(x)].$$
    Wähle Basis $b_1,\ldots,b_n$ von $F|K(x)$, $B \geq 0$ mit $B \geq (b_j)_\infty \forall j$.
    $$ \implies \dim (lA + B) \geq (l+1)n \forall l \in \N$$
    Nach 5.4 ist 
    \begin{align*}
        &\dim (lA+B) - \dim lA \leq \deg (lA + B) - \deg lA = \deg B\\
        \implies& \dim lA \geq (l+1)n - \deg B = \deg lA + n - \deg B\\
        \implies& \deg lA - \dim lA \leq \deg B - n \forall l.
    \end{align*}
    Sei nun $D \in \Div(F|K)$ beliebig.
    \begin{align*}
        \dim (lA - D_+) &\stackrel{5.4}{\geq} \dim lA - \deg D_+\\
        & \geq l \deg A + n - \deg B - \deg D_+\\
        &> 0 \text{ für } l \text{ groß genug.}
    \end{align*}
    Ist $0 \ne z \in \mathcal{L}(lA - D_+)$ ist
    \begin{align*}
        D \leq& D_+ \sim D_+ + (z^{-1}) \leq lA \\
        \implies& \deg D - \dim D \leq \deg D_+ - \dim D_+\\
        &= \deg (D_+ + z^{-1})) - \dim (D_+ + (z^{-1})) \leq \deg lA - \dim lA \leq \deg B -n.
    \end{align*}
\end{proof}

\begin{definition}
    Das \textbf{Geschlecht} von $F|K$ ist
    $$ g := g_{F|K} := \sup \{\deg A - \dim A + 1 \mid A \in \Div(F|K)\} \in \Z_{\geq 0}.$$
\end{definition}

\begin{bemerkungnr}
    Nach Definition gilt also 
    $$ \dim A \geq \deg A + 1 -g $$
    für alle $A$ mit Gleichheit für mindestens ein $A$.
\end{bemerkungnr}
