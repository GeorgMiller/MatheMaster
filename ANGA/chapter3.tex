\chapter{Der Satz von Riemann-Hurwitz}

\section{Erweiterungen}
Sei $F|K$ ein Funktionenkörper.

\begin{definition}
    Ein Funktionenkörper $E|L$ ist eine Erweiterung des Funktionenkörpers $F|K$, wenn $F \subseteq E$ und $K \subseteq L$.
    Die Erweiterung $E|K$ von $F|K$ ist \textbf{algebraisch}, \textbf{separabel}, \textbf{galoisch}, oder \textbf{endlich}, falls
    $E|F$ dies ist. Sie ist eine \textbf{Konstantenkörpererweiterung}, wenn $E = FL$, und \textbf{geometrisch}, wenn $K=L$.
\end{definition}

\begin{bemerkungnr}
    Geometrische Erweiterungen sind stets endlich. Jede Erweiterung $E|L$ von $F|K$ lässt sich wie folgt zerlegen:
    $$\begin{tikzcd}[column sep=10em]
          & E\\
        F \rar[dash]{\text{Konstantenkörpererweiterung}} & FL \uar[dash]{\text{geometrisch}}\\
        K \rar[dash] \uar[dash] & L \uar[dash]
    \end{tikzcd}$$
\end{bemerkungnr}

\begin{beispiel}
    $F\overline{K}|\overline{K}$ ist eine algebraische Konstantenkörpererweiterung von $F|K$.
    Ist $E|F$ endlich, so ist $E|(\overline{K}\cap E)$ eine endliche Erweiterung von $F|K$.
    Ist $K \subsetneqq F_0 \subseteq F$, so ist $F|K$ eine endliche geometrische Erweiterung des Funktionenkörpers $F_0|K$.
\end{beispiel}

\begin{definition}
    Sei $E|L$ eine endliche Erweiterung von $F|K$, $Q \in S(E|L)$, $P \in S(F|K)$, $Q$ \textbf{liegt über} $P$ 
    (i.Z. $Q|P$, $Q|_F = P$) $: \iff$ $\mathcal{O}_{v_Q} \cap F = \mathcal{O}_{v_P}$ ($\mathcal{O}_Q \cap F = \mathcal{O}_P$).
    \begin{align*}
        e_{Q|P} & := e(v_Q | v_Q|_F), \text{ der \textbf{Verzweigungsindex} von } Q \text{ über } P\\
        f_{Q|P} & := f(v_Q | v_Q|_F), \text{ der \textbf{Restklassengrad} von } Q \text{ über } P\\
        Q|P \text{ ist \textbf{verzweigt}} & :\iff e_{Q|P} > 1.
    \end{align*}
\end{definition}

\begin{bemerkungnr}
    Da sowohl $v_P$ als auch $v_Q$ normiert sind, gilt
    $$ v_Q(x) = e_{Q|P} \cdot v_P(x) \text{ für alle } x \in F^\times.$$
    Ist $E|L$ eine endliche Erweiterung von $F|K$, so gibt es für jedes $P \in S(F|K)$ stets mindestens ein, abber immer nur
    endlich viele $e_{Q|P}, f_{Q|P} \leq [E:F]$ (II.3.5).
\end{bemerkungnr}

\begin{theorem}[Fundamentale Gleichung]
    Sei $E|L$ eine endliche Erweiterung von $F|K$, sei $P \in S(F|K)$ und seien $Q_1,\ldots,Q_m \in S(E|L)$ alle Stellen von
    $E$, über $P$. Dann ist
    $$ \sum\limits_{i=1}^m e_{Q_i|P}f_{Q_i|P} = [E:F]. $$
\end{theorem}
\begin{proof}
    Wähle ein $x \in F^\times$ mit einziger Nullstelle $P$ (\#17 liefert Existenz).
    Situation:
    $$\begin{tikzcd}
        & E\\
        F \rar[dash] & FL \uar[dash]\\
        K(x) \uar[dash] \rar[dash] & L(x) \uar[dash]\\
        K \uar[dash] \rar[dash] & L \uar[dash]  
    \end{tikzcd}$$
    \begin{align*}
        [E:K(x)] & = [E:F] \cdot [F:K(x)] \stackeq{II.5.7} \deg (x)_0^F \cdot [E:F] = v_P(x) \cdot \deg P \cdot [E:F]\\
        [E:K(x)] & = [E:L(x)] \cdot [L(x) : K(x)] \stackeq{II.5.7 \& I.1.8} \deg (x)_0^E \cdot [L:K] \\
        & = \sum\limits_{i=1}^m v_{Q_i}(x)\cdot \deg Q_i\cdot [L:K] = \sum\limits_{i=1}^m e_{Q_i|P} \cdot v_P(x) \cdot [E_{Q_i]:L}] \cdot [L:K]\\
        & = v_P(x) \cdot \sum\limits_{i=1}^m e_{Q_i|P} \cdot [E_{Q_i}:F_P] \cdot [F_P : K] = v_P(x) \cdot \deg P \cdot \sum\limits_{i=1}^m e_{Q_i|P} \cdot f_{Q_i|P}
    \end{align*}
\end{proof}

\begin{definition}
    Sei $E|L$ endliche Erweiterung von $F|K$. Für $P \in S(F|K)$ ist 
    \begin{align*}
        \Con_{E|F}(P) := \sum\limits_{Q|P} e_{Q|P} Q \in \Div(E|L), \text{ wobei } Q \in S(E|L)
    \end{align*}
    die \textbf{Konorm} von $P$. Wir setzen dies linear fort zu
    $$ \Con_{E|L} : \Div(F|K) \to \Div(E|L).$$
\end{definition}

\begin{lemma}
    Ist $E|L$ endliche Erweiterung von $F|K$, und $x \in F^\times$, so ist
    \begin{align*}
        \Con_{E|F}(x)^F = (x)^E,& \Con_{E|F} (x)_0^F = (x)_0^E,& \Con_{E|F} (x)_\infty^F = (x)_\infty^E.
    \end{align*}
\end{lemma}
\begin{proof}
    \begin{align*}
        (x)^E & = \sum\limits_Q v_Q(x) Q = \sum\limits_P \sum\limits_{Q|P} e_{Q|P} \cdot v_P(x) \cdot Q\\
        & = \sum\limits_P v_P(x) \cdot \sum\limits_{Q|P} e_{Q|P} \cdot Q = \sum\limits_P v_P(x) \cdot \Con_{E|F} P\\
        & = \Con_{E|F} \sum\limits_P v_P(x) \cdot P = \Con_{E|F} (x)^F.
    \end{align*}
    Da $e_{Q|P} > 0$ folgt $(x)^E_0 = (x)^E_+ = (\Con_{E|F} (x)^F)_+ = \Con_{E|F} (x)^F_+ = \Con_{E|F}(x)^F_0$
    analog für $(x)_\infty$.

\end{proof}

\begin{bemerkungnr}
    Der Morphismus $\Div(F|K) \to \Div(E|L)$ induziert daher einen Homomorphismus
    $\mathscr{C}(F|K) \to \mathscr{C}(E|L)$, der aber im Allgemeinen weder injektiv noch surjektiv sein muss. 
\end{bemerkungnr}

\begin{satz}
    Ist $E|L$ eine endliche Erweiterung von $F|K$, und $A \in \Div(F|K)$, so ist
    $$ \deg (\Con_{E|F}A) ) \frac{[E:F]}{[L:K]} \cdot \deg (A). $$
\end{satz}
\begin{proof}
    O.E. sei $A = P \in S(F|K)$.
    \begin{align*}
        \deg \Con_{E|F}(P) & = \deg ( \sum\limits_{Q|P} e_{Q|P}Q) = \sum\limits_{Q|P}e_{Q|P}\cdot [E_Q:L]\\
        & = \frac{1}{[L:K]}\cdot \sum\limits_{Q|P} e_{Q|P} \cdot [E_Q: F_P] \cdot [F_P : K]\\
        & \stackeq{1.6} \frac{1}{[L:K]} \cdot [E:F] \cdot \deg P.
    \end{align*}
\end{proof}

\section{Holomorphieringe}
Sei $F|K$ ein Funktionenkörper, und $S := S(F|K)$.

\begin{definition}
    Für $S_0 \subseteq S$ sei $$\mathcal{O}_{S_0} := \bigcap_{P \in S_0} \mathcal{O}_P.$$ 
    Ein Teilring $R \subseteq F$ ist ein \textbf{Holomorphiering}, wenn $R = \mathcal{O}_{S_0}$ für ein $\emptyset \ne S_0 \subsetneqq S$.
\end{definition}

\begin{lemma}
    Sei $\emptyset \ne S_0 \subsetneqq S$.
    \begin{enumerate}[label=(\alph*)]
        \item $K \subseteq \mathcal{O}_{S_0} \subsetneqq F$.
        \item $\mathcal{O}_{S_0}$ ist ein ganzabgeschlossener Ring aber kein Körper.
        \item $\Quot(\mathcal{O}_{S_0}) = F$
        \item Für $P \in S$ gilt:
        $$ \mathcal{O}_{S_0} \subseteq \mathcal{O}_P \iff P \in S_0.$$
    \end{enumerate} 
\end{lemma}
\begin{proof}
    (a) $\checkmark$ (b) Jedes $\mathcal{O}_P$ ganzabgeschlossen (II.1.6) $\implies$ $\mathcal{O}_{S_0}$ ganzabgeschlossen.
    Wähle $P_0 \in S_0$. Da $S_0 \subsetneqq S$ existiert nach II.7.12 ein $x \in F^\times$ mit
    $$ v_{P_0} > 0 \text{ und } v_P(x) \geq 0 \quad \forall P \in S_0 \setminus \{P_0\} \implies x \in \mathcal{O}_{S_0}, x^{-1} \notin \mathcal{O}_{S_0}.$$
    (c) Sei $x \in F^\times$. $\stackrel{II.7.12}{\implies}$ ex. $z \in F^\times$ mit $v_P(z)$ für $P$ mit $v_P(x) < 0$
    und $v_P(z) \geq 0$ für $P \in S_0$. $\implies$ $z \in \mathcal{O}_{S_0}$, $xz \in \mathcal{O}_{S_0}$, $\frac{xz}{z} \in \Quot(\mathcal{O}_{S_0})$.\\
    (d) Sei $P_1 \in S \setminus S_0$. Wähle $P_0 \in S_0$.\\
    Nach II.7.12 existiert $x \in F^\times$ mit $v_{P_0}(x) > 0, v_P(x) \geq 0 \quad \forall P \in S \setminus\{P_1\}$
    \begin{align*}
        & \stackrel{\deg (x) = 0}{\implies} v_{P_1} < 0\\
        & \implies x \in \mathcal{O}_{S_0} \setminus \mathcal{O}_{P_1}.
    \end{align*}
\end{proof}

\begin{beispiel}
    Sei $K = \overline{K}, F = K(T), S = \{P_a \mid a \in K\} \cup \{P_\infty\}$.
    \begin{itemize}
        \item $S_0 = S \setminus \{P_\infty\}: \mathcal{O}_{S_0} = \{\frac{f}{g} \mid v_{P_a}(f) - v_{P_a}(g) \geq 0 \quad \forall a \in K\}$
        \item $S_0 = S$: $\mathcal{O}_{S_0} = K$
        \item $S_0 = S \setminus \{P_0\}$: $\mathcal{O}_{S_0}=\{\frac{f}{g} \mid v_P(f) \geq v_P(g) \quad \forall P \in S_0\}$
        \begin{align*}
            & = \{\frac{f}{T^k} \mid v_{P_\infty} (f) \geq v_{P_\infty}(T^k)\}\\
            & = \{\frac{f}{T^k} \mid \deg f \leq k\}\\
            & = K[T^{-1}]
        \end{align*}
    \end{itemize}
\end{beispiel}

\begin{lemma}
    Ist $K \subseteq R \subseteq F$ ein Ring, der kein Körper ist,
    $$ S(R) := \{P \in S \mid R \subseteq \mathcal{O}_P\},$$
    so ist $\emptyset \ne S(R) \subsetneqq S$, und $\mathcal{O}_{S(R)}$ ist der ganze Abschluss von $R$ in $F$.
\end{lemma}
\begin{proof}
    $R$ ist kein Körper $\implies$ es gibt $(0) \neq \mathfrak{p} \in \Spec R$
    $\stackrel{II.2.2}{\implies}$ es gibt $P \in S(F|K)$ mit $R \subseteq \mathcal{O}_P$ (und $m_\mathfrak{p} \cap R = \mathfrak{p}$)
    $\implies S(R) \ne \emptyset$.

    $R$ ist kein Körper $\implies$ $R \ne K$ $\implies$ es gibt $x \in R\setminus K$ $\implies$ $x$ hat Polstelle $P_0 \in S$
    $\implies x \notin \mathcal{O}_{P_0} \supseteq \bigcap\limits_{P \in S} \mathcal{O}_P \implies S(R) \ne S$.
    
    Nach \#14 ist der ganze Abschluss von $R$ in $F$ genau der Durchschnitt aller Bewertungsringe von $F$, die $R$ enthalten, also
    $\mathcal{O}_{S(R)}$.
\end{proof}

\begin{satz}
    Ist $\emptyset \ne S_0 \subseteq S$ endlich, so ist $\mathcal{O}_{S_0}$ ein Hauptidealring.
\end{satz}
\begin{proof}
    Sei $(0) \ne \mathfrak{a} \ideal \mathcal{O}_{S_0}$. Schreibe $S_0 = \{P_1, \ldots, P_n\}$.
    Für jedes $i$ wähle $a_i \in \mathfrak{a}$ mit
    \begin{align*}
        & v_{P_i}(a_i) = k_i := \min \{v_{P_i}(a) \mid a \in \mathfrak{a}\}\\
        \stackrel{II.1.9}{\implies} & \text{es gibt } z_i \in F \text{ mit } v_{P_i}(z_i) = 0, v_{P_j}(z_i) > k_j \quad \forall j\ne i.
    \end{align*} 
    Setze $$x := \sum\limits_{i=1}^n z_i \cdot a_i$$
    Dann ist $z_i \in \O_{S_0}, v_{P_i}(z_i \cdot a_i) = k_i, v_{P_j}(z_i \cdot a_i) > k_j \quad \forall j \ne i$,
    somit $x \in \mathfrak{a}$ und $v_{P_i}(x) = v_{P_i}(x_i\cdot a_i) = k_i$.

    Beh.: $\mathfrak{a} = (x)$: $a \in \mathfrak{a} \implies v_{P_i}(\frac{a}{x}) = v_{P_i}(a) - k_i \geq 0$
    $\implies \frac{a}{x} \in \O_{S_0} \implies a = \frac{a}{x} \cdot x \in (x)$.
\end{proof}

\section{Lokale Ganzheitsbasen}
Sei $F|K$ eine Funktionenkörper, $E|L$ eine endliche Erweiterung von $F|K$, $R \subseteq F$ ein Holomorphiering von $F$,
$R'$ der ganze Abschluss von $R$ in $E$.

\begin{definition}
    Sei $P \in S(F|K)$, $R = \O_P$. Eine Basis $x_1, \ldots, x_n$ von $E|F$ ist eine \textbf{Ganzheitsbasis} für $P$, wenn
    $$ R' = \sum\limits_{i=1}^n R x_i. $$
\end{definition}

\begin{lemma}
    Genau dann ist $x_1,\ldots,x_n\in E$ eine Ganzheitsbasis für $P$, wenn $$R' = \bigoplus\limits_{i=1}^n R x_i.$$
\end{lemma}
\begin{proof}
    $\implies$: $x_1, \ldots, x_n$ $F$-linear unabhängig $\implies$ $x_1, \ldots, x_n$ $R$-linear unabhängig\\
    $\implies$: $x_1,\ldots,x_n$ $R$-linear unabhängig, $F=\Quot R$ $\implies$ $x_1,\ldots,x_n$ $F$-linear unabhängig.
    \begin{align*}
        \bigoplus\limits_{i=1}^n Rx_i = R' \text{ ist ein Ring} & \implies \bigoplus\limits_{i=1}^n Fx_i \text{ ist ein Ring der }F \text{ enthält}\\
        & \implies \bigoplus\limits_{i=1}^n Fx_i \text{ ist ein Körper} \\
        & \implies \bigoplus\limits_{i=1}^n Fx_i \supseteq \Quot \bigoplus\limits_{i=1}^n Rx_i = \Quot R' = E.
    \end{align*}
\end{proof}

\begin{erinnerungnr}
    Sei $M|K$ endlich, $x \in M$. Die $M|K$-Spur $\Tr_{M|K}(x)$ von $x$ kann beschrieben werden als
    \begin{enumerate}[label=(\arabic*)]
        \item $\Tr_{M|K} = \Tr(\mu_x)$, wobei
        $$ \mu_x : \begin{cases}
            M \to M\\
            y \mapsto xy
        \end{cases}$$

        \item $\Tr_{M|K}(x) = - [M:K(x)] \cdot a_{m-1}$, $\MinPol (x|K) = x^m + a_{m-1}x^{m-1} + \ldots + a_0$
        \item $\Tr_{M|K}(x) = \frac{[M:K]}{n} \cdot \sum\limits_{i=1}^n \sigma_i$, $\Hom_K (M, \overline{K}) = \{\sigma_1, \ldots, \sigma_n\}$
    \end{enumerate}
    Genau dann ist $\Tr_{M|K}: M \to K$ nicht die Nullabbildung, wenn $M|K$ separabel ist.
\end{erinnerungnr}

\begin{lemma}
    Sei $M|K$ endlich und 
    $$ \varphi : \begin{cases}
        M \times M \to K\\
        (x,y) \mapsto \Tr_{M|K}(xy)
    \end{cases}$$
    die \textbf{Spurform} von $M|K$.
    \begin{enumerate}[label=(\alph*)]
        \item $\varphi$ ist $K$-bilinear und symmetrisch
        \item $\varphi$ ist genau dann nicht-ausgeartet, wenn $M|K$ separabel ist. In diesem Fall ist
        $$ \varphi': \begin{cases}
            M \to M^\ast\\
            x \mapsto \varphi(x,-)
        \end{cases}$$
        ein Isomorphismus.

        \item Ist $M|K$ separabel und $x_1,\ldots,x_n$ eine Basis von $M|K$, so gibt es eine eindeutig bestimmte Basis
        $x_1',\ldots,x_n'$ von $M|K$ mit $\Tr_{M|K}(x_i\cdot x_j) = \delta_{ij}$ für alle $i,j$ (die \textbf{duale Basis}).
    \end{enumerate}
\end{lemma}
\begin{proof}
    (a) $\checkmark$ (b) $M|K$ separabel $\implies$ $\Tr_{M|K} \not \equiv 0 \implies$ es gibt $a \in M: \Tr_{M|K}(a) \neq 0$
    $\implies \forall 0 \ne x \in M$ ist $\varphi(x, x^{-1}a) = \Tr_{M|K}(a) \ne 0$.\\
    (c) Sei $x_1^\ast, \ldots, x_n^\ast \in M^\ast$ die duale Basis zu $x_1,\ldots,x_n$, d.h. $x_i^\ast(x_j) = \delta_{ij} \forall i,j$.
    Setze $x_i' := (\varphi')^{-1}(x_i^\ast) \in M$. Dann ist $x_1',\ldots,x_n'$ Basis von $M|K$ und 
    $$ \Tr_{M|K}(x_i x_j') = \varphi(x_j',x_i) = \varphi'(x_j')(x_i) = x_j^\ast(x_i) = \delta_{ij}.$$
    Eindeutigkeit folgt aus Eindeutigkeit der dualen Basis $x_1^\ast,\ldots,x_n^\ast$.
\end{proof}

\begin{lemma}
    $$ \Tr_{E|F} (R') \subseteq R.$$
\end{lemma}
\begin{proof}
    $x \in E$ ganz über $R$, $R$ ganzabgeschlossen $\implies$ $\Tr_{E|F}(x) \in R$\\
    aus (2): $\MinPol(x|F) \in R[X] \stackrel{(2)}{\implies} \Tr_{E|F}(x) \in R$\\
    aus (3): $x$ ganz über $R \implies \sigma_i(x)$ ganz über $R$
    $ \implies \Tr_{E|F}(x) \text{ganz über }R \stackrel{R \text{ g.abg.}}{\implies} Tr {E|F} (x) \in R.$
\end{proof}

\begin{lemma}
    Sei $E|F$ separabel.
    \begin{enumerate}[label=(\alph*)]
        \item Ist $x_1,\ldots,x_n$ eine Basis von $E|F$, so existieren $a_1, \ldots, a_n \in R$, \\für die 
        $a_1 x_1,\ldots, a_n x_n$ eine Basis von $E|F$ ist, die in $R'$ enthalten ist.

        \item Ist $x_1,\ldots,x_n \in R'$ eine Basis von $E|F$, so ist
        $$ \sum\limits_{i=1}^nRx_i \subseteq R' \subseteq \sum\limits_{i=1}^n Rx_i'.$$

        \item Ist $R$ ein Hauptidealring, so gibt es eine Basis $x_1,\ldots, x_n \in R'$ von $E|F$ mit 
        $$ \sum\limits_{i=1}^n Rx_i = R'.$$
    \end{enumerate}
\end{lemma}
\begin{proof}
    (a): $x \in E$ $\implies$ $x$ algebraisch über $F$ $\implies$ es gibt $0 \ne a \in R$ mit $ax$ ganz über R.
    \begin{align*}
        \MinPol(x|F) &= X^n + \frac{a_{n-1}}{b_{n-1}}X^{n-1} + \ldots + \frac{a_0}{b_0} \quad \text{mit } a_i,b_i \in R\\
        b &:= b_0 \cdot \ldots \cdot b_{n-1} \implies (bx)^n + \frac{b}{b_{n-1}}a_{n-1}(bx)^{n-1} + \ldots + \frac{b^n}{b_0}a_0 = 0\\
        &\implies bx \text{ ganz über }R.
    \end{align*}
    (b): $x \in R' \implies x = \sum\limits_{i=1}^n \alpha_i x_i' \text{ mit } \alpha_1, \ldots, \alpha_n \in F 
    \implies \alpha_i = \Tr_{E|F}(\underbrace{xx_i}_{\in R'}) \stackrel{3.5}{\in} R \forall i$\\
    (c): Nach (a) und (b) existiert Basis $x_1, \ldots, x_n \in R'$ von $E|F$ mit
    $$ \sum\limits_{i=1}^n Rx_i \subseteq R' \subseteq \sum\limits_{i=1}^n R x_i'.$$
    Da $R$ Hauptidealring ist, ist der Untermodul $R'$ des endlich erzeugten $R$-Moduls $\sum Rx_i'$ wieder endlich erzeugt, genauer
    sogar $R' = \sum\limits_{i=1}^m Ry_i$ mit $m \leq n$. Da $R'$ die Basis $x_1, \ldots, x_n$ von $E|F$ enthält, ist $m=n$ und $y_1, \ldots,y_n$
    ist auch Basis von $E|F$.
\end{proof}

\begin{satz}
    Sei $E|F$ separabel. Für jedes $P\in S(F|K)$ gibt es eine Ganzheitsbasis für $P$ in $E|F$, und jede Basis von $E|F$ ist Ganzheitsbasis,
    für fast alle $P\in S(F|K)$.
\end{satz}
\begin{proof}
    Da $R=\O_P$ Hauptidealring ist, folgt Existenz einer Ganzheitsbasis aus 3.6. Sei $x_1,\ldots,x_n$ Basis von $E|F$.
    Definiere 
    \begin{align*}
        S_0 & = \text{ Menge der Null- und Polstellen in } S(F|K)\\ 
        &\text{ der Koeffizienten der Minimalpolynome von } \\
        &x_1,\ldots,x_n,x'_1,\ldots,x'_n\\
        & \implies S_0 \text{ ist endlich}\\
        \text{Sei } P\in S(F|K)\setminus S_0, R = \O_P & \implies x_1,\ldots,x_n,x'_1,\ldots,x'_n \in R'\\
        & \stackrel{3.6(b)}{\implies} \sum Rx_i \subseteq R' \subseteq \sum Rx'_i \subseteq R' \subseteq \sum Rx''_i = \sum Rx_i\\
        & \implies \sum Rx_i = R'.
    \end{align*}
\end{proof}

\begin{beispiel}
    $F=K(T), R=K[T]=\O_{S(K(T)|K)\setminus\{P_\infty\}}$ ist ein Hauptidealring.
    \begin{align*}
        E|F \text{ endlich, geometrisch und separabel } & \stackrel{3.6(c)}{\implies} R' = \bigoplus\limits_{i=1}^n K[T]x_i\\
        & \implies x_1,\ldots,x_n \text{ ist Ganzheitsbasis für alle } \\&\qquad P \in S(K(T)|K)\setminus\{P_\infty\},
    \end{align*}
    aber \underline{nicht} für $P=P_\infty$, denn sonst wäre $x_i \in \bigcap\limits_{Q \in S(E|L)} \O_Q = L$.
\end{beispiel}

\section{Differente und Kospur}
Sei $F|K$ Funktionenkörper, E endliche, separable und geometrische Erweiterung von $F|K$,
$P \in S(F|K)$, $\O'_P$ der ganze Abschluss von $\O_P$ in $E$.

\begin{definition}
    Der \textbf{Komplementärmodul} zu $P$ ist
    $$ C_P := \left\{z \in E \mid \Tr_{E|F}(z\O'_P)\subseteq\O_P\right\}.$$
\end{definition}

\begin{satz}
    \begin{enumerate}[label=(\alph*)]
        \item $C_P$ ist ein $\O'_P$-Modul, und $\O'_P \subseteq C_P$
        \item Ist $z_1,\ldots,z_n$ Ganzheitsbasis für $P$, so ist 
        $$ C_P = \sum\limits_{i=1}^n \O_P z'_i.$$

        \item $C_P = t_P \cdot \O'_P$ für ein $t_P \in E$ mit $v_Q(t_P) \leq 0~ \forall Q|P$.
        \item Für $t \in E$ gilt:
        $$ C_P = t\cdot\O'_P \iff v_Q(t) = v_Q(t_P)~ \forall Q|P$$

        \item $C_P = \O'_P$ für fast alle $P \in S(F|K)$.
    \end{enumerate}
\end{satz}
\begin{proof}
    (a) 3.5 (b) 
    \begin{align*}
        \underline{\subseteq:~}& z \in C_P \implies z = \sum\limits_{i=1}^n x_iz'_i, x_i \in F \implies x_i = \Tr_{E|F}(\underbrace{z}_{\in C_P}\overbrace{z_i}^{\in \O'_P})\\
        \underline{\supseteq:~}& z = \sum\limits_{i=1}^n x_i z'_i, x_i \in \O_P, u \in \O'_P\\
        & \text{ mit } u = \sum\limits_{i=1}^n y_iz_i, y_i \in \O_P \text{ und } z_1, \ldots, z_n \text{ Ganzheitsbasis von } \O_P\\
        & \implies \Tr_{E|F}(zu) = \sum\limits_{i,j}\Tr_{E|F}(z'_iz_j) = \sum\limits_{i=1}^n x_iy_i \in \O_p
    \end{align*}
    (c) Nach 3.7 und (b) ist $C_P = \sum\limits_{i=1}^n \O_P$ mit $u_1,\ldots,u_n$ Basis von $E|F$. Wähle $x \in F$ mit 
    \begin{align*}
        & v_P(x) \geq -v_Q(u_i)~\forall i\forall Q|P, v_P(x) \geq 0\\
        \implies & v_Q(xu_i) = e_{Q|P}v_P(x) + v_Q(u_i) \geq 0 ~\forall Q|P\\
        \implies & xC_P \subseteq \bigcap\limits_{Q|P} \O_Q = \O'_P (\stackrel{(a)}{\subseteq C_P})\\
        \stackrel{(a)}{\implies} & xC_P \ideal \O'_P
    \end{align*}
    Da $\O'_P$ Hauptidealring ist (2.5) ist $xC_P = y\O'_P$ für ein $y \in \O'_P$. Setze $t_P := \frac{y}{x}$.
    \begin{align*}
        & \implies C_P = t_P \O'_P\\
        \O'_P \stackrel{(a)}{\subseteq} C_P = t_P \O'_P & \implies t^{-1}_P \O'_P \subseteq \O'_P \implies
        t^{-1}_P\in \O'_P=\bigcap_{Q|P} \O_Q\\
        & \implies v_Q(t^{-1}_P) \geq 0 ~\forall Q|P 
    \end{align*}
    (d) 
    \begin{align*}
        C_P = t\cdot \O'_P & \iff t_P \O'_P = t \O'_P \iff t_Pt^{-1} \in (\O'_P)^\times\\
        & \iff v_Q(t_Pt^{-1}) = 0 ~\forall Q|P
    \end{align*}
    (e) Sei $z_1,\ldots,z_n$ Basis von $E|F$ $\stackrel{3.7}{\implies} z_1, \ldots,z_n$ und $z'_1,\ldots,z'_n$ sind Ganzheitsbasen
    für fast alle $P$. Für so ein $P$ ist
    $$ C_P \stackeq{(a)} \sum \O_P z'_i = \O'_P.$$
\end{proof}

\begin{definition}
    Die \textbf{Differente} von $E|F$ ist
    $$ \Diff := \Diff_{E|F} := \sum_{P\in S(F|K)}\sum_{Q|P}\d_{Q|P}Q,$$
    wobei $\d_{Q|P}:= -v_Q(t_P)$ mit $t_P$ wie in (c), der \textbf{Differentenexponent} von $Q|P$ ist.
\end{definition}

\begin{bemerkungnr}
    Nach (d) ist $\d_{Q|P}$ wohldefiniert und $\d_{Q|P} \geq 0$, und $\d_{Q|P} = 0$ für fast alle P und $Q|P$ nach (e).
    Somit ist $\Diff_{E|F}$ ein effektiver Divisor von $E|K$.
\end{bemerkungnr}

\begin{bemerkungnr}
    Für $z \in E$ gilt:
    $$ z \in C_P \iff v_Q(z) \geq - \d_{Q|P} ~\forall Q|P.$$
\end{bemerkungnr}

\begin{definition}
    Der Raum der \textbf{relativen Adele} ist
    $$ \A_{E|F} := \{\alpha \in \A_E \mid \alpha_Q = \alpha_{Q'} \text{ für } Q|F = Q'|F\}.$$
    Für $A \in \Div(E|K)$ setzen wir $\A_{E|F}(A) := \A_E(A) \cap \A_{E|F}$ und wir definieren
    $$ \Tr_{E|F} : \A_{E|F} \to \A_F, (\alpha_Q)_Q \mapsto (\Tr_{E|F}(\alpha_{Q_P}))_P$$
    wobei für jedes $P \in S(F|K)$ ein $Q_P|P$ gewählt wird.
    $$\begin{tikzcd}
      E \rar[hook] & \A_{E|F} \drar[dashed] \dar{\Tr} \rar[hook] & \A_E \\
      F \rar[hook] \uar[dash] & \A_F \rar{\omega} & K  
    \end{tikzcd}$$
\end{definition}

\begin{bemerkungnr}
    $$\begin{tikzcd}
        E \rar[hook] \dar{\Tr_{E|F}}& \A_{E|F} \dar{\Tr_{E|F}} \rar[hook] & \A_E\\
        F \rar[hook] & \A_F
    \end{tikzcd}$$
\end{bemerkungnr}
    
\begin{lemma}
    $\A_E = \A_{E|F} + \A_E(A)$ für jedes $A \in \Div(E|K)$.
\end{lemma}
\begin{proof}
    Sei $\alpha=(\alpha_Q)_Q \in \A_E$. Für jedes $P$ wähle $x_P \in E$ mit
    \begin{align*}
        v_Q(x_P - \alpha_Q) \geq - v_Q(A) ~\forall Q|P \text{(II.1.8)}.
    \end{align*}
    Für $\beta := (x_{Q|_F})_Q \implies$
    \begin{itemize}
        \item $\beta \in \A_{E|F}$: $v_Q(\alpha_Q)=0$ und $v_Q(A)=0$ für fast alle $Q$
        \item $\alpha - \beta \in \A_E(A)$: $v_Q(\alpha_Q -\beta_Q) \geq -v_Q(A)$
    \end{itemize}
\end{proof}

\begin{satz}
    Sei $0 \ne \omega \in \Omega_{F|K}$. Die Abbildung 
    $$\omega_1:= \omega \circ \Tr_{E|F}: \A_{E|F} \to K$$
    lässt sich eindeutig zu einem $\omega' \in \Omega_E$ fortsetzen.
    Es gilt
    $$ (\omega') = \Con_{E|F}((\omega)) + \Diff_{E|F}. $$
\end{satz}
\begin{proof}
    \underline{Eindeutigkeit:} Seien $\omega', \omega'' \in \Omega_E$ mit 
    \begin{align*}
        \omega'|_{\A_{E|F}} = \omega''|_{\A_{E|F}} = \omega_1.
    \end{align*}
    Setze $W:= \inf \{(\omega'), (\omega'')\}$.
    Für $\alpha \in \A_E$ ist nach 4.8
    \begin{align*}
        & \alpha = \beta + \gamma \text{ mit } \beta \in \A_{E|F}, \gamma \in \A_E(W)\\
        \implies & \omega'(\alpha) = \omega'(\beta) + \omega'(\gamma) = \omega_1(\beta)\\
        & \qquad \omega''(\beta) + \omega''(\gamma) = \omega''(\alpha)
    \end{align*}
    \underline{Existenz:} Setze $W':=\Con_{E|F}((\omega)) + \Diff_{E|F}$.\\
    \underline{Behauptung 1:} $\omega_1|_{\A_{E|F}(W')+E} = 0$\\
    \underline{Behauptung 2:} $W'$ ist der größte Divisor mit dieser Eigenschaft.

    Für $\alpha \in \A_E$ schreibe $\alpha = \beta + \gamma$ mit
    $$ \beta \in \A_{E|F}, \gamma \in \A_E(W')~(4.8),$$
    setze $\omega'(\alpha) := \omega_1(\beta)$.
    
    $\omega'$ ist wohldefiniert:
    \begin{align*}
        & \alpha = \beta_1 + \gamma_1 = \beta_2 + \gamma_2, \beta_i \in \A_{E|F}, \gamma_i \in \A_E(W')\\
        \implies & \beta_1 - \beta_2 = \gamma_2 - \gamma_1 \in \A_{E|F}\cap\A_E(W')=\A_{E|F}(W')\\
        \implies & \omega_1(\beta_1) - \omega_1(\beta_2) = \omega_1(\beta_1 - \beta_2) \stackeq{Beh. 1} 0
    \end{align*}

    $\omega'$ ist linear: $\checkmark$

    $\omega'|_{\A_E(W')+E} = 0$: Beh. 1

    $(\omega')=W'$: 
    \begin{align*}
        & \omega'|_{\A_E(A)}=0, z.z. A \subseteq W'\\
        \implies & \omega_1|_{\A_E(A) \cap \A_{E|F}} = 0\\
        \implies & \omega_1|_{\A_{E|F}(A)} = 0\\
        \stackrel{Beh. 2}{\implies} & A \subseteq W'
    \end{align*}
    \textit{Beweis von Beh. 1:} $\omega|_F=0, \Tr_{E|F} \subseteq F \implies \omega_1|_E = 0$
    Sei $\alpha \in \A_{E|F}(W'), P \in S(F|K), Q|P$. Wähle $x \in F$ mit $v_P(x) = v_P((\omega))$.
    \begin{align*}
        \implies v_Q(x\alpha_Q) & = e_{Q|P}v_P(x) + v_Q(\alpha_Q) = e_{Q|P}v_P((\omega)) - v_Q(W')\\
        & = v_Q(\Con_{E|F}((\omega))) - v_Q(W')\\
        & = -v_Q(\Diff_{E|F}) = - \d_{Q|P}\\
    \end{align*}
    \begin{align*}
        \implies & x\alpha_Q \in C_P\\
        \implies & \Tr_{E|F}(x\alpha_Q) \in \O_P\\
        \implies & v_P(\Tr_{E|F}(\alpha_Q)) \geq -v_P(x) = -v_P((\omega))\\
        \implies & \Tr_{E|F}(\alpha) \in \A_F((\omega)), \text{ d.h. } \omega(\Tr_{E|F}(\alpha)) = 0
    \end{align*}
    \textit{Beweis von Beh. 2:} Sei $B \not \leq W'$. Dann existiert $P_0, Q_0|P_0$ mit
    $$ v_{Q_0}(B) > v_{P_0}(\Con_{E|F}((\omega))) + \d_{Q_0|P_0}. $$
    Nach Definition von $(\omega)$ existiert $\alpha \in \A_F((\omega) + P_0)\setminus\A_F((\omega))$ mit
    $\omega(\alpha)\ne 0$.
    Setze $x:= \alpha_{P_0}$
    \begin{align*}
        \implies & v_{P_0}(x) = v_{P_0}(\alpha) = -v_P((\omega)) - 1.
    \end{align*}
    Definiere $\alpha_0 \in \A_F$ durch 
    $$(\alpha_0)_P = \begin{cases}
        x, & P=P_0\\
        0, & P \ne P_0
    \end{cases}$$.
    \begin{align*}
        \implies & 0 \ne \omega(\alpha) = \underbrace{\omega(\underbrace{\alpha-\alpha_0}_{\in \A_F((\omega))})}_{=0}+\omega(\alpha_0)
    \end{align*}
    Wähle $y \in F$ mit $v_{P_0}(y) = v_{P_0}((\omega))$.\\
    $\implies v_{P_0}(xy) = 1$, insbesondere $xy \in t^{-1}\O_{P_0}$, wobei $t \in F$ mit $v_{P_0}(t) = 1$.\\
    Setze $J := \{z \in E \mid v_Q(z) \geq v_Q(\Con_{E|F}((\omega)) - B)~\forall Q|P_0\}$.\\
    Nach I.1.9 existiert $u \in J$ mit $v_Q(u) = v_Q(\Con(\omega)-B)~\forall Q|P_0$, \\
    und da $v_Q(\Con(\omega)-B) < -\d_{Q_0|P_0}$ folgt $J \not \subseteq C_P$.\\
    Da $J\cdot\O'_{P_0} \subseteq J$ folgt $\Tr_{E|F} J \not \subseteq \O_{P_0}$.\\
    Es existiert $r \geq 0$ mit $t^r J \subseteq \O'_{P_0}$
    \begin{align*}
        \implies & t^r\Tr_{E|F}J = \Tr_{E|F} t^r J \subseteq \O_{P_0}\\
        \implies & t^r \Tr_{E|F} \ideal \O_{P_0}, \text{ d.h. } t^r \Tr_{E|F}J = t^s \cdot \O_{P_0} \text{ mit } s \geq 0\\
        \implies & \Tr_{E|F}J = t^m \O_{P_0} \text{ mit } m < 0, \text{ insbesondere } \Tr_{E|F}J \supseteq t^{-1}\O_{P_0}\\
        \implies & xy = \Tr_{E|F} z \text{ mit } z \in J.
    \end{align*}
    Definiere $\beta_0 \in \A_{E|F}$ durch
    $$ (\beta_0)_Q = \begin{cases}
        y^{-1}z, & Q|P_0\\
        0, & \text{sonst}.
    \end{cases}$$
    $\omega_1(\beta) \ne 0:$ $\omega_1(\beta_0) = \omega(\Tr \beta_0) = \omega(\alpha_0)$\\
    $\beta_0 \in \A_{E|F}(B):$ Für $Q|P_0$ ist $v_Q(y^{-1}z) \geq -v_Q(B)$.
\end{proof}

\begin{definition}
    Das $\omega'$ aus 4.9 heißt die \textbf{Kospur} $\Cotr_{E|F}(\omega)$ von $\omega$.
\end{definition}

\begin{theorem}[Riemann-Hurwitz]
    Ist $F|K$ ein Funktionenkörper, $E|K$ eine endliche, separable geometrische Erweiterung von $F|K$, so ist
    $$ 2g_{E|K} - 2 = [E:F] \cdot (2g_{F|K} - 2) + \deg \Diff_{E|F}. $$
\end{theorem}
\begin{proof}
    Sei $0 \ne \omega \in \Omega_F$.
    \begin{align*}
        \stackrel{4.9}{\implies} & (\Cotr_{E|F}(\omega)) = \Con_{E|F}((\omega)) + \Diff_{E|F}\\
        \implies & 2g_E - 2 \stackeq{I.9.7} \deg(\Cotr_{E|F}(\omega)) \\
        & \qquad = \deg \Con_{E|F}((\omega)) + \deg \Diff_{E|F}\\
        & \qquad \stackeq{1.10} [E:F] \cdot \deg ((\omega)) + \deg \Diff\\
        & \qquad \stackeq{I.9.7} [E:F] \cdot (2g_F - 2) + \deg \Diff.
    \end{align*}
\end{proof}

\begin{korollar}
    Ist $E|K$ eine endliche separable geometrische Erweiterung von $F|K$, so ist
    $$ 2g_F - 2 \geq (2g_F - 2)\cdot [E:F].$$
    Insbesondere ist $g_E \geq g_F$.
\end{korollar}

\begin{korollar}
    Ist $\Char(K) = 0$ und $K \subsetneqq F \subseteq K(T)$, so ist $F \cong_K K(T)$.
\end{korollar}
\begin{proof}
    \#23: Funktionenkörper $F|K$ ist rational $\iff$ $g_F=0$ und es gibt $P \in S(F|K)$ mit $\deg P = 1$
    \begin{align*}
        g_{K(T)} = 0 \implies g_F \leq g_{K(T)} = 0, P_\infty|_F \in S(F|K), \deg P_\infty|_F = 1
    \end{align*}
\end{proof}