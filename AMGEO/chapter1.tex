% This work is licensed under the Creative Commons
% Attribution-NonCommercial-ShareAlike 4.0 International License. To view a copy
% of this license, visit http://creativecommons.org/licenses/by-nc-sa/4.0/ or
% send a letter to Creative Commons, PO Box 1866, Mountain View, CA 94042, USA.

\chapter{Lie algebras}
\section{Lie algebras}
\begin{definition}
    A \textbf{Lie algebra} over a field $\F$ 
    is a vector space $L$ together with a bilinear binary operator 
    $$ \bracket : L \times L \to L $$ satisfying for all $x,y,z \in L$:
    \begin{enumerate}[label=(\arabic*)]
        \item $[x,x] = 0$ "' skew symmetry "'\label{item:skew}
        \item $[[x,y],z] + [[y,z],x] + [[z,x],y] = 0$ "'jacobi identity"'.
    \end{enumerate}

    The binary operator $\bracket$ is called \textbf{Lie bracket} or \textbf{commutator}.
\end{definition}

\begin{remark}
    Note that with property \labelcref{item:skew} and the bilinearity we get
    \begin{align*}
        [(x+y),(x+y)] = [x,x] + [x,y] + [y,x] + [y,y] = 0 \\
        \implies [x,y] = -[y,x].
    \end{align*}
    Conversely, if $\bracket$ is bilinear and $[x,y] = -[y,x]$, we get
    \begin{align*}
        \implies 2 [x,x] = 0\\
        \implies [x,x] = 0,
    \end{align*}
    unless $\textrm{char } \F = 2$.
\end{remark}

\begin{notation}
    If we just write $\F$ we mean an algebraically closed field with $\Q \subseteq \F$.
    Or equivalently, $\F$ is algebraically closed and has characteristic $0$.
\end{notation}

\begin{example}
    Let $A$ be any associative algebra over $\F$, e.g. $A = \textrm{End}(V)$, 
    where $V$ is a vector field over $\F$.

    Let $L = A$ as a vector space and $[X,Y] = XY -YX$ the "'commutator"', 
    then $(L, \bracket)$ form a Lie algebra.
\end{example}
\begin{proof} We skip the jacobi identity, as it would lead to 12 different terms, which had to cancel out.
    \begin{itemize}
        \item \underline{bilinearity}:
        \begin{align*}
            [X + Y, Z] &= (X + Y)Z - Z(X + Y) \\
                       &= XZ + YZ - ZX - ZY \\
                       &= [X,Z] + [Y,Z] \\
            [\lambda X, Y] &= \lambda XY - Y\lambda X \\
                           &= \lambda (XY -YX) \\
                           &= \lambda [X,Y]
        \end{align*}

        \item \underline{skew symmetry}:
        $$ [X,X] = XX - XX = 0 $$
    \end{itemize}
\end{proof}

\begin{definition}
    If we consider $\textrm{End}(V)$ as Lie algebra in this way, we write $\gl (V)$, and call it
    "'general linear algebra"'.
\end{definition}

\begin{remark}
    We stick to finite dimensional vector spaces.\\
    If we pick a basis of $V$, this induces an isomorphism of associative algebras
    $$ \textrm{End}(V) \cong M_d(\F), $$ where $d := \dim V$, 
    hence
    $$ \dim \textrm{End}(V) = d^2. $$
\end{remark}

\begin{definition}
    Any Lie subalgebra $L \subseteq \gl(V)$ is called \textbf{linear}, i.e.
    if it is closed under the Lie bracket. 
\end{definition}

\begin{remark}
    Write $e_{ij}$ for the matrix with only zeros, but one $1$ on the $i$-th row and the $j$-th column.
    These form a basis for $\gl(d,\F)$.

    Consider
    $$e_{ij} e_{kl} = \delta_{jk}e_{il}, $$ 
    hence
    $$[e_{ij}, e_{kl}] = \delta_{jk}e_{il} - \delta_{li} e_{kj}.$$
    Note the coeffitients in front of each $e_{rs}$ are 1,0 or  $-1$.
\end{remark}

\begin{example}
    The "'classical algebras"' are for $l \in \N$:
    $$A_l, B_l, C_l, D_l. $$

    \begin{itemize}[label=]
        \item \underline{"special linear algebra" $A_l$}:
        Let $\dim V = l+1$. $A_l$ is the set of all linear maps on $V$ with trace 0
        $$ A_l := \sl(V) := \sl(l+1, \F) := \{f \in \textrm{End}(V) \| \tr(f) = 0\}.$$

        Note the trace is defined as follows:
        $$ \tr : \begin{cases}
            \End(V) \to \F \\
            (a_{ij}) \mapsto \sum\limits_{i=1}^{l+1} a_{ii}             
        \end{cases}. $$
        Note also that this does not depend on the chosen basis, 
        as one may show that for all $g \in \GL(V)$:
        $$ \tr(gag^{-1}) = \tr(a). $$

        If now $a,b \in \sl$ we get
        \begin{align*}
            \tr(ab) = \tr(ba) \implies \tr [a,b] = \tr (ab - ba) = \tr(ab) - \tr(ba) = 0. 
        \end{align*}
        So $\sl(V)$ is linear subalgebra.
        We can pick the following basis for $\sl(V)$
        $$\begin{cases}
            e_{ij}, & i \ne j \\
            e_{ii} - e_{(i+1) (i+1)}, & i=1,...,l. 
        \end{cases}$$
        Hence
        \begin{align*}
            \dim \sl(l+1, \F) &= (l+1)^2 - (l+1) + l\\ 
                              &= l^2 +2l +1 -l -1 + l\\
                              &= l^2 +2l
        \end{align*}
        Note the $\dim \gl(l+1,\F) = l^2 + 2l +1 = (l+1)^2 $.

        \item \underline{"symplectic algebra" $C_l$}:
        \renewcommand{\sp}{\mathfrak{sp}} 
        Note a bilinear form $B$ is called \textbf{nondegenerate}, if 
        \begin{enumerate}
            \item $\forall x: B(x,y) = 0 \implies y = 0$ and
            \item $\forall y: B(x,y) = 0 \implies x = 0$.
        \end{enumerate}

        First we define a nondegenerate bilinear skew symmetric form 
        $$ f: V \times V \to \F,$$
        corresponding to the matrix
        $$ S := \begin{pmatrix}
            0 & E \\
            -E & 0
        \end{pmatrix}$$, that means for all $v,w \in V$
        $$ f(v,w) := v^TSw.$$
        (One can show that the dimension of $V$ has to be even in order to find 
        such an $f$.)
        Such an $f$ satisfies $$f(v,w) = -f(w,v)$$ for all $v,w \in V$.
        
        Define
        \begin{align*}
            C_l &:= \sp(V) := \sp(2l, \F) \\
                &:= \{x \in \End(V) \| \forall v,w \in V: f(x(v),w) = f(v, x(w))\}, 
        \end{align*}
        where $\dim V = 2l$.

        The following equivalences hold:
        \begin{align*}
            x = \begin{pmatrix}
                m & n \\
                p & q
            \end{pmatrix} \in \sp(V) &\iff n^T = n, p^T = p, m^T=-q \\
                                     &\iff Sx = -x^TS.
        \end{align*}
        Hence for a basis we can choose:
        \begin{align*}
            \begin{cases}
                e_{ii} - e_{(l+i)(l+i)}, & \text{if } i=1,...,l\\
                e_{ij} - e_{(j+l)(i+l)}, & \text{if } 1 \leq i\ne j \leq l\\
                e_{i (l+j)} + e_{j (l+i)}, & \text{if } 1 \leq i < j \leq l \\
                e_{(l+i) j} + e_{(l+j) i}, & \text{if } 1 \leq i < j \leq l \\
                e_{i(l+1)}, & \text{if } i=1,...,l \\
                e_{(l+1)i}, & \text{if } i=1,...,l.
            \end{cases}
        \end{align*}
            This leads to $$\dim \sp(2l, \F) = 2l^2 +l.$$

        \item \underline{"orthogonal algebras"} $B_l,D_l$:
        \renewcommand{\o}{\mathfrak{o}}
        Let $f$ be a nondegenerate \underline{symmetric} form corresponding to
        $$ S := \begin{pmatrix}
            1 & 0 & 0 \\
            0 & 0 & E_l \\
            1 & E_l & 0
        \end{pmatrix}.$$
        We define
        $$B_l := \o(2l+1, \F) := \{x \in \End(V) \| \forall v,w \in V: f(x(v),w) = f(v, x(w))\}.$$
        
        For $D_l$ let $f$ be a nondegenerate \underline{symmetric} form corresponding to
        $$ S := \begin{pmatrix}
            0 & E_l \\
            E_l & 0
        \end{pmatrix}.$$
        We define
        $$D_l := \o(2l, \F) := \{x \in \End(V) \| \forall v,w \in V: f(x(v),w) = f(v, x(w))\}.$$
        
        Other Examples of linear Lie algebras we will use are:
        \begin{itemize}
            \item $\mathfrak{t}(d,\F)$ upper triangualar matrices
            \item $\mathfrak{n}(d,\F)$ strictly upper triangualar matrices, i.e. 0 on diagonal
            \item $\mathfrak{d}(d,\F) \subseteq \mathfrak{t}(d,\F)$ diagonal matrices, here we have $\bracket = 0$.
        \end{itemize}
        Note $ \mathfrak{n}(d, \F) = [L,L]$, where $L:= \mathfrak{t}(d,\F)$.
        Warning for $H,K \subseteq L$ sub spaces we define
        $$ [H,K] := \text{span} \{[x,y] \| x\in H,y \in K\}. $$
    \end{itemize}
\end{example}

\section{Lie algebras of derivations}

\begin{definition}
    Let $A$ be a vector space with a bilinear binary operator
    $$ A \tensor A \to A, x\tensor y \mapsto xy$$
    (e.g. associative or Lie).

    Then a linear map $ \partial: A \to A $ is a \textbf{derivation}, if for all $x,y \in A$ 
    $$ \partial(xy) = x\partial(y) + \partial(x)y. $$
    The set of all these is denoted by $\Der(A)$.
\end{definition}

\begin{proposition}
    $\Der(A)$ is a Lie algebra with the bracket
    $$ [\partial, \varphi] := \partial \varphi - \varphi \partial .$$ 
\end{proposition}
\begin{proof}
    Let $\partial, \varphi : A \to A$ derivations. 
    Then
    \newcommand{\p}{\partial}
    \newcommand{\f}{\varphi}
    \begin{align*}   
        (\p \circ \f - \f \circ \p)(xy) =& \p \circ \f(xy) - \f \circ \p(xy)\\
        =& \p(x \f(y) + \f(x)y) - \f(x\p(y) + \p(x)y)\\
        =& \p(x\f(y)) + \p(\f(x)y) - (\f(x\p(y)) + \f(\p(x)y))\\
        =& x (\p \circ \f)(y) + \p(x)\f(y) + (\p \circ \f)(x)y + \f(x)\p(y) \\
        & - \left[\f(x)\p(y) + x(\f \circ\p)(y) + (\f\circ\p)(x)y + \p(x)\f(y)\right] \\
        =& x(\p \circ \f)(y) + (\p \circ \f)(x)y - x (\f\circ\p)(y) - (\f\circ\p)(x)y\\
        =&x\left[(\p \circ \f)(y) - \f\circ\p(y)\right] + \left[(\p\circ\f)(x) - (\f\circ\p)(x)\right]y\\
        =&x[\p,\f](y) + [\p,\f](x)y
    \end{align*}
\end{proof}

\begin{example}
    $A=L$ (with $xy=[x,y]$). Then each $x \in L$ defines a derivation 
    $$\ad(x): L \to L, y \mapsto [x,y].$$
    The Leibniz rule is just the Jacobi identity:
    \begin{align*}
        \ad(x)(yz) =& \ad(x)([y,z]) = [y, \ad(x)(z)] + [\ad(x)(y),z]\\
        \iff& \left[x,[y,z]\right] = \left[y,[x,z]\right] \left[[x,y],z\right] \\
        \iff& 0 = \left[[y,z],x\right] + \left[[z,x],y\right] \left[[x,y],z\right]
    \end{align*}
\end{example}

\begin{definition}
    A derivation $\partial \in \Der(L)$ is called \textbf{inner}, 
    if there is an $x \in L$ s.th.
    $$\partial = [x,-].$$
    We call 
    $$ \ad : L \to \Der(L) \subseteq \gl(L)$$
    the \textbf{adjoint representation} of $L$.
\end{definition}

\section{Abstract Lie agebras and structure constants}
If $L$ is any Lie algebra and $x_1,...,x_n$ is a basis, 
then $\bracket$ is  completly determined by the structure of constants $a_{ij}^k \in \F$:
$$ [x_i,x_j] = a_{ij}^k x_k .$$
Conversely, if $a_{ij}^k \in \F$, then we can use them to define a Lie algebra on $\F^n$, if
\begin{enumerate}[label=(\arabic*)]
    \item $a_{ij}^k = -a_{ji}^k$ and
    \item $a_{ij}^k a_{kl}^m + a_{jl}^k a_{ki}^m + a_{li}^k a_{kj}^m = 0$.
\end{enumerate}

\begin{example}
    Up to isomorphism, there are two non-isomorphic Lie algebras of dimension 2
    \begin{align*}
        [x,y] = 0\\
        [y,x] = 0 &\tab (x,y \text{ basis})
    \end{align*}
\end{example}

\section{Ideals and homomorphisms}
\begin{definition}
    let $L,L^{'}$ be Lie algebras. A \textbf{(homo)-morphism} is a linear map 
    $\varphi : L \to L^{'}$ s.th. for all $x,y \in L$
    $$ [\varphi(x), \varphi(y)]_{L^{'}} = \varphi([x,y]_L). $$
    A \textbf{monomorphism} is an injective Lie algebra morphism.\\ 
    An \textbf{epimorphism} is a surjective Lie algebra morphism.\\
    An \textbf{isomorphism} is a bijective Lie algebra morphism.
\end{definition}

\begin{definition}
    An \textbf{Ideal} $J \ideal L$ is a vector subspace s.th.
    for all $x \in L$ and for all $y \in J$
    $$ [x,y] \in J ,$$
    or equivalently for all $x \in J$ and for all $y \in L$
    $$ [x,y] \in J.$$ 
\end{definition}

\begin{example}
    $J=0, J=L$
    $$ Z(L) := \{x \in L \| \forall y \in L[x,y] = 0\} \ideal L $$
    $$ L^{(1)} := [L,L] \ideal L. \tab \text{"derived Lie algebra"}$$
\end{example}

\begin{definition}
    $L$ is \textbf{abelian}, if $L^{(1)} = 0$.
\end{definition}

\begin{remark}
    For $L = \sl(l+1, \F)$, we have $L^{(1)} = L$.
\end{remark}

\begin{definition}
    If $0,L$ are the only ideals in $L$ and
    $$ L^{(1)} \ne 0$$
    then $L$ is called \textbf{simple}.
\end{definition}

\begin{remark}
    So in a simple Lie algebra,$ Z(L) = 0$ and $ [L,L] = L.$
\end{remark}

\begin{example}
    $\sl(2, \F) = \{x \in M_2(\F) \mid \tr(x) = 0\}$
    $$ x = \begin{pmatrix}
        0 & 1 \\
        0 & 0
    \end{pmatrix}, y = \begin{pmatrix}
        0 & 0 \\
        1 & 0
    \end{pmatrix}, h = \begin{pmatrix}
        1 & 0 \\
        0 & -1
    \end{pmatrix}$$

    $$ \bracket[x][y] = h, \bracket[h][x] = 2x, \bracket[h][y] = - 2y, \bracket[h][h] = 0 $$
    $$ \implies \ad (h) : L \to L \cong \begin{pmatrix}
        2 & 0 & 0 \\
        0 & 0 & 0 \\
        0 & 0 & -2
    \end{pmatrix}$$

    Assume $0 \neq J \ideal L = \sl(2 , \F)$.
    $$ 0 \neq z = \alpha x + \beta y + \gamma h, z\in J $$
    If $\alpha \neq 0$ apply $\ad(y)$ twice zo $z$
    $$ -2 \alpha y = \bracket[y][\bracket[y][z]] \in J $$
    So
    \begin{align*}
        y \in J \implies& \bracket[x][y]=h \in J \implies z - \beta y - \gamma h \in J\\
        \implies & x \in J\\
        \implies & J = L,
    \end{align*}
    hence $L$ is a simple Lie algebra.
\end{example}

\section{Quotients}
\begin{definition}
    If $J \ideal L$ is an ideal, then the quotient vector space $\sfrac{L}{J}$ 
    naturally a Lie algebra with bracket
    $$ \bracket[x + J][y + J] := \bracket[x][y] + J.$$
    $\sfrac{L}{J}$ is called the \textbf{quotient of $L$ by $J$}.
    
    Indeed: If $x + J = J \iff x \in J$ then
    \begin{align*}
        \bracket[x][y] \in J \tab \forall y \in L \text{ and }\\
        \bracket[x + J][y +J] = \bracket[x][y] + J = J
    \end{align*}
\end{definition}

\textbf{Important:} Here are some parts missing. See chapter \ref{chapter0}.