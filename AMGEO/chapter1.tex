\chapter{Lie algebras}
\newcommand{\F}{\mathbb{F}}
\newcommand{\Q}{\mathbb{Q}}
\newcommand{\liebracket}{[\cdot,\cdot]}
\newcommand{\End}{\textrm{End}}
\newcommand{\gl}{\mathfrak{gl}}
\newcommand{\GL}{\textrm{GL}}
\begin{definition}
    A \textbf{Lie algebra} over a field $\F$ 
    is a vector space $L$ together with a bilinear binary operator 
    $$ \liebracket : L \times L \to L $$ satisfying for all $x,y,z \in L$:
    \begin{enumerate}[label=(\arabic*)]
        \item $[x,x] = 0$ "' skew symmetry "'\label{item:skew}
        \item $[[x,y],z] + [[y,z],x] + [[z,x],y] = 0$ "'jacobi identity"'.
    \end{enumerate}

    The binary operator $\liebracket$ is called \textbf{Lie bracket} or \textbf{commutator}.
\end{definition}

\begin{remark}
    Note that with property \labelcref{item:skew} and the bilinearity we get
    \begin{align*}
        [(x+y),(x+y)] = [x,x] + [x,y] + [y,x] + [y,y] = 0 \\
        \implies [x,y] = -[y,x].
    \end{align*}
    Conversely, if $\liebracket$ is bilinear and $[x,y] = -[y,x]$, we get
    \begin{align*}
        \implies 2 [x,x] = 0\\
        \implies [x,x] = 0,
    \end{align*}
    unless $\textrm{char } \F = 2$.
\end{remark}

\begin{notation}
    If we just write $\F$ we mean an algebraically closed field with $\Q \subseteq \F$.
    Or equivalently, $\F$ is algebraically closed and has characteristic $0$.
\end{notation}

\begin{example}
    Let $A$ be any associative algebra over $\F$, e.g. $A = \textrm{End}(V)$, 
    where $V$ is a vector field over $\F$.

    Let $L = A$ as a vector space and $[X,Y] = XY -YX$ the "'commutator"', 
    then $(L, \liebracket)$ form a Lie algebra.
\end{example}
\begin{proof} We skip the jacobi identity, as it would lead to 12 different terms, which had to cancel out.
    \begin{itemize}
        \item \underline{bilinearity}:
        \begin{align*}
            [X + Y, Z] &= (X + Y)Z - Z(X + Y) \\
                       &= XZ + YZ - ZX - ZY \\
                       &= [X,Z] + [Y,Z] \\
            [\lambda X, Y] &= \lambda XY - Y\lambda X \\
                           &= \lambda (XY -YX) \\
                           &= \lambda [X,Y]
        \end{align*}

        \item \underline{skew symmetry}:
        $$ [X,X] = XX - XX = 0 $$
    \end{itemize}
\end{proof}

\begin{definition}
    If we consider $\textrm{End}(V)$ as Lie algebra in this way, we write $\gl (V)$, and call it
    "'general linear algebra"'.
\end{definition}

\begin{remark}
    We stick to finite dimensional vector spaces.\\
    If we pick a basis of $V$, this induces an isomorphism of associative algebras
    $$ \textrm{End}(V) \cong M_d(\F), $$ where $d := \dim V$, 
    hence
    $$ \dim \textrm{End}(V) = d^2. $$
\end{remark}

\begin{definition}
    Any Lie subalgebra $L \subseteq \gl(V)$ is called \textbf{linear}, i.e.
    if it is closed under the Lie bracket. 
\end{definition}

\begin{remark}
    Write $e_{ij}$ for the matrix with only zeros, but one $1$ on the $i$-th row and the $j$-th column.
    These form a basis for $\gl(d,\F)$.

    Consider
    $$e_{ij} e_{kl} = \delta_{jk}e_{il}, $$ 
    hence
    $$[e_{ij}, e_{kl}] = \delta_{jk}e_{il} - \delta_{li} e_{kj}.$$
    Note the coeffitients in front of each $e_{rs}$ are 1,0 or  $-1$.
\end{remark}

\begin{example}
    The "'classical algebras"' are for $l \in \N$:
    $$A_l, B_l, C_l, D_l. $$

    \begin{itemize}[label=]
        \renewcommand{\sl}{\mathfrak{sl}}
        \newcommand{\tr}{\textrm{tr}}
        \item \underline{"special linear algebra" $A_l$}:
        Let $\dim V = l+1$. $A_l$ is the set of all linear maps on $V$ with trace 0
        $$ A_l := \sl(V) := \sl(l+1, \F) := \{f \in \textrm{End}(V) \| \tr(f) = 0\}.$$

        Note the trace is defined as follows:
        $$ \tr : \begin{cases}
            \End(V) \to \F \\
            (a_{ij}) \mapsto \sum\limits_{i=1}^{l+1} a_{ii}             
        \end{cases}. $$
        Note also that this does not depend on the chosen basis, 
        as one may show that for all $g \in \GL(V)$:
        $$ \tr(gag^{-1}) = \tr(a). $$

        If now $a,b \in \sl$ we get
        \begin{align*}
            \tr(ab) = \tr(ba) \implies \tr [a,b] = \tr (ab - ba) = \tr(ab) - \tr(ba) = 0. 
        \end{align*}
        So $\sl(V)$ is linear subalgebra.
        We can pick the following basis for $\sl(V)$
        $$\begin{cases}
            e_{ij}, & i \ne j \\
            e_{ii} - e_{(i+1) (i+1)}, & i=1,...,l. 
        \end{cases}$$
        Hence
        \begin{align*}
            \dim \sl(l+1, \F) &= (l+1)^2 - (l+1) + l\\ 
                              &= l^2 +2l +1 -l -1 + l\\
                              &= l^2 +2l
        \end{align*}
        Note the $\dim \gl(l+1,\F) = l^2 + 2l +1 = (l+1)^2 $.

        \item \underline{"symplectic algebra" $C_l$}:
        \renewcommand{\sp}{\mathfrak{sp}} 
        Note a bilinear form $B$ is called \textbf{nondegenerate}, if 
        \begin{enumerate}
            \item $\forall x: B(x,y) = 0 \implies y = 0$ and
            \item $\forall y: B(x,y) = 0 \implies x = 0$.
        \end{enumerate}

        First we define a nondegenerate bilinear skew symmetric form 
        $$ f: V \times V \to \F,$$
        corresponding to the matrix
        $$ S := \begin{pmatrix}
            0 & E \\
            -E & 0
        \end{pmatrix}$$, that means for all $v,w \in V$
        $$ f(v,w) := v^TSw.$$
        (One can show that the dimension of $V$ has to be even in order to find 
        such an $f$.)
        Such an $f$ satisfies $$f(v,w) = -f(w,v)$$ for all $v,w \in V$.
        
        Define
        \begin{align*}
            C_l &:= \sp(V) := \sp(2l, \F) \\
                &:= \{x \in \End(V) \| \forall v,w \in V: f(x(v),w) = f(v, x(w))\}, 
        \end{align*}
        where $\dim V = 2l$.

        The following equivalences hold:
        \begin{align*}
            x = \begin{pmatrix}
                m & n \\
                p & q
            \end{pmatrix} \in \sp(V) &\iff n^T = n, p^T = p, m^T=-q \\
                                     &\iff Sx = -x^TS.
        \end{align*}
        Hence for a basis we can choose:
        \begin{align*}
            \begin{cases}
                e_{ii} - e_{(l+i)(l+i)}, & \text{if } i=1,...,l\\
                e_{ij} - e_{(j+l)(i+l)}, & \text{if } 1 \leq i\ne j \leq l\\
                e_{i (l+j)} + e_{j (l+i)}, & \text{if } 1 \leq i < j \leq l \\
                e_{(l+i) j} + e_{(l+j) i}, & \text{if } 1 \leq i < j \leq l \\
                e_{i(l+1)}, & \text{if } i=1,...,l \\
                e_{(l+1)i}, & \text{if } i=1,...,l.
            \end{cases}
        \end{align*}
            This leads to $$\dim \sp(2l, \F) = 2l^2 +l.$$

        \item \underline{"orthogonal algebras"} $B_l,D_l$:
        \renewcommand{\o}{\mathfrak{o}}
        Let $f$ be a nondegenerate \underline{symmetric} form corresponding to
        $$ S := \begin{pmatrix}
            1 & 0 & 0 \\
            0 & 0 & E_l \\
            1 & E_l & 0
        \end{pmatrix}.$$
        We define
        $$B_l := \o(2l+1, \F) := \{x \in \End(V) \| \forall v,w \in V: f(x(v),w) = f(v, x(w))\}.$$
        
        For $D_l$ let $f$ be a nondegenerate \underline{symmetric} form corresponding to
        $$ S := \begin{pmatrix}
            0 & E_l \\
            E_l & 0
        \end{pmatrix}.$$
        We define
        $$D_l := \o(2l, \F) := \{x \in \End(V) \| \forall v,w \in V: f(x(v),w) = f(v, x(w))\}.$$
        
        Other Examples of linear Lie algebras we will use are:
        \begin{itemize}
            \item $\mathfrak{t}(d,\F)$ upper triangualar matrices
            \item $\mathfrak{n}(d,\F)$ strictly upper triangualar matrices, i.e. 0 on diagonal
            \item $\mathfrak{d}(d,\F) \subseteq \mathfrak{t}(d,\F)$ diagonal matrices, here we have $\liebracket = 0$.
        \end{itemize}
        Note $ \mathfrak{n}(d, \F) = [L,L]$, where $L:= \mathfrak{t}(d,\F)$.
        Warning for $H,K \subseteq L$ sub spaces we define
        $$ [H,K] := \text{span} \{[x,y] \| x\in H,y \in K\}. $$
    \end{itemize}
\end{example}
