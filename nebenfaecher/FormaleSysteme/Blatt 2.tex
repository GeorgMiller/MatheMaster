% This work is licensed under the Creative Commons
% Attribution-NonCommercial-ShareAlike 4.0 International License. To view a copy
% of this license, visit http://creativecommons.org/licenses/by-nc-sa/4.0/ or
% send a letter to Creative Commons, PO Box 1866, Mountain View, CA 94042, USA.

\documentclass[12pt,a4paper]{article} 

% This work is licensed under the Creative Commons
% Attribution-NonCommercial-ShareAlike 4.0 International License. To view a copy
% of this license, visit http://creativecommons.org/licenses/by-nc-sa/4.0/ or
% send a letter to Creative Commons, PO Box 1866, Mountain View, CA 94042, USA.

% PACKAGES
\usepackage[english, ngerman]{babel}	% Paket für Sprachselektion, in diesem Fall für deutsches Datum etc
\usepackage[utf8]{inputenc}	% Paket für Umlaute; verwende utf8 Kodierung in TexWorks 
\usepackage[T1]{fontenc} % ö,ü,ä werden richtig kodiert
\usepackage{amsmath} % wichtig für align-Umgebung
\usepackage{amssymb} % wichtig für \mathbb{} usw.
\usepackage{amsthm} % damit kann man eigene Theorem-Umgebungen definieren, proof-Umgebungen, etc.
\usepackage{mathrsfs} % für \mathscr
\usepackage[backref]{hyperref} % Inhaltsverzeichnis und \ref-Befehle werden in der PDF-klickbar
\usepackage[english, ngerman, capitalise]{cleveref}
\usepackage{graphicx}
\usepackage{grffile}
\usepackage{setspace} % wichtig für Lesbarkeit. Schöne Zeilenabstände

\usepackage{enumitem} % für custom Liste mit default Buchstaben
\usepackage{ulem} % für bessere Unterstreichung
\usepackage{contour} % für bessere Unterstreichung
\usepackage{epigraph} % für das coole Zitat

\usepackage{tikz}

% This work is licensed under the Creative Commons
% Attribution-NonCommercial-ShareAlike 4.0 International License. To view a copy
% of this license, visit http://creativecommons.org/licenses/by-nc-sa/4.0/ or
% send a letter to Creative Commons, PO Box 1866, Mountain View, CA 94042, USA.

% THEOREM-ENVIRONMENTS

\newtheoremstyle{mystyle}
  {20pt}   % ABOVESPACE \topsep is default, 20pt looks nice
  {20pt}   % BELOWSPACE \topsep is default, 20pt looks nice
  {\normalfont} % BODYFONT
  {0pt}       % INDENT (empty value is the same as 0pt)
  {\bfseries} % HEADFONT
  {}          % HEADPUNCT (if needed)
  {5pt plus 1pt minus 1pt} % HEADSPACE
	{}          % CUSTOM-HEAD-SPEC
\theoremstyle{mystyle}

% Definitionen der Satz, Lemma... - Umgebungen. Der Zähler von "satz" ist dem "section"-Zähler untergeordnet, alle weiteren Umgebungen bedienen sich des satz-Zählers.
\newtheorem{satz}{Satz}[section]
\newtheorem{lemma}[satz]{Lemma}
\newtheorem{korollar}[satz]{Korollar}
\newtheorem{proposition}[satz]{Proposition}
\newtheorem{beispiel}[satz]{Beispiel}
\newtheorem{definition}[satz]{Definition}
\newtheorem{bemerkungnr}[satz]{Bemerkung}
\newtheorem{theorem}[satz]{Theorem}

% Bemerkungen, Erinnerungen und Notationshinweise werden ohne Numerierungen dargestellt.
\newtheorem*{bemerkung}{Bemerkung.}
\newtheorem*{erinnerung}{Erinnerung.}
\newtheorem*{notation}{Notation.}
\newtheorem*{aufgabe}{Aufgabe.}
\newtheorem*{lösung}{Lösung.}
\newtheorem*{beisp}{Beispiel.} %Beispiel ohne Nummerierung
\newtheorem*{defi}{Definition.} %Definition ohne Nummerierung
\newtheorem*{lem}{Lemma.} %Lemma ohne Nummerierung


% SHORTCUTS
\newcommand{\R}{\mathbb{R}}				 % reelle Zahlen
\newcommand{\Rn}{\R^n}						 % der R^n
\newcommand{\N}{\mathbb{N}}				 % natürliche Zahlen
\newcommand{\Z}{\mathbb{Z}}				 % ganze Zahlen
\newcommand{\C}{\mathbb{C}}			   % komplexe Zahlen
\newcommand{\gdw}{\Leftrightarrow} % Genau dann, wenn
\newcommand{\with}{\text{ mit }}   % mit
\newcommand{\falls}{\text{falls }} % falls
\newcommand{\dd}{\text{ d}}        % Differential d

% ETWAS SPEZIELLERE ZEICHEN
%disjoint union
\newcommand{\bigcupdot}{
	\mathop{\vphantom{\bigcup}\mathpalette\setbigcupdot\cdot}\displaylimits
}
\newcommand{\setbigcupdot}[2]{\ooalign{\hfil$#1\bigcup$\hfil\cr\hfil$#2$\hfil\cr\cr}}
%big times
\newcommand*{\bigtimes}{\mathop{\raisebox{-.5ex}{\hbox{\huge{$\times$}}}}} 

% WHITESPACE COMMANDS
%non-restrict newline command
\newcommand{\enter}{$ $\newline} 
%praktischer Tabulator
\newcommand\tab[1][1cm]{\hspace*{#1}}

% TEXT ÜBER ZEICHEN
%das ist ein Gleichheitszeichen mit Text darüber, Beispiel: $a\stackeq{Def} b$
\newcommand{\stackeq}[1]{
	\mathrel{\stackrel{\makebox[0pt]{\mbox{\normalfont\tiny #1}}}{=}}
} 
%das ist ein beliebiges Zeichen mit Text darüber, z. B.  $a\stackrel{Def}{\Rightarrow} b$
\newcommand{\stacksymbol}[2]{
	\mathrel{\stackrel{\makebox[0pt]{\mbox{\normalfont\tiny #1}}}{#2}}
} 

% UNDERLINE
% besseres underline 
\renewcommand{\ULdepth}{1pt}
\contourlength{0.5pt}
\newcommand{\ul}[1]{
	\uline{\phantom{#1}}\llap{\contour{white}{#1}}
}


% hier noch ein paar Commands die nur ich nutze, weil ich sie mir im Laufe der Jahre angewöhnt habe und sie mir jetzt nicht abgewöhnen will:

\newcommand{\gdw}{\Leftrightarrow}   % genau dann, wenn




\author{Willi Sontopski}

\parindent0cm %Ist wichtig, um führende Leerzeichen zu entfernen

\usepackage{pdflscape}
\usepackage{rotating}
\usepackage{scrpage2}
\pagestyle{scrheadings}
\clearscrheadfoot

\ihead{Willi Sontopski}
\chead{Formale Systeme WiSe 18 19}
\ohead{}
\ifoot{Blatt 2}
\cfoot{Version: \today}
\ofoot{Seite \pagemark}

\newcommand{\depth}{\text{depth}}
\newcommand{\length}{\text{length}}

\begin{document}
%\setcounter{section}{1}

\section*{Aufgabe 2.1}
\subsection*{Aufgabe 2.1 (a)}
\begin{align*}
\depth\Big(\big(\neg p\to(\neg p\wedge q)\big)\Big)
&=\max\Big(\depth(\neg p),\depth\big((\neg p\wedge q)\big)\Big)+1\\
&=\max\Big(\depth(p)+1,\max\big(\depth(\neg p),\depth(q)\big)+1\Big)+1\\
&=\max\Big(0+1,\max\big(\depth(p)+1,0\big)+1\Big)+1\\
&=\max\Big(1,\max\big(0+1,0\big)+1\Big)+1\\
&=\max(1,1+1)+1\\
&=3
\end{align*}

\subsection*{Aufgabe 2.1 (b)}
\begin{align*}
&\length:\mathcal{L}(\mathcal{R})\to\N_{>0},\qquad\\
&\varphi\mapsto\left\lbrace\begin{array}{cl}
%0, & \falls \varphi=\Lambda\\
1, & \falls \varphi=p\in\mathcal{R}\\
\length(x) + 1, & \falls \varphi=\neg x\mit x\in\mathcal{L}(\mathcal{R})\\
\length(x_1) + \length(x_2) + 3, & \falls \varphi=(x_1\circ x_2)\mit x_1,x_2\in\mathcal{L}(\mathcal{R})%\text{ und }\circ\in\lbrace\wedge,\vee,\to,\leftrightarrow\rbrace
\end{array}\right.
\end{align*}

\subsection*{Aufgabe 2.1 (c)}
Zu zeigen:
\begin{align*}
\forall\varphi\in\mathcal{L}(\mathcal{R}):\length(\varphi)>\depth(\varphi)
\end{align*}
\textbf{Beweis durch strukturelle Induktion über $\varphi\in\mathcal{L}(\mathcal{R})$:}\\
Induktionsanfang: Sei $\varphi= p\in\mathcal{R}$. Dann gilt:
\begin{align*}
\length(\varphi)=\length(p)\stackeq{\text{Def}}1>0\stackeq{\text{Def}}\depth(p)=\depth(\varphi)
\end{align*}
Induktionsvoraussetzung: Seien $F_1,F_2\in\mathcal{L}(\mathcal{R})$ beliebig aber fest mit
\begin{align*}
\length(F_1)>\depth(F_1)\qquad\text{und}\qquad\length(F_2)>\depth(F_2).
\end{align*}
Induktionsschritt: Sei $\circ\in\lbrace\wedge,\vee,\to,\leftrightarrow\rbrace$. Dann gilt für $\varphi=\neg F_1:$
\begin{align*}
\length(\varphi)
&=\length(\neg F_1)\\
&\stackeq{\text{Def}}
\length(F_1)+1\\
&\stackrel{\text{IV}}{>}
\depth(F_1)+1\\
&\stackeq{\text{Def}}
\depth(\neg F_1)\\
&=\depth(\varphi)
\end{align*}
und für $\varphi=(F_1\circ F_2)$ gilt
\begin{align*}
\length(\varphi)
&=\length((F_1\circ F_2))\\
&\stackeq{\text{Def}}
\length(F_1)+\length(F_2)+3\\
&\stackrel{\text{IV}}{>}
\depth(F_1)+\depth(F_2)+3\\
&\stackrel{\text{Math}}{>}
\max\big(\depth(F_1),\depth(F_2)\big)+1\\
&\stackeq{\text{Def}}
\depth((F_1\circ F_2))\\
&=\depth(\varphi)
\end{align*}

\section*{Aufgabe 2.2}
\subsection*{Aufgabe 2.2 (a)}
\begin{enumerate}[label=(\arabic*)]
\item $\begin{aligned}
\big\lbrace (\neg p\wedge(q\to r)),\neg p, p, (q\to r),q,r\big\rbrace
\end{aligned}$
\item $\begin{aligned}
\big\lbrace p\big\rbrace
\end{aligned}$
\end{enumerate}

\subsection*{Aufgabe 2.2 (b)}
\begin{enumerate}[label=(\arabic*)]
\item $\begin{aligned}\big\lbrace
(p\wedge q)
\big\rbrace\end{aligned}$, da $F$ enthalten und keine Negationen drin.
\item $\begin{aligned}\big\lbrace
(p\wedge q), p, q, \neg m
\big\rbrace\end{aligned}$, da man Eigenschaft 2 wirklich verletzen muss.
\item $\emptyset$
\end{enumerate}

\section*{Aufgabe 2.3}
Beweis durch strukturelle Induktion siehe Musterlösung.\\
Hier: Beweis durch Widerspruch: Angenommen, es existiert eine \underline{kürzeste} Formel $\varphi\in\mathcal{L}(\mathcal{R})$ und Interpretationen $I_1,I_2$ mit 
\begin{align*}
[p]^{I_1}&=[p]^{I_2}\qquad\forall p\in\mathcal{R}_\varphi\text{ aber}\\
[\varphi]^{I_1}&\neq[\varphi]^{I_2}
\end{align*}
Sei o.B.d.A. $[\varphi]^{I_2}=\top$. Dann gilt $[\varphi]^{I_2}=\bot$.\\
Offensichtlich: es existiert kein $q\in\mathcal{R}_\varphi\mit\varphi=q$.\\
\underline{Falls $\varphi=\neg\psi$:}
\begin{align*}
[\varphi]^{I_1}&=[\neg\psi]^{I_1}=\neg^\ast[\psi]^{I_1}\\
[\varphi]^{I_2}&=[\neg\psi]^{I_2}=\neg^\ast[\psi]^{I_2}\\
[\varphi]^{I_1}\neq[\varphi]^{I_2}&\implies[\psi]^{I_1}\neq[\psi]^{I_2}
\end{align*}
Dies ist aber ein Widerspruch zu der Annahme, dass es die kürzeste Formel war.\\

\underline{Falls $\varphi=(\psi_1\wedge\psi_2)$:}
\begin{align*}
[\varphi]^{I_1}&=\big[(\psi_1\wedge\psi_2)\big]^{I_1}=[\psi_1]^{I_1}\wedge^\ast[\psi_2]^{I_1}\\
[\varphi]^{I_2}&=\big[(\psi_1\wedge\psi_2)\big]^{I_2}=[\psi_1]^{I_2}\wedge^\ast[\psi_2]^{I_2}
\end{align*}
Dann folgt aus $[\varphi]^{I_1}\neq[\varphi]^{I_2}$ und $[\varphi]^{I_1}=T$ folgt folgendes:
\begin{align*}
[\psi_1]^{I_1}=[\psi_2]^{I_1}=\top\text{, aber} [\psi_1]^{I_2}=\bot\text{ oder }[\psi_2]^{I_2}=\bot\\
\implies
[\psi_1]^{I_1}=[\psi_1]^{I_2}\text{ oder }[\psi_2]^{I_1}\neq[\psi_2]^{I_2}
\end{align*}
Widerspruch!\\

Analog für $\varphi=(\psi_1\circ\psi_2)\mit\circ\in\lbrace\vee,\to,\leftrightarrow\rbrace.\qquad\qquad\qquad\qquad\qquad\square$

\section*{Aufgabe 2.4}
Beachte $\top:=$ wahr und $\bot:=$ falsch.
\subsection*{Aufgabe 2.4 (a)}
\begin{tabular}{c|c||c|c|c}
$p$ & $q$ & $(p\vee q)$ & $(p\vee q)\to q$ & $(((p\vee q)\to q)\to q)$\\ \hline
$\top$ & $\top$ & $\top$ & $\top$ & $\top$\\
$\bot$ & $\top$ & $\top$ & $\top$ & $\top$\\
$\top$ & $\bot$ & $\top$ & $\bot$ & $\top$\\
$\bot$ & $\bot$ & $\bot$ & $\top$ & $\bot$
\end{tabular}

\subsection*{Aufgabe 2.4 (b)}
Die Formel aus Teil (a) ist
\begin{itemize}
\item erfüllbar, denn sie kann wahr liefern.
\item \underline{nicht} allgemeingültig, weil sie auch falsch liefern kann.
\item widerlegbar, denn sie kann auch falsch liefern.
\item \underline{nicht} unerfüllbar, denn sie kann auch wahr liefern.
\end{itemize}

\subsection*{Aufgabe 2.4 (c)}
\begin{tabular}{c}
$p$\\ \hline
$\top$\\
$\bot$
\end{tabular}

\section*{Aufgabe 2.5}
\subsection*{Aufgabe 2.5 (a)}
\begin{tabular}{c|c||c|c|c}
$p$ & $q$ & $(p\to q)$ & $(p\to q)\to p$ & $(((p\to q)\to p)\to p)$\\ \hline
$\top$ & $\top$ & $\top$ & $\top$ & $\top$ \\
$\bot$ & $\top$ & $\top$ & $\bot$ & $\top$ \\
$\top$ & $\bot$ & $\bot$ & $\top$ & $\top$\\
$\bot$ & $\bot$ & $\top$ & $\bot$ & $\top$
\end{tabular}\\

Die Formel ist allgemeingültig, erfüllbar, \underline{nicht} unerfüllbar und \underline{nicht} widerlegbar.

\subsection*{Aufgabe 2.5 (b)}
%\begin{sidewaystable} %Alternative zu Landscape. Ist geschmackssache was besser ist. Kann man hier umstellen, wenn man möchte.
\begin{landscape}
\begin{tabular}{c|c|c||c|c|c|c|c|c|c}
$p$ & $q$ & $r$ & $(p\to q)$ & $(q\to r)$ & $((p\to q)\wedge(q\to r))$ & $(r\to q)$ & $(q\to p)$ & $((r\to q)\wedge(q\to p))$ & $F$ \\ \hline
$\top$ & $\top$ & $\top$ & $\top$ & $\top$ & $\top$ & $\top$ & $\top$ & $\top$ & $\top$\\
$\bot$ & $\top$ & $\top$ & $\top$ & $\top$ & $\top$ & $\top$ & $\bot$ & $\bot$ & $\top$\\
$\top$ & $\bot$ & $\top$ & $\bot$ & $\top$ & $\bot$ & $\bot$ & $\top$ & $\bot$ & $\bot$\\
$\bot$ & $\bot$ & $\top$ & $\top$ & $\top$ & $\top$ & $\bot$ & $\top$ & $\bot$ & $\top$\\ \hline
$\top$ & $\top$ & $\bot$ & $\top$ & $\bot$ & $\bot$ & $\top$ & $\top$ & $\top$ & $\top$\\
$\bot$ & $\top$ & $\bot$ & $\top$ & $\bot$ & $\bot$ & $\top$ & $\bot$ & $\bot$ & $\bot$\\
$\top$ & $\bot$ & $\bot$ & $\bot$ & $\top$ & $\bot$ & $\top$ & $\top$ & $\top$ & $\top$\\
$\bot$ & $\bot$ & $\bot$ & $\top$ & $\top$ & $\top$ & $\top$ & $\top$ & $\top$ & $\top$
\end{tabular}
\end{landscape}
%\end{sidewaystable}

Die Formel ist \underline{nicht} allgemeingültig, erfüllbar, \underline{nicht} unerfüllbar und widerlegbar.

\subsection*{Aufgabe 2.5 (c)}
\begin{tabular}{c|c||c|c|c|c}
$p$ & $q$ & $(p\to q)$ & $(p\to q)\to p$ & $(((p\to q)\to p)\to p)$ & $\neg(((p\to q)\to p)\to p)$\\ \hline
$\top$ & $\top$ & $\top$ & $\top$ & $\top$ & $\bot$\\
$\bot$ & $\top$ & $\bot$ & $\top$ & $\top$ & $\bot$\\
$\top$ & $\bot$ & $\top$ & $\top$ & $\top$ & $\bot$\\
$\bot$ & $\bot$ & $\bot$ & $\bot$ & $\top$ & $\bot$
\end{tabular}\\

Die Formel ist \underline{nicht} allgemeingültig, \underline{nicht} erfüllbar, unerfüllbar und widerlegbar.

\subsection*{Aufgabe 2.5 (d)}
$\lbrace\varphi_a,\varphi_b\rbrace\rightsquigarrow\varphi_d=(\varphi_a\wedge\varphi_b)$\\
erfüllbar und widerlegbar, da $\varphi_a$ allgemeingültig und $\varphi_b$ erfüllbar und widerlegbar

\end{document}