% This work is licensed under the Creative Commons
% Attribution-NonCommercial-ShareAlike 4.0 International License. To view a copy
% of this license, visit http://creativecommons.org/licenses/by-nc-sa/4.0/ or
% send a letter to Creative Commons, PO Box 1866, Mountain View, CA 94042, USA.

\documentclass[12pt,a4paper]{article} 

% This work is licensed under the Creative Commons
% Attribution-NonCommercial-ShareAlike 4.0 International License. To view a copy
% of this license, visit http://creativecommons.org/licenses/by-nc-sa/4.0/ or
% send a letter to Creative Commons, PO Box 1866, Mountain View, CA 94042, USA.

% PACKAGES
\usepackage[english, ngerman]{babel}	% Paket für Sprachselektion, in diesem Fall für deutsches Datum etc
\usepackage[utf8]{inputenc}	% Paket für Umlaute; verwende utf8 Kodierung in TexWorks 
\usepackage[T1]{fontenc} % ö,ü,ä werden richtig kodiert
\usepackage{amsmath} % wichtig für align-Umgebung
\usepackage{amssymb} % wichtig für \mathbb{} usw.
\usepackage{amsthm} % damit kann man eigene Theorem-Umgebungen definieren, proof-Umgebungen, etc.
\usepackage{mathrsfs} % für \mathscr
\usepackage[backref]{hyperref} % Inhaltsverzeichnis und \ref-Befehle werden in der PDF-klickbar
\usepackage[english, ngerman, capitalise]{cleveref}
\usepackage{graphicx}
\usepackage{grffile}
\usepackage{setspace} % wichtig für Lesbarkeit. Schöne Zeilenabstände

\usepackage{enumitem} % für custom Liste mit default Buchstaben
\usepackage{ulem} % für bessere Unterstreichung
\usepackage{contour} % für bessere Unterstreichung
\usepackage{epigraph} % für das coole Zitat

\usepackage{tikz}

% This work is licensed under the Creative Commons
% Attribution-NonCommercial-ShareAlike 4.0 International License. To view a copy
% of this license, visit http://creativecommons.org/licenses/by-nc-sa/4.0/ or
% send a letter to Creative Commons, PO Box 1866, Mountain View, CA 94042, USA.

% THEOREM-ENVIRONMENTS

\newtheoremstyle{mystyle}
  {20pt}   % ABOVESPACE \topsep is default, 20pt looks nice
  {20pt}   % BELOWSPACE \topsep is default, 20pt looks nice
  {\normalfont} % BODYFONT
  {0pt}       % INDENT (empty value is the same as 0pt)
  {\bfseries} % HEADFONT
  {}          % HEADPUNCT (if needed)
  {5pt plus 1pt minus 1pt} % HEADSPACE
	{}          % CUSTOM-HEAD-SPEC
\theoremstyle{mystyle}

% Definitionen der Satz, Lemma... - Umgebungen. Der Zähler von "satz" ist dem "section"-Zähler untergeordnet, alle weiteren Umgebungen bedienen sich des satz-Zählers.
\newtheorem{satz}{Satz}[section]
\newtheorem{lemma}[satz]{Lemma}
\newtheorem{korollar}[satz]{Korollar}
\newtheorem{proposition}[satz]{Proposition}
\newtheorem{beispiel}[satz]{Beispiel}
\newtheorem{definition}[satz]{Definition}
\newtheorem{bemerkungnr}[satz]{Bemerkung}
\newtheorem{theorem}[satz]{Theorem}

% Bemerkungen, Erinnerungen und Notationshinweise werden ohne Numerierungen dargestellt.
\newtheorem*{bemerkung}{Bemerkung.}
\newtheorem*{erinnerung}{Erinnerung.}
\newtheorem*{notation}{Notation.}
\newtheorem*{aufgabe}{Aufgabe.}
\newtheorem*{lösung}{Lösung.}
\newtheorem*{beisp}{Beispiel.} %Beispiel ohne Nummerierung
\newtheorem*{defi}{Definition.} %Definition ohne Nummerierung
\newtheorem*{lem}{Lemma.} %Lemma ohne Nummerierung


% SHORTCUTS
\newcommand{\R}{\mathbb{R}}				 % reelle Zahlen
\newcommand{\Rn}{\R^n}						 % der R^n
\newcommand{\N}{\mathbb{N}}				 % natürliche Zahlen
\newcommand{\Z}{\mathbb{Z}}				 % ganze Zahlen
\newcommand{\C}{\mathbb{C}}			   % komplexe Zahlen
\newcommand{\gdw}{\Leftrightarrow} % Genau dann, wenn
\newcommand{\with}{\text{ mit }}   % mit
\newcommand{\falls}{\text{falls }} % falls
\newcommand{\dd}{\text{ d}}        % Differential d

% ETWAS SPEZIELLERE ZEICHEN
%disjoint union
\newcommand{\bigcupdot}{
	\mathop{\vphantom{\bigcup}\mathpalette\setbigcupdot\cdot}\displaylimits
}
\newcommand{\setbigcupdot}[2]{\ooalign{\hfil$#1\bigcup$\hfil\cr\hfil$#2$\hfil\cr\cr}}
%big times
\newcommand*{\bigtimes}{\mathop{\raisebox{-.5ex}{\hbox{\huge{$\times$}}}}} 

% WHITESPACE COMMANDS
%non-restrict newline command
\newcommand{\enter}{$ $\newline} 
%praktischer Tabulator
\newcommand\tab[1][1cm]{\hspace*{#1}}

% TEXT ÜBER ZEICHEN
%das ist ein Gleichheitszeichen mit Text darüber, Beispiel: $a\stackeq{Def} b$
\newcommand{\stackeq}[1]{
	\mathrel{\stackrel{\makebox[0pt]{\mbox{\normalfont\tiny #1}}}{=}}
} 
%das ist ein beliebiges Zeichen mit Text darüber, z. B.  $a\stackrel{Def}{\Rightarrow} b$
\newcommand{\stacksymbol}[2]{
	\mathrel{\stackrel{\makebox[0pt]{\mbox{\normalfont\tiny #1}}}{#2}}
} 

% UNDERLINE
% besseres underline 
\renewcommand{\ULdepth}{1pt}
\contourlength{0.5pt}
\newcommand{\ul}[1]{
	\uline{\phantom{#1}}\llap{\contour{white}{#1}}
}


% hier noch ein paar Commands die nur ich nutze, weil ich sie mir im Laufe der Jahre angewöhnt habe und sie mir jetzt nicht abgewöhnen will:

\newcommand{\gdw}{\Leftrightarrow}   % genau dann, wenn




\author{Willi Sontopski}

\parindent0cm %Ist wichtig, um führende Leerzeichen zu entfernen

\usepackage{pdflscape}
\usepackage{rotating}
\usepackage{scrpage2}
\pagestyle{scrheadings}
\clearscrheadfoot

\ihead{Willi Sontopski}
\chead{Formale Systeme WiSe 18 19}
\ohead{}
\ifoot{Blatt 7}
\cfoot{Version: \today}
\ofoot{Seite \pagemark}

\begin{document}
%\setcounter{section}{1}

\section*{Aufgabe $\ast$)}

\section*{Aufgabe $\ast\ast$)}
\textbf{Konvention:} $\cdot$ bindet stärker als $\cup$.

\section*{Aufgabe 1}
Sei $\Sigma$ ein Alphabet und $X_1,X_2,Y_1,Y_2\subseteq\Sigma^\ast$ mit $X_1\subseteq Y_1$ und $X_2\subseteq Y_2$. Dann gilt:
\begin{enumerate}
\item $X_1\cup X_2\subseteq Y_1\cup Y_2$, denn:\\
Sei $x\in X_1\cup X_2$. Dann ist $x\in X_1$ und somit $x\in Y_1$ und $x\in X_2\subseteq Y_2$. Folglich $x\in Y_1\cup Y_2$.
\item $X_1\cdot X_2\subseteq Y_1\cdot Y_2$, denn:\\
Sei $x\in X_1\cdot X_2$, d.h. es existiert $x_1\in X$, $x_2\in X_2$ so, dass $x=x_1\cdot x_2$. Nach Voraussetzung gilt $x_1\in Y_1$ und $x_2\in Y_2$. Folglich ist $x=x_1\cdot x_2\in Y_1\cdot Y_2$
\item $X_1^\ast\cup Y_1^\ast$, denn:\\
Sei $x\in X_1^\ast$. Dann gibt es ein $n\in\N_0$ und $x_0,\ldots, x_n\in X_1$ mit $x=x_1\cdot\ldots\cdot x_n$. Nach Voraussetzung gilt $x_1,\ldots,x_n\in Y_1$ und somit $x\in Y_1^\ast$.
\end{enumerate}

\section*{Aufgabe 2}
Operatorpriorität: $\ast$ vor Exponent vor $\cdot$ vor $\cup,\cap$.

\subsection*{Aufgabe 2 a)}
\begin{align*}
&w\in L_1\cdot(L_2\cup L_3)\\
&\Longleftrightarrow\exists u,v\in\Sigma^\ast:u\in L_1\wedge v\in L_2\cup L_3\wedge w=uv\\
&\Longleftrightarrow\exists u,v\in\Sigma^\ast:u\in L_1\wedge \big(v\in L_2\vee v\in L_3\wedge w=uv\\
&\Longleftrightarrow\exists u,v\in\Sigma^\ast:\big(u\in L_1\wedge v\in L_2\wedge v\in L_3\big)\vee\big(u\in L_1\wedge v\in L_3\wedge w=uv\big)\\
&\Longleftrightarrow\Big(\exists u,v\in\Sigma^\ast:u\in L_1\wedge v\in L_2\wedge v\in L_3\Big)\vee
\Big(\exists u,v\in\Sigma^\ast:u\in L_1\wedge v\in L_3\wedge w=uv\Big)\\
&\Longleftrightarrow w\in L_1\cdot L_2\vee w\in L_1\cdot L_3\\
&\Longleftrightarrow w\in L_1\cdot L_2\cup L_1\cdot L_3
\end{align*}

\subsection*{Aufgabe 2 b)}
Die Aussagen gilt.\\
Zeige zuerst mit natürlicher Induktion, dass für alle $n\in\N$ gilt:
\begin{align*}
\lbrace ab,a\rbrace^n\cdot\lbrace a\rbrace=\lbrace a\rbrace\cdot\lbrace ba,a\rbrace^n
\end{align*}
\ul{IA}: $n=0$: $\lbrace ab,a\rbrace^0\cdot\lbrace a\rbrace=\lbrace\varepsilon\rbrace\cdot\lbrace a\rbrace=\lbrace\varepsilon\cdot a\rbrace=\lbrace a\rbrace=\lbrace a\cdot\varepsilon\rbrace=\lbrace a\rbrace\cdot\lbrace\varepsilon\rbrace=\lbrace a\rbrace\cdot\lbrace ba,a\rbrace^0$\nl
\ul{IH}: Für $n\in\N$ gilt $\lbrace ab,a\rbrace^n\cdot \lbrace a\rbrace=\lbrace a\rbrace\cdot\lbrace ba,a\rbrace^n$\nl
\ul{IS:}
\begin{align*}
\lbrace ab,a\rbrace^{n+1}\cdot\lbrace a\rbrace
&=\lbrace ab,a\rbrace^{n}\cdot\lbrace ab,a\rbrace\cdot\lbrace a\rbrace\\
&=\lbrace ab,a\rbrace^{n}\cdot\lbrace aba,aa\rbrace\\
&=\lbrace ab,a\rbrace^{n}\cdot\lbrace a\rbrace\cdot \lbrace ba,a\rbrace\\
\overset{\text{IV}}&=
\lbrace a\rbrace\cdot\lbrace ba,a\rbrace^n\cdot\lbrace ba,a\rbrace\\
&=\lbrace a\rbrace\cdot\lbrace ba,a\rbrace^{n+1}
\end{align*}
Damit kann nun die ursprüngliche Aussage gezeigt werden:
\begin{align*}
w\in\big(\lbrace a\rbrace\cdot\lbrace b\rbrace\cup\lbrace a\rbrace\big)^\ast\cdot\lbrace a\rbrace\\
&\Longleftrightarrow
w\in \lbrace ab,a\rbrace^\ast\cdot\lbrace a\rbrace\\
&\Longleftrightarrow
\exists k\in\N_0:w\in\lbrace ab,a\rbrace^k\cdot\lbrace a\rbrace\\
\overset{\text{Indu}}&\Longleftrightarrow
\exists k\in\N_0:w\in\lbrace a\rbrace\cdot\lbrace ba,a\rbrace^k\\
&\Longleftrightarrow
w\in\lbrace a\rbrace\cdot\lbrace ba,a\rbrace^\ast\\
&\Longleftrightarrow
w\in\lbrace a\rbrace\cdot\big(\lbrace b\rbrace\cdot\lbrace a\rbrace\cup\lbrace a\rbrace\big)^\ast
\end{align*}

\subsection*{Aufgabe 2 c)}
Falsch, betrachte folgendes Gegenbeispiel:\\
$a\in\big(\lbrace a\rbrace\cup\lbrace b\rbrace\big)\ast=\lbrace a,b\rbrace^\ast$, aber $ab\not\in\lbrace^\ast\cup\lbrace b\rbrace^\ast=\lbrace\varepsilon, a,aa,aaa,\ldots,b,bb,bbb,\ldots\rbrace$

\subsection*{Aufgabe 2 d)}
Für alle $L\subseteq\Sigma^\ast$ gilt $L^0:=\lbrace\varepsilon\rbrace$ und $L\cdot\emptyset=\emptyset\cdot L=\emptyset$. Daher gilt:
\begin{align*}
\emptyset^\ast=\bigcup\limits_{n\geq0}\emptyset^n=\emptyset^0\cup\emptyset^1\cup\emptyset^2\cup\ldots=\lbrace\varepsilon\rbrace\cup\emptyset\cup\emptyset\cup\ldots=\lbrace\varepsilon\rbrace
\end{align*}
Offensichtlich gilt: $L\cup L\cup\lbrace\varepsilon\rbrace$. Damit folgt mit der Monotonie (Aufgabe 1):
\begin{align*}
\big(\cup\lbrace\varepsilon\rbrace\big)^\ast\supseteq L^\ast
\end{align*}
Sei nun $w\in \big(L\cup\lbrace\varepsilon\rbrace\big)^\ast$. Dann existiert $k\in\N$ mit $w\in\big(L\cup\lbrace\varepsilon\rbrace\big)^k$. Zeige nun per Induktion über $n\in\N$: $w\in\big(L\cup\lbrace\varepsilon\rbrace\big)^n\implies L^\ast$.\nl
\ul{IA}: $n=0$: $\big(L\cup\lbrace\varepsilon\rbrace\big)^0=\lbrace\varepsilon\rbrace=L^0\subseteq L^\ast$\nl
\ul{IH:} Für $n\in\N$ gelte: $w\in\big(L\cup\lbrace\varepsilon\rbrace\big)^n\implies w\in L^\ast$\nl
\ul{IS:} Sei $w\in\big(L\cup\lbrace\varepsilon\rbrace\big)^{n+1}=\big(L\cup\lbrace\varepsilon\rbrace\big)^n\cdot\big(L\cup\lbrace\varepsilon\rbrace\big)$. Dann existiert $u\in\big(L\cup\lbrace\varepsilon\rbrace\big)^n$ und $v\in\big(L\cup\lbrace\varepsilon\rbrace\big)$ mit $w=uv$.\\
\begin{itemize}
\item $v=\varepsilon$: Dann $w=u$. Aus IH folgt dann $w\in L^\ast$.
\item $v\in L\setminus\lbrace\varepsilon\rbrace$: Aus IH $u\in L^\ast$, d.h. es existiert $m\in\N$ mit $u\in L^m$. Dann gilt $w=u\cdot v\in L^m\cdot L=L^{m+1}$ und damit auch $w\in L^\ast$
\end{itemize}

\subsection*{Aufgabe 2 e)}
Gilt.

\subsection*{Aufgabe 2 f)}
Gilt.

\section*{Aufgabe 3}
Idee: Füge Zustände ein, so, dass an den Pfeilen Worte von höchstens Länge 1 sind.
%TODO Tikz Bild

\section*{Aufgabe 4}
Ziel: $\varepsilon$-Kanten entfernen: Dazu Anpassung von Übergangsrelation und Endzuständen:
\begin{align*}
\Delta'&:=\left\lbrace(p,a,q)\in Q\times\Sigma\times Q:p\stackrel{a}{\longrightarrow}_\mathcal{A} q\right\rbrace\\
F'&:=\left\lbrace\begin{array}{cl}
F\cup\lbrace q_0\rbrace, &\falls q_0\stackrel{\varepsilon}{\longrightarrow}_\mathcal{A} a\mit q\in F\\
F, &\sonst
\end{array}\right.
\end{align*}
%TODO Tikz Bild

\section*{Aufgabe 5}
Transitionssystem ist NEA mit potenziell mehreren Anfangszuständen und potenziell unendlich viele Zustände akzeptiert
\subsection*{Aufgabe 5 a)}
%TODO Tikz Bild
\subsection*{Aufgabe 5 b)}
%TODO Tikz Bild
\subsection*{Aufgabe 5 c)}
%TODO Tikz Bild
\subsection*{Aufgabe 5 d)}
%TODO Tikz Bild

\end{document}