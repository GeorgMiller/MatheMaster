% This work is licensed under the Creative Commons
% Attribution-NonCommercial-ShareAlike 4.0 International License. To view a copy
% of this license, visit http://creativecommons.org/licenses/by-nc-sa/4.0/ or
% send a letter to Creative Commons, PO Box 1866, Mountain View, CA 94042, USA.

\documentclass[12pt,a4paper]{article} 

% This work is licensed under the Creative Commons
% Attribution-NonCommercial-ShareAlike 4.0 International License. To view a copy
% of this license, visit http://creativecommons.org/licenses/by-nc-sa/4.0/ or
% send a letter to Creative Commons, PO Box 1866, Mountain View, CA 94042, USA.

% PACKAGES
\usepackage[english, ngerman]{babel}	% Paket für Sprachselektion, in diesem Fall für deutsches Datum etc
\usepackage[utf8]{inputenc}	% Paket für Umlaute; verwende utf8 Kodierung in TexWorks 
\usepackage[T1]{fontenc} % ö,ü,ä werden richtig kodiert
\usepackage{amsmath} % wichtig für align-Umgebung
\usepackage{amssymb} % wichtig für \mathbb{} usw.
\usepackage{amsthm} % damit kann man eigene Theorem-Umgebungen definieren, proof-Umgebungen, etc.
\usepackage{mathrsfs} % für \mathscr
\usepackage[backref]{hyperref} % Inhaltsverzeichnis und \ref-Befehle werden in der PDF-klickbar
\usepackage[english, ngerman, capitalise]{cleveref}
\usepackage{graphicx}
\usepackage{grffile}
\usepackage{setspace} % wichtig für Lesbarkeit. Schöne Zeilenabstände

\usepackage{enumitem} % für custom Liste mit default Buchstaben
\usepackage{ulem} % für bessere Unterstreichung
\usepackage{contour} % für bessere Unterstreichung
\usepackage{epigraph} % für das coole Zitat

\usepackage{tikz}

% This work is licensed under the Creative Commons
% Attribution-NonCommercial-ShareAlike 4.0 International License. To view a copy
% of this license, visit http://creativecommons.org/licenses/by-nc-sa/4.0/ or
% send a letter to Creative Commons, PO Box 1866, Mountain View, CA 94042, USA.

% THEOREM-ENVIRONMENTS

\newtheoremstyle{mystyle}
  {20pt}   % ABOVESPACE \topsep is default, 20pt looks nice
  {20pt}   % BELOWSPACE \topsep is default, 20pt looks nice
  {\normalfont} % BODYFONT
  {0pt}       % INDENT (empty value is the same as 0pt)
  {\bfseries} % HEADFONT
  {}          % HEADPUNCT (if needed)
  {5pt plus 1pt minus 1pt} % HEADSPACE
	{}          % CUSTOM-HEAD-SPEC
\theoremstyle{mystyle}

% Definitionen der Satz, Lemma... - Umgebungen. Der Zähler von "satz" ist dem "section"-Zähler untergeordnet, alle weiteren Umgebungen bedienen sich des satz-Zählers.
\newtheorem{satz}{Satz}[section]
\newtheorem{lemma}[satz]{Lemma}
\newtheorem{korollar}[satz]{Korollar}
\newtheorem{proposition}[satz]{Proposition}
\newtheorem{beispiel}[satz]{Beispiel}
\newtheorem{definition}[satz]{Definition}
\newtheorem{bemerkungnr}[satz]{Bemerkung}
\newtheorem{theorem}[satz]{Theorem}

% Bemerkungen, Erinnerungen und Notationshinweise werden ohne Numerierungen dargestellt.
\newtheorem*{bemerkung}{Bemerkung.}
\newtheorem*{erinnerung}{Erinnerung.}
\newtheorem*{notation}{Notation.}
\newtheorem*{aufgabe}{Aufgabe.}
\newtheorem*{lösung}{Lösung.}
\newtheorem*{beisp}{Beispiel.} %Beispiel ohne Nummerierung
\newtheorem*{defi}{Definition.} %Definition ohne Nummerierung
\newtheorem*{lem}{Lemma.} %Lemma ohne Nummerierung


% SHORTCUTS
\newcommand{\R}{\mathbb{R}}				 % reelle Zahlen
\newcommand{\Rn}{\R^n}						 % der R^n
\newcommand{\N}{\mathbb{N}}				 % natürliche Zahlen
\newcommand{\Z}{\mathbb{Z}}				 % ganze Zahlen
\newcommand{\C}{\mathbb{C}}			   % komplexe Zahlen
\newcommand{\gdw}{\Leftrightarrow} % Genau dann, wenn
\newcommand{\with}{\text{ mit }}   % mit
\newcommand{\falls}{\text{falls }} % falls
\newcommand{\dd}{\text{ d}}        % Differential d

% ETWAS SPEZIELLERE ZEICHEN
%disjoint union
\newcommand{\bigcupdot}{
	\mathop{\vphantom{\bigcup}\mathpalette\setbigcupdot\cdot}\displaylimits
}
\newcommand{\setbigcupdot}[2]{\ooalign{\hfil$#1\bigcup$\hfil\cr\hfil$#2$\hfil\cr\cr}}
%big times
\newcommand*{\bigtimes}{\mathop{\raisebox{-.5ex}{\hbox{\huge{$\times$}}}}} 

% WHITESPACE COMMANDS
%non-restrict newline command
\newcommand{\enter}{$ $\newline} 
%praktischer Tabulator
\newcommand\tab[1][1cm]{\hspace*{#1}}

% TEXT ÜBER ZEICHEN
%das ist ein Gleichheitszeichen mit Text darüber, Beispiel: $a\stackeq{Def} b$
\newcommand{\stackeq}[1]{
	\mathrel{\stackrel{\makebox[0pt]{\mbox{\normalfont\tiny #1}}}{=}}
} 
%das ist ein beliebiges Zeichen mit Text darüber, z. B.  $a\stackrel{Def}{\Rightarrow} b$
\newcommand{\stacksymbol}[2]{
	\mathrel{\stackrel{\makebox[0pt]{\mbox{\normalfont\tiny #1}}}{#2}}
} 

% UNDERLINE
% besseres underline 
\renewcommand{\ULdepth}{1pt}
\contourlength{0.5pt}
\newcommand{\ul}[1]{
	\uline{\phantom{#1}}\llap{\contour{white}{#1}}
}


% hier noch ein paar Commands die nur ich nutze, weil ich sie mir im Laufe der Jahre angewöhnt habe und sie mir jetzt nicht abgewöhnen will:

\newcommand{\gdw}{\Leftrightarrow}   % genau dann, wenn




\author{Willi Sontopski}

\parindent0cm %Ist wichtig, um führende Leerzeichen zu entfernen

\usepackage{pdflscape}
\usepackage{rotating}
\usepackage{scrpage2}
\pagestyle{scrheadings}
\clearscrheadfoot

\ihead{Willi Sontopski}
\chead{Formale Systeme WiSe 18 19}
\ohead{}
\ifoot{Blatt 3}
\cfoot{Version: \today}
\ofoot{Seite \pagemark}

\newcommand{\F}{\mathcal{F}}


\begin{document}
%\setcounter{section}{1}

\section*{Aufgabe 3.1}
\subsection*{Aufgabe 3.1 (a)}
Die Aussage stimmt, denn:
\begin{align*}
\F_1\cup\F_2\text{ erfüllbar }
&\stackrel{\text{Def}}{\Longleftrightarrow}
\exists\text{ Modell für }\F_1\cup\F_2\\
&\stackrel{\text{Def}}{\Longleftrightarrow}
\exists\text{ Interpretation }I:\forall F\in\F_1\cup\F_2:I\text{ ist Modell für }F\\
&\implies
\exists\text{ Interpretation }I:\forall F\in\F_1:I\text{ ist Modell für }F\text{ und}\\
&\qquad~~\exists\text{ Interpretation }I:\forall F\in\F_2:I\text{ ist Modell für }F\\
&\stackrel{\text{Def}}{\Longleftrightarrow}
\exists\text{ Modelle für }\F_1,\F_2\\
&\stackrel{\text{Def}}{\Longleftrightarrow}
\F_1,\F_2\text{ sind erfüllbar}
\end{align*}

\subsection*{Aufgabe 3.1 (b)}
Die Aussage stimmt nicht, Gegenbeispiel:
$\F_1:=\lbrace p\rbrace$ und $\F_2:=\lbrace\neg p\rbrace$ sind erfüllbar, aber $\F_1\cup\F_2=\lbrace p ,\neg p\rbrace$ ist nicht erfüllbar.\\

%TODO: das folgende ist falsch.
%\begin{align*}
%&\F_1\text{ erfüllbar }
%\stackrel{\text{Def}}{\implies}
%\exists\text{ Modell für }\F_1
%\stackrel{\text{Def}}{\implies}
%\exists\text{ Interpretation }I:\forall F\in\F_1:I\text{ ist Modell für }F\\
%&F_2\text{ hat Modell }
%\stackrel{\text{Def}}{\implies}
%\exists\text{ Interpretation }\tilde{I}:\forall G\in\F_2:I\text{ ist Modell für }G\\
%\end{align*}
%Dann ist 
%\begin{align*}
%\hat{I}(F):=\left\lbrace\begin{array}{cl}
%I(F), &\falls F\in\F_1\\
%\tilde{I}(F), &\falls F\in\F_2
%\end{array}\right.
%\end{align*}
%eine Interpretation mit der Eigenschaft
%\begin{align*}
%\forall F\in \F_1\cup\F_2:\hat{I}\text{ ist Modell für }F.
%\end{align*}
%Folglich ist $\hat{I}$ ein Modell für $\F_1\cup\F_2$ und somit ist  $\F_1\cup\F_2$ per Definition erfüllbar.

\subsection*{Aufgabe 3.1 (c)}
Nein, denn die Teilmenge $\lbrace p,\neg p\rbrace \subseteq\lbrace p\neq p\rbrace$ ist nicht erfüllbar, da sich die Elemente $p$ und $\neg p$ dieser Teilmenge sich gegenseitig ausschließen.

\section*{Aufgabe 3.2}
\subsection*{Aufgabe 3.2 (a)}
Die Aussage stimmt, denn:
\begin{align*}
F\text{ unerfüllbar }
&\stackrel{\text{Def}}{\Longleftrightarrow}
\forall\text{ Interpretationen } I:[F]^I=\bot\\
&\stackrel{\text{}}{\Longleftrightarrow}
\forall\text{ Interpretationen } I:\neg^\ast([F]^I)=\neg^\ast(\top)\\
&\stackrel{\text{}}{\Longleftrightarrow}
\forall\text{ Interpretationen } I:[\neg F]^I=\top\\
&\stackrel{\text{Def}}{\Longleftrightarrow}
\forall\text{ Interpretationen } I:I\text{ ist Modell für } \neg F
\end{align*}

\subsection*{Aufgabe 3.2 (b)}
Die Aussage stimmt nicht, denn:
\begin{align*}
\F\text{ nicht erfüllbar }
&\stackrel{\text{Def}}{\Longleftrightarrow}
\nexists\text{ Modell für }\F\\
&\stackrel{\text{Def}}{\Longleftrightarrow}
\nexists\text{ Interpretation }I:\forall F\in\F:I\text{ ist Modell für }F\\
&\stackrel{\text{}}{\Longleftrightarrow}
\forall\text{ Interpretation }I:\exists F\in\F:I\text{ ist kein Modell für } F\\
&\stackrel{\text{}}{\Longleftrightarrow}
\forall\text{ Interpretation }I:\exists F\in\F:[F]^I=\bot\\
&\stackrel{\text{}}{\Longleftrightarrow}
\forall\text{ Interpretation }I:\exists F\in\F:[\neg F]^I=\top\\
&\stackrel{\text{}}{\Longleftrightarrow}
\forall\text{ Interpretation }I:\exists F\in\F:I\text{ ist Modell für }\neg F\\
&\Longleftarrow
\forall\text{ Interpretation } I,\forall F\in\F:I\text{ ist Modell für }\neg F
\end{align*}
Der letzte Schritt ist aber keine Äquivalenz und somit gilt die Behauptung nicht. Gegenbeispiel:\\
$\F:=\lbrace\neg p,p\rbrace$ ist nicht erfüllbar und $\neg p$ bzw. $p$ ist nicht allgemeingültig.

\section*{Aufgabe 3.3}
Erfüllungsrelation: $I\models F:\Longleftrightarrow F^I=\top$ ``Interpretation $I$ erfüllt Formel $F$''\\
Logische Konsequenzrelation: $\mathcal{G}\models F:\Longleftrightarrow\forall I:(I\models\mathcal{G}\implies I\models F)$

\subsection*{Aufgabe 3.3 (a)}
\begin{enumerate}[label=(\arabic*)]
\item $F$ ist erfüllbar $\Longleftarrow I\models\top$, die Umkehrung gilt nicht.
\item $F^I=\top\stackrel{\text{Def}}{\Longleftrightarrow} I\models F$
\item $\big(\forall\text{ Interpreation }I:F^I=\top\big)\implies I\models F$
\item $\Big(\exists\text{ Modell für }F \Big)
\stackrel{\text{Def}}{\Longleftrightarrow}
\Big(\exists\text{ Interpretation }\tilde{I}:F^{\tilde{I}}=F\big)\not\Longleftrightarrow I\models F$\\
(gilt g.d.w. $I=\tilde{I}$)
\item $\begin{aligned}
\lbrace F\rbrace\models F
&\stackrel{\text{Def}}{\Longleftrightarrow}
\big(\forall\text{ Interpretation }I:I\models\lbrace F\rbrace\Rightarrow I\models F\big)\\
&\stackrel{\text{Def}}{\Longleftrightarrow}
\big(\forall\text{ Interpretation }I:I\models F\Rightarrow I\models F\big)\\
&\Longleftrightarrow\top
\end{aligned}$\\
Die Aussage ist also nur äquivalent, g.d.w. $I\models F$ wahr ist.
\item $F$ ist unter $I$ wahr $\Longleftrightarrow F^I=\top\stackrel{\text{Def}}{\Longleftrightarrow} I\models F$
\end{enumerate}

\subsection*{Aufgabe 3.3 (b)}
\begin{enumerate}[label=(\arabic*)]
\item $\begin{aligned}
\Big(\forall I\mit (\forall F\in\F:F^I=\top):G^I=\top\Big)
&\Longleftrightarrow
\Big(\forall I:(\forall F\in\F:F^I=\top)\Rightarrow G^I=\top\Big)\\
&\stackrel{\text{Def}}{\Longleftrightarrow}
\Big(\forall I:(\forall F\in\F:I\models F)\Rightarrow I\models G\Big)\\
&\stackrel{\text{Def}}{\Longleftrightarrow}
\Big(\forall I:I\models\F\Rightarrow I\models G\Big)\\
&\stackrel{\text{Def}}{\Longleftrightarrow}
\F\models G
\end{aligned}$
\item $\F,G$ haben gleiche Modell\\
$\stackrel{\text{Def}}{\Longleftrightarrow}
\exists\text{ Interpretation } I:\forall F\in\F:I\models G\wedge I\models F$\\

Es gibt hier keinen Grund für die Äquivalenz. Es sollte in beide Richtungen Gegenbeispiele geben.
\item $\begin{aligned} 
\Big(\big(\F\text{ erfüllbar }\implies G\text{ erfüllbar }\big)
&\stackrel{\text{Def}}{\Longleftrightarrow}
\Big(\big(\exists I:I\models\F\big)\implies\exists\tilde{I}:\tilde{I}\models G\Big)\\
&\Longleftarrow
\Big(\forall I:I\models\F\Rightarrow I\models G\Big)\\
&\stackrel{\text{Def}}{\Longleftrightarrow}
\F\models G
\end{aligned}$\\

Also keine Äquivalenz.
\item $\begin{aligned}
\text{Modelle von $\F$ sind Modelle von $G$ }
&\stackrel{\text{Def}}{\Longleftrightarrow}
\big(I\models\F\implies I\models G\big)\\
&\stackrel{\text{Def}}{\Longleftrightarrow}
\F\models G
\end{aligned}$
\item Betrachte die logische Äquivalenz
\begin{align}\label{eqLogicIndirect}
(A\implies B)\Longleftrightarrow(\neg B\implies\neg A)
\end{align}
für logische Aussagen $A,B$. Damit gilt:
\begin{align*}
&~~\qquad\Big(\forall I:\big(G^I=\bot\implies\exists F\in\F:F^I=\bot\big)\Big)\\
&\stackrel{\text{Def}}{\Longleftrightarrow}
\Big(\forall I:\big(I\not\models G\implies\exists F\in\F:I\not\models F\big)\Big)\\
&\stackrel{\text{}}{\Longleftrightarrow}
\Big(\forall I:\big(\neg(I\models G)\implies\neg(\forall F\in\F:I\models F)\big)\Big)\\
&\stackrel{\eqref{eqLogicIndirect}}{\Longleftrightarrow}
\Big(\forall I:\big((\forall F\in\F:I\models F)\implies I\models G\big)\Big)\\
&\stackrel{\text{Def}}{\Longleftrightarrow}
\Big(\forall I:\big(I\models \F\implies I\models G\big)\Big)\\
&\stackrel{\text{Def}}{\Longleftrightarrow}
\F\models G
\end{align*}
Somit ist (5)$\Longleftrightarrow\F\models G$.
\item $\begin{aligned}
\Big(\forall\text{ Modell $I$ von }G:\exists F\in\F:F^I=\top\Big)
&\stackrel{}{\Longleftrightarrow}
\Big(\forall I:\big(I\models G\implies \exists F\in\F:I\models F\big)\Big)\\
%&\stackrel{}{\Longleftrightarrow}
\end{aligned}$
ist nicht äquivalent. 
\end{enumerate}

\section*{Aufgabe 3.4}
Seien $F,F_1,\ldots F_n\in\mathcal{L}(\mathcal{R}),n\in\N$ aussagenlogische Formeln. Dann gilt:
\begin{align*}
\lbrace F_1,\ldots, F_n\rbrace\models F\Longleftrightarrow\models(((\ldots(F_1\wedge F_2)\ldots)\wedge F_n)\to F)
\end{align*}

\begin{proof}
\underline{Zeige ``$\implies$'':}\\
Sei also 
\begin{align}\label{3.4Hinrichtung}
\lbrace F_1,\ldots, F_n\rbrace\models F
\end{align}
und sei $I$ eine beliebige, aber feste, Interpretation für Formeln aus $\mathcal{L}(\mathcal{R})$. Dann gilt:\\

\underline{Fall 1: $I\not\models\lbrace F_1,\ldots,F_n\rbrace$}\\
Diese Aussage ist äquivalent zu
\begin{align*}
I\not\models((\ldots(F_1\wedge F_2)\ldots)\wedge F_n)
\end{align*}
(wird hier angenommen, nicht bewiesen). Dann ist dies auch äquivalent zu 
\begin{align*}
[((\ldots(F_1\wedge F_2)\ldots)\wedge F_n)]^I=\bot
\end{align*}
Aus der Definition von $\to^\ast$,\\
\begin{tabular}{c|c||c}
$p$ & $q$ & $p\to^\ast q$\\ \hline
$\bot$ & $\bot$ & $\top$\\
$\bot$ & $\top$ & $\top$\\
$\top$ & $\bot$ & $\bot$\\
$\top$ & $\top$ & $\top$
\end{tabular}, 
folgt damit
\begin{align*}
(((\ldots(F_1\wedge F_2)\ldots)\wedge F_n)^I\to^\ast F^I)=\top
\end{align*}
und damit ebenso
\begin{align*}
I\models(((\ldots(F_1\wedge F_2)\ldots)\wedge F_n)\to F)
\end{align*}

\underline{Fall 2: $I\models\lbrace F_1,\ldots,F_n\rbrace$}\\
Das ist äquivalent zu $I\models((\ldots(F_1\wedge F_2)\ldots)\wedge F_n)$ und damit auch zu $((\ldots (F_1^I\wedge^\ast F_2^I)\ldots)\wedge^\ast F_n^I)=\top$. Nach \eqref{3.4Hinrichtung} gilt auch $F^I=\top$. Aus der Definition von $\to^\ast$ folgt dann 
\begin{align*}
(((\ldots(F_1\wedge F_2)\ldots)\wedge F_n)^I\to^\ast F^I)=\top
\end{align*}
und damit
\begin{align*}
I\models(((\ldots(F_1\wedge F_2)\ldots)\wedge F_n)\to F).
\end{align*}

\underline{Zeige ``$\Longleftarrow$'':} Entfällt.
\end{proof}

\section*{Aufgabe 3.5}
\begin{proof}
Mit
\begin{align}\label{3.5}
(A\implies B)\Longleftrightarrow \neg A\vee B
\end{align}
gilt
\begin{align*}
\F\models G 
&\stackrel{\text{Def}}{\Longleftrightarrow}
\forall\text{ Interpretation }I:(I\models \F\implies I\models G)\\
&\stackrel{\eqref{3.5}}{\Longleftrightarrow}
\forall\text{ Interpretation }I:
\neg(I\models \F)\vee I\models G\\
&\stackrel{\text{}}{\Longleftrightarrow}
\forall\text{ Interpretation }I:
I\not\models \F\vee I\not\models \lbrace\neg G\rbrace\\
&\stackrel{\text{}}{\Longleftrightarrow}
\forall\text{ Interpretation }I:
I\not\models \F\cup\lbrace\neg G\rbrace\\
&\stackrel{\text{}}{\Longleftrightarrow}
\F\cup\lbrace\neg G\rbrace\text{ ist unerfüllbar}
\end{align*}
\end{proof}

\section*{Aufgabe 3.6}
\textit{``Aus Falschem folgt beliebiges.''}\\
\begin{proof}
Da $F,\neg F\in\F$, ist $\F$ unerfüllbar. Somit gibt es keine Modell für $\F$. Somit sind alle Modelle von $\F$ (da es keine gibt) auch Modelle von $G$. Damit folgt $F\models G$.
\end{proof}

\end{document}