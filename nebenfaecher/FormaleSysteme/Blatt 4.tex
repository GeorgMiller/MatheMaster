% This work is licensed under the Creative Commons
% Attribution-NonCommercial-ShareAlike 4.0 International License. To view a copy
% of this license, visit http://creativecommons.org/licenses/by-nc-sa/4.0/ or
% send a letter to Creative Commons, PO Box 1866, Mountain View, CA 94042, USA.

\documentclass[12pt,a4paper]{article} 

% This work is licensed under the Creative Commons
% Attribution-NonCommercial-ShareAlike 4.0 International License. To view a copy
% of this license, visit http://creativecommons.org/licenses/by-nc-sa/4.0/ or
% send a letter to Creative Commons, PO Box 1866, Mountain View, CA 94042, USA.

% PACKAGES
\usepackage[english, ngerman]{babel}	% Paket für Sprachselektion, in diesem Fall für deutsches Datum etc
\usepackage[utf8]{inputenc}	% Paket für Umlaute; verwende utf8 Kodierung in TexWorks 
\usepackage[T1]{fontenc} % ö,ü,ä werden richtig kodiert
\usepackage{amsmath} % wichtig für align-Umgebung
\usepackage{amssymb} % wichtig für \mathbb{} usw.
\usepackage{amsthm} % damit kann man eigene Theorem-Umgebungen definieren, proof-Umgebungen, etc.
\usepackage{mathrsfs} % für \mathscr
\usepackage[backref]{hyperref} % Inhaltsverzeichnis und \ref-Befehle werden in der PDF-klickbar
\usepackage[english, ngerman, capitalise]{cleveref}
\usepackage{graphicx}
\usepackage{grffile}
\usepackage{setspace} % wichtig für Lesbarkeit. Schöne Zeilenabstände

\usepackage{enumitem} % für custom Liste mit default Buchstaben
\usepackage{ulem} % für bessere Unterstreichung
\usepackage{contour} % für bessere Unterstreichung
\usepackage{epigraph} % für das coole Zitat

\usepackage{tikz}

% This work is licensed under the Creative Commons
% Attribution-NonCommercial-ShareAlike 4.0 International License. To view a copy
% of this license, visit http://creativecommons.org/licenses/by-nc-sa/4.0/ or
% send a letter to Creative Commons, PO Box 1866, Mountain View, CA 94042, USA.

% THEOREM-ENVIRONMENTS

\newtheoremstyle{mystyle}
  {20pt}   % ABOVESPACE \topsep is default, 20pt looks nice
  {20pt}   % BELOWSPACE \topsep is default, 20pt looks nice
  {\normalfont} % BODYFONT
  {0pt}       % INDENT (empty value is the same as 0pt)
  {\bfseries} % HEADFONT
  {}          % HEADPUNCT (if needed)
  {5pt plus 1pt minus 1pt} % HEADSPACE
	{}          % CUSTOM-HEAD-SPEC
\theoremstyle{mystyle}

% Definitionen der Satz, Lemma... - Umgebungen. Der Zähler von "satz" ist dem "section"-Zähler untergeordnet, alle weiteren Umgebungen bedienen sich des satz-Zählers.
\newtheorem{satz}{Satz}[section]
\newtheorem{lemma}[satz]{Lemma}
\newtheorem{korollar}[satz]{Korollar}
\newtheorem{proposition}[satz]{Proposition}
\newtheorem{beispiel}[satz]{Beispiel}
\newtheorem{definition}[satz]{Definition}
\newtheorem{bemerkungnr}[satz]{Bemerkung}
\newtheorem{theorem}[satz]{Theorem}

% Bemerkungen, Erinnerungen und Notationshinweise werden ohne Numerierungen dargestellt.
\newtheorem*{bemerkung}{Bemerkung.}
\newtheorem*{erinnerung}{Erinnerung.}
\newtheorem*{notation}{Notation.}
\newtheorem*{aufgabe}{Aufgabe.}
\newtheorem*{lösung}{Lösung.}
\newtheorem*{beisp}{Beispiel.} %Beispiel ohne Nummerierung
\newtheorem*{defi}{Definition.} %Definition ohne Nummerierung
\newtheorem*{lem}{Lemma.} %Lemma ohne Nummerierung


% SHORTCUTS
\newcommand{\R}{\mathbb{R}}				 % reelle Zahlen
\newcommand{\Rn}{\R^n}						 % der R^n
\newcommand{\N}{\mathbb{N}}				 % natürliche Zahlen
\newcommand{\Z}{\mathbb{Z}}				 % ganze Zahlen
\newcommand{\C}{\mathbb{C}}			   % komplexe Zahlen
\newcommand{\gdw}{\Leftrightarrow} % Genau dann, wenn
\newcommand{\with}{\text{ mit }}   % mit
\newcommand{\falls}{\text{falls }} % falls
\newcommand{\dd}{\text{ d}}        % Differential d

% ETWAS SPEZIELLERE ZEICHEN
%disjoint union
\newcommand{\bigcupdot}{
	\mathop{\vphantom{\bigcup}\mathpalette\setbigcupdot\cdot}\displaylimits
}
\newcommand{\setbigcupdot}[2]{\ooalign{\hfil$#1\bigcup$\hfil\cr\hfil$#2$\hfil\cr\cr}}
%big times
\newcommand*{\bigtimes}{\mathop{\raisebox{-.5ex}{\hbox{\huge{$\times$}}}}} 

% WHITESPACE COMMANDS
%non-restrict newline command
\newcommand{\enter}{$ $\newline} 
%praktischer Tabulator
\newcommand\tab[1][1cm]{\hspace*{#1}}

% TEXT ÜBER ZEICHEN
%das ist ein Gleichheitszeichen mit Text darüber, Beispiel: $a\stackeq{Def} b$
\newcommand{\stackeq}[1]{
	\mathrel{\stackrel{\makebox[0pt]{\mbox{\normalfont\tiny #1}}}{=}}
} 
%das ist ein beliebiges Zeichen mit Text darüber, z. B.  $a\stackrel{Def}{\Rightarrow} b$
\newcommand{\stacksymbol}[2]{
	\mathrel{\stackrel{\makebox[0pt]{\mbox{\normalfont\tiny #1}}}{#2}}
} 

% UNDERLINE
% besseres underline 
\renewcommand{\ULdepth}{1pt}
\contourlength{0.5pt}
\newcommand{\ul}[1]{
	\uline{\phantom{#1}}\llap{\contour{white}{#1}}
}


% hier noch ein paar Commands die nur ich nutze, weil ich sie mir im Laufe der Jahre angewöhnt habe und sie mir jetzt nicht abgewöhnen will:

\newcommand{\gdw}{\Leftrightarrow}   % genau dann, wenn




\author{Willi Sontopski}

\parindent0cm %Ist wichtig, um führende Leerzeichen zu entfernen

\usepackage{pdflscape}
\usepackage{rotating}
\usepackage{scrpage2}
\pagestyle{scrheadings}
\clearscrheadfoot

\ihead{Willi Sontopski}
\chead{Formale Systeme WiSe 18 19}
\ohead{}
\ifoot{Blatt 3}
\cfoot{Version: \today}
\ofoot{Seite \pagemark}

\newcommand{\F}{\mathcal{F}}
\newcommand{\pos}{\text{pos}}

\usepackage{tikz-qtree}
\usetikzlibrary{positioning,automata}


\begin{document}
%\setcounter{section}{1}

\section*{Aufgabe 4.1}
Zu zeigen:
\begin{align*}
\F'\models G:\Longleftrightarrow\big(\forall I:I\models\F'\implies I\models G\big)
\end{align*}
Sei also $I$ beliebige Interpretation mit $I\models\F'$. Dann gilt nach Definition
\begin{align*}
\forall F'\in\F':I\models F'
\end{align*}
Da $\F\stackrel{\text{Vor}}{\subseteq}\F'$ folgt 
\begin{align*}
\forall F\in\F:I\models F
\end{align*}
Aus der Voraussetzung $\F\models G$ folgt nun $I\models G$. Da die Interpretation $I$ beliebig war, folgt die Behauptung $\F'\models G$. $\qquad\qquad\qquad\qquad\qquad\qquad\qquad\square$

\section*{Aufgabe 4.2}
\subsection*{Aufgabe 4.2 (a)}
\begin{align*}
F_1 &=(m \to\neg s)\\
F_2 &=(\neg m\to (s\wedge t))\\
F_3&=((\neg s\vee t)\to d)\\
F_4&=(d\to a)
\end{align*}

\subsection*{Aufgabe 4.2 (b)}
Wissen über Nessie ist $\F:=\lbrace F_1,\ldots, F_4\rbrace$.
``Folgt daraus, dass Nessie ein Märchenwesen ist?'', d. h.
Frage: Gilt $\F\models m$?\\

Man könnte an dieser Stelle \textit{Resolution} verwenden, aber es geht auch naiv. Betrachte die Interpretation $I_1$ mit\\
$m^{I_1}=\top,~s^{I_1}=\bot,~d^{I_1}=\top,~a^{I_1}=\top$ und $t^{I_1}$ beliebig\\ und die Interpretation $I_2$ mit\\
$m^{I_2}=\bot,~s^{I_2}=\bot,~t^{I_2}=\top, d^{I_2}=\top,a^{I_2}=\top$\\
Dann gilt $I_1\models\F$ und $I_2\models\F$ aber $I_1\models m$ und $I_2\not\models m$. Also folgt $F\not\models m$ und $\F\not\models\neg m$.

%Kann beides nicht gefolgert werden, denn:\\
%\underline{Fall 1:} $m\to \neg s\to d\to a$\\
%\underline{Fall 2:} $\neg m\to s\wedge t\to d\to a$\\

\section*{Aufgabe 4.3}
Beachte zunächst
\begin{align*}
G\equiv F &\stackrel{\text{Def}}{\Longleftrightarrow}
\Big(\forall I:I\models F\gdw I\models G\Big)\\
&\stackrel{\text{Def}}{\Longleftrightarrow}\Big(\forall I:G^I=\top\gdw F^I=\top\Big)\\
&\Longleftrightarrow\forall I:F^I=G^I
\end{align*}

Sei also $I$ eine beliebige Interpretation im Folgenden.

\subsection*{Aufgabe 4.3 (a)}
\begin{align*}
[(F\vee F)]^I=F^I\vee^\ast F^I\stackeq{\text{Tab}} F^I
\end{align*}
\begin{tabular}{c||c}
$F$ & $F\vee^\ast F$\\ \hline
$\top$ & $\top$\\% & $\top$\\
$\bot$ & $\bot$ %& $\top$\\
%$\bot$ & $\top$ & $\top$\\
%$\bot$ & $\bot$ & $\bot$
\end{tabular}

\subsection*{Aufgabe 4.3 (b)}
\begin{align*}
[\neg(F\wedge G)]^I=\neg^\ast\left(F^I\wedge^\ast G^I\right)\stackeq{\text{Tab}}
\left(\neg^\ast F^I\vee^\ast\neg^\ast G^I\right)=\big[(\neg F\vee\neg G)\big]
\end{align*}
\begin{tabular}{c|c||c|c}
$F$ & $G$ & $\neg^\ast(F\wedge^\ast G)$ & $(\neg^\ast F\vee^\ast\neg^\ast G)$\\ \hline
$\top$ & $\top$ & $\bot$ & $\bot$\\
$\top$ & $\bot$ & $\top$ & $\top$\\
$\bot$ & $\top$ & $\top$ & $\top$\\
$\bot$ & $\bot$ & $\top$ & $\top$
\end{tabular}

\subsection*{Aufgabe 4.3 (c)}
Analog zu (a) und (b).
%\begin{align*}
%[(F\to G)]^I=\left(F^I\to^\ast G^I\right)\stackeq{\text{Tab}}
%\left( F^I\to^\ast\neg^\ast G^I\right)=\big[(\neg F\vee\neg G)\big]
%\end{align*}

\subsection*{Aufgabe 4.3 (d)}
Analog zu (a) und (b).

\subsection*{Aufgabe 4.3 (e)}
Analog zu (a) und (b).

\subsection*{Aufgabe 4.3 (f)}
Analog zu (a) und (b).

\subsection*{Aufgabe 4.3 (g)}
Analog zu (a) und (b).

\section*{Aufgabe 4.4}
Sei $\mathcal{R}=\lbrace p_1,p_2,\ldots\rbrace$ eine Menge von aussagenlogischen Variablen, $J_1:=\lbrace J_1^1,\ldots,J_1^4\rbrace$ die Menge der einstelligen Junktoren und $J_2:=\lbrace J_2^1,\ldots,J_2^{16}\rbrace$ die Menge der zweistelligen Junktoren. (Es gibt nach Vorlesung genau 4 einstellige und 16 zweistellige Junktoren). Nun zeigt man durch strukturelle Induktion die Behauptung mit 16 + 4 Induktionsschritten:
\begin{enumerate}
\item $J_1^1 A\equiv\neg A$
\item $J_1^2 A\equiv\neg\neg A$
\item $J_1^3 A\equiv A\vee\neg A$
\item $J_1^4 A\equiv\neg(A\vee\neg A)$
\end{enumerate}
und
\begin{enumerate}
\item $A J_2^1 B\equiv A\vee B$
\item $A J_2^2 B=A\wedge B\equiv\neg(\neg A\vee\neg B)$
\item $A J_2^3 B=A\to B\equiv\neg A\vee B$
\item $\ldots$
\end{enumerate}

\textbf{Mitschrift aus der Übung:}\\
Leite für jeden Junktor aus der Wahrheitswertetabelle eine äquivalente Formel und daraus dann eine Ersetzungsregel ab. Dabei nutzen wir die de'Morgansche Regel:
\begin{align*}
(\varphi_1\wedge\varphi_2)\equiv\neg(\neg\varphi_1\vee\neg\varphi_2)
\end{align*}
Daraus erhält man die Ersetzungsregel
\begin{align}\label{4.4Ersetzungsregel}
\frac{(\varphi_1\wedge\varphi_2)}{\neg(\neg\varphi_1\vee\neg\varphi_2)}
\end{align}
Hier am Beispiel der Implikation:\\
\begin{tabular}{c|c||c}
$\varphi_1^I$ & $\varphi_2^I$ & $\varphi_1^I\to^\ast\varphi_2^I$ \\ \hline
$\top$ & $\top$ & $\top$\\
$\top$ & $\bot$ & $\bot$\\
$\bot$ & $\top$ & $\top$\\
$\bot$ & $\bot$ & $\top$
\end{tabular}
\begin{align*}
(\varphi_1\to\varphi_2)
&\equiv (\varphi_1\wedge\varphi_2)\vee(\neg\varphi_1\wedge\varphi_2)\vee(\neg\varphi_1\wedge\neg\varphi_2)\\
&\stackrel{\eqref{4.4Ersetzungsregel}}{\equiv}\neg(\neg\varphi_1\vee\neg\varphi_2)\vee\neg(\neg\neg\varphi_1\vee\varphi_2)\vee\neg(\neg\neg\varphi_1\vee\neg\neg\varphi_2)\\
&\equiv(\neg\varphi_1\vee\varphi_2)
\end{align*}
Somit erhält man die Ersetzungsregel
\begin{align*}
\frac{(\varphi_1\to\varphi_2)}{(\neg\varphi_1\vee\varphi_2)}
\end{align*}
Erhalte somit alle 16 + 4 Ersetzungsregel erschöpfend auf die Formel an (d.h. bis keine Ersetzungsregel mehr anwendbar ist). Damit erhält man nach endlich vielen Schritten eine semantisch äquivalente Formel, welche nur noch $\neg$ und $\vee$ enthält.

\section*{Aufgabe 4.5}
\subsection*{Aufgabe 4.5 (a)}
\begin{tikzpicture}[node distance=5em]
\tikzstyle{every state}=[shape=circle,draw=black]
\node[state] (1) {$\neg:\Lambda$};
\node[state] (2) [below = of 1] {$\wedge:1$};
\node[state] (3) [below left = of 2] {$p:11$};
\node[state] (4) [below right = of 2] {$\vee:12$};
\node[state] (5) [below left = of 4] {$q:121$};
\node[state] (6) [below right = of 4] {$\neg:122$};
\node[state] (7) [below of =6] {$p:1221$};
\path (2) edge [, -] node {} (1)
      (3) edge [, -] node {} (2)
	  (4) edge [, -] node {} (2)
	  (5) edge [, -] node {} (4)
	  (6) edge [, -] node {} (4)
	  (7) edge [, -] node {} (6)
	;	
\end{tikzpicture}
\begin{align*}
\mathcal{P}_F=\lbrace\Lambda,1,11,12,121,122,1221\rbrace
\end{align*}
Formal:
\begin{align*}
\pos(\neg(p\wedge(q\vee\neg p)))
&\stackeq{2.}\lbrace\Lambda\rbrace\cup\lbrace 1\pi:\pi\in\pos((p\wedge(q\vee\neg p)))\rbrace\\
&=\lbrace\Lambda,1\Lambda,11\Lambda,12\Lambda,121\Lambda,122\Lambda,1221\Lambda\rbrace\\
\pos((p\wedge(q\vee\neg p)))
&\stackeq{3}\lbrace\Lambda\rbrace\cup\lbrace 1\pi_1:\pi_1\in\pos(p)\rbrace\cup\lbrace 2\pi_2:\pi_2\in\pos((q\vee\neg p))\rbrace\\
&=\lbrace\Lambda,1\Lambda,2\Lambda,21\Lambda,22\Lambda,221\Lambda\rbrace\\
\pos(p)&\stackeq{1.}\lbrace\Lambda\rbrace\cup\emptyset=\lbrace\Lambda\rbrace\\
\pos((q\vee\neg p))&\stackeq{3.}\lbrace\Lambda\rbrace\cup\lbrace1\pi_1:\pi_1\in\pos(q)\rbrace\cup\lbrace 2\pi_2:\pi_2\in\pos(\neg p)\rbrace\\
&=\lbrace\Lambda,1\Lambda,2\Lambda,21\Lambda\rbrace\\
\pos(q)&\stackeq{1.}\lbrace\Lambda\rbrace\cup\emptyset=\lbrace\Lambda\rbrace\\
\pos(\neg p)&\stackeq{2.}\lbrace\Lambda\rbrace\cup\lbrace 1\pi:\pi\in\pos(p)\rbrace=\Lambda,1\Lambda\rbrace
\end{align*}

\subsection*{Aufgabe 4.5 (b)}
\begin{tikzpicture}[node distance=5em]
\tikzstyle{every state}=[shape=circle,draw=black]
\node[state] (1) {$\neg:\Lambda$};
\node[state] (2) [below = of 1] {$\neg:1$};
\node[state] (3) [below = of 2] {$\wedge:11$};
\node[state] (4) [left = of 3] {$\vee:111$};
\node[state] (5) [below left = of 4] {$p:1111$};
\node[state] (6) [below right = of 4] {$q:1112$};
\node[state] (7) [right = of 3] {$\vee:112$};
\node[state] (8) [below left = of 7] {$q:1121$};
\node[state] (9) [below right = of 7] {$\neg:1122$};
\node[state] (10) [below = of 9] {$p:11221$};
\path (2) edge [, -] node {} (1)
      (3) edge [, -] node {} (2)
	  (4) edge [, -] node {} (3)
	  (5) edge [, -] node {} (4)
	  (6) edge [, -] node {} (4)
	  (7) edge [, -] node {} (3)
	  (8) edge [, -] node {} (7)
	  (9) edge [, -] node {} (7)
	  (10) edge [, -] node {} (9)
	;	
\end{tikzpicture}
\begin{align*}
G\lceil 1122\Lambda\rceil
&=\Big[\neg\neg((p\vee q)\wedge(q\vee\neg p))\Big]\lceil 1122\Lambda\rceil\\
&\stackeq{\text{2.}}
\Big[\neg((p\vee q)\wedge(q\vee\neg p))\Big]\lceil 122\Lambda\rceil\\
&\stackeq{\text{2.}}
\Big[((p\vee q)\wedge(q\vee\neg p))\Big] \lceil 22\Lambda\rceil\\
&\stackeq{\text{3.}}
\Big[(q\vee\neg p)\Big]\lceil 2\Lambda\rceil\\
&\stackeq{\text{3.}}
[\neg p]\lceil\Lambda\rceil\\
&\stackeq{\text{1.}}[\neg p]=\neg p
\end{align*}

\subsection*{Aufgabe 4.5 (c)}
\begin{tikzpicture}[node distance=3em]
\tikzstyle{every state}=[shape=circle,draw=black]
\node[state] (1) {$\neg:\Lambda$};
\node[state] (2) [below = of 1] {$\wedge:1$};
\node[state] (3) [below left = of 2] {$\neg:11$};
\node[state] (4) [below left = of 3] {$\vee:111$};
\node[state] (5) [below left = of 4] {$p:1111$};
\node[state] (6) [below right= of 4] {$q:1112$};
\node[state] (7) [below right = of 2] {$\neg:12$};
\node[state] (8) [below right = of 7] {$\vee:121$};
\node[state] (9) [below left = of 8] {$q:1211$};
\node[state] (10) [below right = of 8] {$\neg:1212$};
\node[state] (11) [below = of 10] {$p:12121$};
\path (2) edge [, -] node {} (1)
      (3) edge [, -] node {} (2)
	  (4) edge [, -] node {} (3)
	  (5) edge [, -] node {} (4)
	  (6) edge [, -] node {} (4)
	  (7) edge [, -] node {} (2)
	  (8) edge [, -] node {} (7)
	  (9) edge [, -] node {} (8)
	  (10) edge [, -] node {} (8)
	  (11) edge [, -] node {} (10)
	;	
\end{tikzpicture}
\begin{align*}
H\Big\lceil 121\Lambda\mapsto(\neg p\to q)\Big\rceil
&=
\Big[\neg(\neg(p\vee q)\wedge\neg(q\vee\neg p))\Big]\Big\lceil 121\Lambda\mapsto(\neg p\to q)\Big\rceil\\
&\stackeq{\text{2.}}
\neg\Big[(\neg(p\vee q)\wedge\neg(q\vee\neg p))\Big]\Big\lceil 21\Lambda\mapsto(\neg p\to q)\Big\rceil\\
&\stackeq{\text{4.}}
\neg\Big(\neg(p\vee q)\wedge\Big[\neg(q\vee\neg p)\Big]\Big\lceil 1\Lambda\mapsto(\neg p\to q)\Big\rceil\Big)\\
&\stackeq{\text{2.}}
\neg\Big(\neg(p\vee q)\wedge\neg\Big[(q\vee\neg p)\Big]\Big\lceil\Lambda\mapsto(\neg p\to q)\Big\rceil\Big)\\
&\stackeq{\text{1.}}
\neg\Big(\neg(p\vee q)\wedge\neg(\neg p\to q)\Big)
\end{align*}
Zur semantischen Äquivalenz: Das Ersetzungstheorem (Satz 3.23) ist nicht anwendbar, da 
\begin{align*}
(q\vee\neg p)\not\equiv(\neg p\to q)
\end{align*}
Tatsächlich gilt
\begin{align*}
H\not\equiv H\lceil 121\mapsto(\neg p\to q)\rceil
\end{align*}
Betrachte die Interpretation $I$ mit $p^I=q^I=\bot$.

\section*{Aufgabe 4.6}
\subsection*{Aufgabe 4.6 (a)}
Die Aussage gilt nicht, denn betrachte folgendes Gegenbeispiel:\\
$F=(p\wedge p)=p\vee G$ mit $\pi=1$ und $H=\neg p$. Dann sind $F,G,H$ erfüllbar aber \\
$F\lceil\pi\to H\rceil=(p\vee\neg p)$ ist unerfüllbar.\\
Also: Obwohl die Formeln $F,G,H$ unabhängig voneinander erfüllbar sind, können in $H$ Variablen verwendet werden, die auch an anderer Stelle in $F$ vorkommen.

\subsection*{Aufgabe 4.6 (b)}
Die Aussage stimmt, denn:\\
Durch scharfes Hinsehen (oder Wahrheitswertetabelle) sieht man (aufgrund der Unerfüllbarkeit von $G$)
\begin{align*}
G\equiv((H\wedge G)\wedge(H\vee G))
\end{align*}
Nach dem Ersetzungstheorem (Satz 3.23) folgt damit die Behauptung, da in $F$ eine Teilformel durch eine semantisch äquivalente Formel ersetzt wird.

\subsection*{Aufgabe 4.6 (c)}
Die Aussage stimmt nicht. Betrachte\\
$F=\neg p,\pi=1\Lambda,G=p,H=(p\vee q)$, offensichtlich ist $G\to H$ eine Tautologie, aber $F\to F\lceil\pi\mapsto H\rceil=\neg p\to\neg(p\vee q)$ nicht allgemeingültig. Denn $p^I=\bot$, $q^I=\top$)

%$G=H=p\wedge\neg p$ und $F=\neg H$ mit $\pi=\Lambda$. Dann ist $H\to G$ allgemeingültig, aber
%$F\to \underbrace{F\lceil\pi\to G\rceil}_{=H}$ ist unerfüllbar, denn $F$ ist allgemeingültig und $H$ unerfüllbar.\\



\end{document}