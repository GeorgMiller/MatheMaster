% This work is licensed under the Creative Commons
% Attribution-NonCommercial-ShareAlike 4.0 International License. To view a copy
% of this license, visit http://creativecommons.org/licenses/by-nc-sa/4.0/ or
% send a letter to Creative Commons, PO Box 1866, Mountain View, CA 94042, USA.

\documentclass[12pt,a4paper]{article} 

% This work is licensed under the Creative Commons
% Attribution-NonCommercial-ShareAlike 4.0 International License. To view a copy
% of this license, visit http://creativecommons.org/licenses/by-nc-sa/4.0/ or
% send a letter to Creative Commons, PO Box 1866, Mountain View, CA 94042, USA.

% PACKAGES
\usepackage[english, ngerman]{babel}	% Paket für Sprachselektion, in diesem Fall für deutsches Datum etc
\usepackage[utf8]{inputenc}	% Paket für Umlaute; verwende utf8 Kodierung in TexWorks 
\usepackage[T1]{fontenc} % ö,ü,ä werden richtig kodiert
\usepackage{amsmath} % wichtig für align-Umgebung
\usepackage{amssymb} % wichtig für \mathbb{} usw.
\usepackage{amsthm} % damit kann man eigene Theorem-Umgebungen definieren, proof-Umgebungen, etc.
\usepackage{mathrsfs} % für \mathscr
\usepackage[backref]{hyperref} % Inhaltsverzeichnis und \ref-Befehle werden in der PDF-klickbar
\usepackage[english, ngerman, capitalise]{cleveref}
\usepackage{graphicx}
\usepackage{grffile}
\usepackage{setspace} % wichtig für Lesbarkeit. Schöne Zeilenabstände

\usepackage{enumitem} % für custom Liste mit default Buchstaben
\usepackage{ulem} % für bessere Unterstreichung
\usepackage{contour} % für bessere Unterstreichung
\usepackage{epigraph} % für das coole Zitat

\usepackage{tikz}

% This work is licensed under the Creative Commons
% Attribution-NonCommercial-ShareAlike 4.0 International License. To view a copy
% of this license, visit http://creativecommons.org/licenses/by-nc-sa/4.0/ or
% send a letter to Creative Commons, PO Box 1866, Mountain View, CA 94042, USA.

% THEOREM-ENVIRONMENTS

\newtheoremstyle{mystyle}
  {20pt}   % ABOVESPACE \topsep is default, 20pt looks nice
  {20pt}   % BELOWSPACE \topsep is default, 20pt looks nice
  {\normalfont} % BODYFONT
  {0pt}       % INDENT (empty value is the same as 0pt)
  {\bfseries} % HEADFONT
  {}          % HEADPUNCT (if needed)
  {5pt plus 1pt minus 1pt} % HEADSPACE
	{}          % CUSTOM-HEAD-SPEC
\theoremstyle{mystyle}

% Definitionen der Satz, Lemma... - Umgebungen. Der Zähler von "satz" ist dem "section"-Zähler untergeordnet, alle weiteren Umgebungen bedienen sich des satz-Zählers.
\newtheorem{satz}{Satz}[section]
\newtheorem{lemma}[satz]{Lemma}
\newtheorem{korollar}[satz]{Korollar}
\newtheorem{proposition}[satz]{Proposition}
\newtheorem{beispiel}[satz]{Beispiel}
\newtheorem{definition}[satz]{Definition}
\newtheorem{bemerkungnr}[satz]{Bemerkung}
\newtheorem{theorem}[satz]{Theorem}

% Bemerkungen, Erinnerungen und Notationshinweise werden ohne Numerierungen dargestellt.
\newtheorem*{bemerkung}{Bemerkung.}
\newtheorem*{erinnerung}{Erinnerung.}
\newtheorem*{notation}{Notation.}
\newtheorem*{aufgabe}{Aufgabe.}
\newtheorem*{lösung}{Lösung.}
\newtheorem*{beisp}{Beispiel.} %Beispiel ohne Nummerierung
\newtheorem*{defi}{Definition.} %Definition ohne Nummerierung
\newtheorem*{lem}{Lemma.} %Lemma ohne Nummerierung


% SHORTCUTS
\newcommand{\R}{\mathbb{R}}				 % reelle Zahlen
\newcommand{\Rn}{\R^n}						 % der R^n
\newcommand{\N}{\mathbb{N}}				 % natürliche Zahlen
\newcommand{\Z}{\mathbb{Z}}				 % ganze Zahlen
\newcommand{\C}{\mathbb{C}}			   % komplexe Zahlen
\newcommand{\gdw}{\Leftrightarrow} % Genau dann, wenn
\newcommand{\with}{\text{ mit }}   % mit
\newcommand{\falls}{\text{falls }} % falls
\newcommand{\dd}{\text{ d}}        % Differential d

% ETWAS SPEZIELLERE ZEICHEN
%disjoint union
\newcommand{\bigcupdot}{
	\mathop{\vphantom{\bigcup}\mathpalette\setbigcupdot\cdot}\displaylimits
}
\newcommand{\setbigcupdot}[2]{\ooalign{\hfil$#1\bigcup$\hfil\cr\hfil$#2$\hfil\cr\cr}}
%big times
\newcommand*{\bigtimes}{\mathop{\raisebox{-.5ex}{\hbox{\huge{$\times$}}}}} 

% WHITESPACE COMMANDS
%non-restrict newline command
\newcommand{\enter}{$ $\newline} 
%praktischer Tabulator
\newcommand\tab[1][1cm]{\hspace*{#1}}

% TEXT ÜBER ZEICHEN
%das ist ein Gleichheitszeichen mit Text darüber, Beispiel: $a\stackeq{Def} b$
\newcommand{\stackeq}[1]{
	\mathrel{\stackrel{\makebox[0pt]{\mbox{\normalfont\tiny #1}}}{=}}
} 
%das ist ein beliebiges Zeichen mit Text darüber, z. B.  $a\stackrel{Def}{\Rightarrow} b$
\newcommand{\stacksymbol}[2]{
	\mathrel{\stackrel{\makebox[0pt]{\mbox{\normalfont\tiny #1}}}{#2}}
} 

% UNDERLINE
% besseres underline 
\renewcommand{\ULdepth}{1pt}
\contourlength{0.5pt}
\newcommand{\ul}[1]{
	\uline{\phantom{#1}}\llap{\contour{white}{#1}}
}


% hier noch ein paar Commands die nur ich nutze, weil ich sie mir im Laufe der Jahre angewöhnt habe und sie mir jetzt nicht abgewöhnen will:

\newcommand{\gdw}{\Leftrightarrow}   % genau dann, wenn




\author{Willi Sontopski}

\parindent0cm %Ist wichtig, um führende Leerzeichen zu entfernen

\usepackage{pdflscape}
\usepackage{rotating}
\usepackage{scrpage2}
\pagestyle{scrheadings}
\clearscrheadfoot

\ihead{Willi Sontopski}
\chead{Formale Systeme WiSe 18 19}
\ohead{}
\ifoot{Blatt 3}
\cfoot{Version: \today}
\ofoot{Seite \pagemark}

\newcommand{\F}{\mathcal{F}}
\newcommand{\pos}{\text{pos}}

\usepackage{tikz-qtree}
\usetikzlibrary{positioning,automata}


\begin{document}
%\setcounter{section}{1}

\section*{Aufgabe 5.1}
\textbf{Negationsnormalform:} ``Negation immer nur vor Variablen''\\
\begin{align}
&\frac{\neg\neg H}{H}\label{Neg1}\\
&\frac{\neg(G_1\wedge G_2)}{(\neg G_1\vee G_2)}\label{Neg2}\\
&\frac{\neg(G_1\vee G_2)}{(\neg G_1\wedge G_2)}\label{Neg3}
\end{align}
Die Symbole $\to, \leftrightarrow$ müssen zuerst umgeschrieben werden.\\

\textbf{Klauselform / konjunktive Normalform /CNF:}\\
\begin{align}
&\frac{\neg\neg D}{D}\label{KNF1}\\
&\frac{(D_1\wedge D_2)}{D_1\mid D_2}\label{KNF2}\\
&\frac{\neg(D_1\wedge D_2)}{\neg D_1,\neg D_2}\label{KNF3}\\
&\frac{(D_1\vee D_2)}{D_1, D_2}\label{KNF4}\\
&\frac{\neg(D_1\vee D_2)}{\neg D_1\mid\neg D_2}\label{KNF5}\\
\end{align}

\subsection*{Aufgabe 5.1 (a)}
Negationsnormalform:
\begin{align*}
(\underline{\neg (( p\wedge q)\vee p)}\vee p)
&\stackrel{\ref{Neg3}}{\equiv}
((\underline{\neg(p\wedge q)}\wedge\neg p)\vee q)\\
&\stackrel{\ref{Neg2}}{\equiv}
(((\neg p\vee \neg q)\wedge\neg p)\vee p)
\end{align*}
Disjunktive Normalform:
\begin{align*}
\langle[(\neg (( p\wedge q)\vee p)\underline{\vee} p)]\rangle
&\stackrel{\ref{KNF4}}{\equiv}
\langle[(\underline{\neg} (( p\wedge q)\vee p), p)]\rangle\\
&\stackrel{\ref{KNF5}}{\equiv}
\langle[(\neg (( p\wedge q)\vee p)],[\neg p,p)]\rangle\\
&\stackrel{\ref{KNF3}}{\equiv}
\langle[(\neg p,\neg, p)],[\neg p,p)]\rangle\\
\end{align*}
Dies entspricht der Formel
\begin{align*}
(((\neg p\vee\neg q)\vee p)\wedge(\neg p\vee p)))
\end{align*}

\subsection*{Aufgabe 5.1 (b)}
Negationsnormalform:
\begin{align*}
&~~~~\underline{\neg}(((p\vee q)\wedge(q\wedge r))\vee\neg((r\wedge q)\wedge(q\vee p)))\\
&\stackrel{\ref{Neg3}}{\equiv}
(\underline{\neg}((p\vee q)\wedge(q\wedge r))\wedge\neg\neg((r\wedge q)\wedge(q\vee p)))\\
&\stackrel{\ref{Neg2}}{\equiv}
((\underline{\neg}(p\vee q)\vee\underline{\neg}(q\wedge r))\wedge\neg\neg((r\wedge q)\wedge(q\vee p)))\\
&\stackrel{\ref{Neg2},\ref{Neg3}}{\equiv}
(((\neg p\wedge\neg q)\vee(\neg q\vee\neg r))\wedge\underline{\neg\neg}((r\wedge q)\wedge(q\vee p)))\\
&\stackrel{\ref{Neg1}}{\equiv}
(((\neg p\wedge\neg q)\vee(\neg q\vee\neg r))\wedge((r\wedge q)\wedge(q\vee p)))
\end{align*}

Disjunktive Normalform:
\begin{align*}
&~~~~\langle[\underline{\neg}(((p\vee q)\wedge(q\wedge r))\underline{\vee}\neg((r\wedge q)\wedge(q\vee p)))]\rangle\\
&\stackrel{\ref{KNF5}}{\equiv}
\langle[\neg((p\vee q)\wedge(q\wedge r))],[\underline{\neg\neg}((r\wedge q)\wedge(q\vee p))]\rangle\\
&\stackrel{\ref{KNF1}}{\equiv}
\langle[\underline{\neg}((p\vee q)\underline{\wedge}(q\wedge r))],[((r\wedge q)\wedge(q\vee p))]\rangle\\
&\stackrel{\ref{KNF3}}{\equiv}
\langle[\neg(p\vee q),\underline{\neg}(q\wedge r)],[((r\wedge q)\wedge(q\vee p))]\rangle\\
&\stackrel{\ref{KNF3}}{\equiv}
\langle[\neg(p\vee q),\neg q, \neg r)],[((r\wedge q)\underline{\wedge}(q\vee p))]\rangle\\
&\stackrel{\ref{KNF2}}{\equiv}
\langle[\neg(p\vee q),\neg q, \neg r)],[((r\wedge q)],[(q\vee p)]\rangle\\
&\stackrel{\ref{KNF5},\ref{KNF2},\ref{KNF4}}{\equiv}
\langle[\neg p,\neg q,\neg r],[\neg q,,\neg q, \neg r],[r],[q],[q,p]\rangle\\
\end{align*}

\section*{Sudoku $n^2\times n^2$}
Problem: Existiert eine Zuweisung, welche jeder Zelle einen Wert von 1 bis $n^2$ zuweist so, dass in jeder Zeile, in jeder  Spalte und in jedem Block jede Zahl aus $\lbrace1,2,\ldots,n^2\rbrace$ genau einmal vorkommt?

\begin{aufgabe}
Finde eiene Formel (in CNF) für ein gegebenes Sudoku, sodass diese Formel genau dann erfüllbar ist, wenn das Soduku lösbar ist. Aus dem (erfüllenden) Modell soll dann eine konkrete Lösung abgeleitet werden.\\
Variablen: $R=\big\lbrace s_{x,y,z}\in\mathbb{B}^3\mid x,y,z\in\lbrace 1,\ldots,n\rbrace\big\rbrace$. Dabei soll $s_{x,y,z}$ auf wahr abgebildet wird g.d.w. in Zeile $x$, Spalte $y$ die Zahl $z$ steht.\\
Formeln (für $n=3$):
\begin{itemize}
\item ``Jede Zelle bekommt (mindestens) einen Wert zugewiesen.''
\begin{align*}
\bigwedge\limits_{x=1}^{3^2}\bigwedge\limits_{y=1}^{3^2}\bigvee\limits_{z=1}^{3^2} s_{x,y,z}
\end{align*}
\item ``Jede Zelle bekommt höchstens einen Wert zugewiesen.''\\
Meine Lösung (vielleicht richtig:)
\begin{align*}
\bigwedge\limits_{x=1}^{3^2}\bigwedge\limits_{y=1}^{3^2}\bigvee\limits_{z=1}^{3^2}\bigwedge\limits_{\hat{z}\in\lbrace1,2,\ldots,3^2\rbrace\setminus\lbrace z\rbrace} \neg s_{x,y,\hat{z}}
\end{align*}
Tutor-Lösung:
\begin{align*}
\bigwedge\limits_{x=1}^{3^2}
\bigwedge\limits_{y=1}^{3^2}
\bigwedge\limits_{z=1}^{3^2-1}
\bigwedge\limits_{i=z+1}^{3^2}
(\neg s_{x,y,z}\vee\neg s_{x,y,i})
\end{align*}

\item ``Jede Spalte enthält jede Zahl (mindestens) einmal.''
\begin{align*}
\bigwedge\limits_{y=1}^{3^2}\bigwedge\limits_{z=1}^{3^2}\bigvee\limits_{x=1}^{3^2} s_{x,y,z}
\end{align*}
\item ``Jede Zeile enthält jede Zahl (mindestens) einmal.''
\begin{align*}
\bigwedge\limits_{x=1}^{3^2}\bigwedge\limits_{z=1}^{3^2}\bigvee\limits_{y=1}^{3^2} s_{x,y,z}
\end{align*}
\item ``Jeder Block enthält jede Zahl (mindestens) einmal.''\\
Die Menge aller Blöcke ist
\begin{align*}
B:=\left\lbrace 3\cdot(i,j)+\lbrace1,2,\ldots,n\rbrace^2\right\rbrace_{i,j=0}^{n-1}.
\end{align*}
Somit (meine Lösung, k. A: ob es stimmt):
\begin{align*}
\bigwedge\limits_{x\in B}\bigwedge\limits_{z=1}^{3^2}\bigvee\limits_{y=1}^{3^2} s_{x,y,z}
\end{align*}
Richtige Lösung anderer Gruppe:
\begin{align*}
\bigwedge\limits_{n_x=1}^n\bigwedge\limits_{n_y=1}^n\bigwedge\limits_{z=1}^{n^2}\bigvee\limits_{x=n_x\cdot n}^{(n_x\cdot(n+1))}\bigvee\limits_{y=n_y\cdot n}^{(n_y\cdot(n+1))-1} s_{x,y,z}
\end{align*}
Tutor-Lösung:
\begin{align*}
\bigwedge\limits_{z=1}^{3^2}\bigwedge\limits_{i=0}^{2}\bigwedge\limits_{j=1}^2\bigvee\limits_{x=1}^{3}\bigvee\limits_{y=1}^3 s_{3\cdot i+x,3\cdot j+y,z}
\end{align*}
\end{itemize}
\end{aufgabe}


\end{document}