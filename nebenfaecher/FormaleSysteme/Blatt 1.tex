% This work is licensed under the Creative Commons
% Attribution-NonCommercial-ShareAlike 4.0 International License. To view a copy
% of this license, visit http://creativecommons.org/licenses/by-nc-sa/4.0/ or
% send a letter to Creative Commons, PO Box 1866, Mountain View, CA 94042, USA.

\documentclass[12pt,a4paper]{article} 

% This work is licensed under the Creative Commons
% Attribution-NonCommercial-ShareAlike 4.0 International License. To view a copy
% of this license, visit http://creativecommons.org/licenses/by-nc-sa/4.0/ or
% send a letter to Creative Commons, PO Box 1866, Mountain View, CA 94042, USA.

% PACKAGES
\usepackage[english, ngerman]{babel}	% Paket für Sprachselektion, in diesem Fall für deutsches Datum etc
\usepackage[utf8]{inputenc}	% Paket für Umlaute; verwende utf8 Kodierung in TexWorks 
\usepackage[T1]{fontenc} % ö,ü,ä werden richtig kodiert
\usepackage{amsmath} % wichtig für align-Umgebung
\usepackage{amssymb} % wichtig für \mathbb{} usw.
\usepackage{amsthm} % damit kann man eigene Theorem-Umgebungen definieren, proof-Umgebungen, etc.
\usepackage{mathrsfs} % für \mathscr
\usepackage[backref]{hyperref} % Inhaltsverzeichnis und \ref-Befehle werden in der PDF-klickbar
\usepackage[english, ngerman, capitalise]{cleveref}
\usepackage{graphicx}
\usepackage{grffile}
\usepackage{setspace} % wichtig für Lesbarkeit. Schöne Zeilenabstände

\usepackage{enumitem} % für custom Liste mit default Buchstaben
\usepackage{ulem} % für bessere Unterstreichung
\usepackage{contour} % für bessere Unterstreichung
\usepackage{epigraph} % für das coole Zitat

\usepackage{tikz}

% This work is licensed under the Creative Commons
% Attribution-NonCommercial-ShareAlike 4.0 International License. To view a copy
% of this license, visit http://creativecommons.org/licenses/by-nc-sa/4.0/ or
% send a letter to Creative Commons, PO Box 1866, Mountain View, CA 94042, USA.

% THEOREM-ENVIRONMENTS

\newtheoremstyle{mystyle}
  {20pt}   % ABOVESPACE \topsep is default, 20pt looks nice
  {20pt}   % BELOWSPACE \topsep is default, 20pt looks nice
  {\normalfont} % BODYFONT
  {0pt}       % INDENT (empty value is the same as 0pt)
  {\bfseries} % HEADFONT
  {}          % HEADPUNCT (if needed)
  {5pt plus 1pt minus 1pt} % HEADSPACE
	{}          % CUSTOM-HEAD-SPEC
\theoremstyle{mystyle}

% Definitionen der Satz, Lemma... - Umgebungen. Der Zähler von "satz" ist dem "section"-Zähler untergeordnet, alle weiteren Umgebungen bedienen sich des satz-Zählers.
\newtheorem{satz}{Satz}[section]
\newtheorem{lemma}[satz]{Lemma}
\newtheorem{korollar}[satz]{Korollar}
\newtheorem{proposition}[satz]{Proposition}
\newtheorem{beispiel}[satz]{Beispiel}
\newtheorem{definition}[satz]{Definition}
\newtheorem{bemerkungnr}[satz]{Bemerkung}
\newtheorem{theorem}[satz]{Theorem}

% Bemerkungen, Erinnerungen und Notationshinweise werden ohne Numerierungen dargestellt.
\newtheorem*{bemerkung}{Bemerkung.}
\newtheorem*{erinnerung}{Erinnerung.}
\newtheorem*{notation}{Notation.}
\newtheorem*{aufgabe}{Aufgabe.}
\newtheorem*{lösung}{Lösung.}
\newtheorem*{beisp}{Beispiel.} %Beispiel ohne Nummerierung
\newtheorem*{defi}{Definition.} %Definition ohne Nummerierung
\newtheorem*{lem}{Lemma.} %Lemma ohne Nummerierung


% SHORTCUTS
\newcommand{\R}{\mathbb{R}}				 % reelle Zahlen
\newcommand{\Rn}{\R^n}						 % der R^n
\newcommand{\N}{\mathbb{N}}				 % natürliche Zahlen
\newcommand{\Z}{\mathbb{Z}}				 % ganze Zahlen
\newcommand{\C}{\mathbb{C}}			   % komplexe Zahlen
\newcommand{\gdw}{\Leftrightarrow} % Genau dann, wenn
\newcommand{\with}{\text{ mit }}   % mit
\newcommand{\falls}{\text{falls }} % falls
\newcommand{\dd}{\text{ d}}        % Differential d

% ETWAS SPEZIELLERE ZEICHEN
%disjoint union
\newcommand{\bigcupdot}{
	\mathop{\vphantom{\bigcup}\mathpalette\setbigcupdot\cdot}\displaylimits
}
\newcommand{\setbigcupdot}[2]{\ooalign{\hfil$#1\bigcup$\hfil\cr\hfil$#2$\hfil\cr\cr}}
%big times
\newcommand*{\bigtimes}{\mathop{\raisebox{-.5ex}{\hbox{\huge{$\times$}}}}} 

% WHITESPACE COMMANDS
%non-restrict newline command
\newcommand{\enter}{$ $\newline} 
%praktischer Tabulator
\newcommand\tab[1][1cm]{\hspace*{#1}}

% TEXT ÜBER ZEICHEN
%das ist ein Gleichheitszeichen mit Text darüber, Beispiel: $a\stackeq{Def} b$
\newcommand{\stackeq}[1]{
	\mathrel{\stackrel{\makebox[0pt]{\mbox{\normalfont\tiny #1}}}{=}}
} 
%das ist ein beliebiges Zeichen mit Text darüber, z. B.  $a\stackrel{Def}{\Rightarrow} b$
\newcommand{\stacksymbol}[2]{
	\mathrel{\stackrel{\makebox[0pt]{\mbox{\normalfont\tiny #1}}}{#2}}
} 

% UNDERLINE
% besseres underline 
\renewcommand{\ULdepth}{1pt}
\contourlength{0.5pt}
\newcommand{\ul}[1]{
	\uline{\phantom{#1}}\llap{\contour{white}{#1}}
}


% hier noch ein paar Commands die nur ich nutze, weil ich sie mir im Laufe der Jahre angewöhnt habe und sie mir jetzt nicht abgewöhnen will:

\newcommand{\gdw}{\Leftrightarrow}   % genau dann, wenn




\author{Willi Sontopski}

\parindent0cm %Ist wichtig, um führende Leerzeichen zu entfernen

\usepackage{scrpage2}
\pagestyle{scrheadings}
\clearscrheadfoot

\ihead{Willi Sontopski}
\chead{Formale Systeme WiSe 18 19}
\ohead{}
\ifoot{Blatt 1}
\cfoot{Version: \today}
\ofoot{Seite \pagemark}

%\usepackage{enumerate}
\newcommand{\ok}{\text{öK}}
\newcommand{\sk}{\text{sK}}
\renewcommand{\k}{\text{K}}
\newcommand{\bj}{\text{bJ}}

\begin{document}
%\setcounter{section}{1}

\section*{Aufgabe 1.1}
Grammatik für arithmetische Ausdrücke in BNF:
\begin{align*}
\alpha::=|(\alpha_1+\alpha_2)|(\alpha_1-\alpha_2)|(\alpha_1\div\alpha_2)|(\alpha_1\ast\alpha_2)\mit q\in\Q
\end{align*}
\subsection*{Aufgabe 1.1 (a)}
\begin{enumerate}[label=(\arabic*)]
\item  Nein, da, Klammern fehlen. Richtig wäre: $(3-2)+(1-(3\times 4))$ oder $((3-2)+1)-(3\times 4)$
\item Nein, da rationale Zahlen nicht geklammert werden dürfen. Und ein unäres Minus ist auch nicht definiert.
\item Nein, da das Gleichheitszeichen nicht Teil des Alphabets ist.
\item Nein, da es kein unäres Minuszeichen gibt.
\item Nein. Nur falls $p,q\in\Q$.
\end{enumerate}

\subsection*{Aufgabe 1.1 (b)}
%Das ist Abhängig von der Wahl des Alphabets $(\mathcal{R},\mathcal{J},\mathcal{S})$.
Grammatik für aussagenlogische Formeln in BNF:
\begin{align*}
\varphi::=p|\neg \varphi_1|(\varphi_1\wedge\varphi_2)|(\varphi_1\vee\varphi_2)|(\varphi_1\to\varphi_2)|(\varphi_1\leftrightarrow\varphi_2)\mit p\in\mathcal{R}
\end{align*}
\begin{enumerate}[label=(\arabic*)]
\item Nein, da die Zeichen ``1'' und ``2'' auftauchen.
\item Nein, da die Zeichen ``2'' und ``3'' auftauchen.
\item Ja.
\item Nein, weil Klammern fehlen. Richtig wäre $((p\wedge p)\wedge(p\wedge p))$
\item Nein, weil Klammern zu viel sind und fehlen und $\neg$ erwartet Ausdruck.
\item Ja.
\end{enumerate}

\section*{Aufgabe 1.2}
Induktionsanfang: $n=0:~0\leq0=0^2$\\
Induktionsvoraussetzung: Gelte $n\leq n^2$ für beliebiges aber festes $n\in\N$.\\
Induktionsschritt: 
\begin{align*}
n+1\leq
n+\underbrace{2\cdot n}_{\geq0}+1
\stackrel{\text{IV}}{\leq} n^2+2\cdot n+1=(n+1)^2\qquad\square
\end{align*}

\section*{Aufgabe 1.3}
\subsection*{Aufgabe 1.3 (a)}
Beweis durch Induktion, Induktionsanfang: $n=1:$ Wähle $p\in\mathcal{R}$. Dann gilt $p\in\mathcal{L}(\mathcal{R})$.\\
Induktionsvoraussetzung: Für ein beliebiges aber festes $n\in\N$ gibt es eine aussagenlogische Formel der Länge $n$, in Zeichen $\exists\varphi\in\mathcal{L}(\mathcal{R}):|\varphi|=n$.\\
Induktionsschritt: Nach IV ist $x$ aussagenlogische Formel der Länge $n$. Dann ist $\neg x$ nach Def. 3.5.2 auch aussagenlogische Formel, also $\neg\varphi\in\mathcal{L}(\mathcal{R})$ wegen
$|\neg\varphi|=|\neg|+|\varphi|=1+|\varphi|=1+n$ $\square$

\subsection*{Aufgabe 1.3 (b)}
Beweis durch strukturelle Induktion über $\varphi\in\mathcal{L}(\mathcal{R})$ (Menge der aussagenlogischen Formeln).\\
Induktionsanfang: Sei $\varphi=p$ für ein $p\in\mathcal{R}$. Dann $|\varphi|=|p|=1\in\N$\\
Induktionshypothese: für aussagenlogische Formeln $\varphi_1,\varphi_2\in\mathcal{L}(\mathcal{R})$ gilt $|\varphi_1|,|\varphi_2|\in\N$.\\
Induktionsschritt: 
\begin{align*}
\varphi=\neg\varphi_1&\implies|\varphi|=|\neg\varphi_1|=|\neq|+|\varphi_1|=1+|\varphi_1|\stackrel{\text{IH}}{\in}\N\\
\varphi=(\varphi_1\circ\varphi_2)\mit\circ\in\lbrace\wedge,\vee,\to,\leftrightarrow\rbrace&\implies|\varphi|=|(\varphi_1\circ\varphi_2)|=3+|\varphi_1|+|\varphi_2|\in\stackrel{\text{IH}}{\in}\N\qquad\square
\end{align*}

\section*{Aufgabe 1.4}
\subsection*{Aufgabe 1.4 (a)}
Die Aussage stimmt, ja.\\
Betrachte die Funktionen, die aufgehende bzw. schließende Klammern zählen:
\begin{align*}
\sk\equiv\ok:\mathcal{L}(\mathcal{R})\to\N,\qquad
x\mapsto\left\lbrace\begin{array}{cl}
0, & \falls x\in\mathcal{R}\text{ (ist Atom)}\\
\ok(y), & \falls x=\neg y\\
1+\ok(y)+\ok(z), & \falls x=(y\circ z)
\end{array}\right.
\end{align*}
Man sieht schon, dass beide Funktionen für alle $x\in\mathcal{L}(\mathcal{R})$ übereinstimmen. Aber man kann das Offensichtliche natürlich noch induktiv zeigen:\\

Zu zeigen: $\forall\varphi\in\mathcal{L}(\mathcal{R}):\ok(\varphi)=\sk(\varphi)$.\\
Beweis über strukturelle Induktion:\\
Induktionsanfang: Sei $\varphi=p\in\mathcal{R}$ Atom. Dann gilt: $\ok(p)=0=\sk(p)$.\\
Induktionsvoraussetzung: Gelte für beliebiges aber festes $\varphi_1,\varphi_2\in\mathcal{L}(\mathcal{R})$\\ $\ok(\varphi_1)=\sk(\varphi_1)$ und $\ok(\varphi_2)=\sk(\varphi_2)$ .\\ 
Induktionsschritt:
\begin{align*}
\ok(\neg \varphi_1)
\stackeq{\text{Def}}\ok(\varphi_1)
\stackeq{\text{IV}}\sk(\varphi_1)
\stackeq{\text{Def}}\sk(\neg \varphi_1)\\
\ok((\varphi_1\circ \varphi_2))
\stackeq{\text{Def}}1+\ok(\varphi_1)+\ok(\varphi_2)
\stackeq{\text{IV}}1+\sk(\varphi_1)+\sk(\varphi_2)
\stackeq{\text{Def}}\sk((\varphi_1\circ \varphi_2))
\end{align*}

\subsection*{Aufgabe 1.4 (b)}
Ja, diese Aussagen gilt.\\
Betrachte die Funktion, die Klammern bzw. binäre Junktoren zählt:
\begin{align*}
\k:\mathcal{L}(\mathcal{R})\to\N,\qquad
x\mapsto\left\lbrace\begin{array}{cl}
0, & \falls x\in\mathcal{R}\text{ (ist Atom)}\\
\k(y), & \falls x=\neg y\\
2+\k(y)+\k(z), & \falls x=(y\circ z)
\end{array}\right.\qquad \bj:\equiv \ok\equiv\sk
\end{align*}

Beweis durch strukturelle Induktion über $\varphi\in\mathcal{L}(\mathcal{R})$.\\
Induktionsanfang: Sei $\varphi=p\in\mathcal{R}$ Atom. Dann gilt: $\k(\varphi)=0=2\cdot 0=\bj(\varphi)$.\\
Induktionsvoraussetzung: Gelte für beliebige aber feste $\varphi_1,\varphi_2\in\mathcal{L}(\mathcal{R})$\\
$\k(\varphi_1)=2\cdot\bj(\varphi_1)$ und $\k(\varphi_2)=2\cdot\bj(\varphi_2)$.\\ 
Induktionsschritt: betrachte $\varphi=\neg\varphi_1$ und $\varphi=(\varphi_1\circ\varphi_2)\mit\circ\in\lbrace\wedge,\vee,\to,\leftrightarrow\rbrace$: 
\begin{align*}
\k(\neg \varphi_1)
&\stackeq{\text{Def}}\k(\varphi_1)
\stackeq{\text{IV}}2\cdot\bj(\varphi_1)
\stackeq{\text{Def}}\bj(\neg \varphi_1)\\
\k((\varphi_1\circ \varphi_2))
&\stackeq{\text{Def}}2+\k(\varphi_1)+\k(\varphi_2)\\
&\stackeq{\text{IV}}2+2\cdot\bj(\varphi_1)+2\cdot\bj(\varphi_2)\\
&=2\cdot(1+\bj(\varphi_1)+\bj(\varphi_2))\\
&\stackeq{\text{Def}}2\cdot\bj((\varphi_1\circ \varphi_2))
\end{align*}

\subsection*{Aufgabe 1.4 (c)}
Nein, Gegenbeispiel: $\neg\neg\neg p\in\mathcal{L}(\mathcal{R})$ wobei  $p\in\mathcal{R}$ aussagenlogische Variable ist und $\neg$ unärer Junktor.

\section*{Aufgabe 1.5}
\subsection*{Aufgabe 1.5 (a)}
Sei $A\in\mathcal{R}$ aussagenlogische Variable. Setze
\begin{align*}
h_A:\mathcal{L}(\mathcal{R})\to\N_0,\qquad x\mapsto \left\lbrace\begin{array}{cl}
0, & \falls x\in\mathcal{R}\wedge x\neq A\\
1, & \falls x\in\mathcal{R}\wedge x=A\\
h_A(y), & \falls x=\neg y\\
h_A(y)+h_A(z), & \falls x=(y\circ z)
\end{array}\right.
\end{align*}
Ausführlicher Zwischenschritt:
\begin{align*}
&h_{AR}:\mathcal{R}\to\N_0,\qquad x\mapsto \left\lbrace\begin{array}{cl}
1, & \falls x= A\\
0, & \sonst
\end{array}\right.\\
&h_{A\neg}:\N\to\N,\qquad n\mapsto n\\
&h_{A\circ}:\N\times\N\to\N,\qquad(m,n)\mapsto m+n\\
&\implies h_A:\mathcal{L}(\mathcal{R})\to\N_0,\qquad\varphi\mapsto
\left\lbrace\begin{array}{cl}
h_{AR}(p), & \falls \varphi=P\\
h_{A\neg}(h_A(\varphi_1)), & \falls \varphi=\neg\varphi_1\\
h_{A\circ}(h_A(\varphi_1),h_A(\varphi_2)), &\falls \varphi=(\varphi_1,\varphi_2)
\end{array}\right.
\end{align*}

\subsection*{Aufgabe 1.5 (b)}
\begin{align*}
\text{laenge}:\mathcal{L}(\mathcal{R})\to\N_0,\qquad x\mapsto \left\lbrace\begin{array}{cl}
1, & \falls x\in\mathcal{R}\\
1+\text{laenge}(y), & \falls x=\neg y\\
3+\text{laenge}(y)+\text{laenge}(z), & \falls x=(y\circ z)
\end{array}\right.
\end{align*}
Und damit:
\begin{align*}
\text{laenge}((ü\vee(q\wedge\neg p))) 
&=3+\text{laenge}(p)+\text{laenge}((q\wedge\neg p))\\
&=3+1+3+\text{laenge}(q)+\text{laenge}(\neg p)\\
&=7 + 1 + 1+ \text{laenge}(p)\\
&=9+1=10
\end{align*}

\subsection*{Aufgabe 1.5 (c)}
\begin{align*}
\text{du}:\mathcal{L}(\mathcal{R})\to\mathcal{L}(\mathcal{R}),x\mapsto \left\lbrace\begin{array}{cl}
\neg x, & \falls x=p\in\mathcal{R}\\
\neg\text{du}(G), & \falls x=\neg G\\
(\text{du}(G)\vee\text{du}(H)), & \falls x=(G\wedge H)\\
(\text{du}(G)\wedge\text{du}(H)), & \falls x=(G\vee H)\\
(\text{du}(H)\wedge\neg\text{du}(G)), & \falls x=(H\to G)\\
((\text{du}(H)\wedge\neg \text{du}(G))\vee(\text{du}(G)\wedge\neg\text{du}(H))), & \falls x=(H\leftrightarrow G)\\
\end{array}\right.
\end{align*}
Ausführlicher:
\begin{align*}
&\text{du}_R:\mathcal{R}\to\mathcal{L}(\mathcal{R}),\qquad p\mapsto\neg p\\
&\text{du}_{\neg}:\mathcal{L}(\mathcal{R})\to\mathcal{L}(\mathcal{R}),\qquad\varphi\mapsto\neg\varphi\\
&\text{du}_{\wedge}:\mathcal{L}(\mathcal{R})\times\mathcal{L}(\mathcal{R})\to\mathcal{L}(\mathcal{R}),\qquad(\varphi_1,\varphi_2)\mapsto(\varphi_1\vee\varphi_2)\\
&\text{du}_{\vee}:\mathcal{L}(\mathcal{R})\times\mathcal{L}(\mathcal{R})\to\mathcal{L}(\mathcal{R}),\qquad(\varphi_1,\varphi_2)\mapsto(\varphi_1\wedge\varphi_2)\\
&\text{du}_{\to}:\mathcal{L}(\mathcal{R})\times\mathcal{L}(\mathcal{R})\to\mathcal{L}(\mathcal{R}),\qquad(\varphi_1,\varphi_2)\mapsto(\varphi_1\wedge\neg\varphi_2)\\
&\text{du}_{\leftrightarrow}:\mathcal{L}(\mathcal{R})\times\mathcal{L}(\mathcal{R})\to\mathcal{L}(\mathcal{R}),\qquad(\varphi_1,\varphi_2)\mapsto((\varphi_1\wedge\varphi_2)\vee(\varphi_2\wedge\neg\varphi_1)
\end{align*}

Und damit:
\begin{align*}
\text{du}((p\to(p\wedge\neg q)))
&=(\text{du}(p)\wedge\neg\text{du}((p\wedge\neg q)))\\
&=\neg p\wedge\neg(\text{du}(p)\vee\text{du}(\neg q))\\
&=\neg p\wedge\neg\neg p\vee\neg\text{du}(q)\\
&=\neg p\wedge\neg\neg p\vee\neg\neg q\\
\end{align*}
Vielleicht sollte die Funktion noch ein paar Klammern setzen :D

\end{document}