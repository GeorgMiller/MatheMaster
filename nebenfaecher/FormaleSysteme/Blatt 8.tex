% This work is licensed under the Creative Commons
% Attribution-NonCommercial-ShareAlike 4.0 International License. To view a copy
% of this license, visit http://creativecommons.org/licenses/by-nc-sa/4.0/ or
% send a letter to Creative Commons, PO Box 1866, Mountain View, CA 94042, USA.

\documentclass[12pt,a4paper]{article} 

% This work is licensed under the Creative Commons
% Attribution-NonCommercial-ShareAlike 4.0 International License. To view a copy
% of this license, visit http://creativecommons.org/licenses/by-nc-sa/4.0/ or
% send a letter to Creative Commons, PO Box 1866, Mountain View, CA 94042, USA.

% PACKAGES
\usepackage[english, ngerman]{babel}	% Paket für Sprachselektion, in diesem Fall für deutsches Datum etc
\usepackage[utf8]{inputenc}	% Paket für Umlaute; verwende utf8 Kodierung in TexWorks 
\usepackage[T1]{fontenc} % ö,ü,ä werden richtig kodiert
\usepackage{amsmath} % wichtig für align-Umgebung
\usepackage{amssymb} % wichtig für \mathbb{} usw.
\usepackage{amsthm} % damit kann man eigene Theorem-Umgebungen definieren, proof-Umgebungen, etc.
\usepackage{mathrsfs} % für \mathscr
\usepackage[backref]{hyperref} % Inhaltsverzeichnis und \ref-Befehle werden in der PDF-klickbar
\usepackage[english, ngerman, capitalise]{cleveref}
\usepackage{graphicx}
\usepackage{grffile}
\usepackage{setspace} % wichtig für Lesbarkeit. Schöne Zeilenabstände

\usepackage{enumitem} % für custom Liste mit default Buchstaben
\usepackage{ulem} % für bessere Unterstreichung
\usepackage{contour} % für bessere Unterstreichung
\usepackage{epigraph} % für das coole Zitat

\usepackage{tikz}

% This work is licensed under the Creative Commons
% Attribution-NonCommercial-ShareAlike 4.0 International License. To view a copy
% of this license, visit http://creativecommons.org/licenses/by-nc-sa/4.0/ or
% send a letter to Creative Commons, PO Box 1866, Mountain View, CA 94042, USA.

% THEOREM-ENVIRONMENTS

\newtheoremstyle{mystyle}
  {20pt}   % ABOVESPACE \topsep is default, 20pt looks nice
  {20pt}   % BELOWSPACE \topsep is default, 20pt looks nice
  {\normalfont} % BODYFONT
  {0pt}       % INDENT (empty value is the same as 0pt)
  {\bfseries} % HEADFONT
  {}          % HEADPUNCT (if needed)
  {5pt plus 1pt minus 1pt} % HEADSPACE
	{}          % CUSTOM-HEAD-SPEC
\theoremstyle{mystyle}

% Definitionen der Satz, Lemma... - Umgebungen. Der Zähler von "satz" ist dem "section"-Zähler untergeordnet, alle weiteren Umgebungen bedienen sich des satz-Zählers.
\newtheorem{satz}{Satz}[section]
\newtheorem{lemma}[satz]{Lemma}
\newtheorem{korollar}[satz]{Korollar}
\newtheorem{proposition}[satz]{Proposition}
\newtheorem{beispiel}[satz]{Beispiel}
\newtheorem{definition}[satz]{Definition}
\newtheorem{bemerkungnr}[satz]{Bemerkung}
\newtheorem{theorem}[satz]{Theorem}

% Bemerkungen, Erinnerungen und Notationshinweise werden ohne Numerierungen dargestellt.
\newtheorem*{bemerkung}{Bemerkung.}
\newtheorem*{erinnerung}{Erinnerung.}
\newtheorem*{notation}{Notation.}
\newtheorem*{aufgabe}{Aufgabe.}
\newtheorem*{lösung}{Lösung.}
\newtheorem*{beisp}{Beispiel.} %Beispiel ohne Nummerierung
\newtheorem*{defi}{Definition.} %Definition ohne Nummerierung
\newtheorem*{lem}{Lemma.} %Lemma ohne Nummerierung


% SHORTCUTS
\newcommand{\R}{\mathbb{R}}				 % reelle Zahlen
\newcommand{\Rn}{\R^n}						 % der R^n
\newcommand{\N}{\mathbb{N}}				 % natürliche Zahlen
\newcommand{\Z}{\mathbb{Z}}				 % ganze Zahlen
\newcommand{\C}{\mathbb{C}}			   % komplexe Zahlen
\newcommand{\gdw}{\Leftrightarrow} % Genau dann, wenn
\newcommand{\with}{\text{ mit }}   % mit
\newcommand{\falls}{\text{falls }} % falls
\newcommand{\dd}{\text{ d}}        % Differential d

% ETWAS SPEZIELLERE ZEICHEN
%disjoint union
\newcommand{\bigcupdot}{
	\mathop{\vphantom{\bigcup}\mathpalette\setbigcupdot\cdot}\displaylimits
}
\newcommand{\setbigcupdot}[2]{\ooalign{\hfil$#1\bigcup$\hfil\cr\hfil$#2$\hfil\cr\cr}}
%big times
\newcommand*{\bigtimes}{\mathop{\raisebox{-.5ex}{\hbox{\huge{$\times$}}}}} 

% WHITESPACE COMMANDS
%non-restrict newline command
\newcommand{\enter}{$ $\newline} 
%praktischer Tabulator
\newcommand\tab[1][1cm]{\hspace*{#1}}

% TEXT ÜBER ZEICHEN
%das ist ein Gleichheitszeichen mit Text darüber, Beispiel: $a\stackeq{Def} b$
\newcommand{\stackeq}[1]{
	\mathrel{\stackrel{\makebox[0pt]{\mbox{\normalfont\tiny #1}}}{=}}
} 
%das ist ein beliebiges Zeichen mit Text darüber, z. B.  $a\stackrel{Def}{\Rightarrow} b$
\newcommand{\stacksymbol}[2]{
	\mathrel{\stackrel{\makebox[0pt]{\mbox{\normalfont\tiny #1}}}{#2}}
} 

% UNDERLINE
% besseres underline 
\renewcommand{\ULdepth}{1pt}
\contourlength{0.5pt}
\newcommand{\ul}[1]{
	\uline{\phantom{#1}}\llap{\contour{white}{#1}}
}


% hier noch ein paar Commands die nur ich nutze, weil ich sie mir im Laufe der Jahre angewöhnt habe und sie mir jetzt nicht abgewöhnen will:

\newcommand{\gdw}{\Leftrightarrow}   % genau dann, wenn




\author{Willi Sontopski}

\parindent0cm %Ist wichtig, um führende Leerzeichen zu entfernen

\usepackage{pdflscape}
\usepackage{rotating}
\usepackage{scrpage2}
\pagestyle{scrheadings}
\clearscrheadfoot

\ihead{Willi Sontopski}
\chead{Formale Systeme WiSe 18 19}
\ohead{}
\ifoot{Blatt 8}
\cfoot{Version: \today}
\ofoot{Seite \pagemark}

\newcommand{\A}{\mathcal{A}}

\begin{document}
%\setcounter{section}{1}

\section*{Aufgabe $\ast$)}
\begin{itemize}
\item Transitionssystem ist $\A=(Q,\Sigma,I,\Delta,F)$ wobei
\begin{itemize}
\item $Q$ eine Menge von Zuständen ist (nicht notwendigerweise endlich!)
\item $\Sigma$ ein endliches Alphabet
\item $I\subseteq Q$ eine Menge von Anfangszuständen
\item $\Delta\subseteq Q\times\Sigma\times Q$ eine Übergangsrelation (Transitionsrelation) ist
\item $F\subseteq Q$ eine Menge von Endzuständen ist
\end{itemize}
\item Ein NEA ist ein Transitionssystem mit $|Q|<\infty$ und $|I|=1$.
\item Ein NEA mit Wortübergängen hat die Form $\A=(Q,\Sigma,q_0,\Delta,F)$, wobei $Q,\Sigma,q_0,F$ wie beim NEA definiert sind und $\Delta\subseteq Q\times \Sigma^\ast\times Q$ eine endliche Menge von Wortübergängen ist. (Beachte $\Sigma^\ast$!, somit ein NEA mit Wortübergängen \underline{kein} NEA!)
\item Ein $\varepsilon$-NEA ist ein NEA mit Wortübergängen mit
\begin{align*}
\underbrace{|w|\leq1}_{\gdw w\in\Sigma\cup\lbrace\varepsilon\rbrace}\qquad\forall(q,w,q')\in\Delta
\end{align*}
\item $\varepsilon$-Übergang ist v.d.F. $(q,\varepsilon,q')\in Q\times\Sigma^\ast\times Q$
\item Eine Menge $L$ von Wörtern mit $L\subseteq\Sigma^\ast$ heißt formale Sprache über $\Sigma$.
\item unendliche Menge endlicher Wörter ist die Menge aller Wörter (= endliche Zeichenfolge) über $\Sigma$, i.Z. $\Sigma^\ast$.
\end{itemize}

\section*{Aufgabe $\ast\ast$)}
Für $L_1$ betrachte den NEA: $\A_1=(Q=\lbrace q_0,q_1,q_2\rbrace,\Sigma=\lbrace a,b\rbrace,q_0,\Delta,\lbrace q_2\rbrace)$ mit
\begin{align*}
\Delta=\lbrace (q_0,a,q_1),(q_1,a,q_1),(q_1,b,q_1),(q_1,a,q_2)\rbrace\subseteq Q\times\Sigma\times Q
\end{align*}

\usetikzlibrary{positioning,automata}
\begin{tikzpicture}[shorten >=1pt,node distance=2.7cm,on grid]
  \node[state,initial]   (q_0)                {$q_0$};
  \node[state]           (q_1) [right=of q_0] {$q_1$};
  \node[state,accepting] (q_2) [right=of q_1] {$q_2$};
  \path[->] (q_0) edge                node [above] {a} (q_1)
            (q_1) edge [loop above]   node [above] {a,b} ()
                  edge                node [above] {a} (q_2);
\end{tikzpicture}

%TODO Das folgende ist falsch, da Wörter wie abab fehlen.
Für $L_2$ betrachte den NEA: $\A_2=(Q=\lbrace q_0,q_1,q_2\rbrace,\Sigma=\lbrace a,b\rbrace,q_0,\Delta,\lbrace q_0\rbrace)$ mit
\begin{align*}
\Delta=\lbrace (q_0,a,q_1),(q_1,a,q_0),(q_0,b,q_2),(q_2,b,q_0)\rbrace\subseteq Q\times\Sigma\times Q
\end{align*}

\usetikzlibrary{positioning,automata}
\begin{tikzpicture}[shorten >=1pt,node distance=2.7cm,on grid]
  \node[state,initial, accepting]   (q_0)                {$q_0$};
  \node[state]           (q_1) [above right=of q_0] {$q_1$};
  \node[state] (q_2) [below right=of q_0] {$q_2$};
  \node[state] (q_3) [above right=of q_2] {$q_3$};
  \path[->] (q_0) edge [bend left=30] node [above] {a} (q_1)
                  edge [bend left=30] node [above] {b} (q_2)
            (q_1) edge [bend left=30] node [above] {a} (q_0)
            	  edge [bend left=30] node [above] {b} (q_3)
            (q_2) edge [bend left=30] node [above] {b} (q_0)
                  edge [bend left=30] node [above] {a} (q_3)
            (q_3) edge [bend left=30] node [above] {b} (q_1)
                  edge [bend left=30] node [above] {a} (q_2);
\end{tikzpicture}


\section*{Aufgabe 1}
\begin{align*}
V_0:=\lbrace v_0\rbrace,\qquad V_{i+1}:=V_i\cup\big\lbrace v\in V\mid\exists v'\in V_i:(v',v)\in E\big\rbrace
\end{align*}
\begin{proof}
Teil 1 Ist trivial nach Konstruktion (ansonsten kann man das leicht induktiv zeigen)\nl
\ul{Zu Teil 2:}\\
Die Folge $(V_i)_{i\in\N}$ ist nach Teil 1 monoton wachsend und durch die endliche (!) Menge $V$ nach oben beschränkt. Folglich muss es einen Index geben, ab dem die Folge stationär wird.\nl
Alternativ: Beweis durch Widerspruch: Angenommen, $V_i\neq V_{i+1}~\forall i\in\N$. Aus Teil 1 folgt dann $V_i\varsubsetneqq V_{i+1}~\forall i\in\N.$ Man sieht leicht, dass $V_i\subseteq V~\forall i\in\N$. Damit existiert eine injektive Abbildung, welche jeder natürlichen Zahl $n\in\N$ ein Element aus $V_{n+1}\setminus V_n$ zuweist. Damit gilt: $|\N|\leq |V|$. Dies ist ein Widerspruch zur Annahme, dass $V$ endlich ist.\nl
Aufwand: $\mathcal{O}(|V|^2)$ %bin mir nicht ganz sicher
\end{proof}

\section*{Aufgabe 2}
Ein DEA ist ein NEA, falls
\begin{align*}
:\Longleftrightarrow\forall q\in Q,\forall a\in\Sigma:\exists!~q'\in Q:(q,a,q')\in\Delta
\end{align*}
Es erscheint sinnvoll, sich stets zuerst den NEA zu überlegen und danach diesen in einen DEA umzuschreiben.

\subsection*{Aufgabe 2 a)}
\begin{align*}
L_1=\big\lbrace a^n bac^m\mid m,n>0\text{ $n$ ist gerade und $m$ ist ungerade}\big\rbrace
\end{align*}
Wir konstruieren zuerst einen NEA, der $L_a$ akzeptiert:\\

\usetikzlibrary{positioning,automata}
\begin{tikzpicture}[shorten >=1pt,node distance=2.3cm,on grid]
  \node[state,initial]   (q_0)                {$q_0$};
  \node[state]           (q_1) [right=of q_0] {$q_1$};
  \node[state]           (q_2) [below=of q_1] {$q_2$};
  \node[state]           (q_3) [right=of q_1] {$q_3$};
  \node[state]           (q_4) [right=of q_3] {$q_4$};
  \node[state]           (q_5) [right=of q_4] {$q_5$};
  \node[state, accepting](q_6) [right=of q_5] {$q_6$};
  \node[state]           (q_7) [below=of q_6] {$q_7$};
  \path[->] (q_0) edge                node [above] {a} (q_1)
            (q_1) edge                node [above] {a} (q_3)
                  edge [bend left=30] node [right] {a} (q_2)
            (q_2) edge [bend left=30] node [left]  {a} (q_1)
            (q_3) edge                node [above] {b} (q_4)
            (q_4) edge                node [above] {a} (q_5)
            (q_5) edge                node [above] {c} (q_6)
            (q_6) edge [bend left=30] node [right] {c} (q_7)
            (q_7) edge [bend left=30] node [left]  {c} (q_6);
\end{tikzpicture}

Um daraus einen DEA zu machen müssen wir:
\begin{itemize}
\item Papierkorbzustände ergänzen
\item Potenzmengenkonstruktion
\end{itemize}

Hier nun die Lösung aus der Übung (der Tutor stellte direkt den DEA auf und ging nicht obigen Umweg):

\usetikzlibrary{positioning,automata}
\begin{tikzpicture}[shorten >=1pt,node distance=2.3cm,on grid]
  \node[state,initial]   (q_0)                {$q_0$};
  \node[state]           (q_1) [right=of q_0] {$q_1$};
  \node[state]           (q_3) [right=of q_1] {$q_3$};
  \node[state]           (q_4) [right=of q_3] {$q_4$};
  \node[state]           (q_5) [right=of q_4] {$q_5$};
  \node[state, accepting](q_6) [right=of q_5] {$q_2$};
  \node[state]           (q_t) [below=of q_3] {$q_t$};
  \path[->] (q_0) edge                node [above] {a} (q_1)
                  edge [bend right=30] node [below] {b,c} (q_t)
            (q_1) edge [bend left=30] node [above] {a} (q_3)
                  edge [bend right=15]node [above] {b,c} (q_t)
            (q_3) edge                node [above] {b} (q_4)
                  edge [bend left=30] node [above] {a} (q_1)
                  edge                node [right] {c} (q_t)
            (q_4) edge                node [above] {a} (q_5)
            (q_5) edge [bend left=30] node [above] {c} (q_6)
            (q_6) edge [bend left=30] node [above] {c} (q_5)
            (q_t) edge [loop below]   node [below]  {a,b,c,} ();
\end{tikzpicture}

\subsection*{Aufgabe b)}
\begin{align*}
L_2=\big\lbrace w\in\lbrace0,1\rbrace^\ast\mid\exists i\in\lbrace0,1\rbrace:w\text{ endet mit $i$ und enthält eine ungerade Anzahl von $i$'s}\big\rbrace
\end{align*}

Konstruiere zuerst Automaten, welche Wörter über $\Sigma=\lbrace 0,1\rbrace$ erkennen, die mit $i$ enden und eine gerade Anzahl an $i$'s enthalten (für $i\in\lbrace0,1\rbrace$. Dann ``vereinige'' diese beiden Automaten in $\varepsilon$-NEA: 

\section*{Aufgabe 3}
Zwei Transitionssysteme (NEAs sind Spezialfälle) heißen \textbf{äquivalent}, wenn sie dieselbe Sprache akzeptieren.\\
Formaler: Seien $\A=(Q,\Sigma,q_0,\Delta,F),\A'=(Q',\Sigma',q_0',\Delta',F')$ zwei NEAs. Diese sind äquivalent
\begin{align*}
&\Longleftrightarrow L(\A_1)=L(\A_2)\\
&\Longleftrightarrow \big\lbrace w\in\Sigma^\ast:\A_1\text{ akzeptiert }w\big\rbrace
=\big\lbrace w\in\Sigma'^\ast:\A_2\text{ akzeptiert }w\big\rbrace\\
&\Longleftrightarrow\big\lbrace w\in\Sigma^\ast:I\stackrel{w}{\longrightarrow}_{\A_1} F\big\rbrace=
\big\lbrace w\in\Sigma'^\ast:I'\stackrel{w}{\longrightarrow}_{\A_2} F'\big\rbrace
\end{align*}

\section*{Aufgabe 4}
Sei $\A=(Q,\Sigma,I,\delta,F)$ ein DEA.

\subsection*{Aufgabe 4 a)}
\begin{align*}
\forall\lbrace u,v\rbrace\subseteq\Sigma^\ast,\forall q\in Q:\delta(q,uv)=\delta\big(\delta(q,u),v\big)
\end{align*}
\begin{proof}
%Sei $\lbrace u,v\rbrace\subseteq\Sigma^\ast$ beliebige Teilmenge und sei $q\in Q$ beliebig.\nl
Beweis durch Induktion über dieLänge von $v$ beliebigem $u\in\Sigma^\ast$.\\
Erinnerung an die Definition:
\begin{align}
\delta^\ast(q,\varepsilon)&:=q\label{eq1}\\
\delta^\ast(q,wa)&:=\delta\big(\delta^\ast(q,w),a\big)\label{eq2}
\end{align}
\ul{IA:} $|v|=0$, d.h. $v=\varepsilon$:
\begin{align*}
\delta^\ast(q,uv)\overset{u\cdot\varepsilon=u}&{=}
\delta^\ast(q,u)
\overset{\eqref{eq1}}{=}
\delta^\ast\big(\delta^\ast(q,u),\varepsilon\big)
=\delta^\ast\big(\delta^\ast(q,u),v\big)
\end{align*}
\ul{IH:} Für alle $v\in\Sigma^n$ gilt: $\delta^\ast(q,uv)=\delta^\ast\big(\delta^\ast(q,u),v\big)$\nl
\ul{IS:} Sei $v'\in\Sigma^{n+1}$, d.h. es existiert $v\in\Sigma^n$ und $a\in\Sigma$ mit $v'=va$.
\begin{align*}
\delta^\ast\big(q,uv'\big)
&=
\delta^\ast\big(q,uva\big)\\
\overset{\eqref{eq2}}&=
\delta\big(\delta^\ast(q,uv),a\big)\\
\overset{\text{IH}}&=
\delta\big(\delta^\ast\big(\delta^\ast(q,u),v\big),a\big)\\
\overset{\eqref{eq2}}&=
\delta^\ast\big(\delta^\ast(q,u),va\big)\\
&=
\delta^\ast\big(\delta^\ast(q,u),v'\big)
\end{align*}
\end{proof}

\subsection*{Aufgabe 4 b)}
\begin{align*}
\sim_k=\sim_{k+1}\implies\sim_k=\sim_\A
\end{align*}
\begin{proof}
Angenommen es gilt $\sim_k=\sim_{k+1}$ für ein festes $k\in\N$. Dann gilt:
%TODO
\end{proof}

\end{document}