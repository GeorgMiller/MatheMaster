% This work is licensed under the Creative Commons
% Attribution-NonCommercial-ShareAlike 4.0 International License. To view a copy
% of this license, visit http://creativecommons.org/licenses/by-nc-sa/4.0/ or
% send a letter to Creative Commons, PO Box 1866, Mountain View, CA 94042, USA.

\chapter{Nichtlineare Halbgruppen} %4
\setcounter{section}{1}
Sei $(D,d)$ ein metrischer Raum.

\begin{definition}
Eine \textbf{(stark stetige) Halbgruppe} auf $D$ ist eine Funktion\\ $S:[0,\infty]\to C(D,D)$ so, dass 
\begin{enumerate}[label=(\roman*)]
\item $S_0=\id$, d.h. $S_0(x)=x\qquad\forall x\in D$
\item $\begin{aligned}
S_{t+s}=S_t\circ S_s\qquad\forall s,t\geq0
\end{aligned}$
\item $\begin{aligned}
[0,\infty)\to D,\qquad t\mapsto S_t(x)\text{ ist stetig}\qquad\forall x\in D
\end{aligned}$
\end{enumerate}
(alternativer Name: \textbf{topologisches dynamisches System})\nl
Manchmal betrachtet man auch \textbf{degenerierte (stark stetige) Halbgruppen}\\ $S\colon (0,\infty)\to C(D,D)$. Definition wie oben, nur ersetze $[0,\infty]$ durch $(0,\infty)$ (auch in (iii)). Eigenschaft (i) fällt dann weg.
\end{definition}

\begin{beispiel}[Exponentialfunktion]\enter
Sei $X$ ein Banachraum, $A\in\L(X)$ (d.h. beschränkter linearer Operator auf $X$) und 
\begin{align*}
S_t:=\sum\limits_{n=0}^\infty\frac{t^n\cdot A^n}{n!}=:\exp(t\cdot A)\qquad t\in[0,\infty]\text{ oder }t\in\C
\end{align*}
Die Reihe konvergiert absolut
\begin{align*}
\sum\limits_{n=0}^\infty \frac{t^n\cdot\Vert A^n\Vert}{n!}\leq\sum\limits_{n=0}^\infty\frac{t^n\cdot\Vert A\Vert^n}{	n!}=\exp\big(t\cdot\Vert A\Vert\big)<\infty
\end{align*}
und somit konvergiert sie auch in $\L(X)\subseteq C(X,X)$.\\
Die Abbildung $S:\R_+\to\L(X)$ ist (als Potenzreihe mit Konvergenzradius $+\infty$) beliebig oft differenzierbar und 
\begin{align*}
\frac{\d}{\d t}S_t=AS_t=S_tA
\end{align*}
(Differenzierbarkeit bzgl. der Operatornorm in $\L(X)$!)\\
Für alle $u_0\in X$ ist also der Orbit $u(t)=S_t u_0$ eine klassische Lösung des Cauchyproblems
\begin{align*}
\left\lbrace\begin{array}{r}
\dot{u}-Au=0\\
u(0)=u_0
\end{array}\right.
\end{align*}
\end{beispiel}

\begin{beispiel}[$C_0$-Halbgruppen / lineare Halbgruppen]\enter
Eine stark-stetige Halbgruppe $S$ heißt auch \textbf{$C_0$-Halbgruppe} oder \textbf{lineare Halbgruppe}, wenn $D=X$ ein Banachraum ist und alle $S_t$ stetige, lineare Operatoren sind.\\
Für lineare Halbgruppen gelten folgende Eigenschaften:
\end{beispiel}

\begin{lemma}
Ist $S$ eine lineare Halbgruppe auf einem Banachraum $X$, dann existieren Konstanten $M\geq1$ und $\omega\in\R$ so, dass 
\begin{align*}
\big\Vert S_t\big\Vert_{\L(X)}\leq M\cdot\exp(\omega\cdot t)\qquad\forall t\geq0
\end{align*}
\end{lemma}

\begin{proof}
Für alle $x\in X$ ist die Funktion $f\mapsto S_t x$ stetig auf $\R_+$ (Eigenschaft (iii)) und insbesondere auf dem Intervall $[0,1]$. Damit ist die lineare Abbildung 
\begin{align*}
T:X\to C\big([0,1],X\big),\qquad x\mapsto\big(t\mapsto S_t x\big)
\end{align*}
wohldefiniert. 
%Prof: "Alle Abbildungen, die man hinschreiben kann, sind stetig." -> Er meint den Satz vom abgeschlossenen Graphen.
Nun wollen wir den Satz vom abgeschlossenen Graphen anwenden. $T$ ist außerdem abgeschlossen: Es gelte $x_n\stackrel{n\to\infty}{\longrightarrow}x$ in $X$ und $T x_n\stackrel{n\to\infty}{\longrightarrow}f$ in $C\big([0,1],X\big)$. Zu zeigen:\\
Die Konvergenz $T x_n\stackrel{n\to\infty}{\longrightarrow}f$ (gleichmäßige Konvergenz) impliziert punktweise Konvergenz, d.h.
\begin{align*}
\big(\underbrace{T x_n}_{S_t x_n}\big)(t)\stackrel{n\to\infty}{\longrightarrow} f(t)\text{ in }X\qquad\forall t\in[0,1]
\end{align*}
Aus der Stetigkeit von $S_t$ folgt
\begin{align*}
f(t)=\limn S_t x_n=S_t x\qquad\forall t\in[0,1]
\end{align*}
d.h. $f=T x$. Aus dem Satz vom abgeschlossenen Graphen folgt, dass $T$ stetig / beschränkt (äquivalent) ist.\\
Sei $M:=\Vert T\Vert$. Dann ist 
\begin{align*}
\big\Vert S_t x\big\Vert_X&\leq\sup\limits_{s\in[0,1]}\big\Vert S_s x\big\Vert\leq M\cdot\Vert x\Vert&\forall& x\in X,\forall t\in[0,1]\\
\implies\big\Vert S_t\big\Vert_{\L(X)}&\leq M &\forall& t\in [0,1]
\end{align*}
Sei $t\in\R_+$ beliebig. Dann ist $t=n+\delta$ mit $n\in\N_0$ und $\delta\in[0,1[$ und somit
\begin{align*}
\big\Vert S_t\big\Vert
&=\big\Vert S_{n+\delta}\big\Vert
\overset{\text{(ii)}}{=}
\big\Vert S_n S_\delta\big\Vert
=\big\Vert S_1^n S_\delta\big\Vert
\leq\big\Vert S_1\big\Vert^n\cdot\big\Vert S_\delta\big\Vert
\leq M^{n+1}\\
&=\underbrace{M^{n+\delta}}_{=\exp\big((n+\delta)\cdot\log(M)\big)}\cdot M^{1-\delta}
\overset{\omega=\log(M)}\leq
M^{1-\delta}\cdot\exp(\omega\cdot t)
\overset{M\geq1}{\leq}
M\cdot\exp(\omega\cdot t)
\end{align*}
\end{proof}

Sei $S$ eine lineare Halbgruppe auf einem Banachraum $X$. Dann definieren wir den \textbf{Erzeuger} von $S$
\begin{align*}
A&:=\left\lbrace(u,f)\in X\times X~\left|~f=\lim\limits_{t\to0^+}\frac{S_t u-u}{t}\right.\right\rbrace\mit\\
\dom(A)&\hspace{3pt}=\left\lbrace u\in X~\left|~\lim\limits_{t\to0^+}\frac{S_t u-u}{t}\text{ existiert}\right.\right\rbrace
\end{align*}
Der Operator $A$ ist linear und eindeutig.

\begin{lemma}
Sei $S$ eine lineare Halbgruppe auf einem Banachraum $X$, mit Erzeuger $A$. Dann gilt:
\begin{enumerate}[label=(\alph*)]
\item Für alle $u\in\dom(A)$ und alle $t\in\R_+$ ist $S_t u\in\dom(A)$, die Abbildung $t\mapsto S_t$ ist differenzierbar und 
\begin{align*}
\frac{\d}{\d t} S_t u=A S_t u=S_t A u
\end{align*}
\item Für alle $u\in X$ und alle $t\in\R_+$ ist
\begin{align*}
\int\limits_0^t S_s u\d s\in\dom(A)
\end{align*}
und
\begin{align*}
A\int\limits_0^t S_s u\d s=S_t u-u
\end{align*}
\item Der Erzeuger $A$ ist abgeschlossen und $\dom(A)$ ist dicht in $X$.
\end{enumerate}
\end{lemma}

\begin{proof}
\underline{Zeige (a):}\\
sei $u\in\dom(A)$ und $t\in\R_+$. Dann gilt für $h>0$:
\begin{align*}
\frac{S_h\big(S_t u\big)-S_t u}{h}
\overset{\text{(iii)}}&=
\frac{S_{h+t} u-S_t u}{h}
\overset{\text{(iii)}}=
\frac{S_t\big(S_h u\big)-S_t u}{h}
=S_t\frac{S_h u-u}{h}\stackrel{h\to0^+}{\longrightarrow}S_t Au
\end{align*}
Wegen dieser Konvergenz ist $S_t u\in\dom(A)$ und 
\begin{align*}
A S_t u
&=S_t Au
=\frac{\d}{\d t}S_t u
\end{align*}
\underline{Zeige (b):}\\
Sei $u\in X$, $t\in\R_+$. Dann gilt für $h>0$:
\begin{align*}
\frac{S_h\left(\int\limits_0^t S_s u\d s\right)-\int\limits_0^t S_s \d s}{h}
&=\frac{\int\limits_0^t S_{s+h}-\int\limits_0^t S_s u\d s}{h}\\
\overset{\text{Subs}}&=
\frac{1}{h}\cdot\left(\int\limits_h^{t+h} S_s u\d s-\int\limits_0^t S_s u\d s\right)\\
&=\frac{1}{h}\cdot\left(\int\limits_t^{t+h} S_s u\d s-\int\limits_0^h S_s u\d s\right)
\stackrel{h\to0^+}{\longrightarrow} S_t u-u 
\end{align*}
Also ist 
\begin{align*}
\int\limits_0^t S_s u\d s\in\dom(A)\qquad\text{und}\qquad
A\int\limits_0^t S_s u\d s= S_t u-u
\end{align*}

\underline{Zeige (c):}\\
Sei $u\in X$. Dann ist für alle $t\in\R_+$ (siehe (b)):
\begin{align*}
\cdot\int\limits_0^t S_s u \d s\in\dom(A)\\
\overset{\text{linear}}{\implies}
\frac{1}{t}\cdot\int\limits_0^t \underbrace{S_s u}_{\stackrel{t\to 0^+}{\longrightarrow} u} \d s\in\dom(A)\\
\end{align*}
Also ist $u\in\dom(A)$ und somit $\dom(A)$ dicht in $X$. Außerdem ist $A$ abgeschlossen, denn:\\
Es gelte
\begin{align*}
\dom(A)\ni u_n\stackrel{n\to\infty}{\longrightarrow} u\text{ in }X
\qquad\text{und}\qquad
A u_n\stackrel{n\to\infty}{\longrightarrow} f\text{ in }X
\end{align*}
Dann gilt für alle $t\in\R_+$ und alle $n$, wegen (a)
\begin{align*}
S_t A u_n=\frac{\d}{\d t}S_t u_n
\end{align*}
bzw. (Integration)
\begin{align*}
\int\limits_0^t S_s A u_n\d s&=S_t u_n-u_n\\
\downarrow n\to\infty&\qquad\downarrow n\to\infty\\
\overset{n\to\infty}{\implies}
\int\limits_0^t S_s f\d s&=S_t u-u\qquad|:t\\
\implies
\frac{S_t u-u}{t}&=\frac{1}{t}\cdot\int\limits_0^t S_s f\d s\stackrel{t\to0^+}{\longrightarrow} f
\end{align*}
Bei der Grenzwertbildung wird dominierte Konvergenz mit
\begin{align*}
\left\Vert S_s A u_n\right\Vert\leq M\cdot\exp(\omega\cdot s)\cdot\big\Vert A u_n\big\Vert
\end{align*}
und das erste Lemma verwendet.
Damit ist $u\in\dom(A)$ und $Au=f$.
\end{proof}

\begin{lemma}
Sei $S$ eine lineare Halbgruppe auf einem Banachraum $X$, mit Erzeuger $A$. Seien $M\geq1$, $\omega\in\R$ so, dass 
\begin{align*}
\big\Vert S_t\big\Vert\leq M\cdot\exp(\omega\cdot t)\qquad\forall t\in\R_+
\end{align*}
Dann ist für alle $\lambda\in\R\mit\lambda>\omega$ der Operator $\lambda-A$ invertierbar und 
\begin{align*}
(\lambda-a)^{-1}f=\int\limits_0^\infty\exp(-\lambda\cdot t)\cdot S_t f\d t\qquad\forall f\in X
\end{align*}
Es gilt:
\begin{align*}
\left\Vert(\lambda-A)^{-1}\right\Vert_{\L(X)}\leq\frac{M}{\lambda-\omega}
\end{align*}
\end{lemma}

\begin{proof}
Wegen
\begin{align*}
\big\Vert\exp(-\lambda\cdot t)\cdot S_t f\big\Vert_X\leq\exp(-\lambda\cdot t)\cdot M\cdot\exp(\omega\cdot t)\cdot\Vert f\Vert
\end{align*}
ist $t\mapsto\exp(-\lambda\cdot t)\cdot S_t f$ für alle $f\in X$ und alle $\lambda>\omega$ integrierbar.\\
Für alle $f\in X$ und alle $\lambda>\omega$ gilt:
\begin{align*}
(A-\lambda)\int\limits_0^T\exp(-\lambda\cdot t)\cdot S_t f\d t
\overset{(\ast)}&{=}
\exp(-\lambda\cdot t)\cdot S_t f-f\stackrel{T\to\infty}{\longrightarrow}-f
\end{align*}
Bei $(\ast)$ wird verwendet, dass $A-\lambda$ Erzeuger der linearen Halbgruppe $t\mapsto\exp(-\lambda\cdot t)\cdot S_t$ ist.\\
Und weil $A$ abgeschlossen ist, ist
\begin{align*}
\int\limits_0^\infty\exp(-\lambda\cdot t)\cdot S_t f\d t&\in\dom(A)
\qquad\text{und}\qquad
(\lambda-A)\cdot\int\limits_0^\infty\exp(-\lambda\cdot t)\cdot S_t f\d t &=f
\end{align*}
Ähnlich gilt für $f\in\dom(A)$:
\begin{align*}
\int\limits_0^\infty\exp(-\lambda\cdot t)\cdot S_t(\lambda-A)f\d t=f
\end{align*}
Damit ist der Operator
\begin{align*}
R_\lambda:X\to X,\qquad f\mapsto\int\limits_0^\infty\exp(-\lambda\cdot t)\cdot S_t f\d t
\end{align*}
Links- und Rechtsinverse von $\lambda-A$. Außerdem gilt:
\begin{align*}
\Big\Vert(\lambda-A)^{-1}\Big\Vert_{\L(X)}
&=\Vert R_\lambda\Vert_{\L(X)}\\
&=\sup\limits_{\Vert f\Vert\leq 1}\left\Vert\int\limits_0^\infty\exp(-\lambda\cdot t)\cdot S_t f\d t\right\Vert\\
&\leq\sup\limits_{\Vert f\Vert\leq 1}\int\limits_0^\infty\exp(-\lambda\cdot t)\cdot M\cdot\exp(\omega\cdot t)\cdot\Vert f\Vert\d t\\
&=\frac{M}{\lambda-w}
\end{align*}
\end{proof}
