% This work is licensed under the Creative Commons
% Attribution-NonCommercial-ShareAlike 4.0 International License. To view a copy
% of this license, visit http://creativecommons.org/licenses/by-nc-sa/4.0/ or
% send a letter to Creative Commons, PO Box 1866, Mountain View, CA 94042, USA.

\chapter{Wohlgestelltheit des Cauchyproblems} %3
\setcounter{section}{1}
\begin{theorem}[Crandall-Liggett]\label{theoremCrandall-Liggett}\enter
Sei $A\subseteq X\times X$ $m$-akkretiv vom Typ $\omega\in\R$, $X$ ein Banachraum.\\
Dann besitzt das Cauchy-Problem
\begin{align}\label{CPchapter3}\tag{CP}
\text{(CP)}\left\lbrace\begin{array}{rl}
\hat{u}+Au\ni 0&\text{ in }[0,T]\\
u(0)=u_0&
\end{array}\right.
\end{align}
für jedes $u_0\in\overline{\dom(A)}$ genau eine Integrallösung $u\in C\big([0,T],X\big)$.\\
Für je zwei Integrallösungen $u,\hat{u}$ von \eqref{CPchapter3} zu Anfangswerten $u_0$ bzw. $\hat{u}\in\overline{\dom(A)}$ gilt
\begin{align*}
\big\Vert u(t)-\hat{u}(t)\big\Vert
\leq\exp(\omega\cdot t)\cdot\big\Vert u_0-\hat{u}_0\big\Vert\qquad\forall t\in[0,T]
\end{align*}
Setze dann
\begin{align*}
S(t):=u(t)\qquad\forall t\geq0,\forall u_0\in\overline{\dom(A)}
\end{align*}
wobei $u$ eindeutige Integrallösung von \eqref{CPchapter3} mit Anfangswert $u_0$ ist $(T\geq t)$:
\begin{align*}
S(t):\overline{\dom(A)}\to\overline{\dom(A)}
\end{align*}
ist Lipschitz-stetig mit Lipschitzkonstante $S(0)=\id$ und $S(t+a)=S(t)\circ S(s)$
\end{theorem}