% This work is licensed under the Creative Commons
% Attribution-NonCommercial-ShareAlike 4.0 International License. To view a copy
% of this license, visit http://creativecommons.org/licenses/by-nc-sa/4.0/ or
% send a letter to Creative Commons, PO Box 1866, Mountain View, CA 94042, USA.

\setcounter{chapter}{-1}

\chapter{Einführung}
Eine \textbf{Halbgruppe} ist
\begin{align*}
t\mapsto S(t)\\
S(t):C\to C\\
S(0)=\id\\
S(t+s)=S(t)\circ S(s)
\end{align*}

Eine Beobachtung: Wir betrachten das folgende \textbf{Cauchy-Problem} (gewöhnliche GDG)
\begin{align}\label{CauchyProblem0}\tag{CP}
(\text{CP)}\left\lbrace\begin{array}{cl}
	\dot{u}(t)+Au(t)=0,~~~t\geq0\\
	u(0) =u_0
\end{array}\right.
\end{align}

\begin{theorem}[Picard-Lindelöf]
Sei $A\colon\R^N\to\R^N$ eine Lipschitz-stetige Funktion. Dann besitzt das Cauchy-Problem für alle $u_0\in\R^N$ genau eine (globale) Lösung $u\in C^1(\R_+;\R^N)$.
\end{theorem}
\begin{proof}
Die Existenz einer lokalen Lösung folgt bereits aus dem Theorem von Peano. Darauf kommen wir später zurück. Zur Erbringung des Eindeutigkeitsbeweises seien $u,v\in C^1(\R_+;\R^N)$ zwei Lösungen des Cauchy-Problems zum selben Anfangswert. Dann gilt 

\begin{align*}
\frac{1}{2}\frac{\d}{\d t}\Vert u(t)-v(t)\Vert_2^2 
&=\left\langle u(t)-v(t),\dot{u}(t)-\dot{v}(t)\right\rangle_2 \\
\overset{\text{CP}}&=
-\left\langle u(t)-v(t),Au(t)-Av(t)\right\rangle_2 \\
\overset{\text{C.S.}}&{\leq}
\Vert u(t)-v(t)\Vert_2\cdot\Vert Au(t)-Av(t)\Vert_2 \\
\overset{\text{Lip}}&{\leq}
L\cdot\Vert u(t)-v(t)\Vert_2^2
\end{align*}

mit der Cauchy-Schwarz-Ungleichung und der Lipschitz-Stetigkeit von $A$. Wir integrieren über $\left[0,t\right]$ und erhalten

\begin{align*}
	\Vert u(t)-v(t)\Vert_2^2\leq\Vert u(0)-v(0)\Vert_2^2+2L\int_0^t\Vert u(s)-v(s)\Vert_2^2\,\d s,
\end{align*}

wobei $\Vert u(0)-v(0)\Vert_2^2=0$ durch Identität der Anfangswerte für $u$ und $v$. Mit dem Lemma von Gronwall folgt 

\begin{align*}
	\Vert u(t)-v(t)\Vert_2^2\leq\exp(2\cdot L\cdot t)\Vert u(0)-v(0)\Vert_2^2=0\qquad\forall t\in\left[0,T\right].
\end{align*}

Daraus folgt $u\equiv v$. Der Existenzbeweis erfolgt über den Beweis, dass kein "`blow-up"' in endlicher Zeit möglich ist.
\end{proof}

\begin{bemerkung}
Nur Stetigkeit von $A$ reicht nicht für die eindeutige Lösbarkeit. Zum Beispiel sei $N=1$ und $Au=-\operatorname{sgn}(u)\cdot\sqrt{\vert u\vert}$.
\end{bemerkung}

Eine Alternative zur Lipschitz-Stetigkeit:

\begin{definition}[Monotonie]
Die Funktion $A\colon\R^N\to\R^N$ heißt \textbf{monoton (wachsend)}
\begin{align*}
:\gdw\forall u,v\in\R^N\colon\langle Au-Av,u-v\rangle\geq 0
\end{align*}
\end{definition}

\begin{lemma}
Ist $A$ monoton und stetig, dann gibt es zu jedem $u_0\in\R^N$ immer noch genau eine (globale) Lösung des Cauchy-Problems \eqref{CauchyProblem0}.
\end{lemma}
\begin{proof}
\begin{align*}
	\frac{1}{2}\frac{d}{dt}\Vert u(t)-v(t)\Vert_2^2=-\langle u(t)-v(t),Au(t)-Av(t)\rangle\leq 0.\\
	\Longrightarrow t\mapsto\frac{1}{2}\Vert u(t)-v(t)\Vert^2\text{ ist monoton fallend}
\end{align*}

Die Abbildung $t\mapsto 2^{-1}\Vert u(t)-v(t)\Vert_2^2$ ist also monoton fallend, woraus die eindeutige Lösbarkeit folgt. 
\end{proof}

\begin{beispiel}
$N=1$ und $Au=\operatorname{sgn}(u)\cdot\sqrt{\vert u\vert}$
\end{beispiel}

\begin{korollar}[Picard-Lindelöf allgemein]
$A=A_0+A_1$, wobei $A_0\colon\R^N\to\R^N$ monoton sowie stetig und $A_1\colon\R^N\to\R^N$ Lipschitz-stetig sind. Daraus folgt eindeutige (globale) Lösbarkeit und Existenz der Lösung.
\end{korollar}
\begin{proof}
Folgt auf beiden vorherigen Resultaten.
\end{proof}

In dieser Vorlesung behandeln wir die allgemeine Theorie in Banachräumen.