\chapter{Akkretive Operatoren}
Im Folgenden sei $X$ ein Banachraum mit Norm $\Vert\cdot\Vert$ und $H$ ein Hilbertraum mit Skalarprodukt $\langle\cdot,\cdot\rangle$.

\begin{definition}
Ein \textbf{(nichtlinearer) Operator} auf $X$ ist eine Relation $A\subseteq X\times X$. Wir schreiben
\begin{itemize}
\item $Au:=\lbrace f\in X:(u,f)\in A\rbrace~\forall u\in X$
\item $\dom(A):=\lbrace u\in X:Au\neq\emptyset\rbrace$ \textbf{Definitionsbereich} von $A$
\item $\rg(A):=\lbrace f\in X:\exists u\in X:(u,f)\in A\rbrace$ \textbf{Bild} von $A$
\item $A^{-1}:=\lbrace (f,u)\in X\times X:(u,f)\in A\rbrace$ \textbf{inverser Operator}
\item $I:=\lbrace(u,u)\in X\times X:u\in X\rbrace$ \textbf{identischer Operator}
\item Offenbar gilt $\dom(A^{-1})=\rg(A)$
\item Sind $A,B\subseteq X\times X$ zwei Operatoren, $\lambda\in\K\in\lbrace\R,\C\rbrace$, dann ist
\begin{align*}
A+B&:=\lbrace(u,f_1+f_2):f_1\in A,f_2\in B\rbrace\\
&:=\lbrace(u,f)\in X\times X:\exists f_1,f_2\in X:(u,f_1)\in A\wedge(u,f_2)\in B\wedge f=f_1+f_2\rbrace\\
\lambda\cdot A&:=\lbrace(u,\lambda\cdot f:(u,f)\in \rbrace:=\lbrace(u,f)\in X\times X:\exists f_1\in X:(u,f_1)\in X\wedge f=\lambda\cdot f_1\rbrace
\end{align*}
\end{itemize}
\end{definition}

\section{Das ``Bracket''}
Sei $(X,\Vert\cdot\Vert)$ ein Banachraum. Für alle $u,v\in X$ und alle $\lambda\in\R_{>0}$ sei
\begin{align*}
[u,v]_\lambda&:=\frac{\Vert u+\lambda\cdot v\Vert-\Vert u\Vert}{\lambda}\text{ und}\\
[u,v]&:=\inf\limits_{\lambda>0}[u,v]_\lambda.
\end{align*}

Die Abbildung $[\cdot,\cdot]:X\times X\to\R\cup\lbrace-\infty\rbrace$ heißt \textbf{Bracket}. Das Bracket $[u,v]$ ist eine Richtungsableitung der Norm $\Vert\cdot\Vert_X$ im Punkt $u$ in Richtung $v$.

\begin{lemma}[Eigenschaften des Brackets]
Seien $u,v\in X,~\mu>0$. Dann gilt:
\begin{enumerate}[label=(\roman*)]
\item $\left[\cdot,\cdot\right]\colon X\times X\to\R\cup\lbrace-\infty\rbrace$ ist \textbf{oberhalbstetig}, d.h. 
\begin{align*}
(u_n,v_u)_{n\in\N}\to(u,v)\text{ in }X\times X\Longrightarrow\left[u,v\right]\geq\limsup_{n\to\infty}\left[u_n,v_n\right]
\end{align*}

	\item Die Funktion $\left]0,\infty\right[\to\R$, $\lambda\mapsto\left[u,v\right]_\lambda$ ist monoton wachsend und beschränkt durch $\Vert v\Vert$.
	\item $\left[u,v\right]=\lim_{\lambda>0}\left[u,v\right]_\lambda$.
	\item Die Funktion $X\to\R\cup\lbrace-\infty\rbrace$, $v\mapsto\left[u,v\right]$ ist sublinear.
	\item $\left[\mu\cdot u,v\right]=\left[u,v\right]$.
	\item $\left[u,0\right]=0$.
	\item $\left[0,v\right]=\Vert v\Vert$.
	\item $\left[u,u\right]=\Vert u\Vert$.
\end{enumerate}
\end{lemma}

\begin{definition}
Eine Funktion $f\colon M\to\R\cup\lbrace+\infty\rbrace$ auf einem metrischen Raum $M$ heißt \textbf{unterhalbstetig} $:\gdw$
\begin{align*}
\forall(u_n)_{n\in\N}\subseteq M,\forall u\in M:
u_n\stackrel{n\to\infty}{\longrightarrow} u\text{ in }M
\Longrightarrow
f(u)\leq\liminf_{n\to\infty} f(u_n)
\end{align*}
$f$ heißt \textbf{oberhalbstetig}, falls $-f$ unterhalbstetig ist.
\end{definition}

\begin{lemma}
Sei $M$ ein metrischer Raum, $f\colon M\to\R\cup\lbrace+\infty\rbrace$ eine Funktion. Dann sind äquivalent: 
\begin{enumerate}[label=(\roman*)]
	\item $f$ ist unterhalbstetig.
	\item $\forall c\in\R\colon\lbrace f\leq c\rbrace:=\lbrace u\in M\mid f(u)\leq c\rbrace$ ist abgeschlossen.
	\item $\lbrace(u,\lambda)\in M\times\R\mid f(u)\leq\lambda\rbrace=:\operatorname{epi}(f)$ ist abgeschlossen.
\end{enumerate}
\end{lemma}

\begin{proof}
\underline{Zeige (i) $\Rightarrow$ (iii):}\\
Sei $\big((u_n,\lambda_n)\big)_{n\in\N}$ eine konvergente Folge in $\epi(f)\mit(u,\lambda):=\limn(u_n,\lambda_n)$ in $M\times\R$. Dann gilt:
\begin{align*}
f(u)
\stackrel{\text{(i)}}{\leq}
\liminf\limits_{n\to\infty} f(u_n)
\stackrel{f(u_n)\leq\lambda_n}{\leq}
\liminf\limits_{n\to\infty} \lambda_n
=\lambda\\
\implies
(u,\lambda)\in\epi(f)
\end{align*}

\underline{Zeige (iii) $\Rightarrow$ (ii):}\\
Sei $c\in\R$ und sei $(u_n)_{n\in\N}\subseteq\lbrace x\in M:f(x)\leq c\rbrace$ konvergente Folge mit $u:=\lim u_n$ in $M$. Dann ist
\begin{align*}
\limn\underbrace{(u_n,c)}_{=\epi(f)}\mit M\times\R.
\end{align*}
Da $\epi(f)$ abgeschlossen ist, ist $(u,c)\in\epi(f)$, d.h. $f(u)\leq c$ d.h. $u\in\lbrace f\leq c\rbrace$\\

\underline{Zeige (ii) $\Rightarrow$ (i):}\\
Sei $(u_n)_{n\in\N}\subseteq\lbrace x\in M:f(x)\leq c\rbrace$ konvergente Folge mit $u:=\lim u_n$ in $M$. Setze
\begin{align*}
c:=\liminf\limits_{n\to\infty} f(u_n)\in\R\cup\lbrace\pm\infty\rbrace.
\end{align*}
Falls $c>-\infty$, dann enthält $\lbrace f\leq c+\varepsilon\rbrace$ für $\varepsilon>0$ unendlich viele $u_n$. Weil $\lbrace f\leq c+\varepsilon\rbrace$ abgeschlossen ist, ist
\begin{align*}
u=\limn u_n\in\lbrace f\leq c+\varepsilon\rbrace\text{ d.h. } f(u)\leq c+\varepsilon.
\end{align*}
Da $\varepsilon>0$ beliebig ist, ist $f(u)\leq c=\liminf\limits\limits_{n\to\infty} f(u_n)$.\\
Falls $c=-\infty$, dann enthält $\lbrace f\leq K\rbrace\mit K\in\R$ beliebig unendlich viele $u_n$. Und weil $\lbrace f\leq K\rbrace$ abgeschlossen ist, ist $u=\limn u_n\in\lbrace f\leq K\rbrace$, d.h. $f(u)\leq K$.\\
Da $K\in\R$ beliebig ist, ist $f(u)\leq=-\infty$. Dies ist aber ein Widerspruch zur Annahme.
\end{proof}

\begin{lemma}
Sei $M$ ein metrischer Raum und sei $(f_i)_{i\in I}$ eine Familie von unterhalbstetigen Funktionen $f_i:M\to\R\cup\lbrace+\infty\rbrace,~i\in I$.\\
Dann ist das Supremum $f:=\sup\limits_{i\in I} f_i$ unterhalbstetig.
\end{lemma}
\begin{proof}
Es gilt
\begin{align*}
\epi(f)=\epi\left(\sup\limits_{i\in I} f_i\right)=\bigcap\limits_{i\in I}\epi(f_i)
\end{align*}
und beliebige Schnitte abgeschlossener Mengen sind abgeschlossen.
\end{proof}

\begin{definition}
Sei $X$ ein reeller oder komplexer Vektorraum. Eine Funktion $f:X\to\R\cup\lbrace\infty\rbrace$ heißt \textbf{konvex}
\begin{align*}
:\Longleftrightarrow
\forall x,y\in X,\forall\lambda\in[0,1]:f\big(\lambda\cdot x+(1-\lambda)\cdot y\big)
\leq\lambda\cdot f(x)+(1-\lambda)\cdot f(y) 
\end{align*}
Eine Teilmenge $C\subseteq X$ heißt \textbf{konvex}
\begin{align*}
:\Longleftrightarrow
\forall x,y\in C,\forall\lambda\in[0,1]:\lambda\cdot x+(1-\lambda)\cdot y\in C
\end{align*}
\end{definition}

\begin{lemma}
$f$ ist konvex $\Longleftrightarrow\epi(f)$ ist konvex.
\end{lemma}

\begin{lemma}
Sei $f:\R\to\R\cup\lbrace\infty\rbrace$ konvex. Dann gilt
\begin{enumerate}[label=(\alph*)]
\item Für alle $x\in\R\mit f(x)<\infty$ und für alle $y\in\R$ ist
\begin{align*}
(0,\infty)\to\R\cup\lbrace +\infty\rbrace,\qquad
\lambda\mapsto\frac{f(x+\lambda\cdot y)-f(x)}{\lambda}
\end{align*}
monoton wachsend.
\item $\forall x\in\R\mit f(x)<\infty$ existieren die Grenzwerte
\begin{align*}
\lim\limits_{\lambda\to0^+}\frac{f(x+\lambda)-f(x)}{\lambda}\in\R\cup\lbrace\pm\infty\rbrace
\text{ und }
\lim\limits_{\lambda\to0^+}\frac{f(x-\lambda)-f(x)}{-\lambda}\in\R\cup\lbrace\pm\infty\rbrace.
\end{align*}
\item 
$\begin{aligned}
\dom(f):=\lbrace r\in\R:f(r)<\infty\rbrace
\end{aligned}$ 
ist ein Intervall und $f$ ist stetig auf $\dom(f)$.
\end{enumerate}
\end{lemma}
\begin{proof}
\underline{Zeige (a):} O.B.d.A. sei $x=0,~f(x)=0$ und $y=1$.\\
Zu zeigen ist, dass $\lambda\mapsto\frac{f(\lambda)}{\lambda}$ monoton wachsend auf $(0,\infty)$ ist.\\
Sei $0<\lambda_1<\lambda_2$. Dann ist 
\begin{align*}
\lambda_1=(1-\lambda)\cdot 0+\lambda\cdot\lambda_2\text{ für }\lambda:=\frac{\lambda_1}{\lambda_2}\in[0,1]
\end{align*}
und somit
\begin{align*}
f(\lambda_1)
=
f\Big((1-\lambda)\cdot0+\lambda\cdot\lambda_2\Big)
\stackrel{f\text{ konv}}{\leq}
\underbrace{(1-\lambda)\cdot f(0)}_{=0}+\lambda\cdot f(\lambda_2)
=
\frac{\lambda_1}{\lambda_2}\cdot f(\lambda_2).
\end{align*}
Behauptung (c) folgt dann aus (b), welche aus (a) folgt.
\end{proof}

\begin{proof}[Beweis des Lemmas über die Eigenschaften des Brackets]\enter
\underline{Zeige (a):} Das Bracket ist oberhalbstetig, denn
\begin{align*}
[\cdot,\cdot]_\lambda:X\times X\to\R,~(u,v)\mapsto[u,v]_\lambda:=\frac{\Vert u+\lambda\cdot v\Vert - \Vert u\Vert}{\lambda}
\end{align*}
ist stetig (da jede Norm stetig ist) und damit auch oberhalbstetig für alle $\lambda>0$. Das Infimum von oberhalbstetiger Funktion ist wieder oberhalbstetig.\\

\underline{Zeige (b):}\\
Die Funktion $\lambda\mapsto[u,v]_\lambda$ ist monoton wachsend auf $(0,\infty)$, weil $\lambda\mapsto\Vert u+\lambda\cdot v\Vert$ konvex ist, denn jede Norm ist konvex (dies folgt aus der Dreiecksungleichung).\\

\underline{Zeige (c):}\\
$\begin{aligned}[]
[u,v]=\lim\limits_{\lambda\to0^+}[u,v]_\lambda
\end{aligned}$ folgt aus (b) und aus
\begin{align*}
[u,v]_\lambda:=\frac{\Vert u+\lambda\cdot v\Vert-\Vert u\Vert}{\lambda}
\stackrel{\Delta\text{Ungl}}{\leq}
\frac{\Vert u\Vert+\lambda\cdot\Vert v\Vert-\Vert u\Vert}{\lambda}
=\Vert v\Vert.
\end{align*}

\underline{Zeige (d):}\\
$v\mapsto[u,v]$ ist sublinear, denn
\begin{align*}
[u,\mu\cdot v]
&=
\inf\limits_{\lambda>0}[u,\mu\cdot v]_\lambda\\
&=\inf\limits_{\lambda>0}\frac{\Vert u+\lambda\cdot\mu\cdot v\Vert-\Vert u\Vert}{\lambda}\cdot\frac{\mu}{\mu}\\
&\stackeq{Def}
\inf\limits_{\lambda>0}\mu\cdot[u,v]_{\lambda\cdot\mu}\\
&=\mu\cdot[u,v]
\end{align*}
und für alle $\mu\in(0,1)$ gilt
\begin{align*}
[u,v_1+v_2]
&=
\inf\limits_{\lambda>0}\frac{\Vert u+\lambda\cdot(v_1+v_2)\Vert-\Vert u\Vert}{\lambda}\\
&\stackrel{\mu\in(0,1)}{\leq}
\underbrace{\inf\limits_{\lambda>0}}_{=\lim\limits_{\lambda\to0}}
\frac{\Vert\mu\cdot u+\lambda\cdot v_1\Vert+\Vert(1-\mu)\cdot u+\lambda\cdot v_2\Vert-\mu\cdot\Vert u\Vert-(1-\mu)\cdot\Vert u\Vert}{\lambda}\\
&=\lim\limits_{\lambda\to0}\frac{\left\Vert u+\frac{\lambda}{\mu}\cdot v_1\right\Vert-\Vert u\Vert}{\frac{\lambda}{\mu}}+\frac{\left\Vert u+\frac{\lambda}{1-\mu}\cdot v_2\right\Vert-\Vert u\Vert}{\frac{\lambda}{1-\mu}}\\
&=[u,v_1]+[u,v_2]
\end{align*}

\underline{Zeige (e):}\\
Es gilt $[\mu\cdot u,v]=[u,v]$, denn
\begin{align*}
[\mu\cdot u,v]
&=
\inf\limits_{\lambda>0}\frac{\Vert\mu\cdot u+\lambda\cdot v\Vert-\Vert\mu\cdot u\Vert}{\lambda}\\
&=\inf\limits_{\lambda>0}\frac{\left\Vert u+\frac{\lambda}{\mu}\cdot v\right\Vert-\Vert u\Vert}{\frac{\lambda}{\mu}}\\
&=
[u,v]
\end{align*}

\underline{Zeige (f):}
\begin{align*}
[u,0]
&=
\inf\limits_{\lambda>0}\frac{\Vert u+\lambda\cdot0\Vert-\Vert u\Vert}{\lambda}
=0
\end{align*}

\underline{Zeige (g):}
\begin{align*}
[0,v]
&=
\inf\limits_{\lambda>0}\frac{\Vert 0+\lambda\cdot v\Vert-\Vert 0\Vert}{\lambda}
=\Vert v\Vert
\end{align*}

\underline{Zeige (h):}
\begin{align*}
[u,u]
&=
\inf\limits_{\lambda>0}\frac{\Vert u+\lambda\cdot u\Vert-\Vert u\Vert}{\lambda}
=
\inf\limits_{\lambda>0}\frac{(1+\lambda)\cdot\Vert u\Vert-\Vert u\Vert}{\lambda}
=\Vert u\Vert
\end{align*}
\end{proof}

\begin{bemerkung}
Falls $X=H$ ein Hilbertraum mit Skalarprodukt $\langle\cdot,\cdot\rangle$ ist, dann ist
\begin{align*}
[u,v] 
&=
\lim\limits_{\lambda\to0^+}\frac{\sqrt{\langle u+\lambda\cdot v,u+\lambda\cdot v\rangle}-\sqrt{\langle u,u\rangle}}{\lambda}\\
&=\lim\limits_{\lambda\to0^+}\frac{\sqrt{\langle u,u\rangle+2\cdot\lambda\cdot\Re(\langle u,v\rangle)+\lambda^2\cdot\langle v,v\rangle}-\sqrt{\langle u,u\rangle}}{\lambda}\\
&\stackeq{u\neq0}
\frac{1}{2\cdot\Vert u\Vert}\cdot 2\cdot\langle u,v\rangle\\
&=\left\langle\frac{u}{\Vert u\Vert},v\right\rangle
\end{align*}
\end{bemerkung}

\begin{lemma}
Sei $(X,\Vert\cdot\Vert)$ ein Banachraum, $I\subseteq\R$ ein Intervall und sei $u:I\to X$ eine Funktion. Dann gilt:
\begin{enumerate}[label=(\alph*)]
\item Wenn die \textbf{rechtsseitige Ableitung} von $u$, 
\begin{align*}
D_t^R u(t):=\dot{u}(t+):=\lim\limits_{h\to 0^+}\frac{u(t+h)-u(t)}{h},
\end{align*}
existiert, dann existiert die rechtsseitige Ableitung von $\Vert\cdot\Vert\circ u$, also
\begin{align*}
D_t^R \Vert u(t)\Vert:=\dot{u}(t+):=\lim\limits_{h\to 0^+}\frac{\Vert u(t+h)\Vert-\Vert u(t)\Vert}{h}
\end{align*}
und es gilt
\begin{align*}
D_t^R\Vert u(t)\Vert=\left[u(t),D_t^R u(t)\right].
\end{align*}
\item Falls die \textbf{linksseitige Ableitung} von $u$,
\begin{align*}
D_t^L u(t):=\lim\limits_{h\to 0^+}\frac{u(t-h)-u(t)}{-h},
\end{align*}
existiert, dann existiert
\begin{align*}
D_t^L \Vert u(t)\Vert:=\lim\limits_{h\to 0^+}\frac{\Vert u(t-h)\Vert-\Vert u(t)\Vert}{-h}
\end{align*}
und es gilt
\begin{align*}
D_t^R\Vert u(t)\Vert=-\left[u(t),-D_t^R u(t)\right].
\end{align*}
\item Falls die herkömmliche Ableitung von $u$
\begin{align*}
\lim\limits_{h\to0}\frac{u(t+h)-u(t)}{h}=:\dot{u}(t)
\end{align*}
existiert, dann existiert
\begin{align*}
\lim\limits_{h\to0}\frac{\Vert u(t+h)\Vert-\Vert u\Vert}{h}=:D_t\Vert u(t)\Vert
\end{align*}
und es gilt
\begin{align*}
D_t\Vert u(t)\Vert_\lambda=\left[ u(t),\dot{u}(t)\right]=-\big[u(t),-\dot{u}(t)\big]
\end{align*}
\end{enumerate}
\end{lemma}
\begin{proof}
\underline{Zeige (a):}\\
Für $h>0$ gilt die \textit{Weierstraß'sche Zerlegungsformel}
\begin{align*}
u(t+h)=u(t)+h\cdot D_t^R u(t)+o(h)\mit\lim\limits_{h\to0^+}\frac{o(h)}{h}=0
\end{align*}
und somit
\begin{align*}
\frac{\Vert u(t+h)\Vert-\Vert u(t)\Vert}{h}
&=
\frac{\Vert u(t)+h\cdot D_t^R u(t)+ o(h)\Vert-\Vert u(t)\Vert}{h}\\
&\stackeq{\geq\&\leq\mit\Delta\text{-Ungl}}
\frac{\Vert u(t)+h\cdot D_t^R u(t)\Vert-\Vert u(t)\Vert}{h}+\frac{o(h)}{h}\\
&\stackrel{h\to0^+}{\longrightarrow} \left[ u(t), D_t^R u(t)\right]
\end{align*}
Hierbei wird die Dreiecksungleichung und die umgekehrte Dreiecksungleichung benutzt, um in beide Richtung abzuschätzen und Gleichheit zu erzielen.

\underline{Zeige (b):}\\
Analog zu (a) mit $h<0$.\\

\underline{Zeige (c):}\\
Folgt direkt aus (a) und (b).
\end{proof}

\section{Akkretive Operatoren}
\begin{definition}
Sei $(X,\Vert\cdot\Vert)$ ein Banachraum. Ein Operator $A\subseteq X\times X$ heißt \textbf{akkretiv vom Typ $\omega\in\R$}
\begin{align*}
:\Longleftrightarrow\forall (u,v),(\hat{u},\hat{v})\in A:\left[ u-\hat{u},v-\hat{v}\right]+\omega\cdot\left\Vert u-\hat{u}\right\Vert\geq0
\end{align*}
Ein \textbf{akkretiver} Operator  ist ein akkretiver Operator vom Typ $0$.\\
Ein akkretiver Operator vom Typ $\omega$ ist auch ein akkretiver Operator vom Typ $\omega'$ für alle $\omega'\geq\omega$.
\end{definition}

\begin{lemma}
Sei $X$ Banachraum. Für einen Operator $A\subseteq X\times X$ und $\omega\in\R$ sind folgende Aussagen äquivalent:
\begin{enumerate}[label=(\roman*)]
\item $A$ ist akkretiv vom Typ $\omega$
\item $A+\omega\cdot I$ ist akkretiv
\item $\begin{aligned}\forall (u,v),(\hat{u},\hat{v})\in A,\forall\lambda>0:\Vert u-\hat{u}+\lambda\cdot(v-\hat{v})\Vert\geq(1-\lambda\cdot\omega)\cdot\Vert u-\hat{u}\Vert
\end{aligned}$
\end{enumerate}
\end{lemma}
\begin{proof}
\underline{Zeige (i) $\gdw$ (iii):}\\
$A$ ist akkretiv vom Typ $\omega$
\begin{align*}
&\gdw\underbrace{[u-\hat{u},v-\hat{v}]}_{=\int\limits_{\lambda>0}[\ldots]_\lambda}
+ \omega\cdot\Vert u-\hat{u}\Vert\geq0~\forall(u,v),(\hat{u},\hat{v})\in A\\
&\gdw\frac{u-\hat{u}+\lambda\cdot(v-\hat{v}\Vert-\Vert u-\hat{u}\Vert}{\lambda}+\omega\cdot\Vert u-\hat{u}\Vert\geq0~\forall(u,v),(\hat{u},\hat{v})\in A,\forall\lambda>0\\
&\gdw\text{(iii)}
\end{align*}
\underline{Zeige (iii) $\gdw$ (ii):}
\begin{align*}
\text{(iii)} &\Rightarrow\forall(u,v),(\hat{u},\hat{v})\in A,\forall\lambda>0\text{ klein, d. h. }1+\lambda\cdot\omega>0:\\
&\qquad(1+\lambda\cdot\omega)\cdot\Vert u-\hat{u}+\lambda\cdot(v-\hat{v})\Vert\geq(1-\lambda\cdot\omega)\cdot\Vert u-\hat{u}\Vert\cdot(1+\lambda\cdot\omega)\\
&\gdw\forall\ldots:\Big\Vert u-\hat{u}+\lambda\cdot\big((1+\lambda\cdot\omega)\cdot v-\hat{v}+\omega\cdot(u-\hat{u})\big)\Big\Vert
\geq
\left(1-\lambda^2\cdot\omega^2\right)\cdot\Vert u-\hat{u}\Vert\\
&\gdw\forall\ldots:
\Big\Vert u-\hat{u}+\lambda\cdot\big(v-\hat{v}+\omega\cdot(u-\hat{u})\big)+\lambda^2\cdot\omega\cdot(v-\hat{v})\Big\Vert
\geq
\left(1-\lambda^2\cdot\omega^2\right)\cdot\Vert u-\hat{u}\Vert\\
&\gdw\forall\ldots:
\frac{\Big\Vert u-\hat{u}+\lambda\cdot\big(v-\hat{v}+\omega\cdot(u-\hat{u})\big)\Big\Vert-\Vert u-\hat{u}\Vert}{\lambda}+\mathcal{O}(\lambda)\geq0\\
&\stackrel{\lambda\to0}{\Rightarrow}
\forall(u,v),(\hat{u},\hat{v})\in A:\Big[ u-\hat{u},\underbrace{v}_{=Au}-\hat{v}+\omega\cdot(u-\hat{u})\Big]\geq0\\
&\gdw A+\omega\cdot I\text{ ist akkretiv}
\end{align*}
Das die Rückrichtung auch gilt muss man sich überlegen.
\end{proof}

\begin{lemma}
Sei $A\subseteq X\times X$ akkretiv vom Typ $\omega\in\R$.\\
Dann ist der Abschluss $\overline{A}$ akkretiv vom Typ $\omega$.
\end{lemma}
\begin{proof}
Seien $(u,v),(\hat{u},\hat{v})\in\overline{A}$. Dann existieren Folgen 
\begin{align*}
&(u_n,v_n)_{n\in\N},(\hat{u}_n,\hat{v}_n)_{n\in\N}\subseteq A\mit\\
&(u_n,v_n)
\stackrel{n\to\infty}{\longrightarrow}
(u,v)\text{ in } X\times X\\
&(\hat{u}_n,\hat{v}_n)
\stackrel{n\to\infty}{\longrightarrow}
(\hat{u},\hat{v})\text{ in } X\times X
\end{align*}
und wegen der Oberhalbstetigkeit des Brackets gilt
\begin{align*}
[u-\hat{u},v-\hat{v}]+\omega\cdot\Vert u-\hat{u}\Vert
\stackrel{\text{oberhalb stetig}}{\geq}
\limsup\limits_{n\to\infty}\Big(\underbrace{\big[ u_n-\hat{u}_n,v_n-\hat{v}_n\big]+\omega\cdot\Vert u_n-\hat{u}_n\Vert}_{\geq0}\Big)
\geq0
\end{align*}
\end{proof}

\begin{beispiel}\
\begin{enumerate}[label=(\alph*)]
\item In einem Hilbertraum $H$ ist ein Operator $A\subseteq H\times H$ akkretiv vom Typ $\omega\in\R$
\begin{align*}
\Longleftrightarrow\forall(u,v),(\hat{u},\hat{v})\in A:
 \Re\big(\langle u-\hat{u},v-\hat{v}\rangle\big)+\omega\cdot\Vert u-\hat{u}\Vert^2\geq0
\end{align*}
Operatoren $A\subseteq H\times H$, die diese Bedingung erfüllen, heißen auch \textbf{monoton vom Typ $\omega$} bzw. einfach nur \textbf{monoton}, falls $\omega=0$.\\
Es gilt also: $A$ akkretiv vom Typ $\omega\gdw A$ monoton vom Typ $\omega$.\\

Falls $A$ ein linearer (einwertiger) Operator ist, dann ist $A$ akkretiv
\begin{align*}
&\Longleftrightarrow\forall u\in\dom(A):\Re\big(\langle u, Au\rangle\big)\geq0\\
&\Longleftrightarrow -A\text{ ist \textbf{dissipativ}}
\end{align*}
\item $H=L^2(\Omega)\mit\Omega\subseteq\R^n$ offen und 
\begin{align*}
A&=\big\lbrace(u,v)\in L^2\times L^2:u,v\in C_c^\infty(\Omega):v=\Delta u\big\rbrace\\
C_c^\infty(\Omega)&:=\big\lbrace u\in C^\infty(\Omega):\supp(u)\text{ kompakt}\big\rbrace\\
\supp(u)&:=\overline{\big\lbrace x\in\Omega:u(x)\neq0\big\rbrace}^\Omega\text{ Abschluss in }\Omega\\
\Delta u&:=\sum\limits_{i=1}^n\frac{\partial^2 u}{\partial x_i^2}
\end{align*}
Dann gilt für alle $u\in\dom(A)$:
\begin{align*}
\Re\Big(\langle u,-\Delta u\rangle_{L^2}\Big) 
&= \Re\left(-\int\limits_\Omega u\cdot\overline{\Delta u}\right)\\
&\stackeq{\text{part Int}}
\Re\left(\int\limits_{\Omega} \nabla u\cdot\overline{\nabla u}\right)\\
%&=\int\limits_\Omega |\nabla u|^2\\
&\geq0
\end{align*}
Aus dieser partiellen Integration folgt die Akkretivität von $A$. $A$ ist hierbei der negative Laplace-Operator auf den Testfunktionen.
\item Sei $A\subseteq X\times X$ akkretiv vom Typ $\omega\in\R$ und $F:X\to X$ Lipschitzstetig mit Lipschitzkonstante $L\geq0$.\\
Dann ist $A+F$ akkretiv vom Typ $\omega+L$.
\begin{proof}
\begin{align*}
&\forall (u,v),(\hat{u},\hat{v})\in A,\forall\lambda>0:
\big[u-\hat{u},v-\hat{v}+F(u)-F(\hat{u})\big]_\lambda
+(\omega+L)\cdot\Vert u-\hat{u}\Vert\\
&=\frac{\Big\Vert u-\hat{u}+\lambda\cdot\big(v-\hat{v}+F(u)-F(\hat{u})\big)\Big\Vert-\Vert u-\hat{u}\Vert}{\lambda}
+(\omega+L)\cdot\Vert u-\hat{u}\Vert\\
&\stackrel{\Delta\text{-Ungl}}{\geq}
\frac{\big\Vert u-\hat{u}+\lambda\cdot(v-\hat{v})\big\Vert-\Vert u-\hat{u}\Vert}{\lambda}\underbrace{-\big\Vert F(u)-F(\hat{u})\big\Vert}_{\geq -L\cdot\Vert u-\hat{u}\Vert}
+(\omega+L)\cdot\Vert u-\hat{u}\Vert\\
&\geq0
\end{align*}
\end{proof}
\item Sei wieder $\Omega\subseteq\R^n$ offen und sei $f:\R\to\R$ lipschitzstetig mit $f(0)=0$ oder das Maß von $\Omega$ endlich, sei $H=L^2(\Omega)$ der reelle Hilbertraum und
\begin{align*}
A:L^2(\Omega)\supseteq C_c^\infty(\Omega)\to L^2(\Omega),\qquad
u\mapsto Au:=-\Delta u+\underbrace{f(u)}_{f\circ u}.
\end{align*}
Dann ist $A$ akkretiv vom Typ $L_f$, wobei $L_f$ die Lipschitzkonstante von $f$ ist.
\begin{proof}
Die Abbildung
\begin{align*}
F: L^2(\Omega)\to L^2(\Omega),\qquad u\mapsto f(u)
\end{align*}
ist wohldefiniert und Lipschitzstetig.\\
Zur Lipschitzstetigkeit: Es gilt für alle $u,\hat{u}\in L^2(\Omega)$:
\begin{align*}
\big\Vert F(u)-F(\hat{u})\big\Vert^2_{L^2}
&=\int\limits_\Omega\Big|f\big(u(x)\big)-f\big(\hat{u}(x)\big)\Big|^2\d x\\
&\leq
\int\limits_\Omega L^2\cdot\big|u(x)-\hat{u}(x)\big|^2\d x\\
&=L^2\cdot\Vert u-\hat{u}\Vert^2_{L^2}
\end{align*}
Zur Wohldefiniertheit: Für alle $u\in L^2(\Omega)$ gilt:
\begin{align*}
\left(\int\limits_\Omega\Big|f\big(u(x)\big)\Big|^2\d x\right)^{\frac{1}{2}}
&=\left(\int\limits_\Omega\Big|f\big(u(x)\big)-f(0)+f(0)\Big|^2\d x\right)^{\frac{1}{2}}\\
&\leq
\sqrt{\int\limits_\Omega\Big|f\big(u(x)\big)-f(0)\Big|^2\d x}+|f(0)|\cdot\underbrace{|\Omega|}_{\text{Maß von }\Omega}\\
&\leq
L\cdot\Vert u\Vert_{L^2}+|f(0)|\cdot|\Omega|<\infty
\end{align*}
\end{proof}
\item Sei $F:X\to X$ lipschitzstetig mit Lipschitzkonstante $L\geq0$.\\
Dann ist 
\begin{align*}
A=L\cdot I-F
\end{align*}
akkretiv (vom Typ $0$).
\begin{proof}
\begin{align*}
\forall u,\hat{u}\in X,\forall \lambda>0:
&\bigg\Vert u-\hat{u}+\lambda\cdot\Big(L\cdot(u-\hat{u})-\big(F(u)-F(\hat{u})\big)\Big)\bigg\Vert-\Vert u-\hat{u}\Vert\\
&\geq
(1+\lambda\cdot L)\cdot\Vert u-\hat{u}\Vert-\lambda\cdot\Vert F(u)-F(\hat{u})\Vert-\Vert u-\hat{u}\Vert\\
&\geq
\lambda\cdot L\cdot\Vert u-\hat{u}\Vert-\lambda\cdot L\cdot\Vert u-\hat{u}\Vert\\
&=0
\end{align*}
\end{proof}
\end{enumerate}
\end{beispiel}

\begin{theorem}[Abschätzungen für akkretive Operatoren]\enter
Sei $A\subseteq X\times X$ akkretiv vom Typ $\omega\in\Omega$ auf einem Banachraum $(X,\Vert\cdot\Vert)$. Dann gilt:
\begin{enumerate}[label=(\alph*)]
\item Seien $h,\hat{h}>0$ so, dass $h\cdot\omega,\hat{h}\cdot\omega<1$ und seien $(u,f),(\hat{u},\hat{f})\in X\times X$ so, dass 
\begin{align*}
u+h\cdot Au\ni f\qquad\text{und}\qquad\hat{u}+\hat{h}\cdot A\hat{u}\ni\hat{f}
\end{align*}.
Dann gilt:
\begin{align*}
\left(a-\frac{h\cdot\hat{h}}{h+\hat{h}}\cdot\omega\right)\cdot\Vert u-\hat{u}\Vert
\leq
\left\Vert u-\hat{u}+\frac{\hat{h}}{h+\hat{h}}\cdot(f-u)-\frac{h}{h+\hat{h}}\cdot\big(\hat{f}-\hat{u}\big)\right\Vert
\end{align*}
Insbesondere gilt für $h=\hat{h}$ dann
\begin{align*}
\Vert u-\hat{u}\Vert
\leq
\frac{1}{1-h\cdot\omega}\cdot\Vert f-\hat{f}\Vert.
\end{align*}
\item Sei $h>0\mit h\cdot \omega<1$. Für alle $f\in X$ besitzt die Inklusion
\begin{align*}
u+h\cdot Au\ni f
\end{align*}
höchstens eine Lösung $u\in\dom(A)$.
\item Sei $h>0\mit h\cdot\omega<$ und sei $(u,f)\in X\times X$ so, dass 
\begin{align*}
u+h\cdot Au\ni f.
\end{align*}
Dann gilt:
\begin{align*}
\Vert u-\hat{u}\Vert
\leq
\frac{1}{1-h\cdot\omega}\cdot\big\Vert f-h\cdot\hat{f}-\hat{u}\big\Vert
\qquad
\forall(\hat{u},\hat{f})\in A
\end{align*}
Insbesondere, wenn $f\in\dom(A)$, dann ist
\begin{align*}
\Vert u-f\Vert
&\leq
\frac{h}{1-h\cdot\omega}\cdot\Vert Af\Vert
\qquad
\text{ wobei }
\qquad
\Vert Au\Vert:=\inf\big\lbrace\Vert f\Vert:(u,f)\in A\big\rbrace
\end{align*}
\end{enumerate}
\end{theorem}

\begin{bemerkung}
Beachte
\begin{align*}
u+h\cdot Au\ni f\Longleftrightarrow\left(u,\frac{f-u}{h}\right)\in A
\end{align*}
\end{bemerkung}

\begin{proof}
\underline{Zeige (a):} Es ist
\begin{align*}
&\left[u-\hat{u},\frac{f-u}{h}-\frac{\hat{f}-\hat{u}}{\hat{h}}\right]_{\frac{h\cdot\hat{h}}{h+\hat{h}}}+\omega\cdot\Vert u-\hat{u}\Vert\geq0\\
&\gdw
\left\Vert u-\hat{u}+\frac{\hat{h}}{h+\hat{h}}\cdot(f-u)-\frac{h}{h+\hat{h}}\cdot(\hat{f}-\hat{u})\right\Vert
-\Vert u-\hat{u}\Vert+\frac{h\cdot\hat{h}}{h+\hat{h}}\cdot\omega\cdot\Vert u-\hat{u}\Vert>0\\
&\gdw\text{ Behauptung}
\end{align*}
Falls $h=\hat{h}$, dann ist
\begin{align*}
\left(\frac{1}{2}-\frac{h\cdot\omega}{2}\right)\cdot\Vert u-\hat{u}\Vert
&\leq
\Big\Vert u-\hat{u}+\frac{1}{2}\cdot(f-u)-\frac{1}{2}\cdot(\hat{f}-\hat{u})\Big\Vert\\
&=\left\Vert\frac{1}{2}\cdot(u-\hat{u})+\frac{1}{2}\cdot(f-\hat{f})\right\Vert\\
&\leq
\frac{1}{2}\cdot\Vert f-\hat{f}\Vert
\end{align*}
\underline{Zeige (b):} Dies folgt aus (a), denn:\\
Falls $u,\hat{u}\in\dom(A)$ Lösungen von
\begin{align*}
u+h\cdot Au\ni f,\qquad\hat{u}+h\cdot A\hat{u}\ni f
\end{align*}
sind, dann ist
\begin{align*}
\Vert u-\hat{u}\Vert\leq\frac{1}{1-h\cdot\omega}\cdot\Vert f-f\Vert=0
\end{align*}
\underline{Zeige (c):} Seien $h>0\mit h\cdot\omega<1$ und $8u,f)\in X\times X$ so, dass
\begin{align*}
u+h\cdot Au\ni f.
\end{align*}
sei $(\hat{u},\hat{f})\in A$, also äquivalent $A\hat{u}\ni\hat{f}$. Nach Multiplikation beider Seiten mit $h$ und Addition beider Seiten mit $\hat{u}$ erhält man
\begin{align*}
h\cdot A\hat{u}+\hat{u}\ni h\cdot\hat{f}+\hat{u}.
\end{align*}
Dann folgt aus (a)
\begin{align*}
\Vert u-\hat{u}\Vert\leq\frac{1}{1-h\cdot\omega}\cdot\big\Vert f-h\cdot\hat{f}-\hat{u}\big\Vert
\end{align*}
Falls $f\in\dom(A)$, dann gilt für alle $g\in Af$ (ersetze $(\hat{u},\hat{f})$ durch $(f,g)$):
\begin{align*}
\Vert u- f\Vert
\leq
\frac{1}{1-h\cdot\omega}\cdot\big\Vert f-h\cdot g-f\big\Vert
=\frac{h}{1-h\cdot\omega}\cdot\Vert g\Vert
\end{align*}
Nehme $\inf$ über $g\in Af$:
\begin{align*}
\Vert u-f\Vert\leq\frac{h}{1-h\cdot\omega}\cdot\Vert Af\Vert
\end{align*}
\end{proof}

\begin{theorem}
Sei $A\subseteq X\times X$ akkretiv vom Typ $\omega\in\R$ auf einem Banachraum $(X,\Vert\cdot\Vert)$. Dann sind folgende Aussagen äquivalent:
\begin{enumerate}[label=(\roman*)]
\item $\begin{aligned}
\exists h>0\mit h\cdot\omega<1:I+h\cdot A
\end{aligned}$ surjektiv
\item $\begin{aligned}
\forall h>0\mit h\cdot\omega<1: I+h\cdot A
\end{aligned}$ surjektiv
\end{enumerate}
Falls $A$ zusätzlich abgeschlossen ist, dann sind (i) und (ii) äquivalent zu
\begin{enumerate}[label=(iii)]
\item $\begin{aligned}
\exists h>0\mit h\cdot\omega<1:
\end{aligned}$ das Bild von $I+h\cdot A$ dicht in $X$ ist.
\end{enumerate}
\end{theorem}
\begin{proof}
\underline{Zeige (i) $\Rightarrow$ (ii):}\\
Sei $h>0\mit h\cdot\omega<1$ so, dass $I+h\cdot A$ surjektiv ist. Nach dem vorherigen Theorem, Aussage (b), ist $I+h\cdot A$ immer injektiv. Setze
\begin{align*}
J_h:=\big(I+h\cdot A\big)^{-1}.
\end{align*}
Dann ist wegen Theorem, Aussage (a), $J_h$ lipschitzstetig mit Lipschitzkonstante $\frac{1}{1-h\cdot\omega}$. Sei nun 
\begin{align*}
\hat{h}>\frac{h}{2-h\cdot\omega}\mit\hat{h}\cdot\omega<1
\end{align*}
und sei $f\in X$. Betrachte den Operator
\begin{align*}
T:X\to X,\qquad Tv:=T(v)
:=\frac{h}{\hat{h}}\cdot f+\frac{\hat{h}-h}{\hat{h}}\cdot J_h\cdot v\qquad\forall v\in X
\end{align*}
Dann gilt für alle $v,\hat{v}\in X$:
\begin{align*}
\Vert T(v)-T(\hat{v})\Vert
&\leq
\left|\frac{\hat{h}-h}{\hat{h}}\right|\cdot\big\Vert J_h v-J_h \hat{v}\big\Vert\\
&\leq
\underbrace{\frac{1}{1-h\cdot\omega}\cdot\frac{|\hat{h}-h|}{|\hat{h}|}}_{<1}\cdot\Vert v-\hat{v}\Vert
\end{align*}

\end{proof}