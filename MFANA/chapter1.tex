\chapter{Akkretive Operatoren}
Im Folgenden sei $X$ ein Banachraum mit Norm $\Vert\cdot\Vert$ und $H$ ein Hilbertraum mit Skalarprodukt $\langle\cdot,\cdot\rangle$.

\begin{definition}
Ein \textbf{(nichtlinearer) Operator} auf $X$ ist eine Relation $A\subseteq X\times X$. Wir schreiben
\begin{itemize}
\item $Au:=\lbrace f\in X:(u,f)\in A\rbrace~\forall u\in X$
\item $\dom(A):=\lbrace u\in X:Au\neq\emptyset\rbrace$ \textbf{Definitionsbereich} von $A$
\item $\rg(A):=\lbrace f\in X:\exists u\in X:(u,f)\in A\rbrace$ \textbf{Bild} von $A$
\item $A^{-1}:=\lbrace (f,u)\in X\times X:(u,f)\in A\rbrace$ \textbf{inverser Operator}
\item $I:=\lbrace(u,u)\in X\times X:u\in X\rbrace$ \textbf{identischer Operator}
\item Offenbar gilt $\dom(A^{-1})=\rg(A)$
\item Sind $A,B\subseteq X\times X$ zwei Operatoren, $\lambda\in\K\in\lbrace\R,\C\rbrace$, dann ist
\begin{align*}
A+B&:=\lbrace(u,f_1+f_2):f_1\in A,f_2\in B\rbrace\\
&:=\lbrace(u,f)\in X\times X:\exists f_1,f_2\in X:(u,f_1)\in A\wedge(u,f_2)\in B\wedge f=f_1+f_2\rbrace\\
\lambda\cdot A&:=\lbrace(u,\lambda\cdot f:(u,f)\in \rbrace:=\lbrace(u,f)\in X\times X:\exists f_1\in X:(u,f_1)\in X\wedge f=\lambda\cdot f_1\rbrace
\end{align*}
\end{itemize}
\end{definition}

\section{Das ``Bracket''}
Sei $(X,\Vert\cdot\Vert)$ ein Banachraum. Für alle $u,v\in X$ und alle $\lambda\in\R_{>0}$ sei
\begin{align*}
[u,v]_\lambda&:=\frac{\Vert u+\lambda\cdot v\Vert-\Vert u\Vert}{\lambda}\text{ und}\\
[u,v]&:=\inf\limits_{\lambda>0}[u,v]_\lambda.
\end{align*}
Die Abbildung $[\cdot,\cdot]:X\times X\to\R\cup\lbrace-\infty\rbrace$ heißt \textbf{Bracket}. Das Bracket $[u,v]$ ist eine Richtungsableitung der Norm $\Vert\cdot\Vert_X$ im Punkt $u$ in Richtung $v$.

\begin{lemma}[Eigenschaften des Brackets]
Seien $u,v\in X,~\mu>0$. Dann gilt:
\begin{enumerate}[label=(\roman*)]
\item $\left[\cdot,\cdot\right]\colon X\times X\to\R\cup\lbrace-\infty\rbrace$ ist \textbf{oberhalbstetig}, d.h. 
\begin{align*}
(u_n,v_u)_{n\in\N}\to(u,v)\text{ in }X\times X\Longrightarrow\left[u,v\right]\geq\limsup_{n\to\infty}\left[u_n,v_n\right]
\end{align*}

	\item Die Funktion $\left]0,\infty\right[\to\R$, $\lambda\mapsto\left[u,v\right]_\lambda$ ist monoton wachsend und beschränkt durch $\Vert v\Vert$.
	\item $\left[u,v\right]=\lim_{\lambda>0}\left[u,v\right]_\lambda$.
	\item Die Funktion $X\to\R\cup\lbrace-\infty\rbrace$, $v\mapsto\left[u,v\right]$ ist sublinear.
	\item $\left[\mu\cdot u,v\right]=\left[u,v\right]$.
	\item $\left[u,0\right]=0$.
	\item $\left[0,v\right]=\Vert v\Vert$.
	\item $\left[u,u\right]=\Vert u\Vert$.
\end{enumerate}
\end{lemma}

\begin{definition}
Eine Funktion $f\colon M\to\R\cup\lbrace+\infty\rbrace$ auf einem metrischen Raum $M$ heißt \textbf{unterhalbstetig} $:\gdw$
\begin{align*}
\forall(u_n)_{n\in\N}\subseteq M,\forall u\in M:
u_n\stackrel{n\to\infty}{\longrightarrow} u\text{ in }M
\Longrightarrow
f(u)\leq\liminf_{n\to\infty} f(u_n)
\end{align*}
$f$ heißt \textbf{oberhalbstetig}, falls $-f$ unterhalbstetig ist.
\end{definition}

\begin{lemma}
Sei $M$ ein metrischer Raum, $f\colon M\to\R\cup\lbrace+\infty\rbrace$ eine Funktion. Dann sind äquivalent: \begin{enumerate}[label=(\roman*)]
	\item $f$ ist unterhalbstetig.
	\item $\forall c\in\R\colon\lbrace f\leq c\rbrace:=\lbrace u\in M\mid f(u)\leq c\rbrace$ ist abgeschlossen.
	\item $\lbrace(u,\lambda)\in M\times\R\mid f(u)\leq\lambda\rbrace=:\operatorname{epi}(f)$ ist abgeschlossen.
\end{enumerate}
\end{lemma}

\begin{proof}
\underline{Zeige (i) $\Rightarrow$ (iii):}\\
Sei $c\in\R$ und sei $(u_n)_{n\in\N}\subseteq\lbrace x\in M:f(x)\leq c\rbrace$ konvergente Folge mit $u:=\lim u_n$ in $M$. Dann ist
\begin{align*}
\limn\underbrace{(u_n,c)}_{=\epi(f)}\mit M\times\R.
\end{align*}
Da $\epi(f)$ abgeschlossen ist, ist $(u,c)\in\epi(f)$, d.h. $f(u)\leq c$ d.h. $u\in\lbrace f\leq c\rbrace$\\

\underline{Zeige (ii) $\Rightarrow$ (i):}\\
Sei $(u_n)_{n\in\N}\subseteq\lbrace x\in M:f(x)\leq c\rbrace$ konvergente Folge mit $u:=\lim u_n$ in $M$. Setze
\begin{align*}
c:=\liminf\limits_{n\to\infty} f(u_n)\in\R\cup\lbrace\pm\infty\rbrace.
\end{align*}
Falls $c>-\infty$, dann enthält $\lbrace f\leq c+\varepsilon\rbrace$ für $\varepsilon>0$ unendlich viele $u_n$. Weil $\lbrace f\leq c+\varepsilon\rbrace$ abgeschlossen ist, ist
\begin{align*}
u=\limn u_n\in\lbrace f\leq c+\varepsilon\rbrace\text{ d.h. } f(u)\leq c+\varepsilon.
\end{align*}
Da $\varepsilon>0$ beliebig ist, ist $f(u)\leq c=\liminf\limits\limits_{n\to\infty} f(u_n)$.\\
Falls $c=-\infty$, dann enthält $\lbrace f\leq K\rbrace\mit K\in\R$ beliebig unendlich viele $u_n$. Und weil $\lbrace f\leq K\rbrace$ abgeschlossen ist, ist $u=\limn u_n\in\lbrace f\leq K\rbrace$, d.h. $f(u)\leq K$.\\
Da $K\in\R$ beliebig ist, ist $f(u)\leq=-\infty$. Dies ist aber ein Widerspruch zur Annahme.
\end{proof}

\begin{lemma}
Sei $M$ ein metrischer Raum und sei $(f_i)_{i\in I}$ eine Familie von unterhalbstetigen Funktionen $f_i:M\to\R\cup\lbrace+\infty\rbrace,~i\in I$.\\
Dann ist das Supremum $f:=\sup\limits_{i\in I} f_i$ unterhalbstetig.
\end{lemma}
\begin{proof}
Es gilt
\begin{align*}
\epi(f)=\epi\left(\sup\limits_{i\in I} f_i\right)=\bigcap\limits_{i\in I}\epi(f_i)
\end{align*}
und beliebige Schnitte abgeschlossener Mengen sind abgeschlossen.
\end{proof}

\begin{definition}
Sei $X$ ein reeller oder komplexer Vektorraum. Eine Funktion $f:X\to\R\cup\lbrace\infty\rbrace$ heißt \textbf{konvex}
\begin{align*}
:\Longleftrightarrow
\forall x,y\in X,\forall\lambda\in[0,1]:f\big(\lambda\cdot x+(1-\lambda)\cdot y\big)
\leq\lambda\cdot f(x)+(1-\lambda)\cdot f(y) 
\end{align*}
Eine Teilmenge $C\subseteq X$ heißt \textbf{konvex}
\begin{align*}
:\Longleftrightarrow
\forall x,y\in C,\forall\lambda\in[0,1]:\lambda\cdot x+(1-\lambda)\cdot y\in C
\end{align*}
\end{definition}

\begin{lemma}
$f$ ist konvex $\Longleftrightarrow\epi(f)$ ist konvex.
\end{lemma}

\begin{lemma}
Sei $f:\R\to\R\cup\lbrace\infty\rbrace$ konvex. Dann gilt
\begin{enumerate}[label=(\alph*)]
\item Für alle $x\in\R\mit f(x)<\infty$ und für alle $y\in\R$ ist
\begin{align*}
(0,\infty)\to\R\cup\lbrace +\infty\rbrace,\qquad
\lambda\mapsto\frac{f(x+\lambda\cdot y)-f(x)}{\lambda}
\end{align*}
monoton wachsend.
\item $\forall x\in\R\mit f(x)<\infty$ existieren die Grenzwerte
\begin{align*}
\lim\limits_{\lambda\to0^+}\frac{f(x+\lambda\cdot y)-f(x)}{\lambda}\in\R\cup\lbrace\pm\infty\rbrace
\text{ und }
\lim\limits_{\lambda\to0^+}\frac{f(x-\lambda\cdot y)-f(x)}{-\lambda}\in\R\cup\lbrace\pm\infty\rbrace.
\end{align*}
\item 
$\begin{aligned}
\dom(f):=\lbrace r\in\R:f(r)<\infty\rbrace
\end{aligned}$ 
ist ein Intervall und $f$ ist stetig auf $\dom(f)$.
\end{enumerate}
\end{lemma}
\begin{proof}
\underline{Zeige (a):} O.B.d.A. sei $x=0,~f(x)=0$ und $y=1$.\\
Zu zeigen ist, dass $\lambda\mapsto\frac{f(\lambda)}{\lambda}$ monoton wachsend auf $(0,\infty)$ ist.\\
Sei $0<\lambda_1<\lambda_2$. Dann ist 
\begin{align*}
\lambda_1=(1-\lambda)\cdot 0+\lambda\cdot\lambda_2\text{ für }\lambda:=\frac{\lambda_1}{\lambda_2}\in[0,1]
\end{align*}
und somit
\begin{align*}
f(x_1)
=
f\Big((1-\lambda)\cdot0+\lambda\cdot\lambda_2\Big)
\stackrel{f\text{ konv}}{\leq}
\underbrace{(1-\lambda)\cdot f(0)}_{=0}+\lambda\cdot f(\lambda_2)
=
\frac{\lambda_1}{\lambda_2}\cdot f(\lambda_2).
\end{align*}
Behauptung (c) folgt dann aus (b), welche aus (a) folgt.
\end{proof}

\begin{proof}[Beweis des Lemmas über die Eigenschaften des Brackets]\enter
\underline{Zeige (a):} Das Bracket ist oberhalbstetig, denn
\begin{align*}
[\cdot,\cdot]_\lambda:X\times X\to\R,~(u,v)\mapsto[u,v]_\lambda:=\frac{\Vert u+\lambda\cdot v\Vert - \Vert u\Vert}{\lambda}
\end{align*}
ist stetig (da jede Norm stetig ist) und damit auch oberhalbstetig für alle $\lambda>0$. Das Infimum von oberhalbstetiger Funktion ist wieder oberhalbstetig.\\

\underline{Zeige (b):}\\
Die Funktion $\lambda\mapsto[u,v]_\lambda$ ist monoton wachsend auf $(0,\infty)$, weil $\lambda\mapsto\Vert u+\lambda\cdot v\Vert$ konvex ist, denn jede Norm ist konvex (dies folgt aus der Dreiecksungleichung).\\

\underline{Zeige (c):}\\
$\begin{aligned}[]
[u,v]=\lim\limits_{\lambda\to0^+}[u,v]_\lambda
\end{aligned}$ folgt aus (b) und aus
\begin{align*}
[u,v]_\lambda:=\frac{\Vert u+\lambda\cdot v\Vert-\Vert u\Vert}{\lambda}
\stackrel{\Delta\text{Ungl}}{\leq}
\frac{\Vert u\Vert+\lambda\cdot\Vert v\Vert-\Vert u\Vert}{\lambda}
=\Vert v\Vert.
\end{align*}

\underline{Zeige (d):}\\
$v\mapsto[u,v]$ ist sublinear, denn
\begin{align*}
[u,\mu\cdot v]
&=
\inf\limits_{\lambda>0}[u,\mu\cdot v]_\lambda\\
&=\inf\limits_{\lambda>0}\frac{\Vert u+\lambda\cdot\mu\cdot v\Vert-\Vert u\Vert}{\lambda}\cdot\frac{\mu}{\mu}\\
&=\inf\limits_{\lambda>0}\mu\cdot[u,v]_{\lambda\cdot\mu}\\
&=\mu\cdot[u,v]
\end{align*}
und für alle $\mu\in(0,1)$ gilt
\begin{align*}
[u,v_1+v_2]
&=
\inf\limits_{\lambda>0}\frac{\Vert u+\lambda\cdot(v_1+v_2)\Vert-\Vert u\Vert}{\lambda}\cdot\frac{\mu}{\mu}\\
&\stackrel{\mu\in(0,1)}{\leq}
\underbrace{\inf\limits_{\lambda>0}}_{=\lim\limits_{\lambda\to0}}\Vert\mu\cdot u+\lambda\cdot v_1\Vert+\Vert(1-\mu)\cdot u+\lambda\cdot v_2\Vert-\mu\cdot\Vert u\Vert-(1-\mu)\cdot\Vert u\Vert\\
&=\lim\limits_{\lambda\to0}\frac{\left\Vert u+\frac{\lambda}{\mu}\cdot v_1\right\Vert-\Vert u\Vert}{\frac{\lambda}{\mu}}+\frac{\left\Vert u+\frac{\lambda}{1-\mu}\cdot v_2\right\Vert-\Vert u\Vert}{\frac{\lambda}{1-\mu}}\\
&=[u,v_1]+[u,v_2]
\end{align*}

\underline{Zeige (e):}\\
Es gilt $[\mu\cdot u,v]=[u,v]$, denn
\begin{align*}
[\mu\cdot u,v]
&=
\inf\limits_{\lambda>0}\frac{\Vert\mu\cdot u+\lambda\cdot v\Vert-\Vert\mu\cdot u\Vert}{\lambda}\\
&=\inf\limits_{\lambda>0}\frac{\left\Vert u+\frac{\lambda}{\mu}\cdot v\right\Vert-\Vert u\Vert}{\frac{\lambda}{\mu}}\\
&=
[u,v]
\end{align*}

\underline{Zeige (f):}
\begin{align*}
[u,0]
&=
\inf\limits_{\lambda>0}\frac{\Vert u+\lambda\cdot0\Vert-\Vert u\Vert}{\lambda}
=0
\end{align*}

\underline{Zeige (g):}
\begin{align*}
[0,v]
&=
\inf\limits_{\lambda>0}\frac{\Vert 0+\lambda\cdot v\Vert-\Vert 0\Vert}{\lambda}
=\Vert v\Vert
\end{align*}

\underline{Zeige (h):}
\begin{align*}
[u,u]
&=
\inf\limits_{\lambda>0}\frac{\Vert u+\lambda\cdot u\Vert-\Vert u\Vert}{\lambda}
=
\inf\limits_{\lambda>0}\frac{(1+\lambda)\cdot\Vert u\Vert-\Vert u\Vert}{\lambda}
=\Vert u\Vert
\end{align*}
\end{proof}

\begin{bemerkung}
Falls $X=H$ ein Hilbertraum mit Skalarprodukt $\langle\cdot,\cdot\rangle$ ist, dann ist
\begin{align*}
[u,v] 
&=
\lim\limits_{\lambda\to0^+}\sqrt{\langle u+\lambda\cdot v,u+\lambda\cdot v\rangle}-\sqrt{\langle u,u\rangle}\\
&=\lim\limits_{\lambda\to0^+}\frac{\sqrt{\langle u,u\rangle+2\cdot\lambda\cdot\langle u,v\rangle+\lambda^2\cdot\langle v,v\rangle}-\sqrt{\langle u,u\rangle}}{\lambda}\\
\stackeq{u\neq0}
\frac{1}{2\cdot\Vert u\Vert}\cdot 2\cdot\langle u,v\rangle\\
&=\left\langle\frac{u}{\Vert u\Vert},v\right\rangle
\end{align*}
\end{bemerkung}

\begin{lemma}
Sei $X$ ein Banachraum, $I\subseteq\R$ ein Intervall und sei $u:I\to X$ eine Funktion. Dann gilt:
\begin{enumerate}[label=(\alph*)]
\item Wenn die \textbf{rechtsseitige Ableitung} von $u$, 
\begin{align*}
D_t^R u(t):=\dot{u}(t+):=\lim\limits_{h\to 0^+}\frac{u(t+h)-u(t)}{h},
\end{align*}
existiert, dann existiert
\begin{align*}
D_t^R \Vert u(t)\Vert:=\dot{u}(t+):=\lim\limits_{h\to 0^+}\frac{\Vert u(t+h)\Vert-\Vert u(t)\Vert}{h}
\end{align*}
und es gilt
\begin{align*}
D_t^R\Vert u(t)\Vert=\left[u(t),D_t^R u(t)\right\Vert.
\end{align*}
\item
\end{enumerate}
\end{lemma}