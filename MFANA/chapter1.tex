\chapter{Akkretive Operatoren}
Im Folgenden sei $X$ ein Banachraum mit Norm $\Vert\cdot\Vert$ und $H$ ein Hilbertraum mit Skalarprodukt $\langle\cdot,\cdot\rangle$.

\begin{definition}
Ein \textbf{(nichtlinearer) Operator} auf $X$ ist eine Relation $A\subseteq X\times X$. Wir schreiben
\begin{itemize}
\item $Au:=\lbrace f\in X:(u,f)\in A\rbrace~\forall u\in X$
\item $\dom(A):=\lbrace u\in X:Au\neq\emptyset\rbrace$ \textbf{Definitionsbereich} von $A$
\item $\rg(A):=\lbrace f\in X:\exists u\in X:(u,f)\in A\rbrace$ \textbf{Bild} von $A$
\item $A^{-1}:=\lbrace (f,u)\in X\times X:(u,f)\in A\rbrace$ \textbf{inverser Operator}
\item $I:=\lbrace(u,u)\in X\times X:u\in X\rbrace$ \textbf{identischer Operator}
\item Offenbar gilt $\dom(A^{-1})=\rg(A)$
\item Sind $A,B\subseteq X\times X$ zwei Operatoren, $\lambda\in\K\in\lbrace\R,\C\rbrace$, dann ist
\begin{align*}
A+B&:=\lbrace(u,f_1+f_2):f_1\in A,f_2\in B\rbrace\\
&:=\lbrace(u,f)\in X\times X:\exists f_1,f_2\in X:(u,f_1)\in A\wedge(u,f_2)\in B\wedge f=f_1+f_2\rbrace\\
\lambda\cdot A&:=\lbrace(u,\lambda\cdot f:(u,f)\in \rbrace:=\lbrace(u,f)\in X\times X:\exists f_1\in X:(u,f_1)\in X\wedge f=\lambda\cdot f_1\rbrace
\end{align*}
\end{itemize}
\end{definition}

\section{Das ``Bracket''}
Sei $(X,\Vert\cdot\Vert)$ ein Banachraum. Für alle $u,v\in X$ und alle $\lambda\in\R_{>0}$ sei
\begin{align*}
[u,v]_\lambda&:=\frac{\Vert u+\lambda\cdot v\Vert-\Vert u\Vert}{\lambda}\text{ und}\\
[u,v]&:=\inf\limits_{\lambda>0}[u,v]_\lambda.
\end{align*}
Die Abbildung $[\cdot,\cdot]:X\times X\to\R\cup\lbrace-\infty\rbrace$ heißt \textbf{Bracket}. Das Bracket $[u,v]$ ist eine Richtungsableitung der Norm $\Vert\cdot\Vert$ im Punkt $u$ in Richtung $v$.

\begin{lemma}[Eigenschaften des Brackets]
Seien $u,v\in X,~\mu>0$. Dann gilt:
\begin{enumerate}
\item $\left[\cdot,\cdot\right]\colon X\times X\to\R\cup\lbrace-\infty\rbrace$ ist \textbf{oberhalbstetig}, d.h. 
\begin{align*}
(u_n,v_u)_{n\in\N}\to(u,v)\text{ in }X\times X\Longrightarrow\left[u,v\right]\geq\limsup_{n\to\infty}\left[u_n,v_n\right]
\end{align*}

	\item Die Funktion $\left]0,\infty\right[\to\R$, $\lambda\mapsto\left[u,v\right]_\lambda$ ist monoton wachsend und beschränkt durch $\Vert v\Vert$.
	\item $\left[u,v\right]=\lim_{\lambda>0}\left[u,v\right]_\lambda$.
	\item Die Funktion $X\to\R\cup\lbrace-\infty\rbrace$, $v\mapsto\left[u,v\right]$ ist sublinear.
	\item $\left[\mu\cdot u,v\right]=\left[u,v\right]$.
	\item $\left[u,0\right]=0$.
	\item $\left[0,v\right]=\Vert v\Vert$.
	\item $\left[u,u\right]=\Vert u\Vert$.
\end{enumerate}
\end{lemma}

\begin{definition}
Eine Funktion $f\colon M\to\R\cup\lbrace+\infty\rbrace$ auf einem metrischen Raum $M$ heißt \textbf{unterhalbstetig} $:\gdw$
\begin{align*}
\forall(u_n)_{n\in\N}\subseteq M,\forall u\in M:
u_n\stackrel{n\to\infty}{\longrightarrow} u\text{ in }M
\Longrightarrow
f(u)\leq\liminf_{n\to\infty} f(u_n)
\end{align*}
$f$ heißt \textbf{oberhalbstetig}, falls $-f$ unterhalbstetig ist.
\end{definition}

\begin{lemma}
Sei $M$ ein metrischer Raum, $f\colon M\to\R\cup\lbrace+\infty\rbrace$ eine Funktion. Dann sind äquivalent: \begin{enumerate}
	\item $f$ ist unterhalbstetig.
	\item $\forall c\in\R\colon\lbrace f\leq c\rbrace:=\lbrace u\in M\mid f(u)\leq c\rbrace$ ist abgeschlossen.
	\item $\lbrace(u,\lambda)\in M\times\R\mid f(u)\leq\lambda\rbrace=:\operatorname{epi}(f)$ ist abgeschlossen.
\end{enumerate}
\end{lemma}
