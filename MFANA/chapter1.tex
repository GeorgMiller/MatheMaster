% This work is licensed under the Creative Commons
% Attribution-NonCommercial-ShareAlike 4.0 International License. To view a copy
% of this license, visit http://creativecommons.org/licenses/by-nc-sa/4.0/ or
% send a letter to Creative Commons, PO Box 1866, Mountain View, CA 94042, USA.

\chapter{Akkretive Operatoren}
Im Folgenden sei $X$ ein Banachraum mit Norm $\Vert\cdot\Vert$ und $H$ ein Hilbertraum mit Skalarprodukt $\langle\cdot,\cdot\rangle$.

\begin{definition}
Ein \textbf{(nichtlinearer) Operator} auf $X$ ist eine Relation $A\subseteq X\times X$. Wir schreiben
\begin{itemize}
\item $Au:=\lbrace f\in X:(u,f)\in A\rbrace~\forall u\in X$
\item $\dom(A):=\lbrace u\in X:Au\neq\emptyset\rbrace$ \textbf{Definitionsbereich} von $A$
\item $\rg(A):=\lbrace f\in X:\exists u\in X:(u,f)\in A\rbrace$ \textbf{Bild} von $A$
\item $A^{-1}:=\lbrace (f,u)\in X\times X:(u,f)\in A\rbrace$ \textbf{inverser Operator}
\item $I:=\lbrace(u,u)\in X\times X:u\in X\rbrace$ \textbf{identischer Operator}
\item Offenbar gilt $\dom(A^{-1})=\rg(A)$
\item Sind $A,B\subseteq X\times X$ zwei Operatoren, $\lambda\in\K\in\lbrace\R,\C\rbrace$, dann ist
\begin{align*}
A+B&:=\lbrace(u,f_1+f_2):f_1\in A,f_2\in B\rbrace\\
&:=\lbrace(u,f)\in X\times X:\exists f_1,f_2\in X:(u,f_1)\in A\wedge(u,f_2)\in B\wedge f=f_1+f_2\rbrace\\
\lambda\cdot A&:=\lbrace(u,\lambda\cdot f:(u,f)\in \rbrace\\
&:=\lbrace(u,f)\in X\times X:\exists f_1\in X:(u,f_1)\in X\wedge f=\lambda\cdot f_1\rbrace
\end{align*}
\end{itemize}
\end{definition}

\section{Das ``Bracket''}
Sei $(X,\Vert\cdot\Vert)$ ein Banachraum. Für alle $u,v\in X$ und alle $\lambda\in\R_{>0}$ sei
\begin{align*}
[u,v]_\lambda&:=\frac{\Vert u+\lambda\cdot v\Vert-\Vert u\Vert}{\lambda}\text{ und}\\
[u,v]&:=\inf\limits_{\lambda>0}[u,v]_\lambda.
\end{align*}

Die Abbildung $[\cdot,\cdot]:X\times X\to\R\cup\lbrace-\infty\rbrace$ heißt \textbf{Bracket}. Das Bracket $[u,v]$ ist eine Richtungsableitung der Norm $\Vert\cdot\Vert_X$ im Punkt $u$ in Richtung $v$.

\begin{lemma}[Eigenschaften des Brackets]
Seien $u,v\in X,~\mu>0$. Dann gilt:
\begin{enumerate}[label=(\roman*)]
\item $\left[\cdot,\cdot\right]\colon X\times X\to\R\cup\lbrace-\infty\rbrace$ ist \textbf{oberhalbstetig}, d.h. 
\begin{align*}
(u_n,v_u)_{n\in\N}\to(u,v)\text{ in }X\times X\Longrightarrow\left[u,v\right]\geq\limsup_{n\to\infty}\left[u_n,v_n\right]
\end{align*}

	\item Die Funktion $\left]0,\infty\right[\to\R$, $\lambda\mapsto\left[u,v\right]_\lambda$ ist monoton wachsend und beschränkt durch $\Vert v\Vert$.
	\item $\left[u,v\right]=\lim\limits_{\lambda\to0}\left[u,v\right]_\lambda$.
	\item Die Funktion $X\to\R\cup\lbrace-\infty\rbrace$, $v\mapsto\left[u,v\right]$ ist sublinear.
	\item $\left[\mu\cdot u,v\right]=\left[u,v\right]$.
	\item $\left[u,0\right]=0$.
	\item $\left[0,v\right]=\Vert v\Vert$.
	\item $\left[u,u\right]=\Vert u\Vert$.
\end{enumerate}
\end{lemma}

\begin{definition}
Eine Funktion $f\colon M\to\R\cup\lbrace+\infty\rbrace$ auf einem metrischen Raum $M$ heißt \textbf{unterhalbstetig}
\begin{align*}
:\Longleftrightarrow
\forall(u_n)_{n\in\N}\subseteq M,\forall u\in M\mit
u_n\stackrel{n\to\infty}{\longrightarrow} u\text{ in }M:
f(u)\leq\liminf_{n\to\infty} f(u_n)
\end{align*}
$f$ heißt \textbf{oberhalbstetig}
\begin{align*}
&:\Longleftrightarrow -f\text{ unterhalbstetig}\\
&\Longleftrightarrow
\forall(u_n)_{n\in\N}\subseteq M,\forall u\in M\mit
u_n\stackrel{n\to\infty}{\longrightarrow} u\text{ in }M:
-f(u)\leq\liminf\limits_{n\to\infty}\big(-f(u_n)\big)\\
&\Longleftrightarrow
\forall(u_n)_{n\in\N}\subseteq M,\forall u\in M\mit
u_n\stackrel{n\to\infty}{\longrightarrow} u\text{ in }M:
f(u)\geq\limsup\limits_{n\to\infty} f(u_n)
\end{align*}
\end{definition}

\begin{lemma}
Sei $M$ ein metrischer Raum, $f\colon M\to\R\cup\lbrace+\infty\rbrace$ eine Funktion. Dann sind äquivalent: 
\begin{enumerate}[label=(\roman*)]
	\item $f$ ist unterhalbstetig.
	\item $\forall c\in\R\colon\lbrace f\leq c\rbrace:=\lbrace u\in M\mid f(u)\leq c\rbrace$ ist abgeschlossen.
	\item $\lbrace(u,\lambda)\in M\times\R\mid f(u)\leq\lambda\rbrace=:\operatorname{epi}(f)$ ist abgeschlossen.
\end{enumerate}
\end{lemma}

\begin{bemerkung}
Der Epigraph ist die Menge aller Punkte über dem Graphen.
\end{bemerkung}

\begin{proof}
\underline{Zeige (i) $\implies$ (iii):}\\
Sei $\big((u_n,\lambda_n)\big)_{n\in\N}$ eine konvergente Folge in $\epi(f)\mit(u,\lambda):=\limn(u_n,\lambda_n)$ in $M\times\R$. Dann gilt:
\begin{align*}
f(u)
\stackrel{\text{(i)}}{\leq}
\liminf\limits_{n\to\infty} f(u_n)
\stackrel{f(u_n)\leq\lambda_n}{\leq}
\liminf\limits_{n\to\infty} \lambda_n
=\lambda\\
\implies
(u,\lambda)\in\epi(f)
\end{align*}

\underline{Zeige (iii) $\implies$ (ii):}\\
Sei $c\in\R$ und sei $(u_n)_{n\in\N}\subseteq\lbrace x\in M:f(x)\leq c\rbrace$ konvergente Folge mit $u:=\lim u_n$ in $M$. 
Dann ist
\begin{align*}
	\limn\underbrace{(u_n,c)}_{=\epi(f)}\mit M\times\R.
\end{align*}

Da $\epi(f)$ abgeschlossen ist, ist $(u,c)\in\epi(f)$, d.h. $f(u)\leq c$ d.h. $u\in\lbrace f\leq c\rbrace$.\\

\underline{Zeige (ii) $\implies$ (i):}\\
Sei $(u_n)_{n\in\N}\subseteq\lbrace x\in M:f(x)\leq c\rbrace$ konvergente Folge mit $u:=\lim u_n$ in $M$. Setze
\begin{align*}
c:=\liminf\limits_{n\to\infty} f(u_n)\in\R\cup\lbrace\pm\infty\rbrace.
\end{align*}
Falls $c>-\infty$, dann enthält $\lbrace f\leq c+\varepsilon\rbrace$ für $\varepsilon>0$ unendlich viele $u_n$. Weil $\lbrace f\leq c+\varepsilon\rbrace$ abgeschlossen ist, ist
\begin{align*}
u=\limn u_n\in\lbrace f\leq c+\varepsilon\rbrace\text{ d.h. } f(u)\leq c+\varepsilon.
\end{align*}
Da $\varepsilon>0$ beliebig ist, ist $f(u)\leq c=\liminf\limits\limits_{n\to\infty} f(u_n)$.\\
Falls $c=-\infty$, dann enthält $\lbrace f\leq K\rbrace\mit K\in\R$ beliebig unendlich viele $u_n$. Und weil $\lbrace f\leq K\rbrace$ abgeschlossen ist, ist $u=\limn u_n\in\lbrace f\leq K\rbrace$, d.h. $f(u)\leq K$.\\
Da $K\in\R$ beliebig ist, ist $f(u)\leq=-\infty$. Dies ist aber ein Widerspruch zur Annahme.
\end{proof}

\begin{lemma}
Sei $M$ ein metrischer Raum und sei $(f_i)_{i\in I}$ eine Familie von unterhalbstetigen Funktionen $f_i:M\to\R\cup\lbrace+\infty\rbrace,~i\in I$.\\
Dann ist das Supremum $f:=\sup\limits_{i\in I} f_i$ unterhalbstetig.
\end{lemma}
\begin{proof}
Es gilt
\begin{align*}
\epi(f)=\epi\left(\sup\limits_{i\in I} f_i\right)=\bigcap\limits_{i\in I}\epi(f_i)
\end{align*}
und beliebige Schnitte abgeschlossener Mengen sind abgeschlossen.
\end{proof}

\begin{definition}
Sei $X$ ein reeller oder komplexer Vektorraum. Eine Funktion $f:X\to\R\cup\lbrace\infty\rbrace$ heißt \textbf{konvex}
\begin{align*}
:\Longleftrightarrow
\forall x,y\in X,\forall\lambda\in[0,1]:f\big(\lambda\cdot x+(1-\lambda)\cdot y\big)
\leq\lambda\cdot f(x)+(1-\lambda)\cdot f(y) 
\end{align*}
Eine Teilmenge $C\subseteq X$ heißt \textbf{konvex}
\begin{align*}
:\Longleftrightarrow
\forall x,y\in C,\forall\lambda\in[0,1]:\lambda\cdot x+(1-\lambda)\cdot y\in C
\end{align*}
\end{definition}

\begin{lemma}
$f$ ist konvex $\Longleftrightarrow\epi(f)$ ist konvex.
\end{lemma}

\begin{lemma}
Sei $f:\R\to\R\cup\lbrace\infty\rbrace$ konvex. Dann gilt
\begin{enumerate}[label=(\alph*)]
\item Für alle $x\in\R\mit f(x)<\infty$ und für alle $y\in\R$ ist
\begin{align*}
(0,\infty)\to\R\cup\lbrace +\infty\rbrace,\qquad
\lambda\mapsto\frac{f(x+\lambda\cdot y)-f(x)}{\lambda}
\end{align*}
monoton wachsend.
\item $\forall x\in\R\mit f(x)<\infty$ existieren die Grenzwerte
\begin{align*}
\lim\limits_{\lambda\to0^+}\frac{f(x+\lambda)-f(x)}{\lambda}\in\R\cup\lbrace\pm\infty\rbrace
\text{ und }
\lim\limits_{\lambda\to0^+}\frac{f(x-\lambda)-f(x)}{-\lambda}\in\R\cup\lbrace\pm\infty\rbrace.
\end{align*}
\item 
$\begin{aligned}
\dom(f):=\lbrace r\in\R:f(r)<\infty\rbrace
\end{aligned}$ 
ist ein Intervall und $f$ ist stetig auf $\dom(f)$.
\end{enumerate}
\end{lemma}
\begin{proof}
\underline{Zeige (a):} O.B.d.A. sei $x=0,~f(x)=0$ und $y=1$.\\
Zu zeigen ist, dass $\lambda\mapsto\frac{f(\lambda)}{\lambda}$ monoton wachsend auf $(0,\infty)$ ist.\\
Sei $0<\lambda_1<\lambda_2$. Dann ist 
\begin{align*}
\lambda_1=(1-\lambda)\cdot 0+\lambda\cdot\lambda_2\text{ für }\lambda:=\frac{\lambda_1}{\lambda_2}\in[0,1]
\end{align*}
und somit
\begin{align*}
f(\lambda_1)
=
f\Big((1-\lambda)\cdot0+\lambda\cdot\lambda_2\Big)
\stackrel{f\text{ konv}}{\leq}
\underbrace{(1-\lambda)\cdot f(0)}_{=0}+\lambda\cdot f(\lambda_2)
=
\frac{\lambda_1}{\lambda_2}\cdot f(\lambda_2).
\end{align*}
Behauptung (c) folgt dann aus (b), welche aus (a) folgt.
\end{proof}

\begin{proof}[Beweis des Lemmas über die Eigenschaften des Brackets]\enter
\underline{Zeige (a):} Das Bracket ist oberhalbstetig, denn
\begin{align*}
[\cdot,\cdot]_\lambda:X\times X\to\R,~(u,v)\mapsto[u,v]_\lambda:=\frac{\Vert u+\lambda\cdot v\Vert - \Vert u\Vert}{\lambda}
\end{align*}
ist stetig (da jede Norm stetig ist) und damit auch oberhalbstetig für alle $\lambda>0$. Das Infimum von oberhalbstetiger Funktion ist wieder oberhalbstetig.\\

\underline{Zeige (b):}\\
Die Funktion $\lambda\mapsto[u,v]_\lambda$ ist monoton wachsend auf $(0,\infty)$, weil $\lambda\mapsto\Vert u+\lambda\cdot v\Vert$ konvex ist, denn jede Norm ist konvex (dies folgt aus der Dreiecksungleichung).\\

\underline{Zeige (c):}\\
$\begin{aligned}[]
[u,v]=\lim\limits_{\lambda\to0^+}[u,v]_\lambda
\end{aligned}$ folgt aus (b) und aus
\begin{align*}
[u,v]_\lambda:=\frac{\Vert u+\lambda\cdot v\Vert-\Vert u\Vert}{\lambda}
\stackrel{\Delta\text{Ungl}}{\leq}
\frac{\Vert u\Vert+\lambda\cdot\Vert v\Vert-\Vert u\Vert}{\lambda}
=\Vert v\Vert.
\end{align*}

\underline{Zeige (d):}\\
$v\mapsto[u,v]$ ist sublinear, denn
\begin{align*}
[u,\mu\cdot v]
&=
\inf\limits_{\lambda>0}[u,\mu\cdot v]_\lambda\\
&=\inf\limits_{\lambda>0}\frac{\Vert u+\lambda\cdot\mu\cdot v\Vert-\Vert u\Vert}{\lambda}\cdot\frac{\mu}{\mu}\\
&\stackeq{Def}
\inf\limits_{\lambda>0}\mu\cdot[u,v]_{\lambda\cdot\mu}\\
&=\mu\cdot[u,v]
\end{align*}
und für alle $\mu\in(0,1)$ gilt
\begin{align*}
[u,v_1+v_2]
&=
\inf\limits_{\lambda>0}\frac{\Vert u+\lambda\cdot(v_1+v_2)\Vert-\Vert u\Vert}{\lambda}\\
&\stackrel{\mu\in(0,1)}{\leq}
\underbrace{\inf\limits_{\lambda>0}}_{=\lim\limits_{\lambda\to0}}
\frac{\Vert\mu\cdot u+\lambda\cdot v_1\Vert+\Vert(1-\mu)\cdot u+\lambda\cdot v_2\Vert-\mu\cdot\Vert u\Vert-(1-\mu)\cdot\Vert u\Vert}{\lambda}\\
&=\lim\limits_{\lambda\to0}\frac{\left\Vert u+\frac{\lambda}{\mu}\cdot v_1\right\Vert-\Vert u\Vert}{\frac{\lambda}{\mu}}+\frac{\left\Vert u+\frac{\lambda}{1-\mu}\cdot v_2\right\Vert-\Vert u\Vert}{\frac{\lambda}{1-\mu}}\\
&=[u,v_1]+[u,v_2]
\end{align*}

\underline{Zeige (e):}\\
Es gilt $[\mu\cdot u,v]=[u,v]$, denn
\begin{align*}
[\mu\cdot u,v]
&=
\inf\limits_{\lambda>0}\frac{\Vert\mu\cdot u+\lambda\cdot v\Vert-\Vert\mu\cdot u\Vert}{\lambda}\\
&=\inf\limits_{\lambda>0}\frac{\left\Vert u+\frac{\lambda}{\mu}\cdot v\right\Vert-\Vert u\Vert}{\frac{\lambda}{\mu}}\\
&=
[u,v]
\end{align*}

\underline{Zeige (f):}
\begin{align*}
[u,0]
&=
\inf\limits_{\lambda>0}\frac{\Vert u+\lambda\cdot0\Vert-\Vert u\Vert}{\lambda}
=0
\end{align*}

\underline{Zeige (g):}
\begin{align*}
[0,v]
&=
\inf\limits_{\lambda>0}\frac{\Vert 0+\lambda\cdot v\Vert-\Vert 0\Vert}{\lambda}
=\Vert v\Vert
\end{align*}

\underline{Zeige (h):}
\begin{align*}
[u,u]
&=
\inf\limits_{\lambda>0}\frac{\Vert u+\lambda\cdot u\Vert-\Vert u\Vert}{\lambda}
=
\inf\limits_{\lambda>0}\frac{(1+\lambda)\cdot\Vert u\Vert-\Vert u\Vert}{\lambda}
=\Vert u\Vert
\end{align*}
\end{proof}

\begin{bemerkung}
Falls $X=H$ ein Hilbertraum mit Skalarprodukt $\langle\cdot,\cdot\rangle$ ist, dann ist
\begin{align*}
[u,v] 
&=
\lim\limits_{\lambda\to0^+}\frac{\sqrt{\langle u+\lambda\cdot v,u+\lambda\cdot v\rangle}-\sqrt{\langle u,u\rangle}}{\lambda}\\
&=\lim\limits_{\lambda\to0^+}\frac{\sqrt{\langle u,u\rangle+2\cdot\lambda\cdot\Re(\langle u,v\rangle)+\lambda^2\cdot\langle v,v\rangle}-\sqrt{\langle u,u\rangle}}{\lambda}\\
&\stackeq{u\neq0}
\frac{1}{2\cdot\Vert u\Vert}\cdot 2\cdot\langle u,v\rangle\\
&=\left\langle\frac{u}{\Vert u\Vert},v\right\rangle
\end{align*}
\end{bemerkung}

\begin{lemma}
Sei $(X,\Vert\cdot\Vert)$ ein Banachraum, $I\subseteq\R$ ein Intervall und sei $u:I\to X$ eine Funktion. Dann gilt:
\begin{enumerate}[label=(\alph*)]
\item Wenn die \textbf{rechtsseitige Ableitung} von $u$, 
\begin{align*}
D_t^R u(t):=\dot{u}(t+):=\lim\limits_{h\to 0^+}\frac{u(t+h)-u(t)}{h},
\end{align*}
existiert, dann existiert die rechtsseitige Ableitung von $\Vert\cdot\Vert\circ u$, also
\begin{align*}
D_t^R \Vert u(t)\Vert:=\dot{u}(t+):=\lim\limits_{h\to 0^+}\frac{\Vert u(t+h)\Vert-\Vert u(t)\Vert}{h}
\end{align*}
und es gilt
\begin{align*}
D_t^R\Vert u(t)\Vert=\left[u(t),D_t^R u(t)\right].
\end{align*}
\item Falls die \textbf{linksseitige Ableitung} von $u$,
\begin{align*}
D_t^L u(t):=\lim\limits_{h\to 0^+}\frac{u(t-h)-u(t)}{-h},
\end{align*}
existiert, dann existiert
\begin{align*}
D_t^L \Vert u(t)\Vert:=\lim\limits_{h\to 0^+}\frac{\Vert u(t-h)\Vert-\Vert u(t)\Vert}{-h}
\end{align*}
und es gilt
\begin{align*}
D_t^R\Vert u(t)\Vert=-\left[u(t),-D_t^R u(t)\right].
\end{align*}
\item Falls die herkömmliche Ableitung von $u$
\begin{align*}
\lim\limits_{h\to0}\frac{u(t+h)-u(t)}{h}=:\dot{u}(t)
\end{align*}
existiert, dann existiert
\begin{align*}
\lim\limits_{h\to0}\frac{\Vert u(t+h)\Vert-\Vert u\Vert}{h}=:D_t\Vert u(t)\Vert
\end{align*}
und es gilt
\begin{align*}
D_t\Vert u(t)\Vert_\lambda=\left[ u(t),\dot{u}(t)\right]=-\big[u(t),-\dot{u}(t)\big]
\end{align*}
\end{enumerate}
\end{lemma}

\begin{proof}
\underline{Zeige (a):}\\
Für $h>0$ gilt die \textit{Weierstraß'sche Zerlegungsformel}
\begin{align*}
u(t+h)=u(t)+h\cdot D_t^R u(t)+o(h)\mit\lim\limits_{h\to0^+}\frac{o(h)}{h}=0
\end{align*}
und somit
\begin{align*}
\frac{\Vert u(t+h)\Vert-\Vert u(t)\Vert}{h}
&=
\frac{\Vert u(t)+h\cdot D_t^R u(t)+ o(h)\Vert-\Vert u(t)\Vert}{h}\\
&\stackeq{\geq\&\leq\mit\Delta\text{-Ungl}}
\frac{\Vert u(t)+h\cdot D_t^R u(t)\Vert-\Vert u(t)\Vert}{h}+\frac{o(h)}{h}\\
&\stackrel{h\to0^+}{\longrightarrow} \left[ u(t), D_t^R u(t)\right]
\end{align*}
Hierbei wird die Dreiecksungleichung und die umgekehrte Dreiecksungleichung benutzt, um in beide Richtungen abzuschätzen und Gleichheit zu erzielen.\\

\underline{Zeige (b):}
Analog zu (a) mit $h<0$.\\

\underline{Zeige (c):}
Folgt direkt aus (a) und (b).
\end{proof}

\section{Akkretive Operatoren}
\begin{definition}
Sei $(X,\Vert\cdot\Vert)$ ein Banachraum. Ein Operator $A\subseteq X\times X$ heißt \textbf{akkretiv vom Typ $\omega\in\R$}
\begin{align*}
:\Longleftrightarrow\forall (u,v),(\hat{u},\hat{v})\in A:\left[ u-\hat{u},v-\hat{v}\right]+\omega\cdot\left\Vert u-\hat{u}\right\Vert\geq0
\end{align*}
Ein \textbf{akkretiver} Operator  ist ein akkretiver Operator vom Typ $0$.\\
Ein akkretiver Operator vom Typ $\omega$ ist auch ein akkretiver Operator vom Typ $\omega'$ für alle $\omega'\geq\omega$.
\end{definition}

\begin{lemma}
Sei $X$ Banachraum. Für einen Operator $A\subseteq X\times X$ und $\omega\in\R$ sind folgende Aussagen äquivalent:
\begin{enumerate}[label=(\roman*)]
\item $A$ ist akkretiv vom Typ $\omega$
\item $A+\omega\cdot I$ ist akkretiv
\item $\begin{aligned}\forall (u,v),(\hat{u},\hat{v})\in A,\forall\lambda>0:\Vert u-\hat{u}+\lambda\cdot(v-\hat{v})\Vert\geq(1-\lambda\cdot\omega)\cdot\Vert u-\hat{u}\Vert
\end{aligned}$
\end{enumerate}
\end{lemma}
\begin{proof}
\underline{Zeige (i) $\gdw$ (iii):}\\
$A$ ist akkretiv vom Typ $\omega$
\begin{align*}
&\gdw\underbrace{[u-\hat{u},v-\hat{v}]}_{=\inf\limits_{\lambda>0}[\ldots]_\lambda}
+ \omega\cdot\Vert u-\hat{u}\Vert\geq0~\forall(u,v),(\hat{u},\hat{v})\in A\\
&\gdw\frac{\Vert u-\hat{u}+\lambda\cdot(v-\hat{v})\Vert-\Vert u-\hat{u}\Vert}{\lambda}+\omega\cdot\Vert u-\hat{u}\Vert\geq0~\forall(u,v),(\hat{u},\hat{v})\in A,\forall\lambda>0\\
&\gdw\text{(iii)}
\end{align*}
\underline{Zeige (iii) $\gdw$ (ii):}
\begin{align*}
\text{(iii)} &\Rightarrow\forall(u,v),(\hat{u},\hat{v})\in A,\forall\lambda>0\text{ klein, d. h. }1+\lambda\cdot\omega>0:\\
&\qquad(1+\lambda\cdot\omega)\cdot\Vert u-\hat{u}+\lambda\cdot(v-\hat{v})\Vert\geq(1-\lambda\cdot\omega)\cdot\Vert u-\hat{u}\Vert\cdot(1+\lambda\cdot\omega)\\
&\gdw\forall\ldots:\Big\Vert u-\hat{u}+\lambda\cdot\big((1+\lambda\cdot\omega)\cdot (v-\hat{v})+\omega\cdot(u-\hat{u})\big)\Big\Vert
\geq
\left(1-\lambda^2\cdot\omega^2\right)\cdot\Vert u-\hat{u}\Vert\\
&\gdw\forall\ldots:
\Big\Vert u-\hat{u}+\lambda\cdot\big(v-\hat{v}+\omega\cdot(u-\hat{u})\big)+\lambda^2\cdot\omega\cdot(v-\hat{v})\Big\Vert\\
&\qquad\qquad\geq
\left(1-\lambda^2\cdot\omega^2\right)\cdot\Vert u-\hat{u}\Vert\\
&\gdw\forall\ldots:
\frac{\Big\Vert u-\hat{u}+\lambda\cdot\big(v-\hat{v}+\omega\cdot(u-\hat{u})\big)\Big\Vert-\Vert u-\hat{u}\Vert}{\lambda}+\mathcal{O}(\lambda)\geq0\\
&\stackrel{\lambda\to0}{\Rightarrow}
\forall(u,v),(\hat{u},\hat{v})\in A:\Big[ u-\hat{u},\underbrace{v}_{=Au}-\hat{v}+\omega\cdot(u-\hat{u})\Big]\geq0\\
&\gdw A+\omega\cdot I\text{ ist akkretiv}
\end{align*}
Das die Rückrichtung auch gilt muss man sich überlegen.
\end{proof}

\begin{lemma}
Sei $A\subseteq X\times X$ akkretiv vom Typ $\omega\in\R$.\\
Dann ist der Abschluss $\overline{A}$ akkretiv vom Typ $\omega$.
\end{lemma}
\begin{proof}
Seien $(u,v),(\hat{u},\hat{v})\in\overline{A}$. Dann existieren Folgen 
\begin{align*}
&(u_n,v_n)_{n\in\N},(\hat{u}_n,\hat{v}_n)_{n\in\N}\subseteq A\mit\\
&(u_n,v_n)
\stackrel{n\to\infty}{\longrightarrow}
(u,v)\text{ in } \overline{A} \subseteq X\times X\\
&(\hat{u}_n,\hat{v}_n)
\stackrel{n\to\infty}{\longrightarrow}
(\hat{u},\hat{v})\text{ in } \overline{A} \subseteq X\times X
\end{align*}
und wegen der Oberhalbstetigkeit des Brackets gilt
\begin{align*}
[u-\hat{u},v-\hat{v}]+\omega\cdot\Vert u-\hat{u}\Vert
\geq
\limsup\limits_{n\to\infty}\Big(\underbrace{\big[ u_n-\hat{u}_n,v_n-\hat{v}_n\big]+\omega\cdot\Vert u_n-\hat{u}_n\Vert}_{\geq0,\text{ da $A$ akkretiv vom Typ $\omega$}}\Big)
\geq0
\end{align*}
\end{proof}

\begin{beispiel}\
\begin{enumerate}[label=(\alph*)]
\item In einem Hilbertraum $H$ ist ein Operator $A\subseteq H\times H$ akkretiv vom Typ $\omega\in\R$
\begin{align*}
\Longleftrightarrow\forall(u,v),(\hat{u},\hat{v})\in A:
 \Re\big(\langle u-\hat{u},v-\hat{v}\rangle\big)+\omega\cdot\Vert u-\hat{u}\Vert^2\geq0.
\end{align*}
Operatoren $A\subseteq H\times H$, die diese Bedingung erfüllen, heißen auch \textbf{monoton vom Typ $\omega$} bzw. einfach nur \textbf{monoton}, falls $\omega=0$.\\
Es gilt also: $A$ akkretiv vom Typ $\omega\gdw A$ monoton vom Typ $\omega$.

Falls $A$ ein linearer (einwertiger) Operator ist, dann ist $A$ akkretiv
\begin{align*}
&\Longleftrightarrow\forall u\in\dom(A):\Re\big(\langle u, Au\rangle\big)\geq0\\
&\Longleftrightarrow: -A\text{ ist \textbf{dissipativ}}
\end{align*}
\item $H=L^2(\Omega)\mit\Omega\subseteq\R^n$ offen und 
\begin{align*}
A&=\big\lbrace(u,v)\in L^2\times L^2:u,v\in C_c^\infty(\Omega):v=\Delta u\big\rbrace\\
C_c^\infty(\Omega)&:=\big\lbrace u\in C^\infty(\Omega):\supp(u)\text{ kompakt}\big\rbrace\\
\supp(u)&:=\overline{\big\lbrace x\in\Omega:u(x)\neq0\big\rbrace}^\Omega\text{ Abschluss in }\Omega\\
\Delta u&:=\sum\limits_{i=1}^n\frac{\partial^2 u}{\partial x_i^2}
\end{align*}
Dann gilt für alle $u\in\dom(A)$:
\begin{align*}
\Re\Big(\langle u,-\Delta u\rangle_{L^2}\Big) 
&= \Re\left(-\int\limits_\Omega u\cdot\overline{\Delta u}\right) = \Re\left(-\int\limits_\Omega u\cdot\overline{\nabla \Div u}\right)\\
&= -\int\limits_\Omega \Re \left(u\cdot\overline{\nabla \Div u}\right) = -\int\limits_\Omega \Re \left(u\cdot\nabla \Div u\right)\\
&= -\int\limits_{\supp(u)} \Re \left(u\cdot\nabla \Div u\right)\\
&\stackeq{\text{Gauß}}
-\int\limits_{\partial\supp(u)} \Re \left(u\cdot\nabla u\right) + \int\limits_{\supp(u)} \Re \left(\nabla u\cdot\nabla u\right)\\
&=\Re\left(\int\limits_{\Omega} \nabla u\cdot\overline{\nabla u}\right)\\
%&=\int\limits_\Omega |\nabla u|^2\\
&\geq0
\end{align*}
Mithilfe des Gauß'schen Integralsatzes (partielle Integration) folgt die Akkretivität von $A$. $A$ ist hierbei der negative Laplace-Operator auf den Testfunktionen.
Beachte, dass $u$ auf $\partial \supp(u)$ Null sein muss, da $u$ glatt.
\item Sei $A\subseteq X\times X$ akkretiv vom Typ $\omega\in\R$ und $F:X\to X$ Lipschitzstetig mit Lipschitzkonstante $L\geq0$.\\
Dann ist $A+F$ akkretiv vom Typ $\omega+L$.
\begin{proof}
\begin{align*}
&\forall (u,v),(\hat{u},\hat{v})\in A,\forall\lambda>0:
\big[u-\hat{u},v-\hat{v}+F(u)-F(\hat{u})\big]_\lambda
+(\omega+L)\cdot\Vert u-\hat{u}\Vert\\
&=\frac{\Big\Vert u-\hat{u}+\lambda\cdot\big(v-\hat{v}+F(u)-F(\hat{u})\big)\Big\Vert-\Vert u-\hat{u}\Vert}{\lambda}
+(\omega+L)\cdot\Vert u-\hat{u}\Vert\\
&\stackrel{\Delta\text{-Ungl}}{\geq}
\frac{\big\Vert u-\hat{u}+\lambda\cdot(v-\hat{v})\big\Vert-\Vert u-\hat{u}\Vert}{\lambda}\underbrace{-\big\Vert F(u)-F(\hat{u})\big\Vert}_{\geq -L\cdot\Vert u-\hat{u}\Vert}
+(\omega+L)\cdot\Vert u-\hat{u}\Vert\\
&\geq0
\end{align*}
\end{proof}
\item Sei wieder $\Omega\subseteq\R^n$ offen und sei $f:\R\to\R$ lipschitzstetig mit $f(0)=0$ oder das Maß von $\Omega$ endlich, sei $H=L^2(\Omega)$ der reelle Hilbertraum und
\begin{align*}
A:L^2(\Omega)\supseteq C_c^\infty(\Omega)\to L^2(\Omega),\qquad
u\mapsto Au:=-\Delta u+\underbrace{f(u)}_{f\circ u}.
\end{align*}
Dann ist $A$ akkretiv vom Typ $L_f$, wobei $L_f$ die Lipschitzkonstante von $f$ ist.
\begin{proof}
Die Abbildung
\begin{align*}
F: L^2(\Omega)\to L^2(\Omega),\qquad u\mapsto f(u)
\end{align*}
ist wohldefiniert und Lipschitzstetig.\\
Zur Lipschitzstetigkeit: Es gilt für alle $u,\hat{u}\in L^2(\Omega)$:
\begin{align*}
\big\Vert F(u)-F(\hat{u})\big\Vert^2_{L^2}
&=\int\limits_\Omega\Big|f\big(u(x)\big)-f\big(\hat{u}(x)\big)\Big|^2\d x\\
&\leq
\int\limits_\Omega L^2\cdot\big|u(x)-\hat{u}(x)\big|^2\d x\\
&=L^2\cdot\Vert u-\hat{u}\Vert^2_{L^2}
\end{align*}
Zur Wohldefiniertheit: Für alle $u\in L^2(\Omega)$ gilt:
\begin{align*}
\left(\int\limits_\Omega\Big|f\big(u(x)\big)\Big|^2\d x\right)^{\frac{1}{2}}
&=\left(\int\limits_\Omega\Big|f\big(u(x)\big)-f(0)+f(0)\Big|^2\d x\right)^{\frac{1}{2}}\\
&\leq
\sqrt{\int\limits_\Omega\Big|f\big(u(x)\big)-f(0)\Big|^2\d x}+|f(0)|\cdot\underbrace{|\Omega|}_{\text{Maß von }\Omega}\\
&\leq
L\cdot\Vert u\Vert_{L^2}+|f(0)|\cdot|\Omega|<\infty
\end{align*}
\end{proof}
\item Sei $F:X\to X$ lipschitzstetig mit Lipschitzkonstante $L\geq0$.\\
Dann ist 
\begin{align*}
A=L\cdot I-F
\end{align*}
akkretiv (vom Typ $0$).
\begin{proof}
\begin{align*}
\forall u,\hat{u}\in X,\forall \lambda>0:
&\bigg\Vert u-\hat{u}+\lambda\cdot\Big(L\cdot(u-\hat{u})-\big(F(u)-F(\hat{u})\big)\Big)\bigg\Vert-\Vert u-\hat{u}\Vert\\
&\geq
(1+\lambda\cdot L)\cdot\Vert u-\hat{u}\Vert-\lambda\cdot\Vert F(u)-F(\hat{u})\Vert-\Vert u-\hat{u}\Vert\\
&\geq
\lambda\cdot L\cdot\Vert u-\hat{u}\Vert-\lambda\cdot L\cdot\Vert u-\hat{u}\Vert\\
&=0
\end{align*}
\end{proof}
\end{enumerate}
\end{beispiel}

\begin{theorem}[Abschätzungen für akkretive Operatoren]\enter
Sei $A\subseteq X\times X$ akkretiv vom Typ $\omega\in\Omega$ auf einem Banachraum $(X,\Vert\cdot\Vert)$. Dann gilt:
\begin{enumerate}[label=(\alph*)]
\item Seien $h,\hat{h}>0$ so, dass $h\cdot\omega,\hat{h}\cdot\omega<1$ und seien $(u,f),(\hat{u},\hat{f})\in X\times X$ so, dass 
\begin{align*}
u+h\cdot Au\ni f\qquad\text{und}\qquad\hat{u}+\hat{h}\cdot A\hat{u}\ni\hat{f}.
\end{align*}
Dann gilt:
\begin{align*}
\left(a-\frac{h\cdot\hat{h}}{h+\hat{h}}\cdot\omega\right)\cdot\Vert u-\hat{u}\Vert
\leq
\left\Vert u-\hat{u}+\frac{\hat{h}}{h+\hat{h}}\cdot(f-u)-\frac{h}{h+\hat{h}}\cdot\big(\hat{f}-\hat{u}\big)\right\Vert
\end{align*}
Insbesondere gilt für $h=\hat{h}$ dann
\begin{align*}
\Vert u-\hat{u}\Vert
\leq
\frac{1}{1-h\cdot\omega}\cdot\Vert f-\hat{f}\Vert.
\end{align*}
\item Sei $h>0\mit h\cdot \omega<1$. Für alle $f\in X$ besitzt die Inklusion
\begin{align*}
u+h\cdot Au\ni f
\end{align*}
höchstens eine Lösung $u\in\dom(A)$.
\item Sei $h>0\mit h\cdot\omega<$ und sei $(u,f)\in X\times X$ so, dass 
\begin{align*}
u+h\cdot Au\ni f.
\end{align*}
Dann gilt:
\begin{align*}
\Vert u-\hat{u}\Vert
\leq
\frac{1}{1-h\cdot\omega}\cdot\big\Vert f-h\cdot\hat{f}-\hat{u}\big\Vert
\qquad
\forall(\hat{u},\hat{f})\in A
\end{align*}
Insbesondere, wenn $f\in\dom(A)$, dann ist
\begin{align*}
\Vert u-f\Vert
&\leq
\frac{h}{1-h\cdot\omega}\cdot\Vert Af\Vert
\qquad
\text{ wobei }
\qquad
\Vert Au\Vert:=\inf\big\lbrace\Vert f\Vert:(u,f)\in A\big\rbrace
\end{align*}
\end{enumerate}
\end{theorem}

\begin{bemerkung}
Beachte
\begin{align*}
u+h\cdot Au\ni f\Longleftrightarrow\left(u,\frac{f-u}{h}\right)\in A
\end{align*}
\end{bemerkung}

\begin{proof}
\underline{Zeige (a):} Es ist
\begin{align*}
&\left[u-\hat{u},\frac{f-u}{h}-\frac{\hat{f}-\hat{u}}{\hat{h}}\right]_{\frac{h\cdot\hat{h}}{h+\hat{h}}}+\omega\cdot\Vert u-\hat{u}\Vert\geq0\\
&\gdw
\left\Vert u-\hat{u}+\frac{\hat{h}}{h+\hat{h}}\cdot(f-u)-\frac{h}{h+\hat{h}}\cdot(\hat{f}-\hat{u})\right\Vert
-\Vert u-\hat{u}\Vert+\frac{h\cdot\hat{h}}{h+\hat{h}}\cdot\omega\cdot\Vert u-\hat{u}\Vert>0\\
&\gdw\text{ Behauptung}
\end{align*}
Falls $h=\hat{h}$, dann ist
\begin{align*}
\left(\frac{1}{2}-\frac{h\cdot\omega}{2}\right)\cdot\Vert u-\hat{u}\Vert
&\leq
\Big\Vert u-\hat{u}+\frac{1}{2}\cdot(f-u)-\frac{1}{2}\cdot(\hat{f}-\hat{u})\Big\Vert\\
&=\left\Vert\frac{1}{2}\cdot(u-\hat{u})+\frac{1}{2}\cdot(f-\hat{f})\right\Vert\\
&\leq
\frac{1}{2}\cdot\Vert f-\hat{f}\Vert
\end{align*}
\underline{Zeige (b):} Dies folgt aus (a), denn:\\
Falls $u,\hat{u}\in\dom(A)$ Lösungen von
\begin{align*}
u+h\cdot Au\ni f,\qquad\hat{u}+h\cdot A\hat{u}\ni f
\end{align*}
sind, dann ist
\begin{align*}
\Vert u-\hat{u}\Vert\leq\frac{1}{1-h\cdot\omega}\cdot\Vert f-f\Vert=0
\end{align*}
\underline{Zeige (c):} Seien $h>0\mit h\cdot\omega<1$ und $8u,f)\in X\times X$ so, dass
\begin{align*}
u+h\cdot Au\ni f.
\end{align*}
sei $(\hat{u},\hat{f})\in A$, also äquivalent $A\hat{u}\ni\hat{f}$. Nach Multiplikation beider Seiten mit $h$ und Addition beider Seiten mit $\hat{u}$ erhält man
\begin{align*}
h\cdot A\hat{u}+\hat{u}\ni h\cdot\hat{f}+\hat{u}.
\end{align*}
Dann folgt aus (a)
\begin{align*}
\Vert u-\hat{u}\Vert\leq\frac{1}{1-h\cdot\omega}\cdot\big\Vert f-h\cdot\hat{f}-\hat{u}\big\Vert
\end{align*}
Falls $f\in\dom(A)$, dann gilt für alle $g\in Af$ (ersetze $(\hat{u},\hat{f})$ durch $(f,g)$):
\begin{align*}
\Vert u- f\Vert
\leq
\frac{1}{1-h\cdot\omega}\cdot\big\Vert f-h\cdot g-f\big\Vert
=\frac{h}{1-h\cdot\omega}\cdot\Vert g\Vert
\end{align*}
Nehme $\inf$ über $g\in Af$:
\begin{align*}
\Vert u-f\Vert\leq\frac{h}{1-h\cdot\omega}\cdot\Vert Af\Vert
\end{align*}
\end{proof}

\begin{theorem}
Sei $A\subseteq X\times X$ akkretiv vom Typ $\omega\in\R$ auf einem Banachraum $(X,\Vert\cdot\Vert)$. Dann sind folgende Aussagen äquivalent:
\begin{enumerate}[label=(\roman*)]
\item $\begin{aligned}
\exists h>0\mit h\cdot\omega<1:I+h\cdot A
\end{aligned}$ surjektiv
\item $\begin{aligned}
\forall h>0\mit h\cdot\omega<1: I+h\cdot A
\end{aligned}$ surjektiv
\end{enumerate}
Falls $A$ zusätzlich abgeschlossen ist, dann sind (i) und (ii) äquivalent zu
\begin{enumerate}[label=(iii)]
\item $\begin{aligned}
\exists h>0\mit h\cdot\omega<1:
\end{aligned}$ das Bild von $I+h\cdot A$ dicht in $X$ ist.
\end{enumerate}
\end{theorem}
\begin{proof}
\underline{Zeige (i) $\Rightarrow$ (ii):}\\
Sei $h>0\mit h\cdot\omega<1$ so, dass $I+h\cdot A$ surjektiv ist. Nach dem vorherigen Theorem, Aussage (b), ist $I+h\cdot A$ immer injektiv, also bijektiv. Setze
\begin{align*}
J_h:=\big(I+h\cdot A\big)^{-1}.
\end{align*}
Dann ist wegen Theorem, Aussage (a), $J_h$ lipschitzstetig mit Lipschitzkonstante $\frac{1}{1-h\cdot\omega}$. Sei nun 
\begin{align*}
\hat{h}>\frac{h}{2-h\cdot\omega}>0\mit\hat{h}\cdot\omega<1
\end{align*}
und sei $f\in X$. Betrachte den Operator
\begin{align*}
T:X\to X,\qquad Tv:=T(v)
:=\frac{h}{\hat{h}}\cdot f+\frac{\hat{h}-h}{\hat{h}}\cdot J_h\cdot v\qquad\forall v\in X
\end{align*}
Dann gilt für alle $v,\hat{v}\in X$:
\begin{align*}
\big\Vert T(v)-T(\hat{v})\big\Vert_X
&\leq
\left|\frac{\hat{h}-h}{\hat{h}}\right|\cdot\big\Vert J_h v-J_h \hat{v}\big\Vert\\
&\leq
\underbrace{\frac{1}{1-h\cdot\omega}\cdot\frac{|\hat{h}-h|}{|\hat{h}|}}_{<1}\cdot\Vert v-\hat{v}\Vert
\end{align*}
Also ist $T$ Lipschitz-stetig mit Lipschitzkonstante 
\begin{align*}
L_T:=\left|\frac{\hat{h}-h}{\hat{h}}\right|
\cdot\frac{1}{1-h\cdot\omega}.
\end{align*}
Falls $\hat{h}\leq h$ gilt:
\begin{align*}
L_T=\left(\frac{h}{\hat{h}}-1\right)\cdot\frac{1}{1-h\cdot\omega}
<
(2-h\cdot\omega-1)\cdot\frac{1}{1-h\cdot\omega}
=1
\end{align*}
Falls $\hat{h}>h$ gilt hingegen:
\begin{align*}
L_T=\left(1-\frac{h}{\hat{h}}\right)\cdot\frac{1}{1-h\cdot\omega}
\stackrel{-\frac{1}{\hat{h}}<-\omega}{<}
(1-h\cdot\omega)\cdot\frac{1}{1-h\cdot\omega}
=1
\end{align*}
Also ist $T$ eine strikte Kontraktion, da $L_T<1$. Aus dem Banach'schen Fixpunktsatz folgt, dass $T$ (genau) einen Fixpunkt $u\in X$ besitzt. Somit gilt
\begin{align*}
u\stackeq{\text{Fix}}T(u)
=\frac{h}{\hat{h}}\cdot f+\frac{\hat{h}-h}{\hat{h}}\cdot J_h u
\end{align*}
bzw. 
\begin{align*}
f&=\frac{\hat{h}}{h}\cdot u-\frac{\hat{h}-h}{h}\cdot J_h u\\
&=J_h u+\frac{\hat{h}}{h}\cdot\left(u-J_j u\right)\\
&\in J_h u+\hat{h}\cdot A J_h u
\end{align*}
d.h. $f$ ist im Bild von $I+\hat{h}\cdot A$. Da $f\in X$ beliebig war, ist $I+\hat{h}\cdot A$ surjektiv für $\hat{h}>\frac{h}{2-h\cdot\omega}\mit \hat{h}\cdot\omega<1$. Durch Iterieren des Arguments erhält man schließlich, dass $I+\hat{h}\cdot A$ surjektiv ist für alle $\hat{h}>0\mit\hat{h}\cdot\omega<1$.\\

\underline{Zeige (ii) $\implies$ (i):} Ist wirklich trivial.\\

\underline{Zeige (i) $\implies$ (iii):} Trivial, da $Y$ dicht in $Y$ liegt.\\

\underline{Zeige (iii) $\implies$ (i):}\\
Sei $A$ abgeschlossen, $h>0\mit h\cdot\omega<1$ so, dass $I+h\cdot A$ dichtes Bild hat. Die Inverse Relation
\begin{align*}
J_h:=(I+h\cdot A)^{-1}
\end{align*}
ist dann Lipschitz-stetig vom Bild $\underbrace{\rg(I+h\cdot A)}_{\text{dicht in }X}$ in $X$. Damit besitzt $J_h$ eine (eindeutige) Lipschitz-stetige Fortsetzung auf ganz $X$ (Eine gleichmäßige Abbildung lässt sich stets eindeutig auf die Vervollständigung fortsetzen). Aber mit $A$ ist $I+h\cdot A$ abgeschlossen in $X\times X$ und somit auch $J_h$. Also stimmen $J_h$ und die Lipschitz-stetige Fortsetzung überein, d. h. $\rg(I+h\cdot A)=X$. Damit ist Surjektivität gezeigt.
\end{proof}

\begin{definition}
Ein Operator $A\subseteq X\times X$ heißt \textbf{$m$-akkretiv vom Typ $\omega\in\Omega$} $:\gdw\\ A$ akkretiv vom Typ $\omega\in\R$ und $I+h\cdot A$ surjektiv für ein / alle $h>0\mit h\cdot\omega<1$.\\

Ist $A\subseteq X\times X$  $m$-akkretiv vom Typ $\omega\in\R$, dann ist $I+h\cdot A$ bijektiv für alle $h>0\mit h\cdot\omega<1$. Wir schreiben in diesem Fall
\begin{align*}
J_h:=(I+h\cdot A)^{-1}.
\end{align*}
$J_h$ ist einwertiger Operator mit $\dom(J_h)=X$.
\end{definition}

\begin{beispiel}[Beispiele für $X=\R$]\
\begin{enumerate}[label=(\alph*)]
% Hier Abbildung einfügen
\item $\begin{aligned}
A:=\big([-\infty,0]\times\lbrace-1\rbrace\big)\cup\big([0,\infty)\times\lbrace 1\rbrace\big)\cup\big(\lbrace0\rbrace\times[-1,1]\big)
\end{aligned}$
ist akkretiv (``monoton wachsend''). Damit gilt
\begin{align*}
h\cdot A&=\big([-\infty,0]\times\lbrace-h\rbrace\big)\cup\big([0,\infty)\times\lbrace h\rbrace\big)\cup\big(\lbrace0\rbrace\times[-h,h]\big)\\
I+h\cdot A&=\graph\big(\R_{\leq0}\to\R_{\leq0},~x\mapsto x-1\big)\cup\big(\lbrace 0\rbrace\times [-h,h]\big)\cup\graph\big(\R_{\geq0}\to\R_{\geq0},x\mapsto x+1\big)
\end{align*}
Hierbei ist $I+h\cdot A$ $m$-akkretiv (also $m$-akkretiv vom Typ 0).
\begin{align*}
(I+h\cdot A)^{-1}=\graph\left(R\to\R,~x\mapsto\left\lbrace\begin{array}{cl}
x+h, & \falls x\leq -h\\
0, & \falls x\in(-h,h)\\
x-h, & \falls x\geq h
\end{array}\right.\right)
\end{align*}
\item $\begin{aligned}
A=\big(\lbrace a\rbrace\times (-\infty,0]\big)\cup\big(\lbrace b\rbrace\times[0,\infty)\big)
\end{aligned}$
Damit gilt
\begin{align*}
h\cdot A=A\text{ WARUM???}
\end{align*}
und $I+h\cdot A$ ist surjektiv, also ist $A$ akkretiv. $J_h=(I+h\cdot A)^{-1}$ ist eine Funktion.
\end{enumerate}
\end{beispiel}

\begin{theorem}\label{Theorem1.2.9}
Sei $A\subseteq X\times X$ $m$-akkretiv vom Typ $\omega\in\R$. Dann gilt:
\begin{enumerate}[label=(\alph*)]
\item $\begin{aligned}
\forall h,\hat{h}>0\mit h\cdot\omega,\hat{h}\cdot\omega<1,\forall f,\hat{f}\in X:\end{aligned}$
\begin{align*}
\left(1-\frac{h\cdot\hat{h}}{h+\hat{h}}\cdot\omega\right)\cdot\Big\Vert J_h f-J_{\hat{h}}\hat{f}\Big\Vert
\leq
\left\Vert f-\hat{f}-\frac{h}{h+\hat{h}}\cdot\big(f-J_h f\big)+\frac{\hat{h}}{h+\hat{h}}\cdot\big(\hat{f}-J_h \hat{f}\big)\right\Vert
\end{align*}
\item Die \textbf{Resolvente} $J_h$ ist Lipschitz-stetig mit Lipschitzkonstante $\frac{1}{1-h\cdot\omega}$, d. h.
\begin{align*}
\forall h>0\mit h\cdot \omega<1,\forall f,\hat{f}\in X:
\Big\Vert J_h f- J_h\hat{f}\Big\Vert
\leq
\frac{1}{1-h\cdot\omega}\cdot\big\Vert f-\hat{f}\big\Vert
\end{align*}
\item $\begin{aligned}
\forall f\in\dom(A),\forall h>0\mit h\cdot\omega<1:
\big\Vert J_h f-f\big\Vert
\leq
h\cdot\Vert A f\Vert\mit\Vert A f\Vert=\inf\big\lbrace\Vert g\Vert:g\in A f\big\rbrace
\end{aligned}$
\item $\begin{aligned}
\forall f\in\dom(A):\lim\limits_{h\to0^+} J_h f=f
\end{aligned}$
\item Falls $X$ ein Hilbertraum ist, dann ist $\overline{\dom(A)}$ konvex und
\begin{align*}
\lim\limits_{h\to0^+} J_h f=P(f)\qquad\forall f\in X
\end{align*}
wobei $P:H\to H$ die orthogonale Projektion auf $\overline{\dom(A)}$ ist.
\end{enumerate}
\end{theorem}

\begin{lemma}
Sei $H$ ein Hilbertraum und $C\subseteq H$ eine nichtleere, abgeschlossene, konvexe Teilmenge. Dann gilt:
\begin{align*}
\forall u\in H:\exists! u_0\in C:\Vert u-u_0\Vert
=\dist(u,C)
:=\inf\limits_{v\in C}\Vert u-v\Vert
\end{align*}
Dieses $u_0\in C$ ist das eindeutige Element in $C$ so, dass 
\begin{align*}
\forall v\in C:\Re\big(\langle u-u_0,v-u_0\rangle\big)\leq0
\end{align*}
Die Abbildung
\begin{align*}
P:H\to H,\qquad u\mapsto u_0
\end{align*}
heißt \textbf{orthogonale Projektion} auf $C$. Es gilt $\rg(P)=C$ und $P\circ P=P$.
\end{lemma}
\begin{proof}
\underline{Existenz von $u_0$:}\\
Sei $(v_n)_{n\in\N}\subseteq\C$ so, dass
\begin{align*}
\limn\Vert u-v_n\Vert
%\stackrel{n\to\infty}{\longrightarrow}
=\dist(u,C).
\end{align*}
(Diese Folge existiert nach Definition des Infimums.) Mit der Parallelogrammidentität gilt:
\begin{align*}
\forall x,y\in H:\Vert x\Vert^2+\Vert y\Vert^2\stackeq{\text{Parallelo}}\frac{1}{2}\cdot\left(\Vert x+y\Vert^2+\Vert x-y\Vert^2\right)
\end{align*}
Setze $x:=u-v_n$ und $y:=u-v_n$. Dann folgt:
\begin{align*}
\Vert u-v_n\Vert^2+\Vert u-v_m\Vert^2
&=2\cdot\left\Vert u-\frac{v_n+v_m}{2}\right\Vert^2+\frac{1}{2}\cdot\Vert v_n-v_m\Vert^2
\qquad\forall n,m\in\N\\
\implies
\limsup\limits_{m,n\to\infty}\frac{1}{2}\cdot\Vert v_n-v_m\Vert^2
&=\limsup\limits_{m,n\to\infty}\left(\Vert u-v_n\Vert^2+\Vert u-v_m\Vert^2\right)\\
&\qquad-\liminf\limits_{m,n\to\infty} \underbrace{2\cdot\Bigg\Vert u-\underbrace{\frac{v_n+v_m}{2}}_{\in C}\Bigg\Vert^2}_{\geq2\cdot\big(\dist(u,C)\big)^2}\\
&\leq2\cdot\dist\big((u,C)\big)^2-2\cdot\dist\big((u,C)\big)^2\\
&=0
\end{align*}
Also ist $(v_n)_{n\in\N}$ eine Cauchy-Folge in $H$. Setze $u_0:=\limn v_n$. Weil $C$ abgeschlossen ist, ist $u_0\in C$. Nach Wahl von $(v_n)_{n\in\N}$ ist
\begin{align*}
\Vert u-u_0\Vert=\dist(u,C).
\end{align*}

\underline{Eindeutigkeit:}\\
Seien $u_0,u_1\in C$ mit
\begin{align*}
\Vert u-u_0\Vert=\Vert u-u_1\Vert=\dist(u,C).
\end{align*}
Dann ist 
\begin{align*}
\Bigg\Vert u-\underbrace{\frac{u_0+u_1}{2}}_{\in C}\Bigg\Vert^2
&=
\left\Vert\frac{u-u_0}{2}+\frac{u-u_1}{2}\right\Vert^2\\
&=\left\Vert\frac{u-u_0}{2}\right\Vert^2+2\cdot\Re\left(\left\langle\frac{u-u_0}{2},\frac{u-u_1}{2}\right\rangle\right)+\left\Vert\frac{u-u_1}{2}\right\Vert^2\\
&\stackrel{\text{C.S.; }u_0\neq u_1}{<}
\left\Vert\frac{u-u_0}{2}\right\Vert^2+2\cdot\left\Vert\frac{u-u_0}{2}\right\Vert\cdot\left\Vert\frac{u-u_1}{2}\right\Vert+\left\Vert\frac{u-u_1}{2}\right\Vert^2\\
&=
\left(\left\Vert\frac{u-u_0}{2}\right\Vert+\left\Vert\frac{u-u_1}{2}\right\Vert\right)^2\\
&=
\big(\dist(u,C)\big)^2
\end{align*}
Dies ist aber ein Widerspruch zur Annahme. Somit folgt Eindeutigkeit.\\
\underline{Zum zweiten Teil, zeige ``$\implies$'':}\\
Sei $u\in C_0$ so, dass 
\begin{align*}
\Vert u-u_0\Vert=\dist(u,C).
\end{align*}
Dann gilt für alle $v\in C$:
\begin{align*}
\Vert u-u_0\Vert^2 
&\leq
\Vert u-v\Vert^2\\
&=\Vert u-u_0+u_0-v\Vert^2\\
&=\Vert u-u_0\Vert^2+2\cdot\Re\big(\langle u-u_0, u_0-v\rangle\big)+\Vert u_0-v\Vert^2\\
&\implies
2\cdot\Re\big(\langle u-u_0, v-u_0\rangle\big)+\Vert u_0-v\Vert^2
\leq\Vert u_0-v\Vert^2
\end{align*}
Ersetze $v$ durch
\begin{align*}
&u_0+t\cdot(v-u_0)\in C,\qquad\mit t\in]0,1]\
&\implies
2\cdot\Re\big(\langle u-u_0,v-u_0\rangle\leq t\cdot\Vert v-u_0\Vert^2\\
&\stackrel{t\to 0^+}{\implies}
\Re\big(\langle u-u_0,v-u_0\rangle\big)\leq0
\end{align*}
\underline{Zum zweiten Teil, zeige ``$\Longrightarrow$'':}\\
Sei $u_0\in C$ so, dass 
\begin{align*}
\Re\big(\langle u-u_0,v-u_0\rangle\big)\leq\qquad\forall v\in C.
\end{align*}
Dann gilt für alle $v\in C$:
\begin{align*}
\Vert u-v\Vert^2
&=\Vert u-u_0\Vert^2+\underbrace{2\cdot\Re\big(\langle u-u_0,u_0,v\rangle\big)}_{\geq0}+\underbrace{\Vert u_0-v\Vert^2}_{\geq0}\\
&\geq
\Vert u-u_0\Vert^2
\end{align*}
\end{proof}

\begin{proof}[Beweis von Theorem \ref{Theorem1.2.9}]\enter
Die Aussagen (a) bis (d) folgen aus den Abschätzungen für akkretive Operatoren vom Typ (ersetze $u$ und $\hat{u}$ durch $J_hf$ und $J_{\hat{h}}\hat{f}$).\\

\underline{Zu (e):}\\
Sei $D:=\overline{\dom(A)}\neq\emptyset$  und $C:=\overline{\conv(D)}$ der Abschluss der konvexen Hülle. Sei $P$ die orthogonale Projektion  auf $C$ (!). Sei $f\in X$, $h>0\mit h\cdot\omega<1$. Dann gilt nach Definition der Resolvente
\begin{align*}
\left( J_h f,\frac{f-J_h f}{h}\right)\in A.
\end{align*}
Also gilt für alle $(\hat{u},\hat{f})\in A$:
\begin{align*}
\Re\left(\left\langle J_h f-\hat{u},\frac{f-J_h f}{h}-\hat{f}\right\rangle\right)+\omega\cdot\big\Vert J_h f-\hat{u}\big\Vert^2\geq0
\end{align*}
(Beachte: $A$ akkretiv $\gdw A$ monoton)\\
Multiplikation mit $h>0$ und Addition von $\hat{u}$ auf beiden Seiten  liefert:
\begin{align*}
(1-h\cdot\omega)\cdot\big\Vert J_h f-\hat{u}\big\Vert^2
&\leq
\Re\left(\left\langle J_h f-\hat{u},f-\hat{u}-h\cdot \hat{f}\right\rangle\right)\\
&\stackrel{\text{C.S.}}{\leq}
\big\Vert J_h f-\hat{u}\big\Vert\cdot\Vert f-\hat{u}-h\cdot\hat{f}\big\Vert\\
&\implies
\limsup\limits_{h\to 0^+}\big\Vert J_h f-\hat{u}\big\Vert<\infty
\end{align*}
Außerdem folgt aus dieser Ungleichung
\begin{align*}
&(1-h\cdot\omega)\cdot\big\Vert J_h f- P(f)\big\Vert^2\\
&=(1-h\cdot\omega)\cdot\left(\big\Vert J_h f-\hat{u}\big\Vert^2+2\cdot\Re\Big(\big\langle J_h f-\hat{u},\hat{u}-P(f)\big\rangle\Big)+\big\Vert\hat{u}-P(f)\big\Vert^2\right)\\
&\leq
\Re\Big(\big\langle J_h f-\hat{u},f-\hat{u}\big\rangle\Big)+\O(h)
+2\cdot\Re\Big(\big\langle J_h f-\hat{u}\underbrace{-P(f)+P(f)}_{=0},\hat{u}-P(f)\big\rangle\Big)
+\big\Vert\hat{u}-P(f)\big\Vert^2\\
&=
\Re\Big(\big\langle J_h f-P(f),f-\hat{u}\big\rangle\Big)
+\Re\Big(\big\langle P(f)-\hat{u},f-\hat{u}\big\rangle\Big)
+2\cdot\Re\Big(\big\langle J_h f-P(f),\hat{u}-P(f)\big\rangle\Big)\\
&\qquad-2\cdot\big\Vert\hat{u}-P(f)\big\Vert^2+\big\Vert\hat{u}-P(f)\big\Vert^2+\O(h)\\
&=\Re\Big(\big\langle J_h h-P(f),f-\hat{u}\big\rangle\Big)
+\underbrace{\Re\Big(\big\langle P(f)-\hat{u},f-P(f)\big\rangle\Big)}_{\leq0}
+2\cdot\Re\Big(\big\langle J_h f-P(f),\hat{u}-P(f)\big\rangle\Big)+\O(h)\\
&\leq
\Re\Big(\big\langle J_h f-P(f),f-\hat{u}\big\rangle\Big)
+2\cdot\Re\Big(\big\langle J_h f-P(f),\hat{u}-P(f)\big\rangle\Big)+\O(h)
\end{align*}
Sei $v\in C$ ein schwacher Häufungspunkt von $(J_h f)_{h\searrow 0}$, d. h.
\begin{align*}
\exists (h_n)_{n\in\N}\mit \limn h_n=0:\weaklim\limits_{n\to\infty} J_{h_n} f=v
\end{align*}
Dann gilt
\begin{align*}
\Vert v-P(f)\Vert^2 
&\leq\liminf\limits_{n\to\infty}\big\Vert J_{h_n} f-P(f)\big\Vert^2\\
&\leq
\Re\Big(\big\langle v-P(f),f-\hat{u}\big\rangle\Big)
+2\cdot\Re\Big(\big\langle v-P(f),\hat{u}-P(f)\big\rangle\Big)
\end{align*}
für alle $\hat{u}\in\overline{\dom(A)}=D$.
Wegen Linearität des Skalarproduktes in der zweiten Komponente gilt diese Ungleichung schließlich für alle $\hat{u}\in\overline{\conv(D)}=C$. Insbesondere auch für $\hat{u}=P(f)$. Somit folgt:
\begin{align*}
\big\Vert v-P(f)\big\Vert^2
\leq
\Re\Big(\big\langle v-P(f),f-P(f)\big\rangle\Big)
\leq0\\
\implies
v=P(f)
\end{align*}
Da $v$ ein beliebiger Häufungspunkt von $(J_h f)_{h\searrow 0}$ war, gilt 
\begin{align*}
\weaklim\limits_{h\to0^+} J_h f=P(f)
\end{align*}
und aus der Abschätzung oben folgt 
\begin{align*}
\limsup\limits_{h\to 0^+}\big\Vert J_h f-P(f)\big\Vert^2
\leq
0,
\end{align*}
d.h. 
\begin{align*}
\lim\limits_{h\to 0^+} J_h f=P(f)
\end{align*}
in Norm. Wegen $J_h f\in\dom(A)$ folgt damit $P(f)\in D=\overline{\dom(A)}\subseteq C$ für alle $f\in X$. Also ist $D=C$ und somit ist $D$ konvex.
\end{proof}

\begin{definition}
Ein Operator $A\subseteq X\times X$ auf einem Banachraum $X$ ist \textbf{maximal akkretiv vom Typ $\omega\in\R$} $:\Longleftrightarrow A$ ist akkretiv vom Typ $\omega$ und $A$ besitzt keine echte akkretive Erweiterung, d. h.
\begin{align*}
B\subseteq X\times X\text{ akkretiv vom Typ $\omega$ und }A\subseteq B
\implies A=B
\end{align*}
$A$ heißt \textbf{maximal akkretiv} $:\Longleftrightarrow A$ ist maximal akkretiv vom Typ 0.
\end{definition}

Aus einer Standardanwendung des Lemmas von Zorn folgt:

\begin{lemma}
Ist $A\subseteq X\times X$ akkretiv vom Typ $\omega$, dann gilt:
\begin{align*}
\exists B\subseteq X\times X: B\text{ maximal akkretiv vom Typ }\omega\mit A\subseteq B
\end{align*}
\end{lemma}
\begin{proof}
Betrachte die Menge aller akkretiven Operatoren, die $A$ enthalten. Diese Menge ist nicht leer. Die Inklusion definiert eine Teilordnung. Aus dem Lemma von Zorn folgt dann die Behauptung.
\end{proof}

\begin{lemma}
Sei $X$ ein Banachraum, $A\subseteq X\times X$ ein Operator. Dann gilt:
\begin{enumerate}[label=(\alph*)]
\item Es gilt %To Do: Diagramm
\begin{enumerate}[label=(\roman*)]
\item $A$ ist $m$-akkretiv vom Typ $\omega\Longleftrightarrow A+\omega\cdot I$ ist $m$-akkretiv
\item $A$ ist $m$-akkretiv vom Typ $\omega\implies A$ ist maximal akkretiv vom Typ $\omega$
\item $A$ ist maximal akkretiv vom Typ $\omega\Longleftrightarrow A+\omega\cdot I$ ist maximal akkretiv
\item $A$ ist maximal akkretiv vom Typ $\omega\implies A$ ist abgeschlossen
 \end{enumerate}
 \item Ist $A$ maximal akkretiv vom Typ $\omega$ und $(u,f)\in X\times X$, dann gilt:
 \begin{align*}
 (u,f)\in A&\Longleftrightarrow\forall(\hat{u},\hat{f})\in A:\big[u-\hat{u},f-\hat{f}\big]+\omega\cdot\Vert u-\hat{u}\Vert\geq0\\
 &\stackrel{X\text{ HilbertR}}{\Longleftrightarrow}
 \forall(\hat{u},\hat{f})\in A:\big\langle u-\hat{u},f-\hat{f}\big\rangle+\omega\cdot\Vert u-\hat{u}\Vert^2\geq0
 \end{align*}
 \item Sei $A$ maximal akkretiv vom Typ $\omega$. Dann ist $Au\subseteq X$ abgeschlossen für alle $u\in X$. Wenn $X$ ein Hilbertraum ist, dann ist $Au$ auch konvex.
\end{enumerate}
\end{lemma}
\begin{proof}
Die Teile (a)(i) und (a)(iii) sind Übung.\\

\underline{Zu (a)(ii):}\\
Sei $A$ $m$-akkretiv ($\omega=0$) und sei $B$ akkretiv mit $A\subseteq B$. Sei $(u,f)\in B$. Aus der Injektivität folgt:
\begin{align}\label{proof1.2.13}\tag{$\ast$}
\exists v\in\dom(A):v+Bv\supseteq v+Av\ni u+f.
\end{align}
Andererseits gilt
\begin{align*}
u+Bu\ni u+f.
\end{align*}
Weil $B$ eine Erweiterung von $A$ ist, folgt aus den Abschätzungen für akkretive Operatoren (Injektivität von $I+B$) schon $u=v$.\\
Setzt man dies in die Inklusion \eqref{proof1.2.13} ein, dann folgt $f\in Au$ bzw. $(u,f)\in A$. Wir haben also $B\subseteq A$ gezeigt und somit $A=B$. Also ist $A$ maximal akkretiv.\\

\underline{Zeige (a)(iv):}\\
Sei $A$ maximal akkretiv (vom Typ $\omega$). Dann ist der Abschluss $\overline{A}$ akkretiv (vom Typ $\omega$) und $A\subseteq\overline{A}$. Weil $A$ maximal ist, ist $A=\overline{A}$. Also ist $A$ abgeschlossen.\\

\underline{Zu (b), zeige ``$\implies$'':}
Dies folgt aus der Definiton von akkretiv.\\

\underline{Zu (b), zeige ``$\Longleftarrow$'':}\\
Nach Voraussetzung ist $B:=A\cup\big\lbrace(u,f)\big\rbrace$ akkretiv vom Typ $\omega$. Weil $A$ maximal ist, ist $A=B$, d.h. $(u,f)\in A$.\\

\underline{Zu (c):}\\
Der erste Teil der Aussage folgt aus (a). Sei außerdem $X$ ein Hilbertraum, $f_0,f_1\in Au$. Dann gilt für alle $(\hat{u},\hat{f})\in A$:
\begin{align*}
\big\langle u-\hat{u},f_0-\hat{f}\big\rangle+\omega\cdot\Vert u-\hat{u}\Vert^2&\geq0\\
\big\langle u-\hat{u},f_1-\hat{f}\big\rangle+\omega\cdot\Vert u-\hat{u}\Vert^2&\geq0
\end{align*}
Multiplikation der ersten Zeile mit $\lambda\in[0,1]$ und der zweiten Zeile mit $(-\lambda)$ und anschließende Addition der beiden Zeilen liefert für alle $(\hat{u},\hat{f})\in A:$
\begin{align*}
\big\langle u-\hat{u},f_\lambda-\hat{f}\big\rangle +\omega\cdot\Vert u-\hat{u}\Vert^2\geq0\mit f_\lambda:=(1-\lambda)\cdot f_0+\lambda\cdot f_1
\end{align*}
Aus (b) folgt $\big(u,f_{\lambda}\big)\in A$, d.h. $f_\lambda\in Au$ Also ist $Au$ konvex.
\end{proof}

\begin{erinnerung}
Sei $X$ Banachraum. Eine Funktion $F:X\to\R\cup\lbrace\infty\rbrace$ heißt \textbf{koerziv}
\begin{align*}
&:\Longleftrightarrow\forall c\in R:\big\lbrace F\leq c\big\rbrace:=\big\lbrace x\in X:F(x)\leq c\big\rbrace\text{ ist beschränkt}\\
&\Longleftrightarrow\lim\limits_{\Vert u\Vert\to\infty} F(u)=+\infty
\end{align*}
\end{erinnerung}

\begin{theorem}[Minimierung konvexer Funktionen]\label{theoremMinimieurngKonvexerFunktionen}\enter
Sei $X$ ein reflexiver Banachraum und sei $F:X\to\R\cup\lbrace\infty\rbrace$ konvex, unterhalbstetig und koerziv. Dann gilt:
\begin{align*}
\exists u_0\in X:F(u_0)=\inf\limits_{x\in X}F(x)
\end{align*}
\end{theorem}
\begin{proof}
Sei $(c_n)_{n\in\N}\subseteq\R$ monoton fallende Folge in $\R$ mit
\begin{align*}
c_n &> \inf\limits_{x\in X} F(x) \qquad \forall n\in\N\\
c_n &\searrow \inf\limits_{x\in X}F(x)
\end{align*}
Sei $K_n:=\big\lbrace F\leq c_n\big\rbrace$. Dann ist $K_n$ nach Voraussetzung konvex, abgeschlossen und beschränkt. Aus Hahn-Banach folgt, dass $K_n$ abgeschlossen in der schwachen Topologie ist. Weil $X$ reflexiv und $K_n$ beschränkt ist, ist $K_n$ schwach kompakt ( = kompakt bzgl. der schwachen Topologie, siehe Banach-Alaoglu). Schließlich ist $K_n$ nichtleer und $K_{n+1}\subseteq K_n$. Damit ist 
\begin{align*}
\bigcap\limits_{n\in\N}K_n\neq\emptyset.
\end{align*}
Jedes Element $u\in K_n$ ist Minimierer von $F$.
\end{proof}

\begin{theorem}[Minty]\enter
Ein Operator $A\subseteq H\times H$ auf einem Hilbertraum $H$ ist $m$-akkretiv (vom Typ $\omega$) $\Longleftrightarrow\\ A$ ist maximal akkretiv (vom Typ $\omega$).
\end{theorem}
\begin{proof}[Beweis (hier nur der Fall, dass $H$ reeller Hilbertraum ist)]\enter
\underline{Zeige ``$\implies$'':} Folgt aus Lemma oben.\\

\underline{Zeige ``$\Longleftarrow$'':}\\
Sei $A\subseteq H\times H$ maximal akkretiv ($\omega:=0$) (insbesondere ist $A\neq\emptyset$). Betrachte die Funktion 
\begin{align*}
&F_A:H\times H\to\R\cup\lbrace+\infty\rbrace\\
&F_A(u,f):=\sup\Big\lbrace\langle u,\hat{f}\rangle+\langle\hat{u},f\rangle-\langle\hat{u},\hat{f}\rangle:(\hat{u},\hat{f})\in A\Big\rbrace>-\infty
\end{align*}
Beobachtung: 
\begin{align*}
\forall (u,f),(\hat{u},\hat{f})\in A:
&\big\langle u-\hat{u},f-\hat{f}\big\rangle\geq0\\
&\langle u,f\rangle-\Big(\big\langle u,\hat{f}\big\rangle+\big\langle\hat{u},f\big\rangle-\big\langle\hat{u},\hat{f}\big\rangle\Big)\geq0
\end{align*}
Als punktweises Supremum von stetigen, affinen Funktionen (nämlich 
\begin{align*}
(u,f)\mapsto\big\langle u,\hat{f}\big\rangle+\big\langle\hat{u},f\big\rangle-\big\langle\hat{u},\hat{f}\big\rangle~\big)
\end{align*}
ist $F_A$ unterhalbstetig und konvex. Außerdem gilt:
\begin{align*}
\forall (u,f)\in H\times H:F_A(u,f)\geq\langle u,f\rangle
\end{align*}
mit Gleichheit genau dann, wenn $(u,f)\in A$. Verwende hierbei, dass $A$ \ul{maximal} monoton ist und Lemma (b) oben. Die Funktion
\begin{align*}
F&:=F_a+\Vert(\cdot,\cdot)\Vert^2_{H\times H}\text{ d. h. }\\
F(u,f)&:=F(u,f)+\Vert u\Vert^2+\Vert f\Vert^2
\end{align*}
ist damit unterhalbstetig, konvex. Außerdem koerziv, denn: Sei $(u,f)\in\lbrace F\leq c\rbrace$. Dann ist
\begin{align*}
c\geq F(u,f)
&\geq\langle u,f\rangle+\Vert u\Vert^2+\Vert f\Vert^2\\
&\stackrel{\text{C.S.}}{\geq}-\Vert u\Vert\cdot\Vert f\Vert+\Vert u\Vert^2+\Vert f\Vert^2\\
&=\frac{1}{2}\cdot\big(\Vert u\Vert^2-2\cdot\Vert u\Vert\cdot\Vert f\Vert+\Vert f\Vert^2\big)+\frac{1}{2}\cdot\Vert u\Vert^2+\frac{1}{2}\cdot\Vert f\Vert^2\\
&=\frac{1}{2}\cdot\big(\underbrace{\Vert u\Vert-\Vert f\Vert}_{}\big)^2+\frac{1}{2}\cdot\Vert u\Vert^2+\frac{1}{2}\cdot\Vert f\Vert^2\\
&\geq\frac{1}{2}\cdot\Vert u\Vert^2+\frac{1}{2}\cdot\Vert f\Vert^2\\
&=\frac{1}{2}\cdot\big\Vert (u,f)\big\Vert^2_{H\times H}
\end{align*}
Also ist $\lbrace F\leq c\rbrace$ beschränkt. Aus dem Theorem über Minimierung konvexer Funktionen folgt die Existenz eines Elements $(u_0,f_0)\in H\times H$ mit
\begin{align*}
F(u_0,f_0)=\inf\limits_{x\in H\times H}F(x)<\infty
\end{align*}
Da $F_A$ konvex ist, ist 
\begin{align*}
[0,\infty)\to\R\cup\lbrace+\infty\rbrace,\qquad
\lambda\mapsto F_A\big(u_0+\lambda\cdot(u-u_0)\big)
\end{align*}
konvex. Somit ist 
\begin{align*}
\lambda\mapsto\frac{F_A\big(u_0+\lambda\cdot(u-u_0),f_0+\lambda\cdot(f-f_0)\big)-F_A(u_0,f_0)}{\lambda}
\end{align*}
monoton wachsend. Also gilt für diesen Quotienten für $\lambda=1$:
\begin{align*}
&F_A(u,f)-F_A(u_0,f_0)\\
&\geq\lim\limits_{\lambda\to 0^+}\frac{F_A\big(u_0+\lambda\cdot(u-u_0),f_0+\lambda\cdot(f-f_0)\big)-F_A(u_0,f_0)}{\lambda}\\
&=\lim\limits_{\lambda\to 0^+}\Bigg(\underbrace{\frac{F\big(u_0+\lambda\cdot(u-u_0),f_0+\lambda\cdot(f-f_0)\big)-F(u_0,f_0)}{\lambda}}_{\geq0}\\
&\qquad+\frac{1}{2}\cdot\frac{\Vert u_0\Vert^2+\Vert f_0\Vert^2-\Vert u_0+\lambda\cdot(u-u_0)\Vert^2-\Vert f_0+\lambda\cdot(f-f_0)\Vert^2}{\lambda}\Bigg)\\
&\geq\frac{1}{2}\cdot\lim\limits_{\lambda\to 0^+}\\
&\qquad\qquad\frac{-2\cdot\big\langle u_0,\lambda\cdot(u-u_0)\big\rangle-\lambda^2\cdot\Vert u-u_0\Vert^2-2\cdot\big\langle f_0,\lambda\cdot(f-f_0)\big\rangle-\lambda^2\cdot\Vert f-f_0\Vert^2}{\lambda}\\
&=-\big\langle u_0,u-u_0\big\rangle-\big\langle f_0,f-f_0\big\rangle\qquad\forall (u,f)\in H\times H\\
&\implies
\forall(u,f)\in A:
F_A(u,f)\stackeq{(u,f)\in A}\langle u,f\rangle\geq\langle u_0,f_0\rangle-\langle u_0,u-u_0\rangle-\langle f_0,f-f_0\rangle\\
&\implies
\langle u,f\rangle +\langle u_0,f_0\rangle+\langle u_0,u\rangle+\langle f_0,f\rangle\geq\Vert u_0+f_0\Vert^2\geq0\\
&\implies
\big\langle u-(-f_0),f-(-u_0)\big\rangle=
\big\langle u+f_0,f+u_0\big\rangle\geq\big\Vert u_0+f_0\Vert^2\geq0
\end{align*}
Aus der Maximalität von $A$ folgt $(-f_0,-u_0)\in A$ und dann auch $u_0=-f_0$, d.h.\\ $(u_0,-u_0)\in A$.
\begin{align*}
(u_0,-u_0)\in A
&\Longleftrightarrow Au_0\ni -u_0\\
&\Longleftrightarrow u_0+Au_0\ni 0
\end{align*}
Also ist $0\in\rg(I+A)$. Ersetzt man nun den Operator $A$ durch
\begin{align*}
A-\hat{f}=\big\lbrace(u,f-\hat{f}):(u,f)\in A\big\rbrace\qquad\forall \hat{f}\in H,
\end{align*}
dann erhält man
\begin{align*}
u_0+Au_0\ni\hat{f}\text{ (für ein anderes $u_0$)},
\end{align*}
d.h. $\hat{f}\in\rg(I+A)~\forall\hat{f}\in H$. Damit ist $A$ $m$-akkretiv.
\end{proof}

\begin{bemerkung}
Der Satz von Minty ist falsch in Banachräumen. Es gibt ein Gegenbeispiel im $\R^2$ mit der $p$-Norm, $p\in(1,\infty)\setminus\lbrace2\rbrace$ (Grandall-Liggest 1973).
\end{bemerkung}

\section{Subgradienten}
In diesem Abschnitt sei $H$ ein reeller Hilbertraum und $\E:H\to\R\cup\lbrace+\infty\rbrace$ eine konvexe, unterhalbstetige Funktion $\E\not\equiv+\infty$ mit
\begin{align*}
\dom(\E):=\big\lbrace\E<+\infty\big\rbrace=\big\lbrace u\in H:\E(u)<+\infty\big\rbrace
\end{align*}
bezeichnen wir den \textbf{effektiven} Definitionsbereich ($\dom(\E)\neq\emptyset$).\\
Der \textbf{Subgradient} von $\E$ ist der Operator
\begin{align*}
\partial\E&:=\left\lbrace (u,f)\in H\times H\mid u\in\dom(\E)\wedge\forall v\in H:\lim\limits_{\lambda\to0^+}\frac{\E(u+\lambda\cdot v)-\E(u)}{\lambda}\geq\langle f,v\rangle\right\rbrace\\
&=\Big\lbrace(u,f)\in H\times H\mid u\in\dom(\E)\wedge\forall v\in H:\E(u+v)-\E(u)\geq\langle f,v\rangle\Big\rbrace
\end{align*}
Bei der Gleichheit wird verwendet, dass die Funktion $\lambda\mapsto\frac{\E(u\lambda\cdot v)-\E(u)}{\lambda}$ monoton wachsend ist.\\
Ziel: $\partial\E$ ist $m$-akkretiv.

\begin{lemma}
$\partial\E$ ist akkretiv ( = monoton).
\end{lemma}
\begin{proof}
Seien $(u,f),(\hat{u},\hat{f})\in\partial\E$. Dann gilt $u,\hat{u}\in\dom(\E)$ und wegen der Definition von $\partial\E$ auch
\begin{align*}
\E(\hat{u})-\E(u)&\geq\big\langle f,\hat{u}-u\big\rangle\\
\E(u)-\E(\hat{u})&\geq\big\langle \hat{f},u-\hat{u}\big\rangle=\big\langle-\hat{f},\hat{u}-u\big\rangle\\
&\stackrel{\text{addieren}}{\implies}
0\geq\big\langle f-\hat{f},\hat{u}-u\big\rangle
\text{ bzw. }
\big\langle u-\hat{u},f-\hat{f}\big\rangle\geq0
\end{align*}
\end{proof}

\begin{lemma}\label{lemma1.3.2}
\begin{align*}
\exists c\geq0:\forall u\in H:\E(u)\geq -c\cdot\big(1+\Vert u\Vert\big)
\end{align*}
\end{lemma}
\begin{proof}[Beweis (durch Widerspruch).]\enter
O.B.d.A. $0\in\dom(\E)$ und $\E(0)=0$. Angenommen es gibt eine Folge $(u_n)_{n\in\N}$ in $H$ mit
\begin{align*}
\E(u)\leq -n\cdot\big(1+\Vert u_n\Vert\big)\leq -n\cdot\Vert u_n\Vert.
\end{align*}
Insbesondere ist $u_n\in\dom(\E)$. Für $\lambda_n\in]0,1]$ gilt wegen Konvexität,
\begin{align*}
\E(\lambda_n\cdot u_n)
&=\E\big(\lambda_n\cdot u_n+(1-\lambda_n)\cdot 0\big)\\
&\leq\lambda_n\cdot\E(u_n)+\underbrace{(1-\lambda_n)\cdot\E(0)}_{=0}\\
&\leq- n\cdot\Vert\lambda_n\cdot u_n\Vert
\end{align*}
Wähle $\lambda_n$ so, dass $\limn\lambda_n\cdot u_n=0$ und $\limn\big(-n\cdot\Vert\lambda_n\cdot u_n\Vert\big)=-\infty$. Dies liefert einen Widerspruch zur Unterhalbstetigkeit von $\E$, denn
\begin{align*}
0=\E(0)\leq\liminf\limits_{n\to\infty}\E(\lambda_n\cdot u_n)=-\infty
\end{align*}
\end{proof}

\begin{lemma}
Für alle $\lambda>0$ und alle $f\in H$ ist
\begin{align*}
H\to\R\cup\lbrace+\infty\rbrace,\qquad u\mapsto\E(u)+\frac{1}{2\cdot\lambda}\cdot\Vert u-f\Vert^2
\end{align*}
konvex, unterhalbstetig und koerziv.
\end{lemma}
\begin{proof}
\underline{Zur Konvexität:}\\
Die Norm ist konvex und der Vorfaktor ist konvex. Da $\E$ konvex ist und die Summe konvexer Funktionen wieder konvex ist, folgt Konvexität.\\

\underline{Zur Unterhalbstetigkeit:} Ist klar.\\

\underline{Zur Koerzivität:}\\
Sei $v\in\R$. Dann gilt für $u\in H$ mit
\begin{align*}
c\geq\E(u+\frac{1}{2\cdot \lambda}\cdot\Vert u-f\Vert^2
\end{align*}
schon
\begin{align*}
c&\stackrel{\ref{lemma1.3.2}}{\geq} -\big(1+\Vert u\Vert\big)+\frac{1}{2\cdot\lambda}\cdot\Vert u-f\Vert^2\\
&\geq -r-r\cdot\big(\Vert u-f\Vert+\Vert f\Vert\big)+\frac{1}{2\cdot\lambda}\cdot\Vert u-f\Vert^2
\end{align*}
bzw.
\begin{align*}
c+r+r\cdot\Vert f\Vert &\geq -r\cdot\Vert u-f\Vert+\frac{1}{2\cdot\lambda}\cdot\Vert u-f\Vert^2\\
&\stackeq{\text{Q.E.}}\left(\frac{1}{\sqrt{2\cdot\lambda}}\cdot\Vert u-f\Vert-\frac{r\cdot\sqrt{2\cdot\lambda}}{2}\right)^2-\frac{r^2\cdot 2\cdot\lambda}{4}
\end{align*}
Umstellen liefert
\begin{align*}
\Vert u-f\Vert&\leq\sqrt{2\cdot\lambda}\cdot\sqrt{c+r+r\cdot\Vert f\Vert+\frac{r^2\cdot\lambda}{2}}+r\cdot\lambda=:R(c,\Vert f\Vert,\lambda, r)
\end{align*}
Folglich ist $u\in\overline{B}(f,R)$ und somit $\left\lbrace\E(u)+\frac{1}{2\cdot\lambda}\cdot\Vert u-f\Vert^2\leq c\right\rbrace$ ist beschränkt.
\end{proof}

\begin{theorem}
Sei $H$ ein Hilbertraum. Für jede konvexe, unterhalbstetige Funktion $\E:H\to\R\cup\lbrace+\infty\rbrace,~\E\not\equiv+\infty$ ist der Subgradient $\partial\E$ $m$-akkretiv.
\end{theorem}
\begin{proof}
Das $\partial\E$ akkretiv ist, wurde in einem Lemma oben bereits gezeigt.\\
Seien $h>0$ und $f\in H$. Damit ist die Funktion
\begin{align*}
v\mapsto \E(v)+\frac{1}{2\cdot h}\cdot\Vert v-f\Vert^2
\end{align*}
konvex, unterhalbstetig und koerziv. Aus dem Theorem über Minimieurng konvexer Funktionen \ref{theoremMinimieurngKonvexerFunktionen} folgt, dass diese Funktion ein globales Minimum $u\in H$ besitzt. Für alle $v\in H$ und alle $\lambda>0$ gilt dann
\begin{align*}
\E(u+\lambda\cdot v)+\frac{1}{2\cdot h}\cdot\big\Vert u+\lambda\cdot v-f\big\Vert^2
&\geq\E(u)+\frac{1}{2\cdot h}\cdot\Vert u-f\Vert^2
\end{align*}
bzw. nach Umstellen 
\begin{align*}
\E(u+\lambda\cdot v)-\E(u)
&\geq\frac{1}{2\cdot h}\left(\Vert u-f\Vert^2-\Vert u+\lambda\cdot v-f\Vert^2\right)\\
&=\frac{1}{2\cdot h}\cdot\left(\Vert u-f\Vert^2-\Vert u-f\Vert^2-2\cdot\langle u-f,\lambda\cdot v\rangle-\Vert \lambda\cdot v\Vert^2\right)\\
&=\frac{1}{2\cdot h}\cdot\left(-2\cdot\langle u-f,\lambda\cdot v\rangle-\Vert \lambda\cdot v\Vert^2\right)
\end{align*}
bzw.
\begin{align*}
\lim\limits_{\lambda\to 0^+}\frac{\E/u+\lambda\cdot v)-\E(u)}{\lambda}=\left\langle\frac{f-u}{h},v\right\rangle
\end{align*}
d.h. 
\begin{align*}
&\left( u,\frac{f-u}{h}\right)\in\partial\E\\
&\Longleftrightarrow\partial\E(u)\ni\frac{f-u}{h}\\
&\Longleftrightarrow u+h\cdot\partial\E(u)\ni f
\end{align*}
Da $h$ und $f$ beliebig waren, folgt
\begin{align*}
\rg(I+h\cdot\partial\E)=H\qquad\forall h>0\\
\implies\partial\E\text{ ist $m$-akkretiv}
\end{align*}
\end{proof}

\begin{beispiel}\
\begin{enumerate}[label=(\arabic*)]
\item $H=\R$ und $\E(u):=|u|$
%TODO Hier Skizzen einfügen
In $u=0$ ist die rechtsseitige Ableitung -1, die linksseitige Ableitung ist 1.
\begin{align*}
\partial\E=\big((-\infty,0]\times\lbrace-1\rbrace\big)\cup\big(\lbrace0\rbrace\times[-1,1]\big)\cup\big([0,\infty)\times\lbrace1\rbrace\big)
\end{align*}
\item Sei $I=[a,b]\subseteq\R$.
%TODO Hier Skizze einfügen
\begin{align*}
\E(x):=\left\lbrace\begin{array}{cl}
+\infty, &\falls x\not\in I\\
0, &\falls x\in I
\end{array}\right.\\
\implies
\partial\E=\big(\lbrace a\rbrace\times[0,\infty)\big)\cup \big(I\times\lbrace 0\rbrace\big)\cup\big(\lbrace b\rbrace\times[0,\infty)\big)
\end{align*}
\end{enumerate}
\end{beispiel}

\begin{beispiel}[$p$-Laplace-Operator]\enter
Sei $\Omega\subseteq\R^n$ offen. Für $u\in C(\Omega)$ deifnieren wir den \textbf{Träger}
\begin{align*}
\supp(u):=\overline{\big\lbrace x\in\Omega~\big|~u(x)\neq0\big\rbrace}^\Omega
\end{align*}
(Wichtig: Nehme den Abschluss in $\Omega$, nicht in $\R^n$.)\\
Beispiele für $\Omega=]0,1[$
\begin{itemize}
\item $u_1:\Omega\to\R,~x\mapsto 1\implies\supp(u_1)=]0,1[\implies u_1\not\in C_c(\Omega)$
\item $u_2:\Omega\to\R,~x\mapsto x^2-1\implies\supp(u_2)=]0,1[\implies u_2\not\in C_c(\Omega)$
\end{itemize}
\begin{align*}
C_c(\Omega)&:=\big\lbrace u\in C(\Omega):\supp(u)\text{ kompakt}\big\rbrace\\
C_c^k(\Omega)&:=C_c(\Omega)\cap C^k(\Omega)\qquad\forall k\in\N\cup\lbrace\infty\rbrace\\
L^1_{\text{loc}}&:=\Big\lbrace u:\Omega\to\C\text{ messbar }~\big|~\forall K\subseteq\Omega\text{ kompakt}:u\big|_K\in L^1(K)\Big\rbrace
\end{align*}
Es gilt
\begin{align*}
C_c^\infty(\Omega)\subseteq C^1_c(\Omega)\subseteq C_c(\Omega)\begin{array}{l}
\subseteq C(\Omega)\subseteq\\
\subseteq L^p(\Omega)\subseteq
\end{array} L^1_{\text{loc}}(\Omega)\qquad\forall 1\leq p\leq\infty
\end{align*}
Eine Funktion $u\in L^1_{\text{loc}}(\Omega)$ heißt \textbf{schwach partiell nach $x_i$ differenzierbar}
\begin{align*}
:\Longleftrightarrow\exists g\in L^1_{\text{loc}}:\forall\varphi\in C^\infty_c(\Omega):\int\limits_\Omega u(\omega)\cdot\frac{\partial\varphi(\omega)}{\partial x_i}\d\omega=-\int\limits_\Omega g(\omega)\cdot\varphi(\omega)\d\omega
\end{align*}
Schreibweise: $g=:\frac{\partial u}{\partial x_i}$ ist die \textbf{schwache partielle Ableitung} nach $x_i$\\
Ohne Beweis: $g=\frac{\partial u}{\partial x_i}$ ist eindeutig bestimmt.\\
Die Funktion $u\in L^1_{\text{loc}}(\Omega)$ heißt \textbf{schwach differenzierbar} $:\gdw$ sie für alle $i\in\lbrace1,\ldots,n\rbrace$ schwach partiell nach $x_i$ differenzierbar ist. In diesem Fall schreiben wir
\begin{align*}
\nabla u=\begin{pmatrix}
\frac{\partial u}{\partial x_1}\\
\vdots\\
\frac{\partial u}{\partial x_n}
\end{pmatrix}
\end{align*}
Betrachte nun auf dem Hilbertraum $H:=L^2(\Omega)$ und für $p\in]1,\infty[$ die Funktion
\begin{align*}
\E(u):=\left\lbrace\begin{array}{cl}
\frac{1}{p}\cdot\int\limits_\Omega |\nabla u|^p,\d\omega, &\falls \nabla u\in L^p(\Omega,\R^n)\\
+\infty, &\sonst
\end{array}\right.
\end{align*}
Beachte, dass $|\nabla u|$ die euklidische Norm des Gradienten meint. Außerdem ist mit $\nabla u\cdot\nabla v$ das euklidische Skalarprodukt in $\R^n$ gemeint.

\begin{lemma} %innerhalb des Beispiels
Diese Funktion $\E$ ist konvex, unterhalbstetig, $\not\equiv0$ und es gilt
\begin{align*}
\dom(\E)=\Big\lbrace u\in L^2(\Omega)~\Big|~\nabla u\in L^p\big(\Omega,\R^n\big)\Big\rbrace
\end{align*}
und der Subgradient
\begin{align*}
\partial\E=\left\lbrace(u,f)\in L^2(\Omega)\times L^2(\Omega)~\left|~u\in\dom(\E)\wedge\forall v\in\dom(\E):\int\limits_\Omega|\nabla u|^{p-2}\cdot\nabla u\cdot\nabla v=\int\limits_\Omega f\cdot v\right.\right\rbrace
\end{align*}
\end{lemma}
\begin{proof}
Die euklidische Norm $|\cdot|:\R^n\to\R_{\geq0}$ ist konvex (wegen Dreiecksungleichung) und die Funktion
\begin{align*}
\R_{\geq0}\to\R_{\geq0},\qquad s\mapsto s^p
\end{align*}
ist konvex und monoton wachsend ($p\geq1$!) und damit ist die Verknüpfung
\begin{align*}
\R^n\to\R_{\geq0},\qquad x\mapsto |x|^p
\end{align*}
konvex, und schließlich ist $\E$ konvex.\\
$\E\not\equiv0$, denn $\E(0)=0$.\\

\underline{Zeige $\E$ ist unterhalbstetig:}\\
Sei $(u_n)_{n\in\N}\subseteq L^2(\Omega)\mit u:=\limn u_n$ in $L^2(\Omega)$. Zu zeigen:
\begin{align*}
\E(u)\leq\liminf\limits_{n\to\infty}\E(u_n)
\end{align*}
Sei O.B.d.A. (nach Auswahl von Teilfolge):
\begin{align*}
\liminf\limits_{n\to\infty}\E(u_n)<+\infty
\quad\text{und}\quad
\liminf\limits_{n\to\infty}\E(u_n)=\limn\E(u_n)
\quad\text{und}\quad
\E(u_n)\leq C\quad\forall n\in\N
\end{align*}
Dann ist $(\nabla u_n)_{n\in\N}$ beschränkte Folge in $L^p(\Omega,\R^n)$. Der Raum $L^p(\Omega,\R^n)$ ist reflexiv (beachte $1<p<\infty$). Die Folge $(u_n)_{n\in\N}$ besitzt also eine Teilfolge (wieder mit $(u_n)_{n\in\N}$ bezeichnet), sodass $(\nabla u_n)_{n\in\N}$ schwach in $L^p(\Omega,\R^n)$ konvergiert. Sei
\begin{align*}
&g=\begin{pmatrix}
g_1\\ \vdots\\ g_n
\end{pmatrix}:=\weaklim\limits_{n\to\infty}\nabla u_n\\
\Longleftrightarrow\forall i\in\lbrace1,\ldots,n\rbrace,\forall v\in L^{p'}(\Omega)\mit\frac{1}{p}+\frac{1}{p'}=1:
\limn\int\limits_\Omega\frac{\partial u_n}{\partial x_i}\cdot v=\int\limits_\Omega g_i\cdot v
\end{align*}
Zu Erinnerung (schwache Ableitung):
\begin{align*}
\forall i\in\lbrace 1,\ldots,n\rbrace,\forall\varphi\in C_c^\infty(\Omega),\forall n\in\N:
\int\limits_\Omega u_n\cdot\frac{\partial\varphi}{\partial x_i}=-\int\limits_\Omega\frac{\partial u_n}{\partial x_i}\cdot\varphi
\end{align*}
Daraus folgt wegen $C_c^\infty(\Omega)\subseteq L^{p'}(\Omega)$ dann
\begin{align*}
\forall i\in\lbrace1,\ldots,n\rbrace,\forall\varphi\in C_c ^\infty(\Omega):
\int\limits_\Omega u\cdot\frac{\partial\varphi}{\partial x_i}
&=\limn\int\limits_\Omega u_n\cdot\frac{\partial\varphi}{\varphi x_i}\\
&=\limn\left(-\int\limits_\Omega\frac{\partial u_n}{\partial x_i}\cdot\varphi\right)\\
&=-\int\limits_\Omega g_i\cdot\varphi
\end{align*}
Damit ist $u$ schwach differenzierbar und 
\begin{align*}
\nabla u=g\in L^p(\Omega,\R^n)
\end{align*}
Also ist $u\in\dom(\E)$. Allgemein gilt:\\
Ist $X$ ein Banachraum und sei $x_n,x\in X,n\in\N$, dann gilt:
\begin{align*}
x=\weaklim\limits_{n\to\infty} x_n\implies\Vert x\Vert\leq\liminf\limits_{n\to\infty}\Vert x_n\Vert=:c
\end{align*}
\begin{proof}
Die Kugel $\overline{B}(0,c+\varepsilon)$ ist normabgeschlossen und konvex. Nach Hahn-Banach ist diese Kugel dann auch schwach abgeschlossen.
%TODO Hier Skizze einfügen
Damit ist $x\in\overline{B}(0,c+\varepsilon)$ für alle $\varepsilon>0$ (also $\Vert x\Vert\leq c+\varepsilon$) und schließlich ist $\Vert x\Vert\leq c$.
\end{proof}
Aus diesem allgemeinen Prinzip folgt:
\begin{align*}
\Vert\nabla u\Vert_{L^p}&=\Vert g\Vert_{L^p}\leq\liminf\limits_{n\to\infty}\Vert \nabla u_n\Vert_{L^p}\\
\implies
\E(u)=\frac{1}{p}\cdot\Vert\nabla u\Vert_{L^p}^p&=\frac{1}{p}\cdot\Vert g\Vert_{L^p}^p\leq\liminf\limits_{n\to\infty}\underbrace{\frac{1}{p}\cdot\Vert \nabla u_n\Vert_{L^p}^p}_{=\E(u_n)}
\end{align*}
\underline{Charakterisierung von $\partial\E$:} 
Für alle $u,v\in\dom(\E)$ gilt:
\begin{align*}
\lim\limits_{\lambda\to 0^+}\frac{\E(u+\lambda\cdot v)-\E(u)}{\lambda}&=\lim\limits_{\lambda\to 0^+}\int\limits_\Omega\frac{\big|\nabla(u+\lambda\cdot v)\big|^p-\big|\nabla u\big|^p}{\lambda}\\
&\stackeq{!!}
\int\limits_\Omega|\nabla u|^{p-2}\cdot\nabla u\cdot\nabla v
\end{align*}
Ist $(u,f)\in\partial\E$, dann ist $u\in\dom(\E)$ und für alle $v\in L^2(\Omega)$ gilt
\begin{align*}
\lim\limits_{\lambda\to 0^+}\frac{\E(u+\lambda\cdot v)-\E(u)}{\lambda}&\geq\langle f,u\rangle_{L^2}
=\int\limits_\Omega f\cdot v\d x
\end{align*}
Ist $v\not\in\dom(\E)$, dann ist $u+vnot\in\dom(\E)$ für alle $\lambda>0$, d. h.
\begin{align*}
\E(u+\lambda\cdot v)=+\infty\qquad\forall\lambda>0
\end{align*}
und somit immer
\begin{align*}
+\infty=\lim\limits_{\lambda\to 0^+}\frac{\E(u+\lambda\cdot v)-\E(u)}{\lambda}\geq\langle f,v\rangle
\end{align*}
(dieser Fall ist damit uninteressant). Ist $v\in\dom(\E)$, dann ist 
\begin{align*}
\int\limits_\Omega |\nabla u|^{p-2}\cdot\nabla u\cdot\nabla v\geq\int\limits_\Omega f\cdot v
\end{align*}
Ersetze hier $v$ durch $-v$:
\begin{align*}
-\int\limits_\Omega|\nabla u|^{p-2}\cdot\nabla u\cdot\nabla v&\geq-\int\limits f\cdot v\\
\implies
-\int\limits_\Omega |\nabla u|^{p-2}\cdot\nabla u\cdot\nabla v&=\int\limits_\Omega f\cdot v
\end{align*}
Damit ist 
\begin{align*}
\partial\E\subseteq\left\lbrace(u,f)\in L^2(\Omega)\times L^2(\Omega)~\left|~u\in\dom(\E)\wedge\forall v\in\dom(\E):\int\limits_\Omega|\nabla u|^{p-2}\cdot\nabla u\cdot\nabla v=\int\limits_\Omega f\cdot v\right.\right\rbrace
\end{align*}
Da die Inklusion ``$\supseteq$'' klar ist, folgt die Behauptung.
\end{proof}
Spezialfälle:\\
$\Omega=(0,1)$ (offenes Intervall, $n=1$) und $p=2$, d. h. $H=L^2((0,1))$ und 
\begin{align*}
\E(u)&=\left\lbrace\begin{array}{cl}
\frac{1}{2}\cdot\int\limits_0^1|u'|^2,&\falls u'\in L^2((0,1))\\
+\infty, &\sonst
\end{array}\right.\\
\partial\E&=\left\lbrace(u,f)\in L^2(0,1)\times L^2(0,1)~\left|~ u\in H^1(0,1)\wedge\forall v\in H^1(0,1):\int\limits_0^1 u'\cdot v'=\int\limits_0^1 f\cdot v\right.\right\rbrace\\
&\text{wobei } H^1(0,1):=H^1((0,1)):=\dom(\E)=\big\lbrace u\in L^2(0,1)~\big|~u'\in L^2(0,1)\big\rbrace
\end{align*}
Erinnerung (schwache Ableitung)\\
$u\in L^1_{\text{loc}}\big((0,1)\big)$ ist schwach differenzierbar
\begin{align*}
:\Longleftrightarrow\exists g\in L^1_{\text{loc}}\big((0,1)\big):\forall\varphi\in C_c^\infty\big((0,1)\big):
\int\limits_0^1 u(x)\cdot\varphi'(x)\d x=-\int\limits_0^1 g(x)\cdot\varphi(x)\d x
\end{align*}
Im diesem Fall heißt $g=:u'$ \textbf{schwache Ableitung} von $u$. Damit ist für $(u,f)\in\partial\E$ die Ableitung $u'\in L^2(0,1)$ schwach differenzierbar und $u'':=(u')'=-f\in L^2(0,1)$ ($u$ ist zweimal schwach differenzierbar und $u,u',u''\in L^2(0,1)$).
\begin{align*}
\partial\E(u)\hat{=}-u''
\end{align*}
Ohne Beweis: Jede Funktion $u\in H^1(0,1)$ besitzt genau einen stetigen Repräsentanten ($H^1(0,1)\subseteq C\big([0,1]\big)$ !!!) und für alle $u,v\in H^1(0,1)$ gilt:
\begin{align*}
\int\limits_0^1 u\cdot v'=u(1)\cdot v(1)-u(0)\cdot v(0)-\int\limits_0^1 u'\cdot v
\end{align*} 
(Hauptsatz für Differential und Integralrechnung (HDI) für schwach differenzierbare Funktionen)\\
Angewandt auf $u'\in H^1(0,1)$ ergibt sich 
\begin{align*}
\int\limits_0^1 u'\cdot v'=u'(1)\cdot v'(1)-u'(0)\cdot v'(0)\underbrace{-\int\limits_0^1 u''\cdot v}_{=+\int\limits_0^1 f\cdot v}\qquad\forall v\in H^1(0,1)
\end{align*}
Es gilt nach Charakterisierung des Subgradienten auch 
\begin{align*}
u'\cdot v'=\int\limits_0^1 f\cdot v
\end{align*}
Also gilt für $(u,f)\in \partial\E$:
\begin{align*}
u'(1)\cdot v'(1)-u'(0)\cdot v'(0)=0\quad\text{und}\quad -u''\equiv f
\qquad\forall v\in H^1(0,1)
\end{align*}
Betrachte nun die Funktion $v(x):=x$ oder $v(x):=1-x$. Dann erhält man 
\begin{align*}
\left\lbrace\begin{array}{rll}
-u'' &\equiv f &\text{ auf }(0,1)\\
u'(0)&=0\\
u'(1)&=0
\end{array}\right.
\end{align*}
Schließlich gilt:
\begin{align*}
\partial\E=\big\lbrace(u,f)\in L^2(0,1)\times L^2(0,1)~\big|~u\in H^1(0,1),u'\in H^1(0,1)\wedge -u''\equiv f\wedge \underbrace{u'(0)=u'(1)=0}_{\text{(hom.) Neumann-RB}}\big\rbrace
\end{align*}
(negativer Laplace-Operator mit Neumann-Randbedingungen)\\
Der Fall $\Omega=(0,1)$ und $p\in(1,\infty)$:
\begin{align*}
H&=L^2(0,1)\text{ und }\\
\E(u)&=\left\lbrace\begin{array}{cl}
\frac{1}{p}\cdot\int\limits_0^1 |u'|^p, &\falls u'\in L^p(0,1)\\
+\infty, &\sonst
\end{array}\right.\\
\dom(\E)&=\big\lbrace u\in L^2(0,1)~\big|~u'\in L^p(0,1)\big\rbrace\\
\partial\E&=\left\lbrace(u,f)\in L^2(0,1)\times L^2(0,1)\left|u\in\dom(\E)\wedge\forall v\in\dom(\E):\int\limits_0^1|u'|^{p-2}\cdot u'\cdot v'=\int\limits_0^1 f\cdot v\right.\right\rbrace
\end{align*}
Aus $(u,f)\in\partial\E$ folgt wie oben, dass $|u'|^{p-2}\cdot u'$ schwach differenzierbar ist und\\ $\left(|u'|^{p-2}\cdot u'\right)'\equiv -f$. Außerdem folgt mit HDI:
\begin{align*}
\int\limits_0^1 f\cdot v
&=\int\limits_0^1|u'|^{p-2}\cdot u'\cdot v'\\
&\stackeq{\text{HDI}}
\big| u'(1)\big|^{p-2}\cdot u'(1)\cdot v(1)-
\big| u'(0)\big|^{p-2}\cdot u'(0)\cdot v(0)+\int\limits_0^1 f\cdot v
\end{align*}
d.h. $\big|u'(1)\big|^{p-2}\cdot u'(1)=0$ und $\big|u'(0)\big|^{p-2}\cdot u'(0)=0$ bzw. $u'(0)=u'(1)=0$.
\begin{align*}
\partial\E&=\left\lbrace(u,f)\in L^2(0,1)\times L^2(0,1)~\left|~
\begin{array}{c}
u\in\dom(\E),|u'|^{p-2}\cdot u'\text{ schwach diffbar und}\\-\left(|u'|^{p-2}\cdot u'\right)\equiv f\wedge u'(0)=u'(1)=0
\end{array}
\right.\right\rbrace
\end{align*}
Dies ist wieder die (homogene) Neumann-Randbedingung (negativer $p$-Laplace-Operator mit Neumann-Randbedingungen).\\
Der Fall $\Omega\subseteq\R^n$ offen und $p=2$
\begin{align*}
\partial\E=\Bigg\lbrace(u,f)\in L^2(\Omega)\times L^2(\Omega)\Bigg|\overbrace{\nabla u\in L^2(\Omega,\R^n)}^{\gdw u\in H^1(\Omega)}\wedge\forall v\in H^1(\Omega):\underbrace{\int\limits_\Omega\nabla u\cdot \nabla v}_{=\sum\limits_{i=1}^n\int\limits_\Omega\frac{\partial u}{\partial x_i}\cdot\frac{\partial v}{\partial x_i}}=\int\limits_\Omega f\cdot v\Bigg\rbrace
\end{align*}
\underline{Falls} partielle Integration möglich und $v\in C^\infty_c(\Omega)$:
\begin{align*}
\int\limits_\Omega\nabla u\cdot \nabla v=\sum\limits_{i=1}^n\int\limits_\Omega\frac{\partial u}{\partial x_i}\cdot\frac{\partial v}{\partial x_i}=-\int\limits_\Omega\sum\limits_{i=1}^n\frac{\partial^2 u}{\partial x_i^2}\cdot v
=\int\limits_\Omega-\Delta u\cdot v
\end{align*}

Falls $p\in(1,\infty)$ beliebig ist, dann gilt
\begin{align*}
(u,f)\in\partial\E``\Longleftrightarrow\text{''}\left\lbrace\begin{array}{rl}
-\Delta_p u=f&\text{ in }\Omega\\
|\nabla_p u|^{p-2}\cdot\frac{\partial u}{\partial v}=0 &\text{ auf }\partial\Omega
\end{array}\right.\\
\text{ wobei }\Delta_p u:=\div\left(|\nabla u|^{p-2}\nabla u\right)
\end{align*}
der \textbf{$p$-Laplaceoperator} ist.\\
Wichtige Beobachtung: $\partial\E$ ist $m$-akkretiv. Insbesondere ist für alle $h>0$ und $f\in L^2(\Omega)$ die Inklusion $u+h\cdot\partial\E(u)\ni f$ (hier eigentlich eine Gleichung, da $\partial\E$ ``einwertig'' ist) eindeutig lösbar!\\
Damit ist für alle $h>0$, $f\in L^2(\Omega)$ das Problem
\begin{align*}
\left\lbrace\begin{array}{rl}
u-h\cdot\Delta_p u=f &\text{ in }\Omega\\
|\nabla u|^{p-2}\cdot\frac{\partial u}{\partial v}=0 &\text{ auf } \partial\Omega
\end{array}\right.
\end{align*}
eindeutig lösbar. Der Beweis, dass $\partial\E$ $m$-akkretiv ist, zeigt, dass die eindeutige Lösung $u$ dieses Problems genau der (eindeutige) Minimierer der Funktion
\begin{align*}
L^2(\Omega)\to\R\cup\lbrace+\infty\rbrace,\qquad
v\mapsto\E(v)+\frac{1}{2\cdot h}\cdot\Vert v-h\Vert_{L^2}^2
\end{align*}
ist!
%TODO Den Teil von Lukas einfügen. An welche Stelle weiß ich nicht genau.
\end{beispiel}


