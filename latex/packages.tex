%PACKAGES
\usepackage[english, ngerman]{babel}	% Paket für Sprachselektion, in diesem Fall für deutsches Datum etc
\usepackage[utf8]{inputenc}	% Paket für Umlaute; verwende utf8 Kodierung in TexWorks 
\usepackage[T1]{fontenc} %ö,ü,ä werden richtig kodiert
\usepackage{amsmath} %wichtig für align-Umgebung
\usepackage{amssymb} %wichtig für \mathbb{} usw.
\usepackage{amsthm} %damit kann man eigene Theorem-Umgebungen definieren, proof-Umgebungen, etc.
\usepackage[backref]{hyperref} %Inhaltsverzeichnis und \ref-Befehle werden in der PDF-klickbar
\usepackage{graphicx}
\usepackage{grffile}
\usepackage{setspace} %wichtig für Lesbarkeit. Schöne Zeilenabstände

\usepackage{mathrsfs} %für das Hausdorff-H
\usepackage{xifthen} %für das If-Then-Else bei Newcommand
\usepackage{tikz}
\usetikzlibrary{decorations.fractals} %notwendig für Tikz
\usepackage{epigraph} % für das coole Zitat

