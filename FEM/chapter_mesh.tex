\section{Mesh Adaptivity for the FEM}
\subsection{Motivation and general concept}

\begin{example}
	poisson equation, uniform mesh, linear polinomials
\end{example}
\begin{center}
	\begin{tabular}{c | c}
		\underline{Quality} & \underline{Complexity}\\
		higher mesh resolution & higher mesh resolution requires \\
		$\to$ better approximation & additional computaional effort\\
		&  \\
		convergence & linear system \\
		$\|u-u_h\|_{L^2} \leq C h^2$ & $N_{\text{DoF}}\propto h^{-2}$
	\end{tabular}
\end{center}

Consens: Use fine meshes in regions with high gradient(something happens) and coarse meshes in regions with nearly constant gradiants(nothing happening)\vspace{1cm}

Questions:
\begin{itemize}
	\item Where?\\
		\textit{prescribed} (with prior knowledge of the problem) or \\
		\textit{adaptive}(with error estimates,...)
	\item How?\\
		\textit{Red-Green Refinement} or \\
		\textit{Bisection}(we will only deal with this method here)
\end{itemize}

Method of Bisection:
\begin{itemize}
	\item define a refinemnt edge
	\item insert node in the middle
	\item replace old triangle wlth 2 new ones 
\end{itemize}
\begin{figure}[h!]
	\center
	\begin{tikzpicture}[scale=1]
	\def \xone{0};
	\def \yone{0};
	\def \h{3};
	
	% first triangle
	\coordinate (A) at (\xone,\yone);
	\coordinate (B) at ($ (A) + (\h,0) $);
	\coordinate (C) at ($ (A) + (\h/3,2*\h/3) $);
	\coordinate (M) at ($ (A) + (\h/2,0) $);

		
	\draw (A) -- (B) -- (C) --cycle;
	\filldraw (A) circle (1pt)
			  (B) circle (1pt)
			  (C) circle (1pt)
			  (M) circle (1pt);
	
	\draw[-to] (\h,\h/3) -- ++(\h/2,0);
	
	
	% first triangle
	\coordinate (A1) at (\xone +1.7*\h,\yone);
	\coordinate (B1) at ($ (A1) + (\h,0) $);
	\coordinate (C1) at ($ (A1) + (\h/3,2*\h/3) $);
	\coordinate (M1) at ($ (A1) + (\h/2,0) $);


	\draw (A1) -- (B1) -- (C1) --cycle;
	\draw (M1) -- (C1);
	\filldraw (A1) circle (1pt)
			  (B1) circle (1pt)
			  (C1) circle (1pt)
			  (M1) circle (1pt);
	 
	\end{tikzpicture}
	
	\caption{bisection refinement}\label{tikz/chapter_mesh/bisection_refine}
	\label{ch_m_bisection_refine}
	
\end{figure}

Shape regularity:
\begin{center}
	\begin{tabular}{c | c}
		\underline{mathmatical} & \underline{geometric}\\
		$\tau_h$:  triangulation & triangles not strongly stretched \\
		$h_T$:  diameter of $T$ & all angles should be \glqq simialar \grqq \\
		$\rho_T$:  radius of inscribed circle:	& gives the same qualitative description\\
		$\exists C > 0\ \forall T \in \tau_h \colon \rho_T \geq C h_T$ & 
	\end{tabular}
\end{center}
\begin{figure}[H]
	\center
	\begin{tikzpicture}[scale=1]
	\def \xone{0};
	\def \yone{0};
	\def \h{3};
	\def \r{2*\h/9 +\h*0.025}; % nicht schön, aber funktioniert
	
	% first triangle
	\coordinate (A) at (\xone,\yone);
	\coordinate (B) at ($ (A) + (\h,0) $);
	\coordinate (C) at ($ (A) + (\h/2,2*\h/3) $);
	\coordinate (M) at ($  (A) + (\h/2,\r)  $);
		
	\draw (A) -- (B) -- (C) --cycle;
	\draw[red] (A) -- (B);
	\draw[blue] (A) ++(\h/2,0) -- (M);
	\draw (M) circle (\r);
	\filldraw (A) circle (0.8pt)
			  (B) circle (0.8pt)
			  (C) circle (0.8pt)
			  (M) circle (0.8pt);
			  
	
	\def \ra{\h/13}; % nicht schön, aber funktioniert
	% second triangle
	\coordinate (D) at ($ (A) +(2*\h,0) $);
	\coordinate (E) at ($ (D) + (2*\h,0) $);
	\coordinate (F) at ($ (D) + (1*\h,\h/6) $);
	\coordinate (M2) at ($  (D) + (\h,\ra)  $);
	
	\draw (D) -- (E) -- (F) --cycle;
	\draw (M2) circle (\ra);
	\filldraw (D) circle (0.8pt)
			  (E) circle (0.8pt)
			  (F) circle (0.8pt)
			  (M2) circle (0.8pt);
	

	\node[below left] at (\h/3,0) {$h_T$};
	\node[below right] at (\h/2,\r/2) {$\rho_T$};
%	\fill[black,font=\footnotesize] (3.2,0)  node[below] {$h$}
%									(2.7,0.5) node[left] {$h$};
	\end{tikzpicture}
	
	\caption{different shapes}
	\label{ch_m_shape_regularity}
	
\end{figure}

Refinement edge:
\begin{itemize}
	\item chose longest edge for refinement
	\item shape regularity of mesh either remains or gets better
\end{itemize}

Problem: Hanging nodes
\begin{figure}[H]
	\center
	\begin{tikzpicture}[scale=1]
	\def \xone{0};
	\def \yone{0};
	\def \h{3};
	
	% first rectangle
	\coordinate (A) at (\xone,\yone);
	\coordinate (B) at ($ (A) + (\h,0) $);
	\coordinate (C) at ($ (A) + (\h,0.7*\h) $);
	\coordinate (D) at ($ (A) + (0,0.7*\h) $);
	\coordinate (M) at ($ (A) + (\h/2,0.35*\h) $);


	\fill[red!20] (A) --(B) --(D) -- cycle;		
	\draw (A) -- (B) -- (C) --(D) --cycle;
	\draw (B) --(D);
	\filldraw (A) circle (1pt)
			  (B) circle (1pt)
			  (C) circle (1pt)
			  (D) circle (1pt)
			  (M) circle (1pt);
	
	\draw[-to] (11/10*\h,\h/3) -- ++(\h/2,0);
	
	
	% second rectangle
	\coordinate (A1) at (\xone + 1.75*\h,\yone);
	\coordinate (B1) at ($ (A1) + (\h,0) $);
	\coordinate (C1) at ($ (A1) + (\h,0.7*\h) $);
	\coordinate (D1) at ($ (A1) + (0,0.7*\h) $);
	\coordinate (M1) at ($ (A1) + (\h/2,0.35*\h) $);
	
	
	\fill[green!20] (A1) --(B1) --(D1) -- cycle;	
	\fill[red!20] (C1) --(B1) --(D1) -- cycle;	
	\draw (A1) -- (B1) -- (C1) --(D1) --cycle;
	\draw (B1) --(D1);
	\draw (A1) --(M1);
	\filldraw (A1) circle (1pt)
			  (B1) circle (1pt)
			  (C1) circle (1pt)
			  (D1) circle (1pt)
			  (M1) circle (1pt);
	 
	\end{tikzpicture}
	
	\caption{hanging nodes}
	\label{ch_m_hanging_nodes}
	
\end{figure}
Refinement of a single triangle leads to a hanging node that is forbidden.\\
Solution: rekursive Refinement
\begin{enumerate}[label= case \arabic*:]
	\item Hanging node is on the refinement edge of a adjacent triangle
	\begin{figure}[h!]
	\center
	\begin{tikzpicture}[scale=1]
	\def \xone{0};
	\def \yone{0};
	\def \h{3};
	
	% first rectangle
	\coordinate (A) at (\xone,\yone);
	\coordinate (B) at ($ (A) + (\h,0) $);
	\coordinate (C) at ($ (A) + (\h,0.7*\h) $);
	\coordinate (D) at ($ (A) + (0,0.7*\h) $);
	\coordinate (M) at ($ (A) + (\h/2,0.35*\h) $);


	\fill[green!20] (A) --(B) --(D) -- cycle;
	\fill[red!20] (C) --(B) --(D) -- cycle;			
	\draw (A) -- (B) -- (C) --(D) --cycle;
	\draw (B) --(D);
	\draw (A) --(M);
	\filldraw (A) circle (1pt)
			  (B) circle (1pt)
			  (C) circle (1pt)
			  (D) circle (1pt)
			  (M) circle (1pt);
	
	\draw[-to] (11/10*\h,\h/3) -- ++(\h/2,0);
	
	
	% second rectangle
	\coordinate (A1) at (\xone + 1.75*\h,\yone);
	\coordinate (B1) at ($ (A1) + (\h,0) $);
	\coordinate (C1) at ($ (A1) + (\h,0.7*\h) $);
	\coordinate (D1) at ($ (A1) + (0,0.7*\h) $);
	\coordinate (M1) at ($ (A1) + (\h/2,0.35*\h) $);
	
	
	\fill[green!20] (A1) --(B1) --(D1) -- cycle;	
	\fill[green!20] (C1) --(B1) --(D1) -- cycle;	
	\draw (A1) -- (B1) -- (C1) --(D1) --cycle;
	\draw (B1) --(D1);
	\draw (A1) --(M1);
	\draw (C1) --(M1);
	\filldraw (A1) circle (1pt)
			  (B1) circle (1pt)
			  (C1) circle (1pt)
			  (D1) circle (1pt)
			  (M1) circle (1pt);
	 
	\end{tikzpicture}
	
	\caption{hanging nodes: case 1}\label{tikz/chapter_mesh/hanging_nodes_c1}
	\label{ch_m_hanging_nodes_c1}
	
\end{figure}

	\item Hanging node is \textbf{not} on refinement edge
	\begin{figure}[h!]
	\center
	\begin{tikzpicture}[scale=1]
	\def \xone{0};
	\def \yone{0};
	\def \h{3};
	
	% first rectangle
	\coordinate (A) at (\xone,\yone);
	\coordinate (B) at ($ (A) + (1.5*\h,0) $);
	\coordinate (C) at ($ (A) + (1.5*\h,\h) $);
	\coordinate (D) at ($ (A) + (0,\h) $);
	\coordinate (M) at ($ (A) + (0.75*\h,0.5*\h) $);
	\coordinate (AD) at ($ (A) + (0,0.5*\h) $);
	\coordinate (CD) at ($ (A) + (0.75*\h,\h) $);
	\coordinate (BC) at ($ (A) + (1.5*\h,0.5*\h) $);
	\coordinate (MMC) at ($ (A) + (3/4*1.5*\h,0.75*\h) $);

	%draw
	\filldraw[red!20] (M) --(MMC) -- (BC);			
	\draw (A) -- (B) -- (C) --(D) --cycle;
	\draw (B) --(D);
	\draw (A) --(C);
	\draw (AD) --(BC);
	\draw (M) --(CD) -- (BC);
	\foreach \i in {A,B,C,C,D,M,AD,CD,BC,MMC}{
		\filldraw (\i) circle (1.5pt);
	}

	
	\draw[-to] (8/5*\h,\h/2) -- ++(\h/2,0);
	
	
	% second rectangle
	\coordinate (A1) at (\xone +  2.25*\h,\yone);
	\coordinate (B1) at ($ (A1) + (1.5*\h,0) $);
	\coordinate (C1) at ($ (A1) + (1.5*\h,\h) $);
	\coordinate (D1) at ($ (A1) + (0,\h) $);
	\coordinate (M1) at ($ (A1) + (0.75*\h,0.5*\h) $);
	\coordinate (AD1) at ($ (A1) + (0,0.5*\h) $);
	\coordinate (CD1) at ($ (A1) + (0.75*\h,\h) $);
	\coordinate (BC1) at ($ (A1) + (1.5*\h,0.5*\h) $);
	\coordinate (MMC1) at ($ (A1) + (3/4*1.5*\h,0.75*\h) $);
	\coordinate (MMBC1) at ($ (A1) + (3/4*1.5*\h,0.5*\h) $);
	
	%draw
	\fill[red!20] (M1) --(B1) --(BC1) -- cycle;
	\filldraw[green!20] (M1) --(MMC1) -- (BC1);			
	\draw (A1) -- (B1) -- (C1) --(D1) --cycle;
	\draw (B1) --(D1);
	\draw (A1) --(C1);
	\draw (AD1) --(BC1);
	\draw (M1) --(CD1) -- (BC1);
	\draw (MMC1) --(MMBC1);
	\foreach \i in {A1,B1,C1,C1,D1,M1,AD1,CD1,BC1,MMC1,MMBC1}{
		\filldraw (\i) circle (1.5pt);
	}

	\draw[-to] (\xone +  2.1*\h,\yone - 0.1*\h) -- ++(-\h/2,-\h/3);
	
	
	% third rectangle
	\coordinate (A2) at (\xone ,\yone -1.5*\h);
	\coordinate (B2) at ($ (A2) + (1.5*\h,0) $);
	\coordinate (C2) at ($ (A2) + (1.5*\h,\h) $);
	\coordinate (D2) at ($ (A2) + (0,\h) $);
	\coordinate (M2) at ($ (A2) + (0.75*\h,0.5*\h) $);
	\coordinate (AD2) at ($ (A2) + (0,0.5*\h) $);
	\coordinate (CD2) at ($ (A2) + (0.75*\h,\h) $);
	\coordinate (BC2) at ($ (A2) + (1.5*\h,0.5*\h) $);
	\coordinate (MMC2) at ($ (A2) + (3/4*1.5*\h,0.75*\h) $);
	\coordinate (MMBC2) at ($ (A2) + (3/4*1.5*\h,0.5*\h) $);
	\coordinate (MMB2) at ($ (A2) + (3/4*1.5*\h,0.25*\h) $);
	\coordinate (AB2) at ($ (A2) + (0.5*1.5*\h,0) $);
	
	
	%draw
	\fill[red!20] (M2) --(B2) --(BC2) -- cycle;
	\fill[red!20] (A2) --(B2) --(M2) -- cycle;
	\filldraw[green!20] (M2) --(MMC2) -- (BC2);			
	\draw (A2) -- (B2) -- (C2) --(D2) --cycle;
	\draw (B2) --(D2);
	\draw (A2) --(C2);
	\draw (AD2) --(BC2);
	\draw (M2) --(CD2) -- (BC2);
	\draw (MMC2) --(MMBC2);
	\foreach \i in {A2,B2,C2,D2,M2,AD2,CD2,BC2,MMC2,MMBC2,MMB2,AB2}{
		\filldraw (\i) circle (1.5pt);
	}

	\draw[-to] (8/5*\h,-\h) -- ++(\h/2,0);
	
	% fourth rectangle
	\coordinate (A3) at (\xone +  2.25*\h,\yone-1.5*\h);
	\coordinate (B3) at ($ (A3) + (1.5*\h,0) $);
	\coordinate (C3) at ($ (A3) + (1.5*\h,\h) $);
	\coordinate (D3) at ($ (A3) + (0,\h) $);
	\coordinate (M3) at ($ (A3) + (0.75*\h,0.5*\h) $);
	\coordinate (AD3) at ($ (A3) + (0,0.5*\h) $);
	\coordinate (CD3) at ($ (A3) + (0.75*\h,\h) $);
	\coordinate (BC3) at ($ (A3) + (1.5*\h,0.5*\h) $);
	\coordinate (MMC3) at ($ (A3) + (3/4*1.5*\h,0.75*\h) $);
	\coordinate (MMBC3) at ($ (A3) + (3/4*1.5*\h,0.5*\h) $);
	\coordinate (MMB3) at ($ (A3) + (3/4*1.5*\h,0.25*\h) $);
	\coordinate (AB3) at ($ (A3) + (0.5*1.5*\h,0) $);
	
	%draw
	\fill[red!20] (M3) --(MMB3) --(BC3) -- cycle;
	\fill[red!20] (AB3) --(B3) --(M3) -- cycle;
	\filldraw[green!20] (M3) --(MMC3) -- (BC3);	
	\filldraw[green!20] (B3) --(BC3) -- (MMB3);
	\filldraw[green!20] (A3) --(AB3) -- (M3);				
	\draw (A3) -- (B3) -- (C3) --(D3) --cycle;
	\draw (B3) --(D3);
	\draw (A3) --(C3);
	\draw (AD3) --(BC3);
	\draw (M3) --(CD3) -- (BC3);
	\draw (MMC3) --(MMBC3);
	\draw (BC3) --(MMB3);
	\draw (M3) --(AB3);
	\foreach \i in {A3,B3,C3,C3,D3,M3,AD3,CD3,BC3,MMC3,MMBC3,MMB3,AB3}{
		\filldraw (\i) circle (1.5pt);
	}
	
	\draw[-to] (\xone +  2.1*\h,\yone - 1.6*\h) -- ++(-\h/2,-\h/3);
	
	% fifth rectangle
	\coordinate (A4) at (\xone ,\yone -3*\h);
	\coordinate (B4) at ($ (A4) + (1.5*\h,0) $);
	\coordinate (C4) at ($ (A4) + (1.5*\h,\h) $);
	\coordinate (D4) at ($ (A4) + (0,\h) $);
	\coordinate (M4) at ($ (A4) + (0.75*\h,0.5*\h) $);
	\coordinate (AD4) at ($ (A4) + (0,0.5*\h) $);
	\coordinate (CD4) at ($ (A4) + (0.75*\h,\h) $);
	\coordinate (BC4) at ($ (A4) + (1.5*\h,0.5*\h) $);
	\coordinate (MMC4) at ($ (A4) + (3/4*1.5*\h,0.75*\h) $);
	\coordinate (MMBC4) at ($ (A4) + (3/4*1.5*\h,0.5*\h) $);
	\coordinate (MMB4) at ($ (A4) + (3/4*1.5*\h,0.25*\h) $);
	\coordinate (AB4) at ($ (A4) + (0.5*1.5*\h,0) $);
	
	
	%draw
	\fill[red!20] (M4) --(MMB4) --(BC4) -- cycle;
	\fill[red!20] (AB4) --(B4) --(M4) -- cycle;
	\filldraw[green!20] (B4) --(BC4) -- (MMB4);
	\filldraw[green!20] (M4) --(MMC4) -- (BC4);
	\filldraw[green!20] (A4) --(AB4) -- (M4);			
	\draw (A4) -- (B4) -- (C4) --(D4) --cycle;
	\draw (B4) --(D4);
	\draw (A4) --(C4);
	\draw (AD4) --(BC4);
	\draw (M4) --(CD4) -- (BC4);
	\draw (MMC4) --(MMBC4);
	\draw (BC4) --(MMB4);
	\draw (M4) --(AB4);
	\foreach \i in {A4,B4,C4,D4,M4,AD4,CD4,BC4,MMC4,MMBC4,MMB4,AB4}{
		\filldraw (\i) circle (1.5pt);
	}
	
	\draw[-to] (8/5*\h,-2.5*\h) -- ++(\h/2,0);
	
	% sixth rectangle
	\coordinate (A5) at (\xone +  2.25*\h ,\yone -3*\h);
	\coordinate (B5) at ($ (A5) + (1.5*\h,0) $);
	\coordinate (C5) at ($ (A5) + (1.5*\h,\h) $);
	\coordinate (D5) at ($ (A5) + (0,\h) $);
	\coordinate (M5) at ($ (A5) + (0.75*\h,0.5*\h) $);
	\coordinate (AD5) at ($ (A5) + (0,0.5*\h) $);
	\coordinate (CD5) at ($ (A5) + (0.75*\h,\h) $);
	\coordinate (BC5) at ($ (A5) + (1.5*\h,0.5*\h) $);
	\coordinate (MMC5) at ($ (A5) + (3/4*1.5*\h,0.75*\h) $);
	\coordinate (MMBC5) at ($ (A5) + (3/4*1.5*\h,0.5*\h) $);
	\coordinate (MMB5) at ($ (A5) + (3/4*1.5*\h,0.25*\h) $);
	\coordinate (AB5) at ($ (A5) + (0.5*1.5*\h,0) $);
	
	
	%draw
	\fill[green!20] (M5) --(MMB5) --(BC5) -- cycle;
	\fill[green!20] (AB5) --(B5) --(M5) -- cycle;
	\filldraw[green!20] (B5) --(BC5) -- (MMB5);
	\filldraw[green!20] (M5) --(MMC5) -- (BC5);
	\filldraw[green!20] (A5) --(AB5) -- (M5);			
	\draw (A5) -- (B5) -- (C5) --(D5) --cycle;
	\draw (B5) --(D5);
	\draw (A5) --(C5);
	\draw (AD5) --(BC5);
	\draw (M5) --(CD5) -- (BC5);
	\draw (MMC5) --(MMBC5);
	\draw (BC5) --(MMB5);
	\draw (M5) --(AB5);
	\draw (AB5) --(MMB5);
	\draw (MMB5) --(MMBC5);
	\foreach \i in {A5,B5,C5,D5,M5,AD5,CD5,BC5,MMC5,MMBC5,MMB5,AB5}{
		\filldraw (\i) circle (1.5pt);
	}
	
	\end{tikzpicture}
	
	\caption{hanging nodes: case 2}\label{tikz/chapter_mesh/hanging_nodes_c2}
	\label{ch_m_hanging_nodes_c2}
	
\end{figure}
	still better, than refining ehole domain\\
	$\implies$ multiple recursive refinements necessary
\end{enumerate}

\subsection{Data structures}
\underline{direct storage}
\begin{itemize}
	\item different array for nodes and cells(like in the tutorial)
	\begin{figure}[h!]
	\center
	\begin{tikzpicture}[scale=1]
	\def \xone{0};
	\def \yone{0};
	\def \h{3};
	
	% first rectangle
	\coordinate (1) at (\xone,\yone);
	\coordinate (2) at ($ (A) + (\h,0) $);
	\coordinate (3) at ($ (A) + (0,0.7*\h) $);
	\coordinate (4) at ($ (A) + (\h,0.7*\h) $);
	\coordinate (5) at ($ (A) + (\h/2,0.35*\h) $);

	\draw (1) -- (2) -- (4) --(3) --cycle;
	\draw (2) --(3);
	\foreach \i in {1,2}{
		\filldraw (\i) circle (1.5pt);
		\node[below]  at (\i) {\i};
	}
	\foreach \i in {3,4}{
		\filldraw (\i) circle (1.5pt);
		\node[above]  at (\i) {\i};
	}
	
	\end{tikzpicture}
	
	\caption{visualize data structure}\label{tikz/chapter_mesh/data_structure}
	\label{ch_m_data_structure}
	
\end{figure}
	\item cells = 
	$\begin{bmatrix}
		1 & 2 & 3\\
		4 & 3 & 2
	\end{bmatrix} $ (refinement edge is defined by the last 2 entries in corresponding line)
	
\end{itemize}
Bisections on direct storage:
\begin{itemize}
	\item additional node
	\item replace 1 cell by 2 cells
	\begin{figure}[H]
	\center
	\begin{tikzpicture}[scale=1]
	\def \xone{0};
	\def \yone{0};
	\def \h{3};
	
	% first rectangle
	\coordinate (1) at (\xone,\yone);
	\coordinate (2) at ($ (A) + (\h,0) $);
	\coordinate (3) at ($ (A) + (0,0.7*\h) $);
	\coordinate (4) at ($ (A) + (\h,0.7*\h) $);
	\coordinate (5) at ($ (A) + (\h/2,0.35*\h) $);

	\draw (1) -- (2) -- (4) --(3) --cycle;
	\draw (2) --(3);
	\draw (1) --(5);
	\foreach \i in {1,2}{
		\filldraw (\i) circle (1.5pt);
		\node[below]  at (\i) {\i};
	}
	\foreach \i in {3,4}{
		\filldraw (\i) circle (1.5pt);
		\node[above]  at (\i) {\i};
	}
	\filldraw (5) circle (1.5pt);
	\node[above]  at (5) {5};
	
	\end{tikzpicture}
	
	\caption{visualize bisection data structure}
	\label{ch_m_bisection_data_structure}
	
\end{figure}
	\item cells = 
	$\begin{bmatrix}
	5 & 1 & 2\\
	5 & 3 & 1\\
	4 & 3 & 2
	\end{bmatrix} $ (first index is the new node and the structure regarding the refinement edge stays)
	
\end{itemize}

Advantages:
\begin{itemize}
	\item minimal memory consumption
	\item simple implementation
	\item all information directly available
\end{itemize}

Disadvantages:
\begin{itemize}
	\item working on arrays: size-changes are costly computations
	\item coarsening of meshes is very difficult
	\item neighboring relatons have to be computed explicitly after every refinement
\end{itemize}
Thus not so good for moving regions of interest.
\vspace{1cm}

\underline{hierarchical storage}
\begin{itemize}
	\item start from a coarse \glqq macro mesh \grqq
	\item store all refonement steps not only the current mesh
	\item well suited dara structure: \textbf{binary tree}
\end{itemize}

short exkurs: binary trees\\
tree nodes:
\begin{itemize}
	\item pointer to $\leq 2$ children
	\item pointer to $\leq 1$ father
	\item data
	\tikzset
{
	treenode/.style = {circle, draw=blue!60, fill=blue!40, very thick, minimum size=4mm}
}

\begin{figure}[H]
	\center
	\begin{tikzpicture}[->,>=stealth', level/.style={sibling distance = 2cm/#1, level distance = 1.5cm}, scale=0.6,transform shape, scale = 1.5]
	\node [treenode] {}
	child
	{
		node [treenode] {} 
		child
		{
			node [treenode] {} 
			child
			{
				node [treenode] {} 
			}
			child
			{
				node [treenode] {} 
			}
		}
		child
		{
			node [treenode]  {}  
		}
	}
	child
	{
		node [treenode]  {}     
	};
	
	\end{tikzpicture}	
\end{figure}


	

	%%TODO pic
\end{itemize}
properties:
\begin{itemize}
	\item no father: \textit{root}
	\item no children: \textit{leaf}
\end{itemize}

Applications:
\begin{itemize}
	\item search tree: heap
	\item hierarchical mesh
\end{itemize}

hierarchical mesh tree:
Node:
\begin{itemize}
	\item pointer: father,children
	\item data: vertices of the represented mesh cell
\end{itemize}

Mesh:
\begin{itemize}
	\item macro mesh explicitly defined
	\item every macro mesh cell ist \underline{root} of a tree
	\item current mesh os given by all \underline{leafs}
	\tikzset
{
	treenode/.style = {circle, draw=blue!60, fill=blue!40, very thick, minimum size=4mm}
}

\begin{figure}[H]
	\center

	\begin{tikzpicture}[level/.style={sibling distance = 2cm/#1, level distance = 3cm}]
	
	\def \xone{0};
	\def \yone{0};
	\def \h{3};
	
	%first tree (left)
	\coordinate (root1) at (\xone - \h,\yone + \h/3);
	\node [treenode] at (root1) {1}
	child
	{
		node [treenode] {3} 
	}
	child
	{
		node [treenode]  {4}
		child
		{
			node [treenode] {7} 
		}
		child
		{
			node [treenode]  {8}  
		}    
	};

	% second tree (right)
	\coordinate (root2) at (\xone + 2*\h,\yone+ \h/3);
	\node [treenode] at (root2) {2}
	child
	{
		node [treenode] {5} 
	}
	child
	{
		node [treenode]  {6}     
	};
	
	
	
	% first rectangle
	\coordinate (A) at (\xone,\yone);
	\coordinate (B) at ($ (A) + (\h,0) $);
	\coordinate (C) at ($ (A) + (\h,0.7*\h) $);
	\coordinate (D) at ($ (A) + (0,0.7*\h) $);
	\coordinate (M) at ($ (A) + (\h/2,0.35*\h) $);
		
	\draw (A) -- (B) -- (C) --(D) --cycle;
	\draw (B) --(D);
	\foreach \i in {A,B,C,D}{
		\filldraw (\i) circle (1.5pt);
	}

	\node at (\xone + 1/4*\h, \yone + 1/5*\h) {1};
	\node at (\xone + 3/4*\h, \yone + 2/5*\h) {2};


	% second rectangle
	\coordinate (A1) at (\xone,\yone - \h);
	\coordinate (B1) at ($ (A1) + (\h,0) $);
	\coordinate (C1) at ($ (A1) + (\h,0.7*\h) $);
	\coordinate (D1) at ($ (A1) + (0,0.7*\h) $);
	\coordinate (M1) at ($ (A1) + (\h/2,0.35*\h) $);
	
	\draw (A1) -- (B1) -- (C1) --(D1) --cycle;
	\draw (B1) --(D1);
	\draw (A1) --(C1);
	\foreach \i in {A1,B1,C1,D1,M1}{
		\filldraw (\i) circle (1.5pt);
	}

	\node at (\xone + 1/2*\h, \yone -\h + 1/8*\h) {3};
	\node at (\xone + 1/5*\h, \yone -\h + 0.35*\h) {4};
	\node at (\xone + 1/2*\h, \yone -\h + 0.7*3/4*\h) {5};
	\node at (\xone + 4/5*\h, \yone -\h + 0.35*\h) {6};

	% third rectangle
	\coordinate (A2) at (\xone,\yone - 2*\h);
	\coordinate (B2) at ($ (A2) + (\h,0) $);
	\coordinate (C2) at ($ (A2) + (\h,0.7*\h) $);
	\coordinate (D2) at ($ (A2) + (0,0.7*\h) $);
	\coordinate (M2) at ($ (A2) + (\h/2,0.35*\h) $);
	\coordinate (AD2) at ($ (A2) + (0,0.35*\h) $);
	
	\draw (A2) -- (B2) -- (C2) --(D2) --cycle;
	\draw (B2) --(D2);
	\draw (A2) --(C2);
	\draw (AD2) --(M2);
	\foreach \i in {A2,B2,C2,D2,M2,AD2}{
		\filldraw (\i) circle (1.5pt);
	}

	\node at (\xone + 1/2*\h, \yone -2*\h + 1/8*\h) {3};
	\node at (\xone + 1/7*\h, \yone -2*\h + 0.45*\h) {7};
	\node at (\xone + 1/7*\h, \yone -2*\h + 0.7*2/7*\h) {8};
	\node at (\xone + 1/2*\h, \yone -2*\h + 0.7*3/4*\h) {5};
	\node at (\xone + 4/5*\h, \yone -2*\h + 0.35*\h) {6};

	
	\end{tikzpicture}
	
\end{figure}
\end{itemize}

Note that here the seperation of node indices and coordinates is not useful.\vspace{1cm}

Advantages:
\begin{itemize}
	\item large libraries for tree strictires already existent(computation of neighbors, etc.)
	\item coarsening is a trivial task(remove leafs)
	\item refinement inherited from father (determined by tree structure)(compute neighboring relations \glqq on demand\grqq)
	%%TODO add pic
	\begin{figure}[H]
	\center
	\begin{tikzpicture}[scale=1]
	\def \xone{0};
	\def \yone{0};
	\def \h{3};
	
	% first rectangle
	\coordinate (2) at (\xone,\yone);
	\coordinate (1) at ($ (2) + (\h/2,\h/2) $);
	\coordinate (3) at ($ (2) + (\h,0) $);
	\coordinate (4) at ($ (2) + (\h/2,0) $);
	
	
	\draw (1) -- (2) --(3) --cycle;
	\draw (1) -- (4);
	\foreach \i in {2,3,4}{
		\filldraw (\i) circle (1.5pt);
		\node[below]  at (\i) {\i};
	}
	\node[above]  at (1) {1};
	\filldraw (1) circle (1.5pt);
	
	
	\end{tikzpicture}
	
	%\caption{visualize data structure}
	\label{ch_m_hier_mesh}
	
\end{figure}
	\item cells = 
	$\begin{bmatrix}
	1 & 2 & 3
	\end{bmatrix} \implies 
	\begin{bmatrix}
	4 & 1 & 2\\
	4 & 3 & 1
	\end{bmatrix} $ (refinement edge is defined by the last 2 entries in corresponding line)
\end{itemize}

Disadvantages:
\begin{itemize}
	\item memory consumption(storage of unnecessary nodes)
	\item high implementational effort
\end{itemize}

\subsection{Conclusion and Outlook}
\begin{enumerate}[label = \arabic* .]
	\item local mesh refinement can be a powerful tool to find the optimal balence between performance and accuracy
	\item prevention of hanging nodes requieres a recursive refinement strategy
	\item both datastructures have advantages and disadvantages  
\end{enumerate}
Open questions:
\begin{enumerate}[label = \arabic* .]
	\item Where to refine(explicitly give locations, error estimates,...)?
	\item How to prove that it actually works(convergence orders)?
	\item What could have been done differently?\\
	$\implies$ Replace Bisection- by Red-Green Refinement...\\
	$\implies$ $p$-Adaptivity instead of $h$-Adaptivity(take higher order polonomial basisfunctions)
\end{enumerate}


