\section{Mesh Adaptivity for the FEM}
\subsection{Motivation and general concept}

\begin{example}
	poisson equation, uniform mesh, linear polinomials
\end{example}

\begin{tabular}{c | c}
	\underline{Quality} & \underline{Complexity}\\
	higher mesh resolution & higher mesh resolution requires \\
	 $\to$ better approximation & additional computaional effort\\
	 & \\
	 convergence & linear system \\
	 $\|u-u_h\|_{L^2} \leq C h^2$ & $N_{\text{DoF}}\propto h^{-2}$
\end{tabular}

Consens: Use fine meshes in regions woth high gradient(something happens) and coarse meshes in regions with nearly constant gradiants(npthing happening)\\

Questions:
\begin{itemize}
	\item Where?\\
		\textit{prescribed} (with prior knowledge of the problem) or \\
		\textit{adaptive}(with error estimates,...)
	\item How?\\
		\textit{Red-Green Refinement} or \\
		\textit{Bisection}(we will only deal with this method here)
\end{itemize}

Method of Bisection:
\begin{itemize}
	\item define a refinemnt edge
	\item insert node in the middle
	\item replace old triangle wlth 2 new ones 
\end{itemize}
%%TODO add picture

Shape regularity:\\
\begin{tabular}{c | c}
	\underline{mathmatical} & \underline{geometric}\\
	$\tau_h$:  triangulation & triangles not strongly stretched \\
	$h_T$:  diameter of $T$ & all angles should be \glqq simialar \grqq \\
	$\rho_T$:  radius of inscribed circle:	& gives the same qualitative description\\
	$\exists C > 0\ \forall T \in \tau_h \colon \rho_T \geq C h_T$ & 
\end{tabular}
%%TODO add pictures

Refinement edge:
\begin{itemize}
	\item chosse longest edge for refinement
	\item shape regularity of mesh either remains or gets better
\end{itemize}

Problem: Hanging nodes
%%TODO add picture
Refinement of a single triangle leads to a hanging node that is forbidden.\\
Solution: rekursive Refinement\\
\begin{enumerate}[label= case \arabic*:]
	\item Hanging node is on the refinement edge of a adjacent triangle
	%%TODO add picture
	\item Hanging node is \textbf{not} on refinement edge
	%%TODO add picture
	still better,than refining ehole domain\\
	$implies$ multiple recursive refinements necessary
\end{enumerate}

\subsection{Data structures}
\underline{direct storage}\\
\begin{itemize}
	\item different array for nodes and cells(like in the tutorial)
	%%TODO add pic
\end{itemize}
Bisections on direct storage:
\begin{itemize}
	\item additional node
	\item replace 1 cell by 2 cells
	%%TODO add pic
\end{itemize}

Advantages:
\begin{itemize}
	\item minimal memory consumption
	\item simple implementation
	\item all information directly available
\end{itemize}

Disadvantages:
\begin{itemize}
	\item working on arrays: size-changes are costly computations
	\item coarsening of meshes is very difficult
	\item neighboring relatons have to be computed explicitly after every refinement
\end{itemize}
Thus not so good for moving regions of interest

\underline{hierarchical storage}\\
\begin{itemize}
	\item start from a coarse \glqq macro mesh \grqq
	\item store all refonement steps not only the current mesh
	\item well suited dara structure: \textbf{binary tree}
\end{itemize}

short exkurs: binary trees\\
tree nodes:
\begin{itemize}
	\item pointer to $\leq 2$ children
	\item pointer to $\leq 1$ father
	\item data
	%%TODO pic
\end{itemize}
properties:
\begin{itemize}
	\item no father: \textit{root}
	\item no children: \textit{leaf}
\end{itemize}

Applications:
\begin{itemize}
	\item search tree: heap
	\item hierarchical mesh
\end{itemize}

hierarchical mesh tree:
Node:
\begin{itemize}
	\item pointer: father,children
	\item data: vertices of the represented mesh cell
\end{itemize}

Mesh:
\begin{itemize}
	\item macro mesh explicitly defined
	\item every macro mesh cell ist \underline{root} of a tree
	\item current mesh os given by all \underline{leafs}
	%%TODO add pic
\end{itemize}

Advantages:
\begin{itemize}
	\item large libraries for tree strictires already existent(computation of neighbors, etc.)
	\item coarsening is a trivial task(remove leafs)
	\item refinement inherited from father (determined by tree structure)(compute neighboring relations \glqq on demand\grqq)
	%%TODO add pic
\end{itemize}

Disadvantages:
\begin{itemize}
	\item memory consumption(storage of unnecessary nodes)
	\item high implementational effort
\end{itemize}

\subsection{Conclusion and Outlook}
\begin{enumerate}[label = \arabic* .]
	\item local mesh refinement can be a powerful tool to find the optimal balence between performance and accuracy
	\item prevention of hanging nodes requieres a recursive refinement strategy
	\item both datastructures have advantages and disadvantages  
\end{enumerate}
Open questions:
\begin{enumerate}[label = \arabic* .]
	\item Where to refine(explicitly give locations, error estimates,...)?
	\item How to prove that it actually works(convergence orders)?
	\item What could have been done differently?\\
	$\implies$ Replace Bisection- by Red-Green Refinement...\\
	$\implies$ $p$-Adaptivity instead of $h$-Adaptivity(take higher order polonomial basisfunctions)
\end{enumerate}


