% !TeX root= FEM
\section{Quadrature}

\subsection{Motivation}
We want to compute Integrals of the form:
\begin{align*}
	\int \limits_\Omega f(x) \diff x.
\end{align*}
\underline{Problems}
\begin{itemize}
	\item primitive doesn't always exist ($F\colon F'=f$)
	\item analytic computation on PC problematic
\end{itemize}

 \underline{special case: FEM}
 \begin{itemize}
 	\item 
 	\begin{equation*}
 		A_{ij} = \int \limits_\Omega \nabla \varphi_j \cdot \nabla \varphi_i \diff x
 	\end{equation*}
 	\item no closed form of $\varphi_i$ on $\Omega$
 	\item philosophical: FEM is just an approximation
 \end{itemize}

$\implies$ Aim: Approximate $\int \limits_\Omega f(x) \diff x$

\underline{Approach}
\begin{equation*}
	\int \limits_\Omega f(x) \diff x \approx \displaystyle \sum_{k=1}^{N_Q} w_k f(x_k)
\end{equation*}
with $w_k \in \R$ (weights) and $x_k \in \Omega$ (points/nodes). Together the set $\{(w_k,x_k)\}$ is called quadrature rule.\\
\underline{Questions:}
\begin{enumerate}[label=\arabic*)]
	\item Do I need a quadrature rule for every $\Omega$?
	\item How do I choose $\{(w_k,x_k)\}$?
	\item How do I measure the quality of approximation?
\end{enumerate}

\subsection{Transformation}
Integral Transformation, $\hat{\Omega}, \Phi(\hat{\Omega}) = \Omega$
\begin{align}
	\int \limits_{\Phi(\Omega)}  f(y) \diff y &= \int \limits_{\hat{\Omega}} f(\Phi(y)) |det(D\Phi(x))| \diff x \\
	&\approx \displaystyle \sum^{N_Q}_{k=1} f(\underbrace{\Phi(x_k)}_{\in \Omega}) \underbrace{|det(D\Phi)|}_{\in \R} \diff
\end{align}
Note that in our case $\Phi$ is an affine linear function and thus $D\Phi$ does not depend on $x$.
$\{(|det(D\Phi)|w_k,\Phi(x_k))\}$ is a quadrature rule on $\Omega$.
\subsection{Exactness}
Aim: measure the approximation quality\\
\underline{Notation}
\begin{itemize}
	\item $\hat{\Omega}$, $\{(w_k,x_k)\}$ quadrature rule on $\hat{\Omega}$
	\item $I(f):= \int \limits_{\hat{\Omega}} f(x) \diff x$
	\item $Q(f):= \displaystyle \sum_{k=1}^{N_Q} w_k f(x_k)$
	\item $\mathbb{P}_m(\hat{\Omega}) \dots $ polonomials of degree $\leq m$ on $\hat{\Omega}$
\end{itemize}
\begin{definition}
	A quadrature rule is exact of order $m$ $\iff$
	\begin{equation*}
		I(f)= Q(f) \qquad \forall p\in \mathbb{P}_m(\hat{\Omega})
	\end{equation*}
	and there exists $p\in \mathbb{P}_{m+1}$ s.t. $I(f)\neq Q(f)$
\end{definition}
\begin{lemma_}
	exact of order $m$ $\iff$ exact for $\{1,x,x^2,\dots,x^m \}$
\end{lemma_}

\begin{thrm}
	(1D): a quadrature rule with $N_Q$ points has a degree of exactness $\leq 2N_Q -1$
\end{thrm}

\subsection{Optimal Quadrature Rules}
\underline{Gaussian Quadrature Rules}: Aim: Exactness $2N_Q-1$
\vspace{0.5cm}\\
\underline{Dimension 1}, Intervall [0,1]\\
$N_Q = 1$ $\implies$ exact for $\{1,x\}$
\begin{equation*}
	Q(f)= w_1f(x_1)
\end{equation*}
\begin{align*}
	w_1 &= Q(1) = I(1)= 1 &\implies & w_1 = 1\\
	w_1x_1 &= Q(x) = I(x)= \frac{1}{2} &\implies & x_1 = \frac{1}{2}
\end{align*}

$N_Q = 2$ $\implies$ exact for $\{1,x,x^2.x^3\}$
\begin{equation*}
	Q(f)= w_1f(x_1) + w_2f(x_2)
\end{equation*}
\begin{align*}
\begin{rcases}
	1 &= I(1)=Q(1)  \\
	\frac{1}{2} &= I(x)= Q(x)  \\
	\frac{1}{3}&=I(x^2)= Q(x^2) \\
	\frac{1}{4}&=I(x^3)= Q(x^3)
\end{rcases} \implies w_1=w_2= \frac{1}{2}, x_{1,2} = \frac{\sqrt{3}\pm 1}{2\sqrt{3}}
\end{align*}

\underline{Dimension 1}\\
$\hat{\Omega} = [0,1]\times [0,1]$ $\implies \{\{x_k\}\times\{x_k\} \}$ quadrature points on $\hat{\Omega}$\\
$\hat{\Omega} = \hat{T}$\\
First Approach: transform $[0,1]\times [0,1]$ to $\hat{T}$
%%TODO add pic
Second Approach: derive optimal rules on $\hat{T}$\\
$N_Q = 1$ $\implies$ exact for $\{1,x,y\}$
\begin{equation*}
Q(f)= w_1f(x_1,y_1)
\end{equation*}
\begin{align*}
\begin{rcases}
\frac{1}{2}&= I(1)= Q(1) = w_1  \\
\frac{1}{6}&= I(x)= Q(x) = w_1x_1  \\
\frac{1}{6}&= I(y)= Q(y) = w_1y_1\\
\end{rcases} \implies w_1= \frac{1}{2}, x_1=y_1 = \frac{1}{3}
\end{align*}