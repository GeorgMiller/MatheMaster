
\section{weak solutions of elliptic pde}

\begin{example}
	\begin{align}\label{poisson}
	-\laplace u &= f \qquad \text{in } \Omega \\
	u &= 0 \qquad \text{on } \partial \Omega \nonumber
	\end{align}
	
	$\Omega \subset \R^d $ ($d \geq 1$), $\partial \Omega \in C^{0,1}$
\end{example}

If $f \in L^2(\Omega)$, $u \in C^2(\Omega)\cap C(\overline{\Omega})$ is solution of (\ref{poisson}). Now multiply with $v \in C^\infty_0 (\Omega)$ and integrate by parts.

\begin{align*}
	\int \limits_\Omega fv \diff x &= \int \limits_\Omega \laplace uv \diff x\\
								   &= \int \limits_\Omega \nabla u \cdot \nabla v \diff x - \underbrace{\int \limits_{\partial \Omega} \left( \nabla uv\right)\cdot \vartheta \diff x}_{=0,\ v|_{\partial \Omega} = 0} \\
								   &= \int \limits_\Omega \nabla u \cdot \nabla v \diff x
\end{align*}
Because $C^\infty_0(\Omega)$ is dense in $H^1_0(\Omega)$ this also holds for all $v \in H^1_0(\Omega)$. Now define the Hilbert form
\begin{equation*}
	a(u,v) = \int \limits_\Omega \nabla u \cdot \nabla v \diff x \qquad u,v\in H^1_0(\Omega)
\end{equation*}
and linear functional

\begin{equation*}
	F(v) = \int \limits_\Omega fv \diff x \qquad v \in H^1_0(\Omega).
\end{equation*}

Now we call 
\begin{equation}\label{weak_form}
	a(u,v) = F(v) \qquad \forall v \in  H^1_0(\Omega)
\end{equation}
\textbf{weak formulation} of (\ref{poisson}).\enter


We call u
\begin{enumerate}[(a)]
	\item \textbf{weak solution}, if $u \in  H^1_0(\Omega)$ and $u$ is solution of (\ref{weak_form})\\
	\item \textbf{classical solution} of (\ref{poisson}), if \textbf{?missing part?} a weak solution of (\ref{weak_form}). 
\end{enumerate}

If $u$ is a weak solution of (\ref{weak_form}) and $u \in C^2(\Omega)\cap C(\overline{\Omega})$, u is a classical solution of (\ref{poisson}) Because for $v \ in C^\infty_0(\Omega) \subset H^1_0(\Omega)$
\begin{equation*}
	\int \limits_\Omega \left( f + \laplace u \right) v \diff x = \int \limits_\Omega  fv - \nabla u \cdot \nabla v \diff x = 0
\end{equation*}

$\implies \forall v\in L^2(\Omega):$ 
\begin{equation*}
	\int \limits_\Omega \left( f + \laplace u \right) v \diff x = 0.
\end{equation*}

$\implies f + \laplace u = 0$ almost everywhere in $\Omega$.\enter
Because $u \in H^1_0(\Omega)$, there exists $\gamma(u) = 0 = u|_{\partial \Omega}$.

\begin{definition_}
	$V$ is a real Hilbert space, $a: V \times V \to \R$ bilinearform
	\begin{enumerate}[(a)]
		\item $a$ is \textbf{continuous} in $V$ if there exists $K > 0$ s.t. $\forall u,v \in V:$ 
		\begin{equation*}
			|a(u,v)| \leq K \|u\| \|v\|.
		\end{equation*} 
		\item $a$ is \textbf{coercive} (or elliptic) in $V$ if there exists $\lambda > 0$ s.t. $\forall u \in V:$
		\begin{equation*}
			a(u,u)  \geq \lambda \|u\|^2_v
		\end{equation*}
	\end{enumerate}
\end{definition_}

\begin{thrm}
	(Lax-Milgram )\enter
	$V$ real Hilbert space, $a: V \times V \to \R$ is continuous and coercive, $F:V \to \R$. Then there exists a unique solution $u \in V$, s.t.
	\begin{equation*}
		a(u,v) = F(v) \qquad \forall v \in V
	\end{equation*}
	and it holds $\|u\|_V \leq \lambda^{-1}\|F\|_{V'}$.
	%%i think F has to be linear 	
\end{thrm}

\begin{reminder}
	(Representation theorem by Riesz)\enter
	$V$ is a real Hilbertspace with $(\cdot,\cdot)$ scalar product and $F \in \textbf{F} \in V'$. It exists a unique $u \in V$ s.t.
	\begin{equation*}
		(u,v) = F(v) \qquad \forall v\in V
	\end{equation*} 
	and $\|u\|_V = \|F\|_{V'}$.
\end{reminder}

\begin{proof_}
	of Lax-Milgram \enter
	$w \in V$. We have $a(w,\cdot) \in V'$ with Riesz there exists $T(w) \in V$ with 
	\begin{equation}\label{lm_proof}
		(T(w),v) = a(w,v) \qquad \forall v \in V
	\end{equation}
	and 
	\begin{equation*}
		\|T(w)\|_V = \|a(w,\cdot)\|_{V'} \leq K \|w\|_V.
	\end{equation*}
	This shows $T: V \to V$ is continuous and because $a(\cdot,\cdot)$ is bilinear also linear.
\end{proof_}

\par
\underline{\textbf{goal}:}
Show $T^{-1}$ exists and is continuous and linear.\enter

\begin{enumerate}[(a)]
	\item \textbf{existence}:\enter If $g \in V$ is a unique solution of 
	\begin{equation*}
	(g,v) = F(v) \qquad \forall v \in V
	\end{equation*}
	with $\|g\|_V = \|F\|_V$ we obtain with (\ref{lm_proof}) 
	\begin{equation*}
	a(T^{-1}(g),v) = (g,v) = F(v) \qquad \forall v \in V
	\end{equation*}
	and therefor $u = T^{-1}(g)$. It also holds that it is unique, as for $u_1, u_2 \in V$ two solutions $u$ have 
	\begin{equation*}
	a(u_1 - u_2,v) = 0 \qquad \forall v \in V
	\end{equation*}
	and thus for $v = u_1 - u_2$
	\begin{equation*}
	\lambda \|u_1 - u_2\|^2_V \leq a(u_1 -u_2, u_1 -u_2) = 0
	\end{equation*}
	and therefor $u_1 = u_2$.
	
	\item  $T^{-1}$\textbf{ injectiv}: \enter
	$a(\cdot, \cdot)$ is coercive $\implies \forall u \in V$
	\begin{equation*}
		\lambda \|u\|^2_V \leq (T(u),u) \leq \|T(u)\|_V\|u\|_V
	\end{equation*}
	and thus 
	\begin{equation}\label{inverse}
		\lambda \|u\|_V \leq \|T(u)\|_V
	\end{equation}
	$\implies T^{-1}$ is injective and $T^{-1}: T(V) \to V$ is continuous with
	\begin{equation*}
		\|T^{-1}(v)\|_V \leq \lambda^{-1}\|v\|_V \qquad \forall v \in T(V). 
	\end{equation*}
	We need to show
	\begin{equation*}
		T(V) = V.
	\end{equation*}
	\item $T(V)$ \textbf{is closed}:\enter
	 $(v_n) \subset T(V)$ with $v_n \to v$ for $n \to \infty$ and $v \in V$. To show $v \in T(V)$. By definition of $v_n$ exists a sequence $(u_n) \subset V$ with $v_n = T(u_n) \to v$. With (\ref{inverse}) $(u_n)$ is a Cauchy sequence in $V$ and $V$ \textbf{??missing??} follows $u_n \to u \in V$ because $T$ continuous, $T(u_n) \to T(w)$ and thus $T(w)=v$ and $v \in T(V)$\enter
	\item $T(V) = V$:\enter
	 Assume $T(V) \neq V$, as $T(V)$ is a closed linear subspace of $V$, it holds 
	\begin{equation*}
		V = T(V) \oplus T(V)^\perp
	\end{equation*}
	with $T(V)^\perp \neq \emptyset$. If $z \in T(V)^\perp $ with $z \neq 0$
	\begin{equation*}
		(T(v),z) = 0 \qquad \forall v \in V,
	\end{equation*}
	it holds with (\ref{lm_proof}) 
	\begin{equation*}
		0 = (T(z),z) = a(z,z) \geq \alpha \|z\|^2_V
	\end{equation*}
	and thus $z = 0$. \marvosymLightning\enter
	Now (\ref{inverse}) implies
	\begin{equation*}
		\|u\|_V = \|T^{-1}(g)\|_V \leq \lambda^{-1}\|J\|_V = \lambda^{-1}\|F\|_{V'}.
	\end{equation*}
	\textbf{?? J ??}
\end{enumerate}

\textbf{\underline{consider}}
\begin{itemize}
	\item Dirichlet boundary conditions(short BC)\enter
	\item Neumann BC\enter
	\item Robin BC
\end{itemize}


\begin{enumerate}[(a)]
	\item \textbf{Dirichlet BC}:\enter
	look for weak solutions of 
	\begin{align*}
		Lu &= f \qquad \text{in } \Omega\enter
		 u &= g \qquad \text{on } \partial \Omega
	\end{align*}
	with 
	\begin{equation*}
		Lu = - \displaystyle \sum^d_{i,j} \frac{\partial}{\partial x_i} \left(  a_ij(x) \frac{\partial u}{\partial x_j} \right) + c(x)u.
	\end{equation*}
	\begin{example}
		\begin{align*}
			d &= 1,&   a_{11}(x)&=1,              & c(x)&=0 \\
			  &    &            &				  &     &  \\
			d &= 2,&   a_{11}(x)&= a_{22}(x) = 1, & c(x)&=0 \\ 
			  &    &   a_{12}(x)&= a_{21}(x) = 0  &     &   \\
		  	  &    &         Lu &= -\laplace      &     &   \\
		\end{align*}
	\end{example}
	\begin{definition_}
		$L$ is \textbf{elliptic} in $\Omega$ if $\lambda > 0$ exists, s.t. $\forall \xi \in \R^d$ and \textbf{?? missing ??} $x \in \Omega$
		\begin{equation*}
			\displaystyle \sum^d_{i,j} a_{ij}(x)\xi_i\xi_j \geq \lambda |\xi|^2
		\end{equation*}
	\end{definition_}
	weak form 
	\begin{align*}
		a(u,v) &= \displaystyle \sum_{i,j} \int \limits_\Omega a_{ij}(x)\frac{\partial u}{\partial x_j} \frac{\partial v}{\partial x_j} \diff x + \int \limits_\Omega cuv \diff x \\
		F(v)&= \int \limits_\Omega fv \diff x
	\end{align*}
	Look for $u$ with $u = g \in H^1_0(\Omega)$ and transform equation to homogenous BC. $w = u = g \in H^1_0(\Omega)$ is solution of 
	\begin{align*}
		Lw &= f -Lg \qquad \text{in } \Omega\\
		 w &= 0 \qquad \qquad \ \,\text{on } \partial\Omega.
	\end{align*}
	Define right hand side
	\begin{equation*}
		G(v) = F(v) - \displaystyle \sum^d_{i,j} \int \limits_\Omega a_{ij} \frac{\partial g}{\partial x_j}\frac{\partial v}{\partial x_i} \diff x - \int \limits_\Omega cgv \diff x \quad \forall v\in H^1_0(\Omega).
	\end{equation*}
	Weak formulation: find $w\in H^1_0(\Omega)$ 
	\begin{equation*}
		a(w,v) = G(v) \qquad \forall v \in H^1_0(\Omega)
	\end{equation*}
	$\implies u = w +g$ is weak solution of original problem.
	\begin{comment_}
		$u = g $ on $\partial \Omega$ means $\gamma(u-g) = 0$
	\end{comment_}

	\begin{thrm}
		If $\Omega \subset \R^d\ (d\geq1)$ bounded, $a_{ij},c \in L^\infty(\Omega), c\geq 0, f\in L^2(\Omega), g\in H^(\Omega),\ L$ is elliptic, then there exists a unique solution $u \in H^1(\Omega)$ of $Lu=f$ with \textbf{minus or =? } $u-g \in H^1_0(\Omega)$ and 
		\begin{equation*}
			\|u\|_{H^1} \leq C\left( \|f\|_{L^2} \|g\|_{H^1} \right)
		\end{equation*}  
		with $C = C(\lambda, \Omega,a_{ij},c)>0$.
	\end{thrm}

	\begin{proof_}
		Use Lax-Milgram with $V = H^1_0$
		\begin{itemize}
			\item a is continuous:\enter
			Cauchy-Schwartz inequality
			\begin{align*}
				|a(u,v)| &\leq \displaystyle \sum^d_{i,j=1} \int \limits_\Omega |a_{ij}| |\frac{\partial u}{\partial x_i}| |\frac{\partial v}{\partial x_j}| \diff x + \int \limits_\Omega |c||u||v| \diff x \\
				& \leq \underset{i,j = 1,\dots,d}{\text{max}} \Big\{ \|a_{ij}\|_{L^\infty}, \|c\|_{L^\infty} \Big\}\left( \displaystyle \sum^d_{i,j=1} \|\frac{\partial u}{\partial x_j}\| \|\frac{\partial v}{\partial x_i}\| + \|u\|\|v\| \right)\\
				&= K \Big( \|\nabla u\|_{L^2}\|\nabla v\|_{L^2} + \|u\|_{L^2}\|v\|_{L^2} \Big)\\
				&\leq K \|u\|_{H^1}\|v\|_{H^1}
			\end{align*}
			\item a is coercive: \enter
			
		\end{itemize}
	\end{proof_}
\end{enumerate}






