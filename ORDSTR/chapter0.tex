% This work is licensed under the Creative Commons
% Attribution-NonCommercial-ShareAlike 4.0 International License. To view a copy
% of this license, visit http://creativecommons.org/licenses/by-nc-sa/4.0/ or
% send a letter to Creative Commons, PO Box 1866, Mountain View, CA 94042, USA.

\chapter{Notation}
\renewcommand{\|}{~|~}
\begin{definition}
    Seien $P,R$ Mengen. Wir nennen $\mathscr{R} = (P,R)$ ein \textbf{binäres Relat}, 
    wenn $R \subseteq P \times P$.

    Beachte, dass das Kartesische Produkt wie folgt definiert ist:
    $$ P \times Q := \{(p,q) \| p \in P \textrm{ und } q \in Q\}.$$
\end{definition}

\begin{definition}
    Sei $\mathscr{R} = (P,R)$ ein binäres Relat. Es seien $p,t,q \in P$ beliebig.
    Wir nennen $\mathscr{R}$:
    \begin{enumerate}[label=(\arabic*)]
        \item \textbf{reflexiv}, falls $(pRp)$,
        \item \textbf{transitiv}, falls $pRt \land tRq \implies pRq$,
        \item \textbf{antisymmetrisch}, falls $pRq \land qRp \implies p=q$ und
        \item \textbf{symmetrisch}, falls $pRq \implies qRp$.
    \end{enumerate}

    Beachte, dass $pRq$ für $(p,q) \in R$ steht.
\end{definition}

\begin{definition}
    Eine \textbf{Präordnung} ist ein reflexives und transitives binäres Relat. \\
    Eine \textbf{partielle Ordnung} ist eine antisymmetrische Präordnung. \\
    Eine partielle Ordnung $\mathscr{R} = (P,R)$ heißt \textbf{lineare} Ordnung (oder \textbf{Totalordnung}),\\
    falls für alle $p,q \in P$: $$pRq \textrm{ oder } qRp.$$
\end{definition}

\begin{notation}
    Wenn $A,B$ Mengen, so bezeichnet $B^A$ die Menge aller Abbildungen von $A$ nach $B$.
\end{notation}

\begin{aufgabe}
    Sei $N$ eine endliche Menge und 
    sei $\R^N$ mit der \textbf{Dominanzordnung}, d.h. für alle $u,v \in \R^N$ gilt
    $$u \leq v :\iff \forall i \in N (u_i \leq v_i),$$
    gegeben.
    
    Wie lässt sich $\R^N$ linear erweitern?
\end{aufgabe}
\begin{lösung}
    Zunächst wähle eine Totalordnung auf $ N $ und benenne die Elemente so um, dass
    $N = \{1,2,\ldots,n\}$, wobei $n \in \N$.
    Definiere für alle $u,v \in \R^N$:
    $$ I^v_u := \{i \in N \| u_i \leq v_i\}.$$

    Nun definiere die lineare Ordnung wie folgt:
    $$ u \leq_l v :\iff \min I^v_u \leq \min I^u_v.$$
\end{lösung}

\begin{beispiel}
\begin{align*}
(\R^2,\leq)&=(\R,\leq)^2\\
a=(a_1,a_2)&\leq(b_1,b_2)=b:\Longleftrightarrow a_1\leq b_1\wedge a_2\leq b_2
\end{align*}
$(R^2,\sqsubseteq)$ ist \textbf{lineare Erweiterung} von $(\R,\leq)^2$ via 
\begin{align*}
a\sqsubseteq b:\Longleftrightarrow a_1<b_1\vee a_1=b_1\wedge a_2\leq b_2
\end{align*}
Frage: Sei $(S,\leq)$ lineare Ordnung (d.h. reflexiv, transitiv, antisymmetretisch und linear. Dann ist $(S,\leq)^\Omega:=(S^\Omega,\leq)$ Ordnung via
\begin{align*}
\alpha\leq\beta:\Longleftrightarrow\forall\alpha,\beta\in S^\Omega,\forall x\in\Omega:\alpha x\leq\beta x
\end{align*}
\end{beispiel}

\begin{definition}
Ein binäres Relat $(P,R)$ heißt \textbf{linear / total}
\begin{align*}
:\Longleftrightarrow\forall p,q\in P:p~R~q\vee q~R~p
\end{align*}
\end{definition}
\textbf{Frage:} Seien $(\Omega,\leq)$ und $(S,\leq)$ lineare Ordnungen. Existiert dann auch eine lineare Ordnung $(S^\Omega,\sqsubseteq)$, die $(S,\leq)^\Omega$ erweitert, d.h.
\begin{align*}
\forall\alpha,\beta\in S^\Omega:\alpha\leq\beta\implies\alpha\sqsubseteq\beta
\end{align*}
\begin{definition}
$(\Omega,\leq)$ heißt \textbf{Wohlordnung} $:\gdw(\Omega,\leq)$ eine Ordnung derart ist, dass jede nichtleere Teilmenge ein (bzgl. der Ordnung) kleinstes Element enthält.
\end{definition}

\begin{beispiel}
$(\N,\leq)$ ist Wohlordnung.
\end{beispiel}

\begin{bemerkung}
$(\Omega,\leq)$ Wohlordnung $\implies(\Omega,\leq)$ lineare Ordnung.
\end{bemerkung}
\begin{proof}
Sei $x,y\in\Omega$. Dann hat $\lbrace x,y\rbrace$ ein kleinstes Element bzgl. $(\Omega,\leq)$. Also ist $(\Omega,\leq)$ linear.
\end{proof}

\textbf{Mitteilung.} Aus dem Auswahlaxiom bzw. dem Zornschen Lemma folgt, dass jede Menge eine Wohlordnung besitzt (d.h. zu jeder Menge $\Omega$ existiert $\leq$ derart, dass $(\Omega,\leq)$ Wohlordnung ist).

\begin{satz}
Sei $(\Omega,\leq)$ Wohlordnung und sei $(S,\leq)$ lineare Ordnung.\\
Dann ist $(S^\Omega,\sqsubseteq)$ lineare Ordnung vermöge
\begin{align*}
\Omega(\alpha<\beta)&:=\lbrace x\in\Omega:\alpha x<\beta x\rbrace &\forall \alpha,\beta\in S^\Omega\\
\Omega(\alpha=\beta)&:=\lbrace x\in\Omega\mid\alpha x=\beta x\rbrace &\forall \alpha,\beta\in S^\Omega\\
\Omega(\alpha>\beta)&:=\lbrace x\in\Omega:\beta x<\alpha x\rbrace &\forall\alpha,\beta\in S^\Omega
\end{align*}
Setze dann
$\alpha\sqsubseteq\beta:\gdw\Omega(\beta>\alpha)=\emptyset$ oder $\alpha=\beta$ oder $\Omega(\alpha<\beta)$ und $\Omega(\beta>\alpha)$ sind nichtleer und das kleinste Element von $\big(\Omega(\alpha<\beta),\leq\big)$ ist kleiner als das kleinste Element von $\big(\Omega(\beta<\alpha),\leq\big)$.
\end{satz}

%TODO Hier fehlt die 5. Vorlesung

\begin{satz}
Sei $(\Omega,\leq_\Omega)$ Wohlordnung und sei $(S,\leq_S)$ lineare Ordnung.\\
Dann ist $(S^\Omega,\sqsubseteq)$ via
\begin{align*}
\alpha\sqsubseteq\beta:\Longleftrightarrow\alpha=\beta\vee\exists x\in\Omega:\Big(\alpha x<_S\beta x\wedge\forall t\in\Omega:\big(t<_\Omega x\implies \alpha t=\beta t\big)\Big)
\qquad\forall\alpha,\beta\in S^\Omega
\end{align*}
eine lineare Ordnung definiert, die $(S,\leq_S)^\Omega$ erweitert.
\end{satz}

\begin{bemerkung}
$(S^\Omega,\sqsubseteq)$ ist die \textbf{lexikografische} Ordnungserweiterung von $(S,\leq_S)^\Omega$ via $(\Omega,\leq_\Omega)$.
\end{bemerkung}

%TODO Hier fehlt die 7. Vorlesung

\begin{align*}
ID(\Z)=\big(\lbrace n\in\Z\mid n\in\N\rbrace,\subseteq\big)
\end{align*}
ist Idealverband von $\Z=(\Z,+,\cdot,0,1)$ bzw. Untergruppenverband von $(\Z,+,0)$.\\

\textbf{ASCENDING CHAIN CONDITION (ACC) / noethersch} $\bar{\uparrow}$\\
``Jede aufsteigende Kette terminiert'' $\gdw$ jede Teilmenge hat ein maximales Element (aufsteigende Kettenbedingung)\\

\textbf{DESCENDING CHAIN CONDITION (DCC) / artinsch} $\bar{\downarrow}$\\
``Jede absteigende Kette terminiert'' $\gdw$ jede Teilmenge hat ein maximales Element.\\
$(\N,<_\tau)$ mit
\begin{align*}
\leq_\tau:=\big\lbrace %TODO
\end{align*}


%Hier die 13. Vorlesung (20.11.2018).
\begin{bemerkung}
Für $x,y\in[0,\infty]$ mit $x\leq y$ ist also $\delta(x,y)$ das kleinste Element $t$ in $\big([0,\infty],\leq\big)$ mit $x+t=y$, d.h. $t$ ist die kleinste ``Ergänzung'' von $x$ zu $y$ im natürlich geordneten Monoid $\big([0,\infty],+,0\big)$\\
Man erkennt hieraus leicht, dass $\delta$ funktoriell ist:
\begin{align*}
\delta(x,x)=0,\qquad\delta(x,y)+\delta(y,z)=\delta x,z)\qquad\forall x\leq y\leq z
\end{align*}
hingegen $\overline{\delta}$ nicht. Weiterhin ist $\delta$ modular:
\begin{align*}
\delta(x,x\vee y)=\delta x\wedge y,y)
\end{align*}
genauso wie $\overline{\delta}$!\\
Auch wenn $\overline{\delta}$ schon nicht funktoriell ist, dan ngilt zumindest die Dreiecksungleichung
\begin{align*}
\overline{\delta}(x,z)\leq\overline{\delta}(x,y)+\overline{\delta}(y,z)
\end{align*}
Definieren wir $\overline{\delta}^{\text{op}}$ als $\overline{\delta}^{\text{op}}(y,x):=\overline{\delta}(x,y)$. Dann ist
\begin{align*}
d:[0,\infty]^2\to[0,\infty],\qquad (x,y)\mapsto\overline{\delta}(x,y)+\overline{\delta}^{\text{op}}(x,y)
\end{align*}
eine Metrik mit
\begin{align*}
d(x,y)=\left\lbrace\begin{array}{cl}
|x-y|, &\falls x\neq\infty,y\neq0\\
0, &\falls x=y=\infty\\
\infty, &\sonst
\end{array}\right.
\end{align*}
\end{bemerkung}

\begin{definition}
Sei $\P=(P,R)$ eine Präordnung (d.h. reflexiv + transitiv) und sei $\mathbb{M}=(M,\ast,\varepsilon)$ ein Monoid. Dann heiße eine Abbildung
\begin{align*}
\Delta:R\to M
\end{align*}
\textbf{funktoriell} bzgl. $(\P,\mathbb{M}):\gdw\forall p~R~t,t~R~q\text{ in }P:$
\begin{enumerate}
\item $\Delta(p,p)=\varepsilon$
\item $\Delta(p,t)\ast\Delta(t,q)=\Delta(p,q)$
\end{enumerate}
\end{definition}
 
\begin{beispiel}
\begin{align*}
&\left\lbrace\begin{matrix}
\P=\big([0,\infty],\leq\big)\\
\mathbb{M}=\big([0,\infty],+,0\big)
\end{matrix}\right.\implies \mit\Delta:=\delta\\
&\left\lbrace\begin{matrix}
\P=(P,R)\mit P:=[0,\infty]^{\N}\\
\mathbb{M}=\big([0,\infty],+,+\big)\\
\Delta:(\alpha,\beta)\mapsto\sum\limits_{n\in\N}\delta(\alpha n,\beta n)\cdot 2^n
\end{matrix}\right.
\end{align*}
Was ist mit Modularität? Definition ist allgemein. Join $(\vee)$ und meet $(\wedge)$
\end{beispiel}

\section*{Mini-Ausflug ``MEASUREMENT SETUP''}
\begin{definition}
Ein MEASUREMENT SETUP ist erklärt als Tripel $\mathcal{M}=(G,\mathbb{M},\Delta)$ bestehend aus einem \textbf{Aktionsnetzwerk} $\mathbb{G}=(G,\ast,\id)$ mit zugrundeliegendem \textbf{Netzwerk} $G=(V,E,\rho)$, einem Monoid $\mathbb{M}=(M,\ast,\varepsilon)$ und einer \textbf{funktoriellen} Abbildung $\Delta:E\to M$.
\end{definition}
Interpreation:
\begin{itemize}
\item $\mathbb{G}$ ist WAS ich messe.
\item $\mathbb{M}$ ist WORIN ich messe.
\item $\Delta$ ist WIE ich (be)messe.
\end{itemize}
Weitere Einschränkungen: Ein \textbf{Netzwerk} ist definiert als Tripel $G=(V,E,\rho)$ bestehend aus Mengen $V,E$ sowie einer Abbildung $\rho:E\to V\times V$\\
Interpretation: $V$ ist Knotenmenge / Pfeilmenge (vertex set) und $E$ ist Kantenmenge(edge set) und $\rho$ ist Strukturabbilung und wir setzen
\begin{align*}
(\sigma e,\tau e):=\rho e:=\rho(e)\qquad\forall e\in E\\
\mit\sigma:=\pi_1\circ\rho,\qquad\tau:=\pi_2\circ\rho
\end{align*}
wobei $\pi_1,\pi_2$ die Projektionen auf die erste bzw. die zweite Komponente sind.\\
Ein Tripel $\mathbb{G}=(G,\ast,\id)$ heißt \textbf{Aktionsnetzwerk (ANW)}, falls $G=(V,E,\rho)$ Netzwerk ist und 
\begin{align*}
\ast:E^{\langle 2\rangle}\to E, (a,b)\mapsto a\ast b
\end{align*}
Abbildung ist, wobei
\begin{align*}
E^{\langle 2\rangle}:=\big\lbrace (a,b)\in E\times E~\big|~\tau =\tau b\big\rbrace
\end{align*}
die Menge der Pfade der Länge 2 ist und $\id:V\to E$ Abbildung ist, derart, dass die \textbf{Verkettungsaxiome} gelten:
\begin{align*}
\rho(a\ast b)&=(\sigma a,\tau b) &\forall& (a,b)\in E^{\langle 2\rangle}\\
(a\ast b)\ast c&=a\ast(b\ast c) &\forall& (a,b,c)\in E^{\langle 3\rangle}\text{ Assoziativität}
\end{align*}
wobei 
\begin{align*}
E^{\langle 3\rangle}:=\big\lbrace(a,b,c)\in E^3~\big|~\tau a=\sigma b\text{ und }\tau b=\sigma c\big\rbrace
\end{align*}
die Menge der Pfade der Länge 3 in $G$.
\textbf{Neutralitäts-Axiome:}
\begin{align*}
\rho (\id p)&=(p,p) &\forall& p\in V\\
\id(\sigma e)\ast e&=e=e\ast\id(\tau e) &\forall& e\in E
\end{align*}
Kanten interpretieren wir als Aktionen und Knoten als Zustände.

\begin{beispiel}\
\begin{itemize}
\item Pfadaktionsnetzwerk zu einem Netzwerk
\item logistisches Aktionsnetzwerk zu einer Menge $P$:
\begin{align*}
\mathbb{G}_P&:=(G_P,\ast,\id)\\
\id p:=\id(p)&:=(p,p)\\
G_P&:=(P,P\times P,\id_{P\times P})\\
(p,t)\ast(t,q)&:=(p,q)\text{ das Weglassprodukt}
\end{align*}
\item Monoide (1-wertige ANWs), der ``Schreibtischtäter''
\end{itemize}
\end{beispiel}

Die Abbildung $\Delta:E\to M$ heiße \textbf{funktoriell} bzgl. $\mathbb{G},\mathbb{M}):\gdw$ 
\begin{enumerate}
\item $\Delta \id(p)=\varepsilon\qquad\forall p\in V$
\item $\Delta a\ast b=\Delta a\ast\Delta b\qquad\forall(a,b)\in E^{\langle2\rangle}$
\end{enumerate}


