\chapter{Notation}
\renewcommand{\|}{~|~}
\begin{definition}
    Seien $P,R$ Mengen. Wir nennen $\mathscr{R} = (P,R)$ ein \textbf{binäres Relat}, 
    wenn $R \subseteq P \times P$.

    Beachte, dass das Kartesische Produkt wie folgt definiert ist:
    $$ P \times Q := \{(p,q) \| p \in P \textrm{ und } q \in Q\}.$$
\end{definition}

\begin{definition}
    Sei $\mathscr{R} = (P,R)$ ein binäres Relat. Es seien $p,t,q \in P$ beliebig.
    Wir nennen $\mathscr{R}$:
    \begin{enumerate}[label=(\arabic*)]
        \item \textbf{reflexiv}, falls $(pRp)$,
        \item \textbf{transitiv}, falls $pRt \land tRq \implies pRq$,
        \item \textbf{antisymmetrisch}, falls $pRq \land qRp \implies p=q$ und
        \item \textbf{symmetrisch}, falls $pRq \implies qRp$.
    \end{enumerate}

    Beachte, dass $pRq$ für $(p,q) \in R$ steht.
\end{definition}

\begin{definition}
    Eine \textbf{Präordnung} ist ein reflexives und transitives binäres Relat. \\
    Eine \textbf{partielle Ordnung} ist eine antisymmetrische Präordnung. \\
    Eine partielle Ordnung $\mathscr{R} = (P,R)$ heißt \textbf{lineare} Ordnung (oder \textbf{Totalordnung}),\\
    falls für alle $p,q \in P$: $$pRq \textrm{ oder } qRp.$$
\end{definition}

\begin{notation}
    Wenn $A,B$ Mengen, so bezeichnet $B^A$ die Menge aller Abbildungen von $A$ nach $B$.
\end{notation}

\begin{aufgabe}
    Sei $N$ eine endliche Menge und 
    sei $\R^N$ mit der \textbf{Dominanzordnung}, d.h. für alle $u,v \in \R^N$ gilt
    $$u \leq v :\iff \forall i \in N (u_i \leq v_i),$$
    gegeben.
    
    Wie lässt sich $\R^N$ linear erweitern?
\end{aufgabe}
\begin{lösung}
    Zunächst wähle eine Totalordnung auf $ N $ und benenne die Elemente so um, dass
    $N = \{1,2,...,n\}$, wobei $n \in \N$.
    Definiere für alle $u,v \in \R^N$:
    $$ I^v_u := \{i \in N \| u_i \leq v_i\}.$$

    Nun definiere die lineare Ordnung wie folgt:
    $$ u \leq_l v :\iff \min I^v_u \leq \min I^u_v.$$
\end{lösung}
 