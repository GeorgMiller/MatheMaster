

\subsection{Forster}

Die folgenden S\"atze sind aus \textit{Analysis 3} von \textit{Otto Forster} und werden hier zur Wiederholung und Übersicht kurz beleuchtet.
\enter

\begin{satz}
	Sei $p > 1$ und $f \in L^p(\Omega)$ mit $\Omega \subset \R^n$. 
	\renewcommand{\labelenumi}{(\alph{enumi})}
	\begin{enumerate}
		\item Sei $q$ konjugierter Exponent zu $p$ (d.h. $\frac{1}{q} + \frac{1}{p}=1$)
			$\implies fg \in L^q(\Omega)$.
		\item Falls $|\Omega| < \infty$ gilt $f \in L^1(\Omega)$
		\item $ L^1(\Omega) \cap L^{\infty}(\Omega) \subset L^p(\Omega) \quad \forall p\in [1,\infty) $
	\end{enumerate}
\end{satz}

\begin{proof}
	\enter
	\renewcommand{\labelenumi}{(\alph{enumi})}
	\begin{enumerate}
		\item H\"older
		\item folgt aus (a) mit $g\equiv1$
		\item
			\begin{align*}
				\int |g|^p &= \int |g|^{p-1}|g| \\
									 &\leq \|g\|^{p-1}_{L^\infty(\Omega)}\int |g| \\
									 &= \|g\|^{p-1}_{L^\infty(\Omega)}\|g\|_{L^1}(\Omega)
			\end{align*}
	\end{enumerate}

\end{proof}

\begin{satz}\textbf{(Majorierte Konvergenz für $L^p$)} \enter
	Sei $p \geq 1$ und $f_k \in L^p(\Omega)$, sodass eine Funktion $f:\Omega\to [-\infty,\infty]$ ex. mit
	\[f_k \to f\text{ punktweise f.\"u.}\]
	Zudem ex. $F: \Omega \to [0,\infty]$ mit $F \in L^p(\Omega)$ sd. 
	\[|f_k| \subset F \quad \text{ f.\"u. } \quad \forall k\]
	Dann ist $f\in L^p(\Omega)$ und 
	\[f_k \to f \text{ in }L^p(\Omega)\]
\end{satz}

\begin{proof}
	Forster
\end{proof}

\begin{satz} \enter
	Sei $p \geq 1$ und $(f_m) \subset L^p(\Omega)$ eine $L^p$-Cauchy-Folge(d.h.
	bzgl. $L^p$-Norm).\enter
	Dann ex. eine Teilfolge ($f_{m_k}$) und eine Funktion
	$f:\Omega\to \R$ sd. 
	\[f_{m_k} \overset{k\to \infty}{\longrightarrow} f \quad \text{ f.ü.}\]
	Zudem ist $f\in L^p(\Omega)$ und
	\[f_m \stackrel{m\rightarrow \infty}{\longrightarrow} f \quad \text{ in }
	L^p(\Omega)\]
	M.a.W. ist $\left(L^p(\Omega), ||\cdot||_{L^p(\Omega)}\right)$ vollst\"andig 
	\[\implies \text{ also ein Banachraum } \]
\end{satz}

\begin{proof}
	Forster
\end{proof}

\begin{satz} \enter
	Für jedes $p \geq 1$ liegt $C_c(\R^n)$ dicht in $L^p(\R^n)$. \enter
	D.h. für alle $f \in L^p(\R^n)$ und für alle $\varepsilon > 0$ ex. ein
	$\varphi \in C^0_c(\R^n)$ mit 
	\[||f-\varphi||_{ L^p(\R^n)} < \varepsilon\]
\end{satz}

\begin{proof}
	Approximiere $f$ durch eine Treppenfunktion mit endlich vielen Werten und \glqq sch\"onen\grqq\ Urbildmengen. Dann approximiere diese Treppenfunktionen durch stetige Funktionen
\end{proof}

\begin{satz}\enter
	Der vorherige Satz gilt auch mit $C^\infty_c(\R^n)$ statt $C^0_c(\R^n)$.
\end{satz}

\begin{proof}
	siehe Forster oder Evans via \glqq mollification\grqq = Faltung mit Glättungskern
\end{proof}


\subsection{Evans Appendix C}

\begin{itemize}
	\item \textbf{Definition Glättungskerne} \enter
		zum Beispiel:
		\[\eta(x):= \frac{1}{|x|^2-1}\]
		\[\rightarrow \eta_\epsilon(x):=\frac{1}{\epsilon^n}\eta\left(\frac{x}{\epsilon})\]
			(vgl. PDF-Vertiefungslorlesung)
		\item \textbf{Glättung von $f$} \enter
			Die Glättung von $f$ wird definiert durch:
			\[f^\epsilon:=f\ast\eta_\epsilon\]
		\item \textbf{Eigenschaften von Glättungen}
			\newpage
		\item \textbf{lokale $L^p$-Konvergenz}\enter
			\[f_k \rightarrow f \text{ in } L^p_{loc}(\Omega)\]
			($:\iff$ für alle $V$ offen und $\overline{V}\subset\Omega$ gilt
				$f_k\rightarrow f \text{ in } L^p(V)$)
\end{itemize}
