% This work is licensed under the Creative Commons
% Attribution-NonCommercial-ShareAlike 4.0 International License. To view a copy
% of this license, visit http://creativecommons.org/licenses/by-nc-sa/4.0/ or
% send a letter to Creative Commons, PO Box 1866, Mountain View, CA 94042, USA.

\chapter*{Wiederholung}
Die folgenden Begriffe werden und Zusammenhänge werden als bekannt
vorausgesetzt und können jederzeit in \textit{L. C. Evans , Partial
Differential Equations} nachgelesen werden.
\enter
\section{Evans Appendix E}
\begin{itemize}
	\item Lebesgue Maß
\item $\sigma$-Algebra
\item Lebesgue-messbare Mengen
\item Messbarkeit
\item Lebesgue Integral
\item $\text{ess}\sup$ (für $L^\infty$)
\item Grenzwerttheorie
\begin{itemize}
\item Fatou
\item Monotone Konvergenz
\item dominierte Konvergenz
\end{itemize}
\item Differentiationstheoreme:\enter
Für fast alle $x_0 \in \R$ gelten:
\[\aveint{B(x_0,r)}{} f(x) \d x \stackrel{r\rightarrow 0}{\longrightarrow} f(x_0)\]
und
\[\aveint{B(x_0,r)}{} |f(x) - f(x_0)| \d x \stackrel{r\rightarrow 0}{\longrightarrow} 0\]
	$x_0$ heißt dann Lebesgue-Punkt
\end{itemize}
Als nächstes folgt Theorie zu den $L1p$-Räumen. Siehe dafür
\textit{Forster, Analysis 3}
\section{Forster § 12}
\begin{itemize}
	\item \textbf{Definition $L^p$ Norm}\enter
		Für $p\in[1,\infty]$ und messbare Funktionen
\[f:\Omega \rightarrow [-\infty,\infty]\]
definiert man
\[\|f\|_{L^p(\Omega)}:=\left(\int_\Omega |f|^p\right)^{\frac{1}{p}}\]
($\Omega$ ist Maßraum, bei uns offene Teilmenge des $\R^n$)
\item \textbf{Minkowski-Ungleichung}\enter
	Seien $f,g:\Omega\rightarrow[-\infty,\infty]$ messbar, dann
	\[\|f+g\|_{L^p(\Omega)} \leq \|f\|_{L^p(\Omega)} + \|g\|_{L^p(\Omega)}\]
	Dies ist eine Folgerung aus der Hölder-Ungleichung (siehe Forster): \enter
	für $p\in[1,\infty]$ und $q$ mit $\frac{1}{p} + \frac{1}{q} = 1$:
	\[\|fg\|_{L^1(\Omega)} \leq \|f\|_{L^p(\Omega)}\|g\|_{L^q(\Omega)}\]
\item \textbf{Definition $\mathcal{L}^p$ und $L^p$} \enter
	Definiere:
	\[\mathcal{L}^p(\Omega):=\{f:\Omega\rightarrow \R \text{ messbar mit } \|f\|_{L^p(\Omega)}<\infty\}\]
	Auf $\mathcal{L}^p(\Omega)$ ist $\|\cdot\|_{L^p(\Omega)}$ eine Seminorm bzw. Halbnorm wegen der Minkowski-Ungleichung und 
	\[\|cf\|_{L^p(\Omega)}=|c|\cdot\|f\|_{L^p(\Omega)}\]
	aber keine Norm, da 
	\[\|f\|_{L^p(\Omega)}= 0 \nRightarrow f(x) = 0 \qquad \forall x \in \Omega\]
	Um aus $\|\cdot\|_{L^p(\Omega)}$ eine Norm zu erhalten, bilden wir den Quotientenraum
	\[L^p(\Omega) := \mathcal{L}^p(\Omega)/\sim\]
	für die Relation $\sim$, die definiert wird über
	\[f\sim g :\iff f=g \text{ fast überall}\]
	Insbesondere gelten zum Beispiel punktweise Aussagen über $f$ für jeden Repräsentanten oder für einen Repräsentanten. Im Zweifelsfall wähle immer den Repräsentanten
	\[f^*(x):=\begin{cases}
			\lim\limits_{r\downarrow 0}\aveint{B(0,r)}{}f & \text{, falls $\lim$ ex.} \\
			0 & \text{, sonst}
	\end{cases}\]
	Dies nutzt unter anderem aus, dass für jedes $p\in[1,\infty]$ gilt
	\[\|f\|_{L^p(\Omega)}=0 \iff f = 0 \text{ fast überall}\]
	\item \textbf{Beziehung zwischen $L^p$ und $L^1$ für $p>1$} \enter
		\begin{itemize}
			\item wenn $|\Omega| < \infty$, dann $L^p(\Omega) \subset L^1(\Omega)$
			\item i.A. ist $L^1(\Omega) \neq L^p(\Omega)$
		\end{itemize}

		\beisp \enter
		Für $\Omega=\R^n$ und $f(x)=\frac{1}{1+|x|^n}$ gilt
		\[f \in L^p(\R^n) \quad \forall p>1, \text{ aber } f\not\in L^1(R^n)\]
		Umgekehrt sei nun $\Omega = B(0,1)$ und $g:B(0,1)\rightarrow \R$ definiert durch
		\[g(x)=\begin{cases}
				\frac{1}{|x|^{\frac{n}{p}}} & ,\forall x \neq 0 \\
				0 & ,x=0 
		\end{cases}\]
		Dann gilt $g\in(L^1\setminus L^p)(\Omega)$
\end{itemize}


Die folgenden S\"atze stammen ebenfalls aus \textit{Analysis 3} von \textit{Otto Forster} und werden hier zur Wiederholung und Übersicht kurz beleuchtet.
\enter

\begin{satz}
	Sei $p > 1$ und $f \in L^p(\Omega)$ mit $\Omega \subset \R^n$. 
	\renewcommand{\labelenumi}{(\alph{enumi})}
	\begin{enumerate}
		\item Sei $q$ konjugierter Exponent zu $p$ (d.h. $\frac{1}{q} + \frac{1}{p}=1$)
			$\implies fg \in L^q(\Omega)$.
		\item Falls $|\Omega| < \infty$ gilt $f \in L^1(\Omega)$
		\item $ L^1(\Omega) \cap L^{\infty}(\Omega) \subset L^p(\Omega) \quad \forall p\in [1,\infty) $
	\end{enumerate}
\end{satz}

\begin{proof}
	\enter
	\renewcommand{\labelenumi}{(\alph{enumi})}
	\begin{enumerate}
		\item H\"older
		\item folgt aus (a) mit $g\equiv1$
		\item
			\begin{align*}
				\int |g|^p &= \int |g|^{p-1}|g| \\
									 &\leq \|g\|^{p-1}_{L^\infty(\Omega)}\int |g| \\
									 &= \|g\|^{p-1}_{L^\infty(\Omega)}\|g\|_{L^1}(\Omega)
			\end{align*}
	\end{enumerate}

\end{proof}

\begin{satz}\textbf{(Majorierte Konvergenz für $L^p$)} \enter
	Sei $p \geq 1$ und $f_k \in L^p(\Omega)$, sodass eine Funktion $f:\Omega\to [-\infty,\infty]$ ex. mit
	\[f_k \to f\text{ punktweise f.\"u.}\]
	Zudem ex. $F: \Omega \to [0,\infty]$ mit $F \in L^p(\Omega)$ sd. 
	\[|f_k| \subset F \quad \text{ f.\"u. } \quad \forall k\]
	Dann ist $f\in L^p(\Omega)$ und 
	\[f_k \to f \text{ in }L^p(\Omega)\]
\end{satz}

\begin{proof}
	Forster
\end{proof}

\begin{satz} \enter
	Sei $p \geq 1$ und $(f_m) \subset L^p(\Omega)$ eine $L^p$-Cauchy-Folge(d.h.
	bzgl. $L^p$-Norm).\enter
	Dann ex. eine Teilfolge ($f_{m_k}$) und eine Funktion
	$f:\Omega\to \R$ sd. 
	\[f_{m_k} \overset{k\to \infty}{\longrightarrow} f \quad \text{ f.ü.}\]
	Zudem ist $f\in L^p(\Omega)$ und
	\[f_m \stackrel{m\rightarrow \infty}{\longrightarrow} f \quad \text{ in }
	L^p(\Omega)\]
	M.a.W. ist $\left(L^p(\Omega), ||\cdot||_{L^p(\Omega)}\right)$ vollst\"andig 
	\[\implies \text{ also ein Banachraum } \]
\end{satz}

\begin{proof}
	Forster
\end{proof}

\begin{satz} \enter
	Für jedes $p \geq 1$ liegt $C_c(\R^n)$ dicht in $L^p(\R^n)$. \enter
	D.h. für alle $f \in L^p(\R^n)$ und für alle $\varepsilon > 0$ ex. ein
	$\varphi \in C^0_c(\R^n)$ mit 
	\[||f-\varphi||_{ L^p(\R^n)} < \varepsilon\]
\end{satz}

\begin{proof}
	Approximiere $f$ durch eine Treppenfunktion mit endlich vielen Werten und \glqq sch\"onen\grqq\ Urbildmengen. Dann approximiere diese Treppenfunktionen durch stetige Funktionen
\end{proof}

\begin{satz}\enter
	Der vorherige Satz gilt auch mit $C^\infty_c(\R^n)$ statt $C^0_c(\R^n)$.
\end{satz}

\begin{proof}
	siehe Forster oder Evans via \glqq mollification\grqq = Faltung mit Glättungskern
\end{proof}


\section{Evans Appendix C}

\begin{itemize}
	\item \textbf{Definition Glättungskerne} \enter
		zum Beispiel:
		\[\eta(x):= \frac{1}{|x|^2-1}\]
		\[\rightarrow \eta_\epsilon(x):=\frac{1}{\epsilon^n}\eta\left(\frac{x}{\epsilon}\right)\]
			(vgl. PDF-Vertiefungslorlesung)
		\item \textbf{Glättung von $f$} \enter
			Die Glättung von $f$ wird definiert durch:
			\[f^\epsilon:=f\ast\eta_\epsilon\]
		\item \textbf{Eigenschaften von Glättungen}\enter
			Symmetrie, etc.
		\item \textbf{lokale $L^p$-Konvergenz}\enter
			\[f_k \rightarrow f \text{ in } L^p_{loc}(\Omega)\]
			($:\iff$ für alle $V$ offen und $\overline{V}\subset\Omega$ gilt
				$f_k\rightarrow f \text{ in } L^p(V)$)
\end{itemize}
