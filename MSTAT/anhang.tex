% This work is licensed under the Creative Commons
% Attribution-NonCommercial-ShareAlike 4.0 International License. To view a copy
% of this license, visit http://creativecommons.org/licenses/by-nc-sa/4.0/ or
% send a letter to Creative Commons, PO Box 1866, Mountain View, CA 94042, USA.

\setcounter{chapter}{0}
\renewcommand{\thechapter}{\Alph{chapter}}
\chapter{Anhang}
\setcounter{equation}{1}
\section{Grundlagen, die man kennen sollen}
\begin{itemize}
\item $f:\R\to\R$ heißt \textbf{Borel-messbar} $:\gdw\forall M\in\B(\R):f^{-1}(M)\in\B(\R)$
\item Sei $(\Omega,\A, \P)$ WRaum und $(S,\B)$ Messraum. Eine \textbf{Zufallsvariable (ZV)} $X$ ist eine $(\A,\B)$-messbare Abbildung $X:\Omega\to S$\\
Wenn $S=\R$, dann heißt $X$ \textbf{Zufallsgröße (ZG)}
\item Die \textbf{Verteilung} einer ZG $X$ ist W-Maß auf $\B$:
\begin{align*}
\mu_X:\B\to[0,1],~
\mu_X(B):=\P[X\in B]:=\P(X^{-1}(B))=\P(\lbrace w\in\Omega:X(w)\in B\rbrace)\\\forall B\in\B
\end{align*}
\item Die \textbf{Dichte} ist
\begin{align*}
p:\Rn\to[0,\infty]\mit \mu_X(A)=\int\limits_A p(x)\d\mathcal{L}(x)\qquad\forall A\in\B(\Rn)
\end{align*}
\item \textbf{Verteilungsfunktion} von $X$ ist
\begin{align*}
F_X:\R\to[0,1],\qquad F_X(x)=\P[X<x]=\mu_X(]-\infty,x[)=\int\limits_{-\infty}^x p(t)\d t
\end{align*}
mit der Eigenschaft
\begin{align*}
F'(x)=p(x)\qquad\forall x\in\R~~~\mathcal{L}\text{fast überall}
\end{align*}
\item Zwei Zufallsvariablen $X_1:\Omega\to S_1$ und $X_2:\Omega\to S_2$ heißen \textbf{unabhängig} 
\begin{align*}
:\Longleftrightarrow
\forall B_1\in\B_1,\forall B_2\in\B_2:\\
\P[X_1\in B_1]\cdot \P[X_2\in B_2]
&=
\P[X_1\in B_1, X_2\in B_2]\\
&:=\P\big(\lbrace\omega\in\Omega:X_1(\omega)\in B_1\wedge X_2(\omega)\in B_2\rbrace\big)
\end{align*}
\item Sei $X$ eine $\P$-integrierbare oder nichtnegative Zufallsgröße. Dann ist der \textbf{Erwartungswert von $X$} definiert als
\begin{align*}
\E(X):=\int\limits_\Omega X(\omega)\d\P(\omega)
\stackeq{\text{\ref{eqTrafo}}}
\int\limits_\R x\d\mu_X (x)
\stackeq{p\text{ Dichte}}
\int\limits_\R x\cdot p(x)\d x
\end{align*}
und hat folgende Eigenschaften ($X,Y$ seien Zufallsgrößen):
\begin{enumerate}
\item Linearität: $\forall a,b\in\R:\E(a\cdot X+b\cdot Y)=a\cdot \E(X)+b\cdot\E(Y)$
\item $X=c\in\R$ fast sicher konstant $\Longrightarrow\E(X)=c$
\item $a\leq X\leq b$ fast sicher konstant $\Longrightarrow a\leq\E(X)\leq b$
\item $|\E(X)|\leq\E(|X|)$
\item $X\geq0$ fast sicher und $\E(X)=0\Rightarrow X=0$ fast sicher
\item $X,Y$ unabhängig $\Longrightarrow\E(X\cdot Y)=\E(X)\cdot E(Y)$
\end{enumerate}
\item Zwei ZG heißen \textbf{unkorreliert} $:\gdw\E[X\cdot Y]=\E[X]\cdot\E[Y]$
\item Für $X\in L^2(\P)$ ist die \textbf{Varianz} 
\begin{align*}
\Var(X):=\E[(X-\E[X])^2]=\int\limits_\Omega(X-\E[X])^2\d\P=E[X^2]-(\E[X])^2
\end{align*}
mit den Eigenschaften
\begin{align*}
\Var(a\cdot X+b)&=a^2\cdot\Var(X)\\
\Var(X)&=0\Longleftrightarrow X\text{ ist konstant fast sicher}\\
\Var(X+Y)&=\Var(X)+\Var(Y)+\underbrace{2\E[(X-\E[X])\cdot(Y-\E[Y])]}_{=0\text{, falls $X,Y$ unkorreliert}}
\end{align*}
\end{itemize}

\section{Wichtige Sätze}
\begin{satz}[Korrespondenzsatz]\label{satzKorrespondenzsatz}\enter
Jede Verteilungsfunktion $F$ ist Verteilungsfunktion eines eindeutigen Wahrscheinlichkeitsmaßes $\P$. Dieses Maß $\P$ ist durch
\begin{align*}
\P_F((-\infty,x]):=F_(x)
\end{align*}
eindeutig bestimmt.\\
Umgekehrt bestimmt jedes Wahrscheinlichkeitsmaß eine eindeutige Verteilungsfunktion über
\begin{align*}
F_{\P}(x):=\P((-\infty,x])
\end{align*}

Somit ist die Zuordnung der Verteilungsfunktionen zu den Wahrscheinlichkeitsverteilungen bijektiv. 
\end{satz}

\begin{notation}
$\P(\d x):=:\d\P(x)$. Außerdem bedeutet $F(\d x)$ oftmals auch, dass man bzgl. dem Maß $Q$ integriert, was durch $F$ eindeutig festgelegt ist.
\end{notation}

\begin{satz}[Transformationssatz]\label{satzTransformationssatz}\enter
Seien $(\Omega,\A,\P)$ ein Wahrscheinlichkeitsraum und $(S,\mathcal{F})$ ein Messraum.
Sei $u:S\to\R$ eine messbare Funktion und $X:\Omega\to S$ eine Zufallsvariable. Dann gilt:
\begin{align}\label{eqTrafo}\tag{Trafo}
\int\limits_\Omega u(X(\omega))~\P(d\omega)=\int\limits_S u(s)~\P_X(\d s)
\end{align}
\end{satz}

\begin{satz}[Lebesgue / dominierte Konvergenz / majorisierte Konvergenz]\label{satzMajorisierteKonvergenz}\enter
Sei $(\Omega,\A,\P)$ ein Maßraum und sei $(f_n)_{n\in\N}$ eine Folge von $\P$-mesbaren Funktionen (z. B. ZV) $f_n:\Omega\to\R\cup\lbrace\infty\rbrace$. Die Folge $(f_n)_{n\in\N}$ konvergiere $\P$-fast überall gegen eine $\P$-messbare Funktion $f$ und es gilt $|f_n|\leq g$ $\P$-fast überall für alle $n\in\N$ für eine $\P$-integrierbare Funktion $g:\Omega\to\R$.\\
Dann sind $f_n$ und $f$ $\P$-integrierbar und es gilt
\begin{align*}
\int\limits_\Omega f(\omega)\d\P(\omega)=\int\limits_\Omega \limn f_n(\omega)\d\P(\omega)
=\limn\int\limits_\Omega f_n(\omega)\d\P(\omega)
\end{align*}
\end{satz}

\begin{satz}[Monotone Konvergenz]\label{satzMonotoneKonvergenz}\enter
Sei $(\Omega,\A,\P)$ ein Maßraum. Ist $(f_n)_{n\in\N}$ eine Folge nichtnegativer, messbarer Funktionen $f_n:\Omega\to[0,\infty]$, die $\P$-fast-überall monoton wachsend gegen eine messbare Funktion $f:\Omega\to[0,\infty]$ konvergiert, so gilt:
\begin{align*}
\int\limits_\Omega f(\omega)\d\P(\omega)=\int\limits_\Omega \limn f_n(\omega)\d\P(\omega)
=\limn\int\limits_\Omega f_n(\omega)\d\P(\omega)
\end{align*}
\end{satz}

%\begin{satz}[Gliwenko-Cantelli]\enter

%\end{satz}

