% This work is licensed under the Creative Commons
% Attribution-NonCommercial-ShareAlike 4.0 International License. To view a copy
% of this license, visit http://creativecommons.org/licenses/by-nc-sa/4.0/ or
% send a letter to Creative Commons, PO Box 1866, Mountain View, CA 94042, USA.

\chapter{Einführung}
Viele Schätzer in der Statistik sind definiert als Minimal- oder Maximalstelle von bestimmten \textit{Kriteriumsfunktionen}, z. B. der \textit{Maximum-Likelihood-Schätzer (MLS)} oder \textit{Minimum-Qudrat-Schätzer (MQS, KQS)} oder \textit{Bayes-Schätzer}. Allgemein nennt man solche Schätzer \textbf{M-Schätzer}.\\
Ziel: Untersuchung des asymptotischen Verhaltens ($n\to\infty$) von M-Schätzern über einen \textit{funktionalen Ansatz}. Als Beispiel:

\section{Der Median}
Sei $X:(\Omega,\A,\P)\to\R$ eine reelle Zufallsvariable mit Verteilungsfunktion\\ $F_X:\R\to[0,1],~F_X(x):=\P[X\leq x]$, also $X\sim F_X$. Definiere

\begin{align}
Y(t)&:=\E\left(|X-t|\right)\label{DefY}\tag{1.0}\\ \nonumber
&=\int\limits_\Omega |X(\omega)-t|\d\P(\omega)\\ \nonumber
&\stackeq{\text{Trafo}}
\int\limits_\R|x-t|\Big(\P\circ X\d x\Big)\\ \nonumber
&=\int\limits_\R|x-t|(F\d x) \nonumber
\qquad\forall t\in\R\\
m&:=\arg\min\limits_{t\in\R}Y(t):=\text{ (irgendeine) Minimalstelle der Funktion}\nonumber
\end{align} 

\begin{notation}
$F(m-):=F(m-0):=\lim\limits_{t\uparrow m} F(t)$
\end{notation}

Charakterisierung der Menge aller Mediane in folgendem kleinen Lemma:

\begin{lemma}\label{lemmaMedian}
Sei $X\sim F_X$ integrierbar und $m\in\R$. Dann äquivalent:
\begin{enumerate}[label=(\alph*)]
\item $F(m-)\leq\frac{1}{2}\leq F(m)$
\item $\E[|X-t|]\geq\E[|X-m|]\qquad\forall t\in\R$
\item $m$ ist Median
\end{enumerate}
\end{lemma}

\begin{proof}
\underline{Zeige (a) $\Rightarrow$ (b):}\\
Setze $h(t):=\E[|X-t|-|X-m|]\stackeq{\text{Lin}}Y(t)-Y(m)$. Dann ist 2. äquivalent zu $h(t)\geq0~\forall t\in\R$. Dies ist noch zu zeigen.\\

\underline{Fall 1: $t<m$}
\begin{align*}
h(t)&\stackeq{\text{Trafo}}
\int\limits_{\R}|x-t|-|x-m| Q_F(\d x)\\
&=\int\limits_{(-\infty,t]}\underbrace{|x-t|-|x-m|}_{=t-x-(m-x)=-(m-t)} F(\d x)
+\int\limits_{(t,m)}\underbrace{\underbrace{|x-t|-|x-m|}_{\underbrace{x-t}_{\geq0}-\underbrace{(m-x)}_{\leq m-t}}}_{\geq-(m-t)} F(\d x)
+\int\limits_{[m,\infty)}\underbrace{|x-t|-|x-m|}_{x-t-(x-m)=m-t} F(\d x)\\
&\geq-(m-t)\cdot \underbrace{Q((-\infty,t])}_{F(t)}+
\Big(-(m-t)\cdot F(m-)-F(t)\Big)
+(m-t)\cdot\underbrace{Q([m,\infty))}_{1-\underbrace{Q((-m,m)}_{F(m-)}}\\
&=-\underbrace{(m-t)}_{\geq0}\cdot(\underbrace{1-2\cdot F(m-)}_{\stackrel{1.}{\geq}0})\\
&\geq0
\end{align*}

\underline{Fall 2: $t> m$}
\begin{align*}
h(t)&=\int\limits_{(-\infty,m]}\ldots F(\d x)+\int\limits_{(m,t]}\ldots+\int\limits_{(t,\infty)}\ldots F(\d x)\\
&\ldots\\
&\geq(t-m)\cdot(\underbrace{2\cdot F(m)-1}_{\stackrel{1.}{\geq}0})\\
&\geq0
\end{align*}

\underline{Fall 3: $t=m$} ist trivial. $\#$\\

\underline{Zeige (b) $\Rightarrow$ (a):}\\
Nach Annahme ist $h(t)\geq0~\forall t\in\R$.\\

\underline{Fall 1: $t<m$} Die obige Rechnung im Fall 1 bei 1. $\Rightarrow$ 2. zeigt:
\begin{align*}
0\leq h(t)&=-(m-t)\cdot F(t)+\int\limits_{(t,m)}\underbrace{\underbrace{x}_{<m}-t-(m-x) }_{=2x-t-m\leq m-t}F(\d x)+(m-t)\cdot(1-F(m-))\\
&\leq-(m-t)\cdot\Big(F(t)-1\underbrace{+F(m-)-F(m-)}_{=0}+F(t)\Big)\\
&=\underbrace{(m-t)}_{>0}\cdot(1-2\cdot F(t))\\
&\Longrightarrow\forall t<m:0\leq(m-t)\cdot(1-2\cdot F(t))\\
&\Longrightarrow\forall t<m:0\leq 1-2\cdot F(t)\\
&\Longrightarrow\forall t<m:F(t)\leq\frac{1}{2}\\
&\stackrel{t\uparrow m}{\Longrightarrow}F(m-)\leq\frac{1}{2}
\end{align*}
\underline{Fall 2: $t> m$} Siehe 2. Fall, analog.\\

\underline{Zeige (a) $\gdw$ (c):}
(b) ist offensichtlich äquivalent zur Definition des Medians.
\end{proof}

\begin{bemerkungnr}\
\begin{enumerate}
\item Lemma \ref{lemmaMedian} (a) besagt, dass $\lbrace m\in\R: m\text{ erfüllt } 1.\rbrace$ die Menge aller Mediane von $F$ ist.
\item Im Allgemeinen gibt es mehrere Mediane. Üblicherweise \underline{wählt} man $m:=F^{-1}(\frac{1}{2})$, wobei
\begin{align*}
F^{-1}(u):=\inf\left\lbrace x\in\R:F(x)\geq u\right\rbrace\quad\forall u\in (0,1)
\end{align*}
die \textbf{Quantilfuntion / die verallgemeinerte Inverse} ist. Da
\begin{align*}
F\left(F^{-1}(u)-\right)\leq u\leq F\left(F^{-1}(u)\right)\qquad\forall u\in (0,1),
\end{align*}
erfüllt $m=F^{-1}\left(\frac{1}{2}\right)$ die Bedingung (a) in Lemma \ref{lemmaMedian} und ist somit ein Median, nämlich der kleinste.
\item Die obige Funktion \eqref{DefY}, also
\begin{align*}
Y:\R\to\R,\qquad Y(t)=\int\limits|x-t|~F(\d x)\qquad\forall t\in\R,
\end{align*}
ist stetig (nutze Folgenkriterium + dominierte Konvergenz bzw. Satz von Lebesgue), aber im Allgemeinen nicht differenzierbar, z. B. falls $F\sim X$ eine diskrete Zufallsvariable ist. In diesem Fall ist somit die Minimierung über Differentiation nicht möglich!
\end{enumerate}
\end{bemerkungnr}

Zur Schätzung von $m$ seien $X_1,\ldots, X_n$ i.i.d.$\sim F$ mit zugehöriger \textbf{empirischer Verteilungsfunktion}
\begin{align*}
F_n(x):=\frac{1}{n}\cdot\sum\limits_{i=1}^n\indi_{\lbrace X_i\leq x\rbrace}\qquad\forall x\in\R.
\end{align*}
Tatsächlich ist $F_n$ die Verteilungsfunktion zum \textbf{empirischen Maß}
\begin{align*}
Q_n:=\frac{1}{n}\cdot\sum\limits_{i=1}^n\delta_{X_i}\text{ wobei }\delta_x\text {das Dirac-Maß in}x\in\R
\end{align*}

Gemäß dem Satz von Gliwenko-Cantelli gilt:
\begin{align*}
\sup\limits_{x\in\R}|F_n(x)-F(x)|\stackrel{n\to\infty}{\longrightarrow}0\text{ konvergiert $\P$-fast sicher für alle Vereteilungsfunktionen }F
\end{align*}

\begin{erinnerung}
Für das Dirac-Maß $\delta_x:\A\to\R_+,\qquad\delta_x(A):=\indi_A(x)$ gilt:
\begin{align*}
\int\limits f(t)~\delta_x(\d t)=f(x)
\end{align*}
\end{erinnerung}

Ein vages Stetigkeitsargument motiviert folgenden Schätzer für $m$:
\begin{align*}
\hat{m}_n&:=\arg\min\limits_{t\in\R}Y_n(t):=\text{ (irgendeine) Minimalstelle der Funktion}\\
Y_n(t)&:=\int\limits_\Omega |x-t|F_n(\d x)\\ 
&=\int\limits_\Omega |x-t|Q_n(\d x)\\ 
&=\frac{1}{n}\cdot\sum\limits_{i=1}^n\int\limits|x-t|~\delta_{X_i}(\d x)\\
&=\frac{1}{n}\cdot\sum\limits_{i=1}^n|X_i-t|
\end{align*} 

$\hat{m}_n$ heißt \textbf{empirischer Median} von $X_1,\ldots,X_n$ mit üblicher Auswahl $\hat{m}_n=F_n^{-1}\left(\frac{1}{2}\right)$ gemäß Lemma \ref{lemmaMedian} (da empirische Verteilungsfunktion eine Verteilungsfunktion ist).

%Hier wäre Abbildung 1

\begin{bemerkung}\
\begin{itemize}
\item Wenn man eine ungerade Anzahl von Daten hat, ist der Median der mittlere Wert, nachdem man die Daten der Größe nach geordnet hat.
\item Hat man hingegen eine gerade Anzahl an Daten, dann ist der Median der kleinere der beiden mittleren Werte.
\end{itemize}
\end{bemerkung}

Mit dem starken Gesetz der großen Zahlen (SGGZ) gilt
\begin{align}\label{eq1.1}
\forall t\in\R: \Big(Y_n(t)\stackrel{n\to\infty}{\longrightarrow}
\E[|X_1-t|]=Y(t)\text{ fast sicher}\Big)
\end{align}
Problem: Folgt aus \eqref{eq1.1} bereits, dass
\begin{align*}
\arg\min\limits_{t\in\R} Y_n(t)
\stackrel{n\to\infty}{\longrightarrow}
\arg\min\limits_{t\in\R}
Y(t)\text{ fast sicher?}
\end{align*}
Dann folgte:
\begin{align*}
\hat{m}_n
\stackrel{n\to\infty}{\longrightarrow}
m\text{ fast sicher (\textbf{starke Konvergenz})}
\end{align*}

Wir formalisieren und verallgemeinern:
\begin{align*}
&X_i:(\Omega, \A,P)\to(\R,\B(\R))\text{ messbar},\qquad\omega\mapsto X_i(\omega)\\
&\Longrightarrow
Y_n(t):=Y_n(t,\omega)=
\frac{1}{n}\cdot\sum\limits_{i=1}^n\left|X_i(\omega)-t\right|\\
&\Longrightarrow
Y_n(t,\cdot):(\Omega,\A)\to(\R,\B(\R))\text{ messbar }\forall t\in\R
\end{align*}

\begin{defi}
Die \textbf{Kollektion}
\begin{align*}
Y_n:=\lbrace Y_n(t,\cdot):t\in\R\rbrace
=\lbrace Y_n(t):t\in\R\rbrace
\end{align*}
heißt \textbf{stochastischer Prozess (SP)}. Die Abbildung
\begin{align*}
X_n(\cdot,\omega):\R\to\R,\qquad t\mapsto Y_n(t,\omega)
\end{align*}
heißt \textbf{Trajektorie / Pfad} des SP $Y_n$ zu festem $\omega\in\Omega$.
\end{defi}

In unserem Beispiel sind für \underline{alle} $\omega\in\Omega$ die Pfade stetig auf $\R$. Die Abbildung
\begin{align*}
Y_n:\Omega\to X C(\R,\R),\qquad\omega\mapsto Y_n(\cdot,\omega)
\end{align*}
heißt \textbf{Pfadabbildung} des SP $Y_n$. Wir identifizieren also den SP $Y_n$ mit seiner Pfadabbildung. Damit ist $Y_n$ eine Abbildung von $\Omega$ in den Funktionenraum 
\begin{align*}
C(\R):=C(\R,\R):=\lbrace f:\R\to\R: f\text{ ist stetig }\rbrace.
\end{align*}

Sei $d:C(\R)\times C(\R\to\R$ die Metrik der gleichmäßigen Konvergenz der Kompakta auf $\C(\R)$ (formale Definition kommt später) und sei
\begin{align*}
\B(C(\R)):=\B_d\big(C(\R)\big)=\sigma\big(\lbrace G\subseteq C(\R):G\text{ ist offen bzgl. }d\rbrace\big)
\end{align*}
die von $d$ induzierte \textbf{Borel-$\sigma$-Algebra}.\\
Wir werden sehen, dass die Abbildung
\begin{align*}
Y_n:(\Omega,\A,\P)\to\Big(C(\R),\B\big(C(\R)\big)\Big)
\end{align*}
messbar ist. $Y_n$ ist also eine Zufallsvariable mit Werten im metrischen Raum $\big(C(\R),d\big)$.

Formulierung des Problems im allgemeinen Rahmen:
Seien $Y_n,~n\in\N$ mit $Y$ SP mit stetigen Pfaden (stetige SP).
Was lässt sich sagen über die Gültigkeit der folgenden Implikationen?
\begin{align}
Y
\stackrel{n\to\infty}{\longrightarrow}
Y\text{ fast sicher }
\Longrightarrow
\arg\min\limits_{t\in\R} Y(t)
\stackrel{n\to\infty}{\longrightarrow}
\arg\min\limits_{t\in\R} Y(t)\text{ fast sicher}
\end{align}
Ziel: Welche Art der Konvergenz $Y_n\stackrel{n\to\infty}{\longrightarrow} Y$ reicht für obige Implikation aus? Gleichmäßige Konvergenz, gleichmäßige Konvergenz auf Kompakta punktweise Konvergenz oder sogar nur \eqref{eq1.1}?\\
$Y$ besitzt womöglich (unter positiven Wahrscheinlichkeiten) keine eindeutige Minimalstelle. Und dann?\\

Für die Konstruktion von (asymptotischen) Konfidenzintervallen für $m$ benötigt man \textbf{Verteilungskonvergenz}:
\begin{align}\label{eq1.3}
a_n(\hat{m}_n-m)
\stackrel{\mathcal{L}}{\longrightarrow}\xi\text{ in }\R
\end{align}
wobei $a_n\to\infty$ in $\xi$ Grenzvariable, die es zu identifizieren gilt. Für die Herleitung von \eqref{eq1.3} favorisiere wieder einen \textit{funktionalen Ansatz}. Sei
\begin{align*}
Z_n(t):=\beta_n\cdot\left( Y_n\left(m+\frac{t}{a_n}\right)-Y_n(m)\right)\qquad t\in\R
\end{align*}
der sogenannte \textbf{reskalierte Prozess zu $Y_n$}, wobei $\beta_n$ deine geeignete positive Folge ist. Damit folgt
\begin{align}\label{1.4}
a_n(\hat{m}_n-m)=\arg\min\limits_{t\in\R} Z_n(t)
\end{align}
Klar: $Z_n$ ist wieder ein stetiger stochastischer Prozess und damit $(Z_n)_{n\in\N}$ eine Folge von Zufallsvariablen in $\big(C(\R),d\big)$. Wünschenswert auch hier wäre die Gültigkeit folgender Implikation:
\begin{align}\label{eqSternchen}\tag{$\ast$}
Z_n
\stackrel{\mathcal{L}}{\longrightarrow}
Z\text{ in } \big(C(\R),d\big)
\Longrightarrow
\arg\min\limits_{t\in\R} Z_n(t)
\stackrel{\mathcal{L}}{\longrightarrow}
\arg\min\limits_{t\in\R} Z(t)
\end{align}
Dazu erforderlich ist das Konzept der \textbf{Verteilungskonvergenz} von Zufallsvariablen in metrischen Räumen, damit \eqref{eqSternchen} eine wohldefinierte Bedeutung erhält. Dies folgt später. Natürlich auch hier wieder das Problem: $Z$ besitzt mit positiver Wahrscheinlichkeit mindestens 2. Minimalstellen. Und dann?\\

Im Falle einer fast sicher eindeutigen Minimalstelle von $Z$ würde aber aus (1.4) und (1.5) folgen:
\begin{align*}
a_n(\hat{m}_n-m)
\stackrel{\mathcal{L}}{\longrightarrow}
\arg\min\limits_{t\in\R} Z(t)
\end{align*}


\section{Konzepte aus metrischen Räumen}
Sei $(\mathcal{S},d)$ metrischer Raum.

\begin{beispiel}[Supremums-Metrik] %2.1
\begin{align*}
\mathcal{S}=C([0,1]):=\big\lbrace f:[0,1]\to\R: f\text{ stetig}\big\rbrace\\
d(f,g):=\sup\limits_{t\in[0,1]}\big|f(t)-g(t)\big|,\qquad\forall f,g\in C([0,1])
\end{align*}
\end{beispiel}

\begin{definition}\
\begin{enumerate}[label={(\arabic*)}]
\item Für $x\in\mathcal{S},~r>0$ ist
\begin{align*}
B(x,r):=B_d(x,r):=\lbrace y\in\mathcal{S}:d(x,y)<r\rbrace
\end{align*}
die offene Kugel um Mittelpunkt $x$ und Radius $r$.
\item Sei $A\subseteq\mathcal{S}$. Dann:
\begin{align*}
\stackrel{\circ}{A}&:=\inner(A):=\text{ das Innere von }A\\
\overline{A}&:=\text{ Abschluss von }A\\
\partial A&:=\overline{A}\cap\overline{A^C}=\overline{A}\setminus\stackrel{\circ}{A}\text{ ist der Rand von }A\\
A^C&:=\mathcal{S}\setminus A
\end{align*}
\item \begin{align*}
\mathcal{G}:=\mathcal{G}(\mathcal{S})&:=\big\lbrace G\subseteq\mathcal{S}: G\text{ ist offen bzgl. }d\big\rbrace\\
&=\big\lbrace G\subseteq\mathcal{S}:\forall x\in G:\exists r>0:B_d(x,r)\subseteq G\big\rbrace
\end{align*}
ist die durch $d$ induzierte Topologie.
\begin{align*}
\mathcal{F}:=\mathcal{F}(\mathcal{S}):=\big\lbrace F\subseteq\mathcal{S}:F\text{ ist abgeschlossen}\big\rbrace
\end{align*}
\item Sei $\emptyset\neq A\subseteq\mathcal{S},~x\in\mathcal{S}$. Dann ist
\begin{align*}
d(x,A):=\inf\lbrace d(x,a):a\in A\rbrace\geq0
\end{align*}
der Abstand von $x$ zu $A$.
\item $C(\mathcal{S}):=\lbrace f:S\to\R:f\text{ stetig}\rbrace$
\begin{align*}
C^b(\mathcal{S}):=\lbrace f\in C(\mathcal{S}):f\text{ beschränkt}\rbrace\\
\Vert f\Vert:=\Vert f\Vert_\infty:=\sup\limits_{x\in\mathcal{S}}|f(x)|
\end{align*}
\end{enumerate}
\end{definition}

\begin{lemma}\label{lemma2.3}\ %2.3
\begin{enumerate}[label={(\arabic*)}]
\item $\begin{aligned}
x\in\overline{A}\Longleftrightarrow d(x,A)=0
\end{aligned}$
\item $\begin{aligned}
\big| d(x,A)-d(y,A)\big|\leq d(x,y)\qquad\forall x,y\in\mathcal{S}
\end{aligned}$
\item $\begin{aligned}
d(\cdot, A):\mathcal{S}\to\R,\qquad x\mapsto d(x,A)
\end{aligned}$ ist gleichmäßig stetig ($A\neq\emptyset$).
\end{enumerate}
\end{lemma}

\begin{proof}
\underline{Zeige (1) ``$\Rightarrow$'':} Sei $x\in\overline{A}$. Dann gilt:
\begin{align*}
&\forall\varepsilon>0:\exists a\in A: d(x,a)<\varepsilon\\
&\implies d(x,A)\leq d(x,a)<\varepsilon~\forall\varepsilon>0\\
&\stackrel{\varepsilon\to0}{\implies}
d(x,A)=0
\end{align*}
\underline{Zeige (1) ``$\Leftarrow$'':}
Sei $d(x,A)=0$. Dann folgt aus der Infimumseigenschaft:
\begin{align*}
&\forall\varepsilon>0:\exists a\in A;0\leq d(x,a)\leq0+\varepsilon=\varepsilon\\
&\implies x\in\overline{A}
\end{align*}
\underline{Zeige (2):} Seien $x,y\in\mathcal{S}$. Dann gilt:
\begin{align*}
&d(x,a)
\stackrel{\Delta\text{Ungl}}{\leq}
d(x,y)+d(y,a)\qquad\forall a\in A\\
&\implies
d(x,A)\leq d(x,y)+d(y,A)\implies d(x,A)-d(y,A)\leq d(x,y)
\end{align*}
Vertauschen von $x$ und $y$ liefert:
\begin{align*}
d(y,A)-d(x,A)\leq d(y,x)=d(x,y)\implies\text{ Behauptung}
\end{align*}
\underline{Zeige (3):} Folgt aus (2) da, die Funktion $d(\cdot,A)$ Lipschitz-stetig und damit gleichmäßig stetig ist.
\end{proof}

\begin{satz}\label{Satz2.4} %2.4
Zu $A\subseteq\mathcal{S}$ und $\varepsilon>0$ existiert ein gleichmäßig stetige Funktion 
\begin{align*}
f:\mathcal{S}\to[0,1]\text{ mit der Eigenschaft} f(x)=\left\lbrace\begin{array}{cl}
1, & \falls x\in A\\
0, & \falls d(x,A)\geq\varepsilon
\end{array}\right.
\end{align*}
\end{satz}
\begin{proof}
Setze
\begin{align*}
\varphi:\R\to[0,1],\qquad \varphi(t):=\left\lbrace\begin{array}{cl}
1 , & \falls t\leq0\\
1-t, & \falls 0<1<1\\
0, & \falls t\geq1
\end{array}\right.
\end{align*}
Dann ist $\varphi$ gleichmäßig stetig auf $\R$. Sei
\begin{align*}
f(x):=\varphi\left(\frac{1}{\varepsilon}\cdot d(x,A)\right)\qquad\forall x\in\S
\end{align*}
Dann hat dieses $f$ die gewünschte Eigenschaft wegen Lemma \ref{lemma2.3}.
\end{proof}

\begin{definition} %2.5
Ein metrischer Raum $(\S,d)$ heißt \textbf{separabel}
\begin{align*}
&:\Longleftrightarrow\exists\text{ abzählbares } S_0\subseteq\S:\S\subseteq\overline{S_0}\\
&\Longleftrightarrow\exists\text{ abzählbares } S_0\subseteq\S:\S=\overline{S_0}\\
&\Longleftrightarrow\exists\text{ abzählbares } S_0\subseteq\S:S_0\text{ liegt dicht in }\S
\end{align*}
\end{definition}

\begin{beispiel} %2.6
$C([0,1])$ mit Supremums-Metrik ist separabal.
\begin{proof}
\begin{align*}
S_0:=\big\lbrace P:P\text{ ist Polynom mit \underline{rationalen} Koeffizienten}\big\rbrace
\end{align*}
$S_0$ ist abzählbar. Aus dem \textit{Approximationssatz von Weierstraß} und der Dichtheit von $\Q$ folgt die Behauptung.
\end{proof}
\end{beispiel}

\begin{definition} %2.7
$\G_0\subseteq\G$ heißt \textbf{Basis} von $\G:\Longleftrightarrow\forall G\in\G:G$ ist Vereinigung von Mengen aus $\G_0$, so genannte \textbf{$\G_0$-Mengen}.
\end{definition}

\begin{beispiel} %2.8
Die Menge
\begin{align*}
\big\lbrace B(x,r):x\in\S,0<r\in\Q\big\rbrace
\end{align*}
ist Basis von $\G$, denn:
\begin{proof}
Sei $G\in\G$. Dann gilt:
\begin{align*}
&\forall x\in G:\exists 0<r_x\in\Q:B(x,r_x)\subseteq G\\
&\implies
G=\bigcup\limits_{x\in G}\underbrace{\lbrace x\rbrace}_{\subseteq B(x,r_x)}\subseteq\bigcup\limits_{x\in G} \underbrace{B(x,r_x)}_{\subseteq G}\subseteq G\implies G=\bigcup\limits_{x\in G} \underbrace{B(x,r_x)}_{\in\G_0}
\end{align*}
\end{proof}
\end{beispiel}

\begin{satz}\label{satz2.9}
$\S$ separabel $\Longleftrightarrow\G$ hat abzählbare Basis
\end{satz}
\begin{proof}
\underline{Zeige ``$\Rightarrow$'':}\\
Sei $S_0\subseteq\S$ abzählbar und dicht in $\S$. Zeige:
\begin{align*}
\G_0:=\big\lbrace B(x,r):x\in S_0,0<r\in\Q\big\rbrace\subseteq\G\text{ ist Basis.}
\end{align*}
Sei also $G$ offen. Dann folgt aus Beispiel 2.8:
\begin{align}\label{proof2.9Sternchen}\tag{$\ast$}
G=\bigcup\limits_{x\in G} B(x,r_x),\qquad 0<r_x\in\Q,\forall x\in G
\end{align}
Da $\overline{S_0}=\S$ gilt:
\begin{align*}
&\forall x\in G:\exists y_x\in S_0: d(x,y_x)<\frac{r_x}{2}\\
&\implies d(x,y)
\stackrel{\Delta\text{Ungl}}{\leq}
d(x,y_x)+d(y_x,x)< \underbrace{\frac{r_x}{2}+\frac{r_x}{2}}_{=r_x}\qquad\forall y\in B\left(y_x,\frac{r_x}{2}\right)\\
&\implies B\left(y_x,\frac{r_x}{2}\right)\subseteq B(x,r_x)\qquad\forall x\in G\\
&\implies G\stackrel{\eqref{proof2.9Sternchen}}{\supseteq}
\bigcup\limits_{x\in G}\underbrace{B\left(y_x,\frac{r_x}{2}\right)}_{\supseteq\lbrace x\rbrace}
\supseteq\bigcup\limits_{x\in G}\lbrace x\rbrace=G\\
&\implies G=\bigcup	\limits_{x\in G}\underbrace{B\left(y_x,\frac{r_x}{2}\right)}_{\in\G_0}
\end{align*}
Also ist $\G_0$ einen Basis. Da $S_0$ abzählbar ist $\G_0$ abzählbar.\\

\underline{Zeige ``$\Leftarrow$'':}\\
Sei $\G_0$ abzählbare Basis von $\G$ und sei o.B.d.A. $\emptyset\notin\G_0$. Wähle für jedes $G\in\G_0$ ein $x_G\in G$ fest aus. Setze
\begin{align*}
S_0:=\lbrace x_G:G\in\G_0\rbrace.
\end{align*}
$S_0$ ist auch abzählbar. Bleibt Dichtheit zu zeigen.\\
Sei $x\in\S$ und $\varepsilon>0$. Da $B(x,\varepsilon)$ offen und $\G_0$ Basis, gilt: 
\begin{align*}
&\exists\G_{x,\varepsilon}\subseteq\G_0\mit B(x,\varepsilon)=\bigcup\limits_{G\in\G_{x,\varepsilon}} G\\
&\implies G\subseteq B(x,\varepsilon)\qquad\forall G\in\G_{x\varepsilon}
\end{align*}
Wähle ein $G$ von diesen aus. Dann gilt:
\begin{align*}
x_G\in G\subseteq B(x,\varepsilon)
\implies x_G\in B(x,\varepsilon)
\implies d(\underbrace{x_G}_{\in S_0},x)<\varepsilon
\end{align*}
\end{proof}

\begin{satz}\label{Satz2.10} %2.10
Seien $(\S,d)$ und $(\S',d')$ metrische Räume.
\begin{enumerate}[label={(\arabic*)}]
\item Auf $\S\times\S'$ sind Metriken definiert durch
\begin{align*}
d_1\Big((x,x'),(y,y')\Big)&:=\left( \big(d(x,y)\big)^2+\big(d'(x',y')\big)^2\right)^{\frac{1}{2}} &\forall(x,x'),(y,y')\in \S\times\S'\\
d_2\Big((x,x'),(y,y')\Big)&:=\max \left\lbrace d(x,y),d'(x',y')\right\rbrace &\forall(x,x'),(y,y')\in \S\times\S'\\
d_3\Big((x,x'),(y,y')\Big)&:=d(x,y)+d'(x',y') &\forall(x,x'),(y,y')\in \S\times\S'
\end{align*}
\item Die Metriken $d_1,d_2,d_3$ induzieren dieselbe Topologie $\mathcal{G}(\S\times \S')$ auf $\S\times\S'$, die sogenannte \textbf{Produkttopologie} von $\mathcal{G}(\S)$ und $\mathcal{G}(\S')$.
\item $\begin{aligned}
\mathcal{G}(\S\times\S')=\left\lbrace\bigcup\limits_{\begin{subarray}{c}G\in\mathcal{O}\\ G'\in\mathcal{O}'\end{subarray}}G\times G':\mathcal{O}\subseteq\mathcal{G}(\S),\mathcal{O}'\subseteq\mathcal{G}(\S')\right\rbrace
\end{aligned}$\\
d.h.
\begin{align*}
\big\lbrace G\times G':G\in\mathcal{G}(\S),G'\in\mathcal{G}(\S')\big\rbrace
\end{align*}
bildet eine Basis von $\mathcal{G}(S\times\S')$.
\end{enumerate}
\end{satz}
\begin{proof}\enter
\underline{Zu (1):} Überprüfung der Eigenschaften einer Metrik (zur Übung).\\
\underline{Zu (2):} Punktweise gelten die Beziehungen:
\begin{align*}
d_2\leq d_1\leq\sqrt{2}\cdot d_2,\qquad
\frac{1}{\sqrt{2}}\cdot d_3\leq d_1\leq d_3,\qquad
d_2\leq d_3\leq 2\cdot d_2
\end{align*}
Beachte beim Nachweis, dass die $d_i$'s als Metriken größer Null sind. Aus obigen Beziehungen folgt u. a.:
\begin{align*}
B_{d_2}\left(x,\frac{r}{\sqrt{2}}\right)\subseteq B_{d_1}(x,r)
\end{align*}
denn:
\begin{align*}
r>\sqrt{2}\cdot d_2(y,x)\geq d_1(y,x)
\end{align*}
\underline{Zu (3), zeige ``$\subseteq$'':}\\
Sei $G^\ast\in\mathcal{G}(\S\times\S')$. Dann gilt:
\begin{align*}
\forall x^\ast=(x,y)\in G^\ast:\exists r=r_{x^\ast}>0:
G^\ast=\bigcup\limits_{x^\ast\in G^\ast} B\big(x^\ast,r_{x^\ast}\big)
\end{align*}
Wegen Teil (2) sei o.B.d.A. $\S^\ast:=\S\times\S'$ versehen mit der Metrik $d_2$. Dann gilt:
\begin{align*}
B_{d_2}\big(x^\ast,r_{x^\ast}\big)&=\Big\lbrace(y,y')\in \S\times\S':\max\big\lbrace d(x,y),d'(x',y')\big\rbrace<r_{x^\ast}\Big\rbrace\\
&=\Big\lbrace(y,y')\in\S\times\S':d(x,y)<r_{x^\ast}\wedge d'(x',y')<r_{x^\ast}\Big\rbrace\\
&= \underbrace{B_d\big(x,r_{x^\ast}\big)}_{\in\mathcal{G}(\S)}\times \underbrace{B_{d'}\big(x', r_{x^\ast}\big)}_{\in\mathcal{G}(\S')}
\end{align*}
\underline{Zu (3), zeige ``$\supseteq$'':}\\
Sei zunächst $G\times G'\mit G,G'$ offen und $x^\ast=(x,x')\in G\times G'$. Also ist $x\in G$ und $x'\in G'$ und somit
\begin{align*}
\exists r,r'>0:B_d(x,r)\subseteq G\wedge B_{d'}(x',r')\subseteq G'
\end{align*}
Setze $r^\ast:=\min\lbrace r,r'\rbrace>0$. Damit folgt

\begin{align*}
B_{d_2}\big( x^\ast,r^\ast\big)&\subseteq B_d(x,r)\times B_{d'}\big(x',r'\big)\\
&\subseteq
G\times G'=G^\ast\\
&\implies
G\times G'\in\mathcal{G}(\S\times\S')\\
&\implies
\bigcup\limits_{\begin{subarray}{c} G\in\mathcal{O}\\G'\in\mathcal{O}'\end{subarray}}G\times G'\subseteq\mathcal{G}(\S\times\S')
\qquad\forall\mathcal{O}\subseteq\mathcal{G}(\S),\mathcal{O}'\subseteq\mathcal{G}(\S')
\end{align*}
da die Produkttopologie vereinigungsstabil ist.
\end{proof}

\begin{defi}
Die Metriken $d_1,d_2,d_3$ heißen \textbf{Produktmetriken}. Daher alternative Schreibweise $d\times d'$, also z. B. 
\begin{align*}
d\times d':=\max\lbrace d,d'\rbrace
\end{align*}
usw.
\end{defi}

\begin{bemerkungnr} %2.11
Analog lassen sich Produktmetriken für \underline{endlich viele} metrische Räume $(S_i,d_i)_{i\in\lbrace1,\ldots,k\rbrace}$ definieren, z. B.
\begin{align*}
d_1\times\ldots\times d_k:=\left(\sum\limits_{i=1}^k d_i^2\right)^{\frac{1}{2}},
\end{align*}
die wiederum dieselbe Produkttopologie induzieren.
\end{bemerkungnr}

\section{Zufallsvariablen in metrischen Räumen}
\begin{definition} %3.1
Die \textbf{Borel-$\sigma$-Algebra} auf dem metrischen Raum $(\S,d)$ ist %definiert als
\begin{align*}
\B(\S):=\sigma\big(\mathcal{G}(\S)\big).
\end{align*}
Elemente $B\in\B(\S)$ heißen \textbf{Borel-Mengen} in $\S$.\\
Beachte: $\B(\S)=\B_d(\S)$ hängt i. A. von der Metrik $d$ ab.
\end{definition}

\begin{lemma}\label{Lemma3.2} %3.2
Es gilt:
\begin{enumerate}[label=(\arabic*)]
\item 
$\begin{aligned}
\B(\S)=\sigma\big(\mathcal{F}(\S)\big)
\end{aligned}$
\item $\begin{aligned}
f:(\S,d)\to(\S',d)
\end{aligned}$ ist stetig und damit $\B_d(\S)-\B_d(\S')$ ToDo
\item Sei $\mathcal{G}_0$ abzählbare Basis von $\mathcal{G}(\S)$. Dann gilt:
\begin{align*}
\sigma(\mathcal{G}_0)=\B(\S)
\end{align*}
\end{enumerate}
\end{lemma}

\begin{proof}\enter
\underline{Zu (1), zeige ``$\subseteq$'':}
\begin{align*}
G^C\in\mathcal{F}(\S)\subseteq\sigma\big(\mathcal{F}(\S)\big)
\implies
G=\big(G^C\big)^C\in\mathcal{F}(\S)
\end{align*}
da $\sigma\big(\mathcal{F}(\S)\big)$ Komplement-stabil ist. Also folgt
\begin{align*}
\mathcal{G}\subseteq\sigma(\mathcal{F})
\implies\sigma(\mathcal{G})\subseteq\sigma(\mathcal{F})
\end{align*}
\underline{Zu (1), zeige ``$\supseteq$'':} Analog.\\

\underline{Zeige (2):}
\begin{align*}
f^{-1}\big(\B_{d'}(\S')\big)&=f^{-1}\Big(\sigma\big(\mathcal{G}(\S')\big)\Big)\\
&=\sigma\Big(\underbrace{f^{-1}\big(\mathcal{G}(\S')\big)}_{\stackrel{f\text{ stetig}}{\subseteq}\mathcal{G}(S)}\Big)\\
&\subseteq\sigma\big(\mathcal{G}(\S)\big)\\
&=\B(\S)
\end{align*}
\underline{Zu (3), zeige ``$\subseteq$'':}\\
Klar wegen $\mathcal{G}_0\subseteq\mathcal{G}$ und $\sigma$ monoton.\\
\underline{Zu (3), zeige ``$\supseteq$'':} Sei $G\in\mathcal{G}$. Dann:
\begin{align*}
G&=\bigcup\limits_{i\in\N} G_i\mit\text{geeigneten }G_i\in\mathcal{G}_0\subseteq\sigma(\mathcal{G}_0)\\
&\implies
G\in\sigma(\mathcal{G}_0)
\end{align*}
Aus der Stabilität unter Vereinigungen folgt die Behauptung.
\end{proof}

\begin{satz}\label{satz3.3} %3.3
Sei $(\S,d)$ separabler metrischer Raum. Dann gilt:
\begin{align*}
\B_{d\times d}(\S\times\S)=\B(S)\otimes\B(\S)
\end{align*}
\end{satz}
\begin{proof}
Seien
\begin{align*}
&\pi_1:\S\times\S\to\S,\qquad \pi_1(x,y):=x\qquad\forall(x,y)\in\S\times S\\
&\pi_2:\S\times\S\to\S,\qquad \pi_2(x,y):=y\qquad\forall(x,y)\in\S\times S
\end{align*}
die \textbf{Projektionsabbildungen}. Dann gilt
\begin{align*}
\B(\S)\otimes\B(\S) 
&\stackeq{\text{Def}}
\sigma(\pi_1,\pi_2)\\
&\stackeq{\text{Def}}
\sigma\Big(\pi_1^{-1}\big(\sigma(\mathcal{G})\big)\cup\pi_2^{-1}\big(\sigma(\mathcal{G})\big)\Big)\\
&\stackeq{(+)}
\sigma\Big(\sigma\big(\pi_1^{-1}(\mathcal{G})\big)\cup\sigma\big(\pi_2^{-1}(\mathcal{G})\big)\Big)\\
&=
\sigma\Big(\pi_1^{-1}(\mathcal{G})\big)\cup\pi_2^{-1}(\mathcal{G})\Big)\\
&=\sigma\Big(\big\lbrace G\times S,S\times G':G,G'\in\mathcal{G}\big\rbrace\Big)\\
&=
\sigma\Big(\big\lbrace \overbrace{G\times G'}^{=(G\times S)\cap(S\times G')}:G,G'1\in\mathcal{G}\big\rbrace\Big)\\
&\stackeq{\text{($\ast$)}}
\sigma\left(\left\lbrace\bigcup\limits_{\begin{subarray}{c}G\in\mathcal{O}\\G'\in\mathcal{O}'\end{subarray}}G\times G':\mathcal{O},\mathcal{O}'\subseteq\mathcal{G}\right\rbrace\right)\\
&\stackeq{2.10~(3)}
\sigma\Big(\mathcal{G}(\S\times\S)\Big)\\
&\stackeq{\text{Def}}
\B(\S\times\S)
\end{align*}
Zum Nachweis von (+):\\
Zeige ``$\supseteq$'': Setze
\begin{align*}
\xi&:=
\underbrace{\sigma\big(\pi_1^{-1}(\mathcal{G})\big)}_{\supseteq \pi_1^{-1}(\mathcal{G})}\cup\underbrace{\sigma\big(\pi_2^{-1}(\mathcal{G})\big)}_{\pi_2^{-1}(\mathcal{G})}\\
&\supseteq
\pi_1^{-1}(\mathcal{G})\cup\pi_2^{-1}(\mathcal{G})\\
&=:\mathcal{H}\\
&\implies\sigma(\xi)\supseteq\sigma(\mathcal{H})
\end{align*}
Zeige ``$\subseteq$'': Es gilt
\begin{align*}
&\pi_1^{-1}(\mathcal{G})\subseteq\big(\pi_1^{-1}(\mathcal{G})\cup\pi_2^{-1}(\mathcal{G})\big)=\mathcal{H}\\
&\implies
\sigma\big(\pi_1^{-1}(\mathcal{G})\big)\subseteq\sigma(\mathcal{H})\text{ und analog }\\
&\implies
\sigma\big(\pi_2^{-1}(\mathcal{G})\big)\subseteq\sigma(\mathcal{H})\\
&\implies
\xi=\underbrace{\sigma\big(\pi_1^{-1}(\mathcal{G})\big)}_{\subseteq\sigma(\mathcal{H})}\cup\underbrace{\sigma\big(\pi_2^{-1}(\mathcal{G})\big)}_{\subseteq\sigma(\mathcal{H})}\subseteq\sigma(\mathcal{H})\\
&\implies
\sigma(\xi)\subseteq\sigma(\mathcal{H})
\end{align*}

Bleibt Nachweis von ($\ast$):\\
``$\subseteq$'': ist klar (gilt auch ohne Separabilität)\\
``$\supseteq$'': Gemäß 2.9 existiert abzählbare Basis $\mathcal{G}_0$  von $\mathcal{G}$. Sei
\begin{align*}
G^\ast&=\bigcup\limits_{\begin{subarray}{c}G\in\mathcal{O}\\G'\in\mathcal{O}'\end{subarray}}G\times G'\text{ und }\mathcal{O},\mathcal{O}'\subseteq\mathcal{G}\\
&\stackeq{(!)}
\bigcup\limits_{\begin{subarray}{c}
G,G'\text{ offen}\\
G,G'\subseteq G^\ast
\end{subarray}}
G\times G'\\
&\stackeq{(!)}
\bigcup\limits_{\begin{subarray}{c}
G_0,G_0'\in\mathcal{G}_0\\
G\times G_0'\subseteq G^\ast
\end{subarray}}
G_0\times G_0'\\
&=\text{ abzählbare Vereinigung, da $\mathcal{G}_0$ abzählbare Basis }\\
&\implies
G^\ast\in\sigma\Big(\big\lbrace G\times G':G,G'\in\mathcal{G}\big\rbrace\Big)
\end{align*}
\end{proof}

\begin{definition} %3.4
Sei $(\Omega,\A)$ ein Messraum. Eine Abbildung
$X:\Omega\to\S$, die $\A$-$\B(\S)$-messbar ist, heißt \textbf{Zufallsvariable (ZV)} in den metrischen Raum $(\S,d)$ über $(\Omega,\A)$.\\

Sei $\P$ ein Wahrscheinlichkeitsmaß auf $(\Omega,\A)$, also $(\Omega,\A,\P)$ ein Wahrscheinlichkeitsraum. Das Bildmaß
\begin{align*}
\P\circ X^{-1}&:=:\P_X:=:\mathcal{L}:=:\mathcal{L}(X~|~\P)\\
(\P\circ X^{-1})(B)&:=\P\left(X^{-1}(B)\right)=\P\Big(\big\lbrace\omega\in\Omega:X(\omega)\in B\big\rbrace\Big)
=: \P[X\in B]
\qquad\forall B\in\B(\S)
\end{align*}
heißt \textbf{Verteilung} von $X$ unter $\P$.
\end{definition}

\begin{satz}\label{Satz3.5} %3.5
Sei $(\S,d)$ separabler metrischer Raum und seien $X,Y$ Zufallsvariablen in $(\S,d)$ über $(\Omega,\A)$.\\
Dann ist $d(X,Y)$ eine reelle Zufallsvariable.
\end{satz}

\begin{proof}
\begin{align*}
X,Y:(\Omega,\A)\to(\S,\B(\S))\text{ sind messbar }\\
\stackrel{\text{MINT}}{\Longleftrightarrow}
(X,Y):(\Omega,\A)\to\big(\S\times\S,\underbrace{\B(\S)\otimes\B(\S)}_{\stackeq{\ref{satz3.3}}\B(\S\times\S)}\big)\text{ ist messbar}\\
\end{align*}
Jede Metrik ist bekanntlich stetig, also auch
\begin{align*}
d:\big(\S\times\S,\G(\S\times\S)\big)\to\R.
\end{align*}
Dann folgt aus Lemma \ref{Lemma3.2}, dass
\begin{align*}
d:\B(\S\times\S)\to\B(\R)
\end{align*}
messbar ist. Damit folgt die Behauptung, denn $d(X,Y)=d\circ(X,Y)$ ist messbar als Komposition von messbaren Abbildungen.
\end{proof}

\subsection*{Fast sichere Konvergenz} %NoNumber
\begin{definition} %3.6
Seien $X,X_n,n\in\N$ Zufallsvariablen in $(\S,d)$ über $(\Omega,\A,\P)$. Dann:
\begin{align*}
X_n\stackrel{n\to\infty}{\longrightarrow} X\quad\P\text{-fast sicher }:\Longleftrightarrow
\P\Big(\underbrace{\big\lbrace\omega\in\Omega:d\big(X_n(\omega),X(\omega)\big)\stackrel{n\to\infty}{\longrightarrow} 0\big\rbrace}_{=:M}\Big)=1
\end{align*}
Beachte: Die Definition von Konvergenz mengentheoretisch aufgeschrieben (Schnitt $\hat{=}$ ``für alle''; Vereinigung $\hat{=}$ ``Es gibt''):
\begin{align*}
\bigcap\limits_{0<\varepsilon\in\Q}
\bigcup\limits_{m\in\N}
\bigcap\limits_{n\geq m}
\big\lbrace\underbrace{d(X_n,X)}_{=:\xi_n}<\varepsilon\big\rbrace\stackrel{\ref{Satz3.5}}{\in}\A\\
\text{denn }\xi_n^{-1}\big((-\infty,\varepsilon)\big)\in\A
\end{align*}
\end{definition}

Die bekannten Regeln (Ergebnisse) für \underline{reelle} Zufallsvariablen lassen sich mühelos verallgemeinern. Dazu z. B.:

\begin{satz}\label{Satz3.7} %3.7
\begin{align*}
X_n
\stackrel{n\to\infty}{\longrightarrow}
X\quad\P\text{-fast sicher }\wedge
X_n
\stackrel{n\to\infty}{\longrightarrow}
X'\quad\P\text{-fast sicher }
\implies
X=X'\quad\P\text{-fast sicher}
\end{align*}
\end{satz}
\begin{proof}
\begin{align*}
\lbrace X\neq X'\rbrace
&\subseteq\lbrace X_n
\stackrel{n\to\infty}{\not\longrightarrow}
X\rbrace
\cup\lbrace X_n
\stackrel{n\to\infty}{\not\longrightarrow}
X'\rbrace\\
&\implies
\P[X_n\not\to X]+\P[X_n\not\to X']=0+0\\
&\implies
\P[X\neq X']=0
\end{align*}
\end{proof}

\begin{satz}\label{Satz3.8} %3.8
Seien $X,X_n,n\in\N$ Zufallsvariablen im metrischen Raum $(\S,d)$ und sei $f:(\S,d)\to(\S',d')$ messbar und stetig in $X$ $\P$-fast sicher.
Dann gilt:
\begin{align*}
X_n
\stackrel{n\to\infty}{\longrightarrow}
X\quad\P\text{-fast sicher }
\implies
f(X_n)
\stackrel{n\to\infty}{\longrightarrow}
f(X)\quad\P\text{-fast sicher}
\end{align*}
\end{satz}
\begin{proof}
\begin{align*}
\lbrace X_n
\stackrel{n\to\infty}{\longrightarrow}
X\rbrace\cap\lbrace f\text{ stetig in }X\rbrace\rbrace
\stackrel{\text{Folgen-Stetigkeit}}{\subseteq}
\lbrace f(X_n)
\stackrel{n\to\infty}{\longrightarrow}
f(X)\rbrace
\end{align*}
Damit folgt die Behauptung, denn zur Erinnerung:
\begin{align*}
\big(\forall i\in\N:\P(E_i)=1\big)\implies\P\left(\bigcap\limits_{i\in\N} E_i\right)=1
\end{align*}
\end{proof}

\begin{satz}[Konvergenz-Kriterium]\label{Satz3.9} %3.9
\begin{align*}
X_n
\stackrel{n\to\infty}{\longrightarrow}
X\quad\P\text{-fast sicher}
\Longleftrightarrow
\forall\varepsilon>0:\limn\P\left(\sup\limits_{m\geq n} d(X_m,X)>\varepsilon\right)=0
\end{align*}
\end{satz}
\begin{proof}
Man ersetze im Beweis für den Fall reeller Zufallsvariablen $|X_n-X|$ durch $d(X_n,X)$. Und beachte, dass alle Schlussfolgerungen bestehen bleiben.
\end{proof}

Ein sehr nützliches Kriterium ist Folgendes:
\begin{satz}\label{Satz3.10} %3.10
\begin{align*}
\sum\limits_{n\in\N_{>0}}\P\big(d(X_n,X)>\varepsilon\big)<\infty\qquad\forall\varepsilon>0
\implies
X_n
\stackrel{n\to\infty}{\longrightarrow}
X\quad\P\text{-fast sicher}
\end{align*}
\end{satz}
\begin{proof}
Setze
\begin{align*}
A_n(\varepsilon):=\big\lbrace d(X_n,X)>\varepsilon\big\rbrace\stackrel{\ref{Satz3.5}}{\in}\A
\end{align*}
Dann folgt aus dem \textit{ersten Borel-Cantelli-Lemma}:
\begin{align*}
&\P\left(\limsup\limits_{n\to\infty} A_n(\varepsilon)\right)=0\qquad\forall\varepsilon>0
\end{align*}
Mit
\begin{align*}
\liminf\limits_{n\to\infty}\big(A_n(\varepsilon)\big)
\stackeq{\text{Def}}
\bigcup\limits_{m\in\N}\bigcap\limits_{n\geq m}\big(A_n(\varepsilon)\big)^C
=
\bigcup\limits_{m\in\N}\bigcap\limits_{n\geq m}\big\lbrace d(X_n,X)\leq\varepsilon\big\rbrace
\end{align*}
folgt dann
\begin{align*}
1=\P\left(\left(\limsup\limits_{n\to\infty} A_n(\varepsilon)\right)^C\right)
=\P\left(\liminf\limits_{n\to\infty}\big(A_n(\varepsilon)\big)\right)\qquad\forall\varepsilon>0
\end{align*}
Da Abzählbare Durchschnitte von Eins-Mengen (also Mengen mit $\P$-Maß 1) wieder Eins-Mengen sind, folgt schließlich:
\begin{align*}
\P\Bigg(\underbrace{\bigcap\limits_{0<\varepsilon\in\Q}\bigcup\limits_{n\geq m}\big\lbrace d(X_n,X)\leq\varepsilon\big\rbrace}_{\lbrace X_n\to X\rbrace=\lbrace d(X_n,X)\to0\rbrace}\Bigg)=1
\end{align*}
\end{proof}
Weitere Eigenschaften der fast sicheren Konvergenz von Zufallsvariablen in metrischen Räumen finden sich z. B. in \textit{``Wahrscheinlichkeitstheorie'' Gäussler u. Stute (1977), Kapitel 8.2}

\subsection*{Stochastische Konvergenz} %NoNumber
\begin{definition} %3.11
\begin{align*}
X_n
\stackrelnew{n\to\infty}{\P}{\longrightarrow}
X:\Longleftrightarrow\forall\varepsilon>0:
\P\Big(\big\lbrace d(X_n,X)>\varepsilon\big\rbrace\Big)
\stackrel{n\to\infty}{\longrightarrow}
0
\end{align*}
\end{definition}

\begin{satz}\label{Satz3.12}
\begin{align*}
X_n
\stackrel{n\to\infty}{\longrightarrow}
X\quad\P\text{-fast sicher }
\implies X_n
\stackrelnew{n\to\infty}{\P}{\longrightarrow}
X
\end{align*}
\end{satz}
\begin{proof}
\begin{align*}
\forall\varepsilon>0:
0\leq\P\big(d(X_n,X)>\varepsilon\big)
\leq\P\left(\sup\limits_{m\geq n}d(X_m,X)>\varepsilon\right)
\stackrelnew{n\to\infty}{\P}{\longrightarrow}
0
\end{align*}
gemäß Satz \ref{Satz3.9}.
\end{proof}

Die Umkehrung in 3.12 gilt i. A. \underline{nicht}, aber es gilt das folgende Teilfolgenkriterium:

\begin{satz}[Teilfolgenkriterium für stochastische Konvergenz]\label{satz3.13}\enter
Folgende Aussagen sind äquivalent:
\begin{enumerate}[label=(\arabic*)]
\item $\begin{aligned}
X_n
\stackrelnew{n\to\infty}{\P}{\longrightarrow}
X\end{aligned}$
\item Zu jeder Teilfolge (TF) $(X_{n'})$ von $(X_n)_{n\in\N}$ existiert eine Teilfolge $(X_{n''})$ von $(X_{n'})$ derart, dass $X_{n''}
\stackrel{n''\to\infty}{\longrightarrow} X$ $\P$-fast sicher.
\end{enumerate}
\end{satz}
\begin{proof}
Wie im Reellen.
\end{proof}

Mit dem Teilfolgenkriterium lassen sich Rechenregeln für fast sichere Konvergenz auf stochastische Konvergenz übertragen.

\begin{korollar}\label{Korollar3.14}\
\begin{enumerate}[label=(\arabic*)]
\item $\begin{aligned}
X_n
\stackrelnew{n\to\infty}{\P}{\longrightarrow}
X\wedge X_n
\stackrelnew{n\to\infty}{\P}{\longrightarrow}
X'
\implies X=X'\quad\P\text{-fast sicher}
\end{aligned}$
\item $\begin{aligned}
X_n
\stackrelnew{n\to\infty}{\P}{\longrightarrow}
X\text{ in }(\S,d),~f:(\S,d)\to(\S',d')\text{ messbar mit $f$ stetig in $X$ $\P$-fast sicher }
\end{aligned}$
\begin{align*}
\implies f(X_n)
\stackrelnew{n\to\infty}{\P}{\longrightarrow}
f(X)
\end{align*}
\end{enumerate}
\end{korollar}
\begin{proof}
\underline{Zeige (1):}
\begin{align*}
X_n
\stackrelnew{n\to\infty}{\P}{\longrightarrow}
X
\stackrel{\ref{satz3.13}}{\implies}
\exists\text{ TF }(X_{n'})\subseteq(X_n)_{n\in\N}\mit X_{n'}
\stackrel{n'\to\infty}{\longrightarrow}
X\text{ fast sicher}
\end{align*}
Zu $(X_{n'})$ existiert (wegen $X_n\stackrelnew{n\to\infty}{\P}{\longrightarrow} X'$ und Satz \ref{satz3.3}) eine Teilfolge $(X_{n''})\subseteq(X_{n'})\mit$
\begin{align*}
X_{n''}
\stackrel{n\to\infty}{\longrightarrow} X'\text{ fast sicher}
\stackrel{\ref{Satz3.7}}{\implies}
X=X'\text{ fast sicher }
\end{align*}

\underline{Zeige (2):} Zur Übung.
\end{proof}

\subsection*{Konvergenz in Produkträumen} %noNumber
Seien $(\S,d)$ und $(\S',d')$ separable metrische Räume. Dann ist auch $(\S\times\S',d\times d')$ ein metrischer Raum. Dies folgt z. B. aus dem \textit{Satz von der koordinatenweise Konvergenz}:

\begin{align}\label{eq3.1KoordinatenweissKonvergenz}\tag{3.1}
\big(a_n,a_n'\big)
\stackrelnew{d\times d'}{n\to\infty}{\longrightarrow}
(a,a')
\Longleftrightarrow
(a_n)
\stackrelnew{d}{n\to\infty}{\longrightarrow}
a
\wedge
(a_n')
\stackrelnew{d'}{n\to\infty}{\longrightarrow}
(a')
\end{align}

Es ``stochastische Versionen'' dieses Satzes.

\begin{satz}\label{satz3.15}\
\begin{enumerate}[label=(\arabic*)]
\item $\begin{aligned}
(X_n,X_n')
\stackrelnew{}{n\to\infty}{\longrightarrow}
(X,X')~\P\text{-fast sicher}
\Longleftrightarrow
X_n
\stackrelnew{}{n\to\infty}{\longrightarrow}
X~\P\text{-fast sicher }\wedge
X_n'
\stackrelnew{}{n\to\infty}{\longrightarrow}
X'~\P\text{-fast sicher}
\end{aligned}$
\item $\begin{aligned}
(X_n,X_n')
\stackrelnew{n\to\infty}{\P}{\longrightarrow}
(X,X')
\Longleftrightarrow
X_n
\stackrelnew{n\to\infty}{\P}{\longrightarrow}
X\wedge
X_n'
\stackrelnew{n\to\infty}{\P}{\longrightarrow}
X'
\end{aligned}$
\end{enumerate}
\end{satz}
\begin{proof}
\underline{Zu (1):}
\begin{align*}
(1)\text{, linke Seite }
&\stackrel{\eqref{eq3.1KoordinatenweissKonvergenz}}{\Longleftrightarrow}
X_n\to X,~X_n'\to X'\quad\P\text{-fast sicher}\\
&\stackrel{\cap\text{ Eins-Mengen}}{\Longleftrightarrow}
(1),\text{ rechte Seite}
\end{align*}

\underline{Zu (2):}
\begin{align*}
(2)\text{, linke Seite}
&\stackrel{\eqref{eq3.1KoordinatenweissKonvergenz}}{\Longleftrightarrow}
\forall\text{ TF }(X_{n'},X_{n'}')\subseteq(X_n,X_n'):\\
&\qquad\exists\text{ TTF }(X_{n''},X_{n''}')\subseteq(X_{n'},X_{n'}'):
(X_{n''},X_{n''}')\stackrel{n\to\infty}{\longrightarrow}\text{ f.s.}
\end{align*}
Also wegen Teil (1) mit
\begin{align*}
X_{n''}\to X\text{ f.s. und }X_{n''}'\to X'\text{ f.s.}
\end{align*}
Somit:
\begin{align*}
\forall\text{ TF }(X_{n'})\subseteq(X_n):\exists\text{ TTF }(X_{n''})\subseteq(X_{n'}):X_{n''}\to X\text{ f.s. }
&\stackrel{\eqref{eq3.1KoordinatenweissKonvergenz}}{\Longleftrightarrow}
X_n\stackrelnew{n\to\infty}{\P}{\longrightarrow} X
\end{align*}
Und es gilt analog: $X_n'\to X'$.
\end{proof}

\subsection*{Gleichheit in Verteilung} %NoNumber
\begin{definition}\label{def3.16} %3.16
Zufallsvariablen $X,Y$ in $(\S,d)$ über $(\Omega,\A,\P)$ heißen \textbf{gleich in Verteilung}, in Zeichen $X\stackeq{\mathcal{L}} Y$
\begin{align*}
:\Longleftrightarrow \P\circ X^{-1}\equiv\P\circ Y^{-1}
\end{align*}
\end{definition}

\begin{bemerkung} %noNomber
Def 3.16 kann erweitert werden auf Zufallsvariablen $X:(\Omega,\A,\P)\to(\S,d)$ und $Y:(\tilde{\Omega},\tilde{\A},\tilde{\P})\to(\S,d)$ durch
\begin{align*}
X\stackeq{\mathcal{L}} Y:\Longleftrightarrow\P\circ X^{-1}\equiv\tilde{\P}\circ Y^{-1}
\end{align*}
\end{bemerkung}

Charakterisierung von Verteilungsgleichheit in folgendem Satz:

\begin{satz}\label{satz3.17}\
\begin{enumerate}[label=(\arabic*)]
\item Seien $\P,Q$ Wahrscheinlichkeitsmaße auf $\B(\S)$. Dann gilt:
\begin{align*}
\P\equiv Q\Longleftrightarrow
\int\limits f\d\P=\int\limits f\d Q\qquad\forall f\in C^b(\S)\text{ glm. stetig}
\end{align*}
\item $\begin{aligned}
X\stackeq{\mathcal{L}} Y\Longleftrightarrow
\E\big[f(X)\big]=\E\big[f(Y)\big]\qquad\forall f\in C^b(\S)\text{ glm. stetig}
\end{aligned}$
\end{enumerate}
\end{satz}
\begin{proof}
\underline{Zu (1) zeige ``$\implies$'':} Klar.\\

\underline{Zu (1) zeige ``$\Longleftarrow$'':}
\begin{align*}
\B(\S)\stackeq{\text{3.2 (1)}}\sigma\big(\mathcal{F}(\S)\big)
\end{align*}
und $\F$ ist Durchschnittsstabil. Wegen dem Maßeindeutigkeitssatz reicht es zu zeigen:
\begin{align*}
\P(F)=Q(F)\qquad\forall F\in\mathcal{F}(\S)
\end{align*}
Sei $F\subseteq\S$ abgeschlossen. Setze
\begin{align*}
f_k(x):=\varphi\Big(k\cdot d(x,F)\Big)
\end{align*}
(vgl. Satz 2.4). Aus dem Lemma \ref{lemma2.3} folgt, dass die $f_k$ beschränkt und gleichmäßig stetig sind mit $f_k\stackrel{k\to\infty}{\downarrow}\indi_F$. Also gilt:
\begin{align*}
\P(F)
&\stackeq{\text{Def}}
\int\limits\indi_F\d\P
=\int\limits\lim\limits_{k\to\infty} f_k\d\P
\stackeq{\text{Mono-Konv}}
\lim\limits_{k\to\infty}\int\limits f_k\d\P
\stackeq{\text{Vor}}
\lim\limits_{k\to\infty}\int\limits f_k\d Q\\
&\stackeq{\text{Mono-Konv}}
\int\limits\lim\limits_{k\to\infty} f_k\d Q
=\int\limits\indi_F\d Q
\stackeq{\text{Def}} Q(F)
\end{align*}
Da $F$ beliebig war, folgt die Behauptung.\\

\underline{Zu (2):} folgt aus (1) mit dem Transformationssatz \eqref{eqTrafo}:
\begin{align*}
X\stackeq{\mathcal{L}} Y
&\stackrel{\ref{def3.16}}{\Longleftrightarrow}
\underbrace{\P\circ X^{-1}}_{}=\underbrace{\P\circ Y^{-1}}_{}\\
&\stackrel{(1)}{\Longleftrightarrow}
\int\limits_{\S} f\d(\P\circ X^{-1})=\int\limits_{\S} f\d(\P\circ Y^{-1})
\end{align*}
und
\begin{align*}
\int\limits_{\S} f\d(\P\circ X^{-1})
\stackeq{\text{Trafo}}
\int\limits_{\Omega}\underbrace{f\circ X}_{=:f(X)}\d\P
\stackeq{\text{Def}}
\E\big[f(X)\big]
\end{align*}
\end{proof}

\section{Verteilungskonvergenz von Zufallsvariablen in metrischen Räumen}
Seien $X,X_n,n\in\N$ Zufallsvariablen in $(\S,d)$ über $(\Omega,\A,\P)$. Dann sind
%st die Verteilung von $X$, 
\begin{align*}
P:=\P\circ X^{-1},\qquad P_n:=\P\circ X_n^{-1},\qquad n\in\N
\end{align*}
sind Wahrscheinlichkeitsmaße auf $\B(\S)$.

\begin{definition}[Verteilungskonvergenz]\label{def4.1}\
\begin{enumerate}[label=(\arabic*)]
\item Seien $P,P_n,n\in\N$ Wahrscheinlichkeitsmaße auf $\B(S)$. Dann \textbf{konvergiert $P_n$ schwach gegen $P$}, in Zeichen
\begin{align*}
P_n\stackrelnew{w}{n\to\infty}{\longrightarrow} P
:\Longleftrightarrow
\int\limits f\d P_n\stackrel{n\to\infty}{\longrightarrow}\int\limits f\d P\qquad\forall f\in C^b(\S)
\end{align*}
Das $w$ steht für ``weakly''.
\item $X_n$ \textbf{konvergiert in Verteilung gegen $X$ in Raum $(\S,d)$}, in Zeichen
\begin{align*}
X_n\stackrel{\mathcal{L}}{\longrightarrow} X\text{ in }(\S,d)
:\Longleftrightarrow
\P\circ X_n^{-1}\stackrelnew{w}{n\to\infty}{\longrightarrow}\P\circ X^{-1}
\end{align*}
Alternative Schreibweise: $X_n\stackrel{\d}{\longrightarrow} X$. Das $\L$ steht für ``law''.
\end{enumerate}
\end{definition}

Äquivalente Charakterisierung von $\stackrelnew{w}{}{\longrightarrow}$ bzw. $\stackrel{\L}{\longrightarrow}$ in folgendem Satz:

\begin{satz}[Portmanteau-Theorem]\enter\label{satz4.2}
Folgende Aussagen sind äquivalent:
\begin{enumerate}[label=(\arabic*)]
\item $\begin{aligned}
P_n\stackrelnew{w}{}{\longrightarrow} P
\end{aligned}$
\item $\begin{aligned}
\int\limits f\d P_n\stackrel{}{\longrightarrow}\int\limits f\d P\qquad\forall f\in C^b(\S)\text{ glm. stetig}
\end{aligned}$
\item $\begin{aligned}
\limsup\limits_{n\to\infty} P_n(F)\leq P(F)\qquad\forall F\in\F(\S)
\end{aligned}$
\item $\begin{aligned}
\liminf\limits_{n\to\infty} P_n(G)\geq P(G)\qquad\forall G\in\G(\S)
\end{aligned}$
\item $\begin{aligned}
\limn P_n(B)=P(B)\qquad\forall B\in\B(\S)\mit P(\underbrace{\partial B}_{\in\F(\S)})=0
\end{aligned}$\\
Mengen $B\in\B(\S)$ mit $P(\partial B)=0$ heißen \textbf{$P$-randlos}.
\end{enumerate}
\end{satz}
\begin{proof}
\underline{Zeige (1) $\implies$ (2):}\\
Folgt aus der Definition \ref{def4.1} (1).\\

\underline{Zeige (2) $\implies$ (3):}\\
Sei $F\in\F(\S)$ (also abgeschlossen), Der Beweis von Satz \ref{satz3.17} zeigt: Es gibt eine Folge $(f_k)_{k\in\N}$ von gleichmäßigen stetigen, beschränkten Funktionen auf $\S$ mit $f_k\downarrow\indi_F$. Dann gilt:
\begin{align*}
\limsup\limits_{n\to\infty} P_n(F)
=\limsup\limits_{n\to\infty}\int\limits\underbrace{\indi_F}_{\leq f_k~\forall k\in\N}\d P_n
&\stackrel{\text{Mono}}{\leq}
\limsup\limits_{n\to\infty}\int\limits f_k\d P_n
\stackrel{\text{Vor (2)}}{=}
\int\limits f_k\d P\quad\forall k\in\N\\
&\stackrel{\text{Mono Konv}}{\implies}
\int\limits f_k\d P\stackrel{k\to\infty}{\longrightarrow}
\int\limits\indi_F\d P=P(F)\\
&\stackrel{k\to\infty}{\implies}~(3)
\end{align*}

\underline{Zeige (3) $\Longleftrightarrow$ (4):}\\
Nutze Übergang zum Komplement sowie Rechenregeln für $\liminf$ und $\limsup$:
\begin{align*}
\liminf\limits_{n\to\infty} P_n(G)
&=\liminf\limits_{n\to\infty} \big(1-P_n(G^C)\big)\\
&=1-\underbrace{\limsup\limits_{n\to\infty} P_n(\underbrace{G^C}_{\in\F})}_{\leq P(G^C)}\\
&\geq 1-P(G^C)\\
&=P(G)
\end{align*}

\underline{Zeige (3) $\implies$ (1):}\\
Sei $f\in C^b(\S)$ beliebig. Zu zeigen:
\begin{align}\label{eqProof1.4.2Sternchen}
\limsup\limits_{n\to\infty}\int\limits f\d P_n\leq\int\limits f\d P
\end{align}
\underline{1. Schritt:} Sei $0\leq f<1$. Setze
\begin{align*}
F_i:=\left\lbrace f\geq\frac{i}{k}\right\rbrace=\left\lbrace x\in\S:f(x)\geq\frac{i}{k}\right\rbrace,\qquad \forall 0\leq i\leq k,k\in\N
\end{align*}
Dann gilt $F_i\in\F~\forall i$, da $f$ stetig. Da 
\begin{align*}
\int\limits_{\S}f\d P
\stackeq{\text{Lin}}
\sum\limits_{i=1}^k\int\limits\indi_{\left\lbrace\frac{i-1}{k}\leq f<\frac{i}{k}\right\rbrace}\cdot f\d P
\end{align*}
folgt wegen Monotonie
\begin{align}\label{eqProof1.4.2Plus}
\sum\limits_{i=1}^k\underbrace{\frac{i-1}{k}}_{=\frac{i}{k}-\frac{1}{k}}\cdot P\left(\frac{i-1}{k}\leq f<\frac{i}{k}\right)
\leq
\int\limits f\d P
\leq
\sum\limits_{i=1}^k \frac{1}{k}\cdot P\Big(\underbrace{\frac{i-1}{k}\leq f<\frac{i}{k}}_{F_{i-1}\setminus F_i}\Big)
\end{align}
Die rechte Summe in \eqref{eqProof1.4.2Plus} ist gleich
\begin{align*}
&\frac{1}{k}\cdot\sum\limits_{i=1}^k i\cdot\big( P(F_{i-1}-P(F_i)\big)\\
&=\frac{1}{k}\cdot\Big(P(F_0)-P(F_1)+2\cdot P(F_1-2\cdot P(F_2)+3\cdot P(F_2)-3\cdot P(F_3)+\\
&\qquad+\ldots+(k-1)\cdot P(F_{k-2})-(k-1)\cdot P(F_{k-1})+k\cdot P(F_{k-1})-k\cdot P(F_k)\Big)\\
&=\frac{1}{k}\cdot\Big(\underbrace{P(F_0)}_{=1}+P(F_1)+P(F_2)+\ldots+P(F_{k-1})-k\cdot k\cdot \underbrace{P(F_k)}_{=0}\Big)\\
&=\frac{1}{k}+\frac{1}{k}\cdot\sum\limits_{i=1}^{k-1} P(F_i)
\end{align*}
Da die linke Summe in \eqref{eqProof1.4.2Plus} gleich der rechten Summe minus $\frac{1}{k}$, folgt
\begin{align}\label{eqProof1.4.2DoppelSternchen}
\sum\limits_{i=1}^{k-1} P(F_i)
\leq\int\limits f\d P
\leq\frac{1}{k}+\sum\limits_{i=1}^{k-1} P(F_i)
\end{align}
Beachte, \eqref{eqProof1.4.2DoppelSternchen} gilt für \ul{jedes} Wahrscheinlichkeitsmaß $P$, also auch für $P_n$. Damit folgt:
\begin{align*}
\limsup\limits_{n\to\infty}\int\limits f\d P_n
&\leq\frac{1}{k}+\sum\limits_{i=1}^{k-1}\underbrace{\limsup\limits_{n\to\infty} P_n(F_i)}_{\stackrel{(3)}{\leq}P(F_i)~\forall i}\\
&\leq\frac{1}{k}+\underbrace{\sum\limits_{i=1}^{k-1} P(F_i)}_{\stackrel{\eqref{eqProof1.4.2DoppelSternchen}}{\leq}\int\limits f\d P}\\
&\leq\frac{1}{k}+\int\limits f\d P\qquad\forall k\in\N
\end{align*}
Grenzwertbildung $k\to\infty$ liefert \eqref{eqProof1.4.2Sternchen}.\\

\ul{2. Schritt:} Da $f\in C^b(\S)$ beliebig, gilt:
\begin{align*}
\exists a<b:a\leq f<b
\implies g(x):=\frac{f(x)-a}{b-a}\text{ ist stetig und } 0\leq g<1
\end{align*}
Daraus folgt
\begin{align*}
\limsup\limits_{n\to\infty}\int\limits f\d P_n
&=\limsup\limits_{n\to\infty}\int\limits (b-a)\cdot g+a\d P_n\\
&=\limsup\limits_{n\to\infty}\left((b-a)\cdot\int\limits g\d P_n+a\right)\\
&\leq(b-a)\cdot\underbrace{\limsup\limits_{n\to\infty}\int\limits g\d P_n}_{\leq\int\limits g\d P\text{, wg. 1. Schritt}}+a\\
&\leq(b-a)\cdot\int\limits g\d P+0\\
&\stackeq{\text{Lin}}
\int\limits f\d P
\end{align*}
Damit ist \eqref{eqProof1.4.2Sternchen} gezeigt. Übergang zu $-f$ in \eqref{eqProof1.4.2Sternchen} liefert
\begin{align*}
\int\limits f\d P
&\leq
\liminf\limits_{n\to\infty}\int\limits f\d P_n\\
&=\liminf\limits_{n\to\infty}-\int\limits -f\d P_n\\
&=-\limsup\limits_{n\to\infty}-\int\limits \underbrace{-f}_{\in C^b(\S)}\d P\\
&\stackrel{\eqref{eqProof1.4.2Sternchen}}{\geq}
-\int\limits -f\d P\\
&\stackeq{\text{Lin}}
\int\limits f\d P
\qquad\forall f\in C^b(\S)\\
&\implies(1)
\end{align*}

\underline{Zeige (3) $\implies$ (5):}\\
Sei $B\in\B(\S)\mit P(\partial B)=0$. Dann gilt:
\begin{align*}
P(\overline{B})
&\stackrel{(3)}{\geq}
\limsup\limits_{n\to\infty} \underbrace{P_n(\overbrace{\overline{B}}^{\supseteq B})}_{\geq P_n(B)}\\
&\geq\limsup\limits_{n\to\infty} P_n(B)\\
&\stackrel{\text{stetig}}{\geq}
\liminf\limits_{n\to\infty} \underbrace{P_n(\overbrace{B}^{\supseteq\stackrel{\circ}{B}})}_{P_n(\stackrel{\circ}{B}}\\
\stackrel{(3)\gdw(4)}{\geq}
P(\stackrel{\circ}{B})\\
&=P(\overline{B})\\
&=P(B)
\end{align*}, denn:
\begin{align}\label{eqProof1.4.2SternchenZwei}
0
=P(\overbrace{\partial B}^{\overline{B}\setminus\stackrel{\circ}{B}}
)=P(\overline{B})-P(\stackrel{\circ}{B})
\implies
P(\stackrel{\circ}{B})\leq P(B)\leq P(\overline{B})=P(\stackrel{\circ}{B})
\end{align}
Da $\liminf=\limsup$ folgt $\limn P_n(B)=P(B)$.\\

\underline{Zeige (5) $\implies$ (3):}\\
Sei $F\in\F$ (abgeschlossen) beliebig. Dann gilt $\forall\varepsilon>0:$
\begin{align}\label{eqProof1.4.2SternchenUnten}
\partial\Big(\big\lbrace x\in\S:d(x,F)\leq\varepsilon\big\rbrace\Big)
\subseteq\big\lbrace x\in\S:d(x,F)=\varepsilon\big\rbrace
\end{align}
denn: Sei $x\in\partial\Big(\big\lbrace x\in\S:d(x,F)\leq\varepsilon\big\rbrace\Big)$. Dann gilt:
\begin{align*}
&\exists (x_n)_{n\in\N}:\forall n\in\N:d(x_n,F)\leq\varepsilon\wedge \limn x_n=x\\
&\exists (\xi_n)_{n\in\N}:\forall n\in\N:d(\xi_n,F)>\varepsilon\wedge\limn\xi_n=x
\end{align*}
Da $d(\cdot,F)$ stetig gemäß \ref{lemma2.3} (3), folgt
\begin{align*}
\varepsilon\leq d(x,F)\leq\varepsilon.
\end{align*}
Wegen \eqref{eqProof1.4.2SternchenUnten} sind 
\begin{align*}
A_\varepsilon:=\partial\Big(\big\lbrace x\in\S:d(x,F)\leq\varepsilon\big\rbrace\Big)\qquad\forall\varepsilon>0
\end{align*}
paarweise disjunkt, da bereits die Obermengen paarweise disjunkt sind. Dann folgt
\begin{align}\label{eqProof1.4.2DoppelSternchenUnten}
E:=\big\lbrace\varepsilon>0:P(A_\varepsilon)>0\big\rbrace\text{ ist höchstens abzählbar},
\end{align}
denn:
\begin{align*}
E=\bigcup\limits_{n\in\N}\underbrace{\left\lbrace\varepsilon>0:P(A_\varepsilon)\geq\frac{1}{m}\right\rbrace}_{=:E_m}
\end{align*}
Es gilt $|E_m|\leq m$, weil: Angenommen es existieren $0<\varepsilon_1<\ldots<\varepsilon_{m+1}$ mit 
\begin{align*}
&P(A_{\varepsilon_i})\geq\frac{1}{m}\qquad\forall 1\leq i\leq m+1\\
&\implies
1\geq P\left(\bigcup\limits_{i=1}^{m+1} A_{\varepsilon_i}\right)
\stackeq{\text{pw. disj.}}
\sum\limits_{i=1}^{m+1}\underbrace{P\big(A_{\varepsilon_i}\big)}_{\geq\frac{1}{m}}\geq(m+1)\cdot\frac{1}{m}>1
\end{align*}
Das ist ein Widerspruch. Damit ist $E$ höchstens abzählbar unendlich. Damit liegt das Komplement
\begin{align*}
E^C=\big\lbrace\varepsilon>0: P(A_\varepsilon)=0\big\rbrace
\end{align*}
dicht in $[0,\infty)$. (dies kann man auch durch Widerspruch zeigen)\\
Daraus folgt insbesondere:
\begin{align*}
\exists(\varepsilon_k)_{k\in\N}\subseteq\R\mit\varepsilon_k\downarrow0:\forall	 k\in\N: F_k:=\big\lbrace x\in\S:d(x,F)\leq\varepsilon_k\big\rbrace\text{ ist $P$-randlos}
\end{align*}
Beachte $A_{\varepsilon_k}=\partial F_k$. Da $F\subseteq F_k~\forall k\in\N$,gilt:
\begin{align*}
&\limsup\limits_{n\to\infty} P_n(F)
\leq\limsup\limits_{n\to\infty} \underbrace{P_n(F_k)}_{\text{konv.}}
\stackeq{(5)}
P(F_k)\qquad\forall k\in\N\\
&\stackrel{k\to\infty}{\implies}
\limsup\limits_{n\to\N} P_n(F)
\leq\lim\limits_{k\to\infty} P(F_k)
=P(F)
\end{align*}
Die letzte Gleichheit gilt, weil $P$ $\sigma$-stetig von oben ist und $F_k\downarrow F$. $F_k\downarrow F$, denn $F_1\supseteq F_2\supseteq\ldots$, da $\varepsilon_k$ monoton fallende Folge ist und
\begin{align*}
\bigcap\limits_{k\in\N}F_k=F, 
\end{align*}
denn: 
\begin{align*}
x\in\bigcap\limits_{k\in\N}F_k
&\Longleftrightarrow
x\in F_k\qquad\forall k\in\N\\
&\Longleftrightarrow
d(x,F)\leq\varepsilon_k\qquad\forall k\in\N\\
&\implies
d(x,F)=0\\
&\stackrel{\ref{lemma2.3}~(1)}{\Longleftrightarrow}
x\in \overline{F}\stackeq{F\in\F}F
\end{align*}
\end{proof}

Mitunter folgt schwache Konvergenz aus $P_n(A)\stackrel{n\to\infty}{\longrightarrow} P(A)$ für eine spezielle Klasse von Mengen $A$.

\begin{theorem}\label{theorem4.3}
Sei $\U\subseteq\B(\S)$ mit
\begin{enumerate}[label=(\roman*)]
\item $\begin{aligned}
A,B\in \U\implies A\cap B\in\U
\end{aligned}$, also $\U$ ist endlich $\cap$-stabil
\item Jedes offene $G$ ist abzählbare Vereinigung von Mengen aus $\U$.
\end{enumerate}
Dann gilt:
\begin{align*}
\Big(\forall A\in\U:P_n(A)\stackrel{n\to\infty}{\longrightarrow} P(A)\Big)\implies P_n\stackrelnew{w}{}{\longrightarrow} P
\end{align*}
\end{theorem}

\begin{proof}
Seien $A_1,\ldots,A_m\in\U$. Dann gilt:
\begin{align*}
P_n\left(\bigcup\limits_{i=1}^m A_i\right)
&\stackeq{\text{allg. Add-Formel}}
\sum\limits_{k=1}^m (-1)^{k-1}\cdot\sum\limits_{1\leq i_1<\ldots<i_k\leq m}\underbrace{P_n\big(\underbrace{A_{i_1}\cap\ldots\cap A_{i_k})}_{\in\U}}_{\stackrel{n\to\infty}{\longrightarrow} P(A_{i_1}\cap\ldots\cap A_{i_k})}\\
&\stackrel{n\to\infty}{\longrightarrow}
\sum\limits_{k=1}^m\sum\limits_{1\leq i_1<\ldots<i_k\leq m} P\left(A_{i_1}\cap\ldots\cap A_{i_k}\right)\\
&\stackeq{\text{all. Add.}}
P\left(\bigcup\limits_{i=1}^m A_i\right)
\end{align*}
Sei $G\in\G$. Dann gilt wegen Voraussetzung (ii):
\begin{align*}
&\exists(A_i)_{i\in\N}\subseteq\U:G=\bigcup\limits_{i\in\N} A_i\\
&\implies
G_m:=\bigcup\limits_{i=1}^m A_i\uparrow G,~m\to\infty\\
&\stackrel{P~\sigma\text{-stetig}}{\implies}
\forall \varepsilon>0:\exists m_0\in\N:P(G)-P(G_m)\leq\varepsilon\\
&\implies
P(G)-\varepsilon\leq P(G_m)
\stackeq{\text{s. o.}}
\limn P_n(\underbrace{G_m}_{\subseteq G})\leq\liminf\limits_{n\to\infty} P_n(G)\qquad\forall\varepsilon>0\\
&\stackrel{\varepsilon\to0}{\implies}
\liminf\limits_{n\to\infty} P_n(G)\geq P(G)\qquad\forall G\in\G
\end{align*}
Nun folgt die Behauptung aus dem Theorem \ref{satz4.2}.
\end{proof}

\begin{korollar}\label{korollar4.4}
Sei $\U$ endlich Durschnittsstabil mit
\begin{enumerate}[label=(\roman*)]
\item $\begin{aligned}
\forall x\in S,\forall\varepsilon>0:\exists A\in\U:x\in\stackrel{\circ}{A}\subseteq A\subseteq B(x,\varepsilon)
\end{aligned}$
\end{enumerate}
Ist $(\S,d)$ separabel, so gilt
\begin{align*}
\Big(\forall A\in\U:P_n(A)\stackrel{n\to\infty}{\longrightarrow} P(A)\Big)
\implies P_n-P
\end{align*}
\end{korollar}

\begin{proof}
Gemäß Satz \ref{satz2.9} hat $\G$ abzählbare Basis. Dem dem Satz von Lindelöf:
\begin{align}\label{eqSatzVonLindelöf}\tag{L}
\text{Für jede offene Überdeckung einer beliebigen Teilmenge von $\S$ existiert}\\\nonumber\text{eine abzählbare Teilüberdeckung.}
\end{align}
Sei nun $G\in\G$ beliebig. Für alle $x\in G$ existiert ein $\varepsilon_x>0$ mit $B(x,\varepsilon_x)\subseteq G$ gemäß (i) findet man ein $A_x\in\U$ mit $x\in\stackrel{\circ}{A}_x\subseteq A_x\subseteq B(x,\varepsilon_x)\subseteq G$. Also folgt
\begin{align*}
G=\bigcup\limits_{x\in G}\lbrace x\rbrace\subseteq\bigcup\limits_{x\in G}\stackrel{\circ}{A}_x\subseteq G
\end{align*}
Somit ist $\left\lbrace\stackrel{\circ}{A}_x:x\in G\right\rbrace$ eine offene Teilüberdeckung von $G$. Aus \eqref{eqSatzVonLindelöf} folgt nun: Es existieren $A_{x_i}\in\U,~i\in\N$ mit
\begin{align*}
G\subseteq\bigcup\limits_{i\in\N}\stackrel{\circ}{A}_{x_i}\subseteq 
\bigcup\limits_{i\in\N}A_{x_i}\subseteq 
G
\implies
G=\bigcup\limits_{i\in\N}A_{x_i}
\end{align*}
Also erfüllt $\U$ die Voraussetzung (i) und (ii) in Theorem \ref{theorem4.3} und es folgt die Behauptung.
\end{proof}

Als Anwendung / Beschreibung der schwachen Konvergenz im $\S=\R$. Erinnere an \textbf{schwache Konvergenz von Verteilungsfunktionen (VF)} $(F_n)_{n\in\N}$ gegen $F$, in Zeichen
\begin{align*}
F_n\rightharpoonup F
:\Longleftrightarrow
F_n(x)\stackrel{n\to\infty}{\longrightarrow} F(x)\qquad\forall x\in C_F
\mit C_F:=\big\lbrace x\in\R:F\text{ ist stetig in }x\big\rbrace
\end{align*}

\begin{korollar}\label{korollar4.5}
Seien $P,P_n,n\in\N$ Wahrscheinlichkeitsmaße auf $\B(\R)$ mit zugehörigen Verteilungsfunktionen $F$ und $F_n,n\in\N$. Dann gilt:
\begin{align*}
P_n\stackrelnew{w}{}{\longrightarrow}P
\Longleftrightarrow
F_n\rightharpoonup F
\end{align*}
\end{korollar}
\begin{proof}
\underline{Zeige ``$\implies$'':}\\
Sei $B:=(-\infty,x],~x\in\R$. Dann gilt:
\begin{align*}
P(\underbrace{\partial B}_{=\lbrace x\rbrace})=0
&\Longleftrightarrow P(\lbrace x\rbrace)=F(x)-\underbrace{F(x-0)}{\text{Grenzwert}}=0\\
& x\in C_F
\end{align*}
Somit folgt für $x\in C_F$:
\begin{align*}
F_n(x)&\stackeq{\text{Def}}
P_n\big(\underbrace{(-\infty,x]}_{=B}\big)=P_n(B)\stackrel{n\to\infty}{\longrightarrow} P(B)\stackeq{\text{Def}} F(x)\qquad\forall x\in C_F
\end{align*}
gemäß Satz \ref{satz4.2}.\\

\underline{Zeige ``$\Longleftarrow$'':} Sei
\begin{align*}
\U:=\big\lbrace (a,b]:a,b\in C_F\big\rbrace.
\end{align*}
Dann ist $\U$ endlich durchschnittsstabil. Ferner: Die Menge 
\begin{align*}
D_F:=\big\lbrace x\in\R: F\text{ \underline{nicht} stetig in }x\big\rbrace
=\big\lbrace x\in\R:P(\lbrace x\rbrace)>0\big\rbrace
\end{align*}
ist höchstens abzählbar (vgl. \eqref{eqProof1.4.2DoppelSternchen} 
%keine Ahnung welches DoppelSternchen er hier meint mit der Referenz To Do
im Beweis von Satz \ref{satz4.2}). Also folgt
\begin{align*}
\forall x\in\R:\forall\varepsilon>0:\exists A=(a,b]\in\U:x\in (a,b)=\stackrel{\circ}{A}\subseteq A=(a,b]\subseteq B(x,\varepsilon)=(a-\varepsilon,x+\varepsilon)
\end{align*}
denn: In $(x-\varepsilon,x+\varepsilon)$ muss ein $a\in C_F$ existieren, denn sonst wäre $(x-\varepsilon, x)\in D_F$. Analog findet man ein $b\in(x,x+\varepsilon)$. Somit erfüllt $\U$ die Voraussetzungen von Korollar \ref{korollar4.4}. Klar: $\S=\R$ ist separabel, da $\Q$ abzählbar und dicht in $\R$. Schließlich gilt:
\begin{align*}
P_n\big((a,b]\big)&=
F_n(\underbrace{b}_{\in C_F})-F_n(\underbrace{a}_{\in C_F})\stackrel{n\to\infty}{\longrightarrow} F(b)-F(a)=P\big((a,b]\big)\qquad\forall a,b\in C_F\\
&\stackrel{\ref{korollar4.4}}{\implies}
P_n\longrightarrow P
\end{align*}
\end{proof}

\begin{bemerkung}\ %4.6
\begin{enumerate}[label=(\arabic*)]
\item Seien $X,X_n,n\in\N$ reelle Zufallsvariablen über $(\Omega,\A,\P)$. Dann:
\begin{align}\label{eqBemerkung4.6}\tag{$\ast$} 
X_n\stackrel{\L}{\longrightarrow} X
\stackrel{\text{Def}}{\Longleftrightarrow}
\underbrace{\P\circ X_n^{-1}}_{\hat{=}P_n}
\stackrelnew{w}{}{\longrightarrow} \underbrace{\P\circ X^{-1}}_{\hat{=}P}
\stackrel{\ref{korollar4.5}}{\Longleftrightarrow}
\underbrace{\P(X_n\leq x)}_{\hat{=}F_n(x)}
\stackrel{n\to\infty}{\longrightarrow}
\underbrace{\P(X\leq x)}_{\hat{=}F(x)}
\end{align}
für alle $x$, die Stetigkeitsstellen der Verteilungsfunktion von $X$ sind.
\item Es gibt Verallgemeinerung von \ref{korollar4.5} bzw \eqref{eqBemerkung4.6} auf $\S=\R$:\\
Seien 
\begin{align*}
X=\left(X^{(1)},\ldots, X^{(k)}\right),X_n=\left(X^{(1)},\ldots,X^{(k)}_n\right),\qquad(\Omega,\A)\to\left(\R^k,\B(\R^k)\right)
\end{align*}
Zufallsvariablen in $\R^k$. Dann gilt:
\begin{align*}
X_n\stackrel{\L}{\longrightarrow} X\text{ in }\R^k
\Longleftrightarrow
\P(X_n\leq x)\stackrel{n\to\infty}{\longrightarrow}\P(X\leq x)\qquad\forall x=\big(x_1,\ldots,x_k\big)\in\R^k
\end{align*}
wobei $x_i$ Stetigkeitsstelle der Verteilungsfunktion von $X^{(i)}$ ist für alle $i\in\lbrace1,\ldots,k\rbrace$. Beweis ist analog zu \ref{korollar4.5}.
\end{enumerate}
\end{bemerkung}

Der schwache Limes einer Folge $(P_n)_{n\in\N}$ ist eindeutig, denn es gilt:

\begin{lemma}\label{lemma4.6Einhalb}
\begin{align*}
P_n\stackrelnew{w}{}{\longrightarrow} P,~P_n\stackrelnew{w}{}{\longrightarrow} Q\implies P=Q
\end{align*}
\end{lemma}
\begin{proof}
Gemäß Definition gilt:
\begin{align*}
\int\limits f\d P_n&\stackrel{}{\longrightarrow}\int\limits f\d P\qquad\forall f\in C^b(\S)\\
\int\limits f\d P_n&\stackrel{}{\longrightarrow}\int\limits f\d Q\qquad\forall f\in C^b(\S)
\end{align*}
Der Grenzwert von reellen Zahlenfolgen eindeutig ist, folgt
\begin{align*}
\int\limits f\d P=\int\limits f\d Q\qquad\forall f\in C^b(\S)\\
\stackrel{\ref{satz3.17}}{\implies}
P=Q
\end{align*}
\end{proof}

Im Folgenden ist das Ziel die Übertragung unserer Resultate auf Verteilungskonvergenz.

\begin{satz}[Portmanteau-Theorem]\label{satz4.7}\enter
Folgende Aussagen sind äquivalent:
\begin{enumerate}[label=(\arabic*)]
\item $\begin{aligned}
X_n\stackrel{\L}{\longrightarrow} X\text{ in }(\S,d)
\end{aligned}$
\item $\begin{aligned}
\E\big[f(X_n)\big]\stackrel{n\to\infty}{\longrightarrow}\E\big[f(X)\big]\qquad\forall f\in C^b(\S)
\end{aligned}$ gleichmäßig stetig
\item $\begin{aligned}
\limsup\limits_{n\to\infty}\P(X_n\in F)\leq\P(X\in F)\qquad\forall F\in\F
\end{aligned}$
\item $\begin{aligned}
\liminf\limits_{n\to\infty}\P(X_n\in G)\geq\P(X\in G)\qquad\forall G\in\G
\end{aligned}$
\item $\begin{aligned}
\P(X_n\in B)\stackrel{n\to\infty}{\longrightarrow}\P(X\in B)\qquad\forall B\in\B(\S)\mit\P(X\in\partial B)=0
\end{aligned}$
\end{enumerate}
\end{satz}
\begin{proof}
Wende Satz \ref{satz4.2} an auf $P_n:=\P\circ X^{-1}_n,~P:=\P\circ X^{-1}$ (wegen Def $\stackrel{\L}{\longrightarrow}$). Beachte z. B.
\begin{align*}
P_n(f)&=\P\circ X^{-1}_n(F)
\stackeq{\text{Def}}
\P\left(X_n^{-1}(F)\right)
=\P\big(\lbrace\omega\in\Omega:X_n(\omega)\in F\rbrace\big)
=\P(X_n\in F)
\end{align*}
und 
\begin{align*}
\int\limits f\d P_n
=
\int\limits_{\S} f\d(\P\circ X_n^{-1})
\stackeq{\text{Trafo}}
\int\limits_\Omega f(X_n)\d\P
=\E\big[f(X_n)\big]
\end{align*}
\end{proof}

