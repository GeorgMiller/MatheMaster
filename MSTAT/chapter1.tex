\chapter{Einführung}
Viele Schätzer in der Statistik sind definiert als Minimal- oder Maximalstelle von bestimmten \textit{Kriteriumsfunktionen}, z. B. der \textit{Maximum-Likelihood-Schätzer (MLS)} oder \textit{Minimum-Qudrat-Schätzer (MQS, KQS)} oder \textit{Bayes-Schätzer}. Allgemein nennt man solche Schätzer \textbf{M-Schätzer}.\\
Ziel: Untersuchung des asymptotischen Verhaltens ($n\to\infty$) von M-Schätzern über einen \textit{funktionalen Ansatz}. Als Beispiel:

\section{Der Median}
Sei $X:(\Omega,\A,\P)\to\R$ eine reelle Zufallsvariable mit Verteilungsfunktion $F_X$, also $X\sim F$. Definiere

\begin{align*}
Y_F(t)&:=\E\left(|X-t|\right)\\
&=\int\limits_\Omega |X(\omega)-t|\d\P(\omega)\\
&\stackeq{Trafo}\int\limits_\R|x-t|\cdot\P\circ X^{-1}\d x\\
&=\int\limits_\R|x-t|\cdot F\d x
\qquad\forall t\in\R\\
m&:=\arg\min\limits_{t\in\R}Y(t):=\text{ (irgendeine) Minimalstelle der Funktion}
\end{align*} 

Charakterisierung der Menge aller Mediane in folgendem kleinen Lemma:

\begin{notation}
$F(m-):=F(m-0):=\lim\limits_{t\uparrow m} F(t)$
\end{notation}

\begin{lemma}\label{lemmaMedian}
Sei $X\sim F_X$ integrierbar und $m\in\R$. Dann äquivalent:
\begin{enumerate}
\item $F(m-)\leq\frac{1}{2}\leq F(m)$
\item $\E[|X-t|]\geq\E[|X-m|]\qquad\forall t\in\R$
\item $m$ ist Median
\end{enumerate}
\end{lemma}

\begin{proof}
\underline{Zeige 1. $\Rightarrow$ 2.:}\\
Setze $h(t):=\E[|X-t|-|X-m|]\stackeq{Lin}Y(t)-Y(m)$. Dann ist 2. äquivalent zu $h(t)\geq0~\forall t\in\R$. Dies ist noch zu zeigen.\\

\underline{Fall 1: $t<m$}
\begin{align*}
h(t)&\stackeq{Trafo}\int\limits_{\R}|x-t|-|x-m| Q_F(\d x)\\
&=\int\limits_{(-\infty,t]}\underbrace{|x-t|-|x-m|}_{=t-x-(m-x)=-(m-t)} F(\d x)
+\int\limits_{(t,m)}\underbrace{\underbrace{|x-t|-|x-m|}_{\underbrace{x-t}_{\geq0}-\underbrace{(m-x)}_{\leq m-t}}}_{\geq-(m-t)} F(\d x)
+\int\limits_{[m,\infty)}\underbrace{|x-t|-|x-m|}_{x-t-(x-m)=m-t} F(\d x)\\
&\geq-(m-t)\cdot \underbrace{Q((-\infty,t])}_{F(t)}+
\Big(-(m-t)\cdot F(m-)-F(t)\Big)
+(m-t)\cdot\underbrace{Q([m,\infty))}_{1-\underbrace{Q((-m,m)}_{F(m-)}}\\
&=-\underbrace{(m-t)}_{\geq0}\cdot(\underbrace{1-2\cdot F(m-)}_{\stackrel{1.}{\geq}0})\\
&\geq0
\end{align*}

\underline{Fall 2: $t> m$}
\begin{align*}
h(t)&=\int\limits_{(-\infty,m]}\ldots F(\d x)+\int\limits_{(m,t]}\ldots+\int\limits_{(t,\infty)}\ldots F(\d x)\\
&\ldots\\
&\geq(t-m)\cdot(\underbrace{2\cdot F(m)-1}_{\stackrel{1.}{\geq}0})\\
&\geq0
\end{align*}

\underline{Fall 3: $t=m$} ist trivial. $\#$

\underline{Zeige 2. $\Rightarrow$ 1:}\\
Nach Annahme ist $h(t)\geq0~\forall t\in\R$.\\

\underline{Fall 1: $t<m$} Die obige Rechnung im Fall 1 bei 1. $\Rightarrow$ 2. zeigt:
\begin{align*}
0\leq h(t)&=-(m-t)\cdot F(t)+\int\limits_{(t,m)}\underbrace{\underbrace{x}_{<m}-t-(m-x) }_{=2x-t-m\leq m-t}F(\d x)+(m-t)\cdot(1-F(m-))\\
&\leq-(m-t)\cdot\Big(F(t)-1\underbrace{+F(m-)-F(m-)}_{=0}+F(t)\Big)\\
&=\underbrace{(m-t)}_{>0}\cdot(1-2\cdot F(t))\\
&\Longrightarrow\forall t<m:0\leq(m-t)\cdot(1-2\cdot F(t))\\
&\Longrightarrow\forall t<m:0\leq 1-2\cdot F(t)\\
&\Longrightarrow\forall t<m:F(t)\leq\frac{1}{2}\\
&\stackrel{t\uparrow m}{\Longrightarrow}F(m-)\leq\frac{1}{2}
\end{align*}
\underline{Fall 2: $t> m$} Siehe 2. Fall, analog.
\end{proof}

\begin{bemerkung} TODO Nr 1.2\\
\begin{enumerate}
\item Lemma \ref{lemmaMedian} 1. besagt, dass $\lbrace m\in\R: m\text{ erfüllt } 1.\rbrace$ = Menge aller Mediane von $F$.
\item Im Allgemeinen gibt es mehrere Mediane. Üblicherweise \underline{wählt} man $m:=F^{-1}(\frac{1}{2})$, wobei
\begin{align*}
F^{-1}(u):=\inf\left\lbrace x\in\R:F(x)\geq u\right\rbrace\quad\forall u\in (0,1)
\end{align*}
die \textbf{Quantilfuntion / die verallgemeinerte Inverse} ist. Da
\begin{align*}
F\left(F^{-1}(u)-\right)\leq u\leq F\left(F^{-1}(u)\right)\qquad\forall u\in (0,1),
\end{align*}
erfüllt $m=F^{-1}\left(\frac{1}{2}\right)$ die Bedingung 1. in Lemma \ref{lemmaMedian} und ist somit ein Median, nämlich der kleinste.
\item Die Funktion 
\begin{align*}
Y:\R\to\R,\qquad Y(t)=\int\limits|x-t|~F(\d x)\qquad\forall t\in\R
\end{align*}
TODO die obige; ist stetig (nutze Folgenkriterium + dominierte Konvergenz bzw. Satz von Lebesgue), aber im Allgemeinen nicht differenzierbar, z. B. falls $F\sim X$ eine diskrete Zufallsvariable ist. In diesem Fall ist somit die Minimierungüber Differentiation nicht möglich!
\end{enumerate}
\end{bemerkung}
