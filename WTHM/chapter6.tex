% This work is licensed under the Creative Commons
% Attribution-NonCommercial-ShareAlike 4.0 International License. To view a copy
% of this license, visit http://creativecommons.org/licenses/by-nc-sa/4.0/ or
% send a letter to Creative Commons, PO Box 1866, Mountain View, CA 94042, USA.

\chapter{Fouriertransformation und charakteristische Funktionen} %6
Ziel: Mit Zufallsvariablen analytisch / deterministisch zu rechnen \nl
Wichtige Anwendungen:
\begin{itemize}
\item Charakterisierung von Unabhängigkeit
\item schwache Konvergenz
\item Zentrale Grenzwertsätze
\end{itemize}

\begin{defi}\
\begin{enumerate}[label=(\alph*)]
\item Für $\mu\in\mathcal{M}_b(\R^d)$ (beschränkte Borelmaße auf $\R^d$) ist die \textbf{Fouriertransformation (FT)} definiert als
\begin{align*}
\hat{\mu}:\R^d\to\C,\qquad
\hat{\mu}(\xi):=\int\limits_{\R^d}\exp(i\cdot\xi^T\cdot x)~\mu(\d x)\qquad\forall\xi\in\R^d
\end{align*}
\item Für $f\in L_1(\R^d,\d x)$ ist die \textbf{FT} definiert als
\begin{align*}
\hat{f}:\R^d\to\C,\qquad
\hat{f}(\xi):=\int\limits_{\R^d}\exp(i\cdot\xi^T\cdot x)\cdot f(x)\d x\qquad\forall\xi\in\R^d
\end{align*}
\item Für Zufallsvariable $X:\Omega\to\R^d$ ist die \textbf{charakteristische Funktion (CF)} definiert als
\begin{align*}
\Phi_X:\R^d\to\C,\qquad
\Phi_X(\xi):=\E\Big[\exp\big(i\cdot X^T\cdot\xi\big)\Big]\qquad\forall\xi\in\R^d
\end{align*} 
\end{enumerate}
\end{defi}

\begin{bemerkung}
(a) ist die allgemeinste Definition. (b) und (c) können als Spezialfälle betrachtet werden:\\
(b): Setze $\mu(\d x):=f(x)\d x$, d.h. $f$ ist Dichte von $\mu$ bzgl. Lebesgue-Maß. Damit gilt:
\begin{align*}
\hat{\mu}(\xi)
&=\int\limits_{\R^d}\exp(i\cdot x^T\cdot\xi)~\mu(\d x)
=\int\limits_{\R^d}\exp(i\cdot x^T\cdot\xi)\cdot f(x)\d x=\hat{f}(\xi)
\end{align*}
(c): Sei $\P_X$ Bildmaß von $X$ unter $\P$ (d.h. Wahrscheinlichkeitsmaß auf $\R^d$). Dann gilt:
\begin{align*}
\hat{\mu}(\xi)=\int\limits_{\R^d}\exp(i\cdot x^T\cdot\xi)~\P_X(\d x)=\E\big[\exp(i\cdot X^T\cdot\xi)\big]
=\Phi_X(\xi)
\end{align*}
Weitere Notation: 
\begin{align*}
\hat{\mu}=:\F(\mu)\qquad\hat{f}=:\F(f)
\end{align*}
Hier steht $\F$ also nicht für eine $\sigma$-Algebra, sondern für die Fouriertransformation. Die Schreibweise ist günstig, um große ``Dächer'' zu vermeiden. 
\end{bemerkung}

\begin{beisp}\
\begin{enumerate}[label=(\alph*)]
\item Sei $X\sim\mathcal{N}(0,1)$. Folglich ist
\begin{align*}
\Phi_X(\xi)=\exp\left(-\frac{\xi^2}{2}\right)
\end{align*}
denn:
\begin{align*}
\Phi_X(\xi)
&=\int\limits_{-\infty}^\infty \exp(i\cdot x\cdot\xi)\cdot\frac{1}{\sqrt{2\cdot\pi}}\cdot\exp\left(-\frac{x^2}{2}\right)\d x\\
\Phi_X'(\xi)
&=\frac{1}{\sqrt{2\pi}}\cdot\int\limits_{-\infty}^\infty
i\cdot x\cdot\exp\left(i\cdot x\cdot\xi-\frac{x^2}{2}\right)\d x\\
&=i\cdot\frac{1}{\sqrt{2\cdot\pi}}\cdot\int\limits_{-\infty}^\infty\exp(i\cdot x\cdot\xi)\cdot\underbrace{x\cdot\exp\left(-\frac{x^2}{2}\right)}_{=-\frac{\partial}{\partial x}\cdot\exp\left(-\frac{x^2}{2}\right)}\\
&=\underbrace{\left[i\cdot\frac{1}{\sqrt{2\cdot\pi}}\cdot\exp(i\cdot x\cdot\xi)\cdot\exp\left(-\frac{x^2}{2}\right)\right]_{-\infty}^\infty}_{=0}\\
&\qquad-\xi\cdot\underbrace{\frac{1}{\sqrt{2\cdot\pi}}\cdot\int\limits_{-\infty}^\infty\exp(i\cdot x\cdot\xi)\cdot\exp\left(-\frac{x^2}{2}\right)\d x}_{=\Phi_X(\xi)}\\
&=-\xi\cdot\Phi_X(\xi)\\
\end{align*}
Also erfüllt $\Phi_X(\xi)$ die gewöhnliche Differentialgleichung
\begin{align*}
\Phi_X'(\xi)=-\xi\cdot\Phi_X(\xi),\qquad\Phi_X(0)=1
\end{align*}
Eindeutige Lösung ist $\Phi_X(\xi)=\exp\left(-\frac{\xi^2}{2}\right)$.\nl
\textbf{Merke:} Die Charakteristische Funktion der Standardnormalverteilung ist proportional zu ihrer Dichte.
\item Sei $X\sim\mathcal{N}(0,I_d)$ multivariat. Dann haben wir
\begin{align*}
\Phi_X(\xi)
&=\E\Big[\exp(i\cdot\xi^T\cdot X)\Big]\\
&=\E\left[\prod\limits_{j=1}^d\exp(i\cdot\xi_j\cdot X_j)\right]\\
\overset{(X_j)\text{ unab}}&=
\prod\limits_{j=1}^d\underbrace{\E\Big[\exp(i\cdot\xi\cdot X_j)\Big]}_{=\Phi_{X_j}(\xi_j)}\\
&=\prod\limits_{j=1}^d\exp\left(-\frac{\xi_j^2}{2}\right)\\
&=\exp\left(-\frac{\xi^T\cdot\xi}{2}\right)\\
&=\exp\left(-\frac{|\xi|^2}{2}\right)
\end{align*}
\end{enumerate}
\end{beisp}

\begin{theorem}[Eigenschaften der FT / CT]\label{theorem6.1EigenschaftenDerFTCF}\enter
Sei $\mu\in\mathcal{M}_b(\R^d)$ bzw. $X:\Omega\to\R^d$ eine Zufallsvariable. Dann gilt:
\begin{enumerate}[label=(\alph*)]
\item $\begin{aligned}
\hat{\mu}(0)=\mu(\R^d)\qquad\text{bzw.}\qquad\Phi_X(0)=1
\end{aligned}$
\item $\begin{aligned}
\Big|\hat{\mu}(\xi)\Big|\leq\mu(\R^d)\qquad\text{bzw.}\qquad\Big|\Phi_X(\xi)\Big|\leq \qquad\forall\xi\in\R^d
\end{aligned}$ (Beschränktheit)
\item $\xi\mapsto\hat{\mu}(\xi)$ bzw. $\xi\mapsto\Phi_X(\xi)$ ist stetig
\item $\begin{aligned}
\hat{\mu}(-\xi)=\overline{\hat{\mu}(\xi)}\qquad\text{bzw.}\qquad\Phi_X(-\xi)=\overline{\Phi_X(\xi)}\qquad\forall\xi\in\R^d
\end{aligned}$ (Hermitsch-Symmetrie)
\item Sei $T(x)=A\cdot x+b$ eine lineare Transformation mit $A\in\R^{m\times d}$ und $b\in\R^m$. Dann gilt:
\begin{align*}
\F\left(\mu\circ T^{-1}(\xi)\right)&=\exp\left(i\cdot b^T\cdot\xi\right)\cdot\hat{\mu}\left(A^T\cdot\xi\right)\qquad\forall\xi\in\R^m\text{ bzw.}\\
\Phi_{(A\cdot X+b)}(\xi)
&=\exp\left(i\cdot b^T\cdot\xi\right)\cdot\Phi_X\left(A^T\cdot\xi\right)\qquad\forall\xi\in\R^m
\end{align*}
\end{enumerate}
\end{theorem}

\begin{proof}
\underline{Zu (a):}
\begin{align*}
\hat{\mu}(0)&=\int\limits_{\R^d}1~\mu(\d x)=\mu(\R^d)
\end{align*}
\underline{Zu (b):}
\begin{align*}
\Big|\hat{\mu}(\xi)\Big|
&=\left|\int\limits_{\R^d}\exp(i\cdot x^T\cdot\xi)~\mu(\d x)\right|\\
&\leq
\int\limits_{\R^d}\underbrace{\Big|\exp(i\cdot x^T\cdot\xi)\Big|}_{=1}~\mu(\d x)
=\mu(\R^d)
\end{align*}
\underline{Zu (c):}\\
$\xi\mapsto\exp(i\cdot x^T\cdot\xi)$ ist stetig für alle $x$ und $\Big|\exp(i\cdot x^T\cdot\xi)\Big|\leq1$. Somit folgt aus dominierter Konvergenz die Stetigkeit von
\begin{align*}
\xi\mapsto\int\limits\exp\left(i\cdot x\cdot\xi\right)~\mu(\d x).
\end{align*}
\underline{Zu (d):}
\begin{align*}
\hat{\mu}(-\xi)
&=\int\limits_{\R^d}\exp(-i\cdot x^T\cdot\xi)~\mu(\d x)\\
&=\int\limits_{\R^d}\overline{\exp(i\cdot x^T\cdot\xi)}~\mu(\d x)\\
&=\overline{\int\limits_{\R^d}\exp(i\cdot x^T\cdot\xi)~\mu(\d x)}\\
&=\overline{\hat{\mu}(\xi)}
\end{align*}
\underline{Zu (e):}
\begin{align*}
\F\left(\mu\circ T^{-1}\right)(\xi)
&=\int\limits_{\R^m}\exp(i\cdot\xi^T\cdot x)~\big(\mu\circ T^{-1}\big)(\d x)\\
\overset{\text{Trafo}}&=
\int\limits_{\R^d}\exp\big(i\cdot\xi^T\cdot T(x)\big)~\mu(\d x)\\
&=\int\limits_{\R^d}\exp\left(i\cdot\xi^T\cdot (A\cdot x)+i\cdot\xi^T\cdot b\right)~\mu(\d x)\\
&=\exp\left(i\cdot\xi^T\cdot b\right)\cdot\int\limits_{\R^d}\exp\left(i\cdot(A^T\cdot\xi)^T\cdot x\right)~\mu(\d x)\\
&=\exp\left(i\cdot\xi^T\cdot b\right)\cdot\hat{\mu}\left(A^T\cdot\xi\right)
\end{align*}
Alle Aussagen für $X$ folgen als Spezialfall $\mu=\P_X$.
\end{proof}

\begin{beisp}[Multivariate Normalverteilung]\enter
Sei $X\sim\mathcal{N}(0,I_d)$, $\mu\in\R^d$, $\Sigma\in\R^{d\times d}$ positiv semidefinit mit $\Sigma=A^T\cdot A$ (z. B. \textit{Cholesky-Zerlegung}). Dann gilt:
\begin{align*}
Y&:=\underbrace{\mu+A\cdot X}_{=T(X)}\sim\mathcal{N}(\mu,\Sigma)\\
\overset{\ref{theorem6.1EigenschaftenDerFTCF}}{\implies}
\Phi_Y(\xi)&=\exp\left(i\cdot\mu^T\cdot \xi\right)\cdot\Phi_X\left(A^T\cdot\xi\right)\\
&=\exp\left(i\cdot\mu^T\cdot\xi-\frac{1}{2}\cdot\frac{\xi^T\cdot A\cdot A^T\cdot\xi}{\Sigma}\right)\\
&=\exp\left(-i\cdot\mu^T\cdot\xi-\frac{1}{2}\cdot\xi^T\cdot\Sigma\cdot\xi\right)
\end{align*}
Das ist die CF von $\mathcal{N}(\mu,\Sigma)$.
\end{beisp}
