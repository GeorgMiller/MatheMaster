% This work is licensed under the Creative Commons
% Attribution-NonCommercial-ShareAlike 4.0 International License. To view a copy
% of this license, visit http://creativecommons.org/licenses/by-nc-sa/4.0/ or
% send a letter to Creative Commons, PO Box 1866, Mountain View, CA 94042, USA.

\chapter{Martingalkonvergenz und gleichgradige Integrierbarkeit} %4
%\setcounter{section}{4} %kleiner Hack

\section{Martingalkonvergenz}
\underline{Vorüberlegung:} Sei $(a_n)_{n\in\N}\subseteq\R$ deterministische Folge in $\R$.
\begin{itemize}
\item $\liminf\limits_{n\to\infty} a_n$ und $\limsup\limits_{n\to\infty} a_n$ immer wohldefiniert mit Werten in $\overline{\R}:=\R\cup\lbrace\pm\infty\rbrace$
\item Grenzwert $\limn a_n$ existiert in $\overline{\R}$
\begin{align*}
\Longleftrightarrow\liminf\limits_{n\to\infty} a_n=\limsup\limits_{n\to\infty} a_n
\end{align*}
\item Mit Kontraposition gilt also 
\begin{align*}
&\lim a_n\text{ existiert \ul{nicht} in }\overline{\R}\\
&\Longleftrightarrow\liminf\limits_{n\to\infty} a_n<\limsup\limits_{n\to\infty} a_n\\
&\Longleftrightarrow
\exists p,q\in\Q\mit p<q:\liminf\limits_{n\to\infty} a_n<p<q<\limsup\limits_{n\to\infty} a_n\\
&\Longleftrightarrow
\exists p,q\in\Q\mit p<q:\\
&\qquad(a_n)_{n\in\N}\text{ unendlich oft den Streifen }[p,q]\times\N\text{ ``aufsteigend'' überquert}\\
&\Longleftrightarrow
\exists p,q\in\Q\mit p<q:U[p,q]=\infty
\end{align*}
wobei $U[p,q]$ die \textbf{upcrossings} (aufsteigende Überquerungen) des Streifens $[p,q]\times\N$ bezeichnet.
\end{itemize}

%TODO (für Robert oder Henrik): Hier Skizze einfügen

\begin{defi}
Sei $(X_n)_{n\in\N}$ ein adaptierter stochastischer Prozess und $p,q\in\R, p<q,N\in\N$. Setze
\begin{align*}
U_N[p,q]&:=\max\Big\lbrace k\in\N_0~\Big|~\exists\text{ Stoppzeiten }0<\sigma_1<\tau_1<\sigma_2<\tau_2<\ldots<\tau_k\le n:\\
&\qquad\qquad\qquad\qquad\forall i\in\lbrace 1,\ldots,k\rbrace: x_{\sigma_i}<p\wedge X_{\tau_i}>q\Big\rbrace
\end{align*}
$U_N[p,q]$ ist die Anzahl der \textbf{Upcrossings} von $[p,q]\times\lbrace0,1,\ldots,N\rbrace$ durch $(X_n)_{n\in\N}$ und 
\begin{align*}
U[p,q]=\limsup\limits_{n\to\infty} U_n[p,q]
\end{align*}
die Anzahl der \textbf{Upcrossings} von $[p,q]\times\N$.
\end{defi}

\begin{lemma}[Doob's Upcrossing Lemma]\enter\label{lemma4.1DoobsUpcrossingLemma}
Sei $(X_n)_{n\in\N}$ Sub-Martingal und $p,q\in\R\mit p<q$. Dann gilt:
\begin{align*}
\E\big[U_N[p,q]\big]&\leq\frac{\E\big[(X_N-p)^+\big]}{q-p}\qquad\forall N\in\N
\end{align*}
\end{lemma}

\begin{bemerkung}
\textbf{Positivteil} und \textbf{Negativteil} einer Zufallsvariblen $X$ ist definiert als
\begin{align*}
X^+(\omega):=\max\big\lbrace X(\omega),0\big\rbrace
\qquad\text{ und }\qquad
X^-(\omega):=-\min\big\lbrace X(\omega),0\big\rbrace
\end{align*}
Es gilt $X=X^+-X^-$ und $|X|=X^++X^-$.
\end{bemerkung}

\begin{proof}
Setze der Kürze halber $k(\omega):=U_N[p,q](\omega)$. Klarerweise ist $k\leq N$. Definiere
\begin{align*}
\tau_0&:=0\\
\sigma_j&:=\min\big\lbrace k\geq \tau_{j-1}:X_k<p\big\rbrace\wedge N\text{ ``Erstaustrittszeit''}\\
\tau_j&:=\max\big\lbrace k\geq\sigma_j:X_k>q\big\rbrace\wedge N
\end{align*}
d. h. $\tau_0<\overbrace{\sigma_1<\tau_1}^{\text{1. Upcrossing}}<\overbrace{\sigma_2<\ldots}^{\text{2. Upcrossing}}\ldots<\tau_k$ und $\tau_{k+1}=\sigma_{k+1}=\tau_{k+2}=\ldots=\tau_N=N$. Es gilt:
\begin{align}\label{eqProof4.1.1Stern}\tag{$\ast$}
(q-p)\cdot U_N[p,q] &\leq
\overbrace{\underbrace{(X_{\tau_1}-p)}_{>q-p}+\underbrace{(X_{\tau_2}-X_{\sigma_2})}_{>q-p}+\ldots+\underbrace{(X_{\tau_k}-X_{\tau_k}-X_{\sigma_k})}_{>q-p}}^{U_N[p,q]\text{ Stück}}
\end{align}
Außerdem gilt
\begin{align}\label{eqProof4.1.1ZweiStern}\tag{$\ast\ast$}
\min\lbrace X_n-p,0\rbrace&\leq x_n-X_{\sigma_{k+1}},
\end{align}
denn: 
\begin{align*}
X_n-p<0&\implies X_{\sigma_{k+1}}<p&\implies X_N-p\leq X_{\sigma_{k+1}}-p\\
X_n-p\geq0&\implies \sigma_{k+1}=N&\implies X_N-p- X_{\sigma_{k+1}}=0
\end{align*}
Addiere von \eqref{eqProof4.1.1Stern} auf \eqref{eqProof4.1.1ZweiStern}
\begin{align*}
(q-p)\cdot U_N[p,q]+\min\lbrace X_N-p,0\rbrace\leq (X_{\tau_1}-p)
\end{align*}
Bilden von Erwartungswert und Umordnen der Summe liefert
\begin{align*}
(q-p)\cdot\E\big[U_N[p,q]\big]-\E\big[(X_N-p)^-\big]
&\leq-p+\sum\limits_{j=1}^{N-1}\Big(\underbrace{\E\big[X_{\sigma_{j+1}}\big]-\E\big[X_{\tau_j}\big]}_{\leq 0\text{, wegen Theorem \ref{theorem3.4}}}\Big)+\E[X_n]\\
&\leq\E\big[(X_n-p)\big]\\
\implies
(q-p)\cdot\E\big[U_N[p,q]\big]&\leq\E\big[(X_n-p)+(X_n-p)^-\big]\\
&=\E\big[(X_n-p)^+\big]
\end{align*}
\end{proof}

\begin{theorem}[Martingalkonvergenz]\label{theorem4.2Martingalkonvergenz}\enter
Sei $(X_n)_{n\in\N}$ ein Submartingal mit $\sup\limits_{n\in\N}\E[X_n^+]<\infty$. Dann gilt:
\begin{align*}
\exists X_\infty\in L_1(\Omega,\A,\P):\limn X_n=X_\infty\text{ fast sicher}
\end{align*}
\end{theorem}
\begin{proof}
nächste VL.
\end{proof}




%%%%%%%%%%
%Abschließbar: $X_n=\E[X_\infty~|~\F_n]$?


