% This work is licensed under the Creative Commons
% Attribution-NonCommercial-ShareAlike 4.0 International License. To view a copy
% of this license, visit http://creativecommons.org/licenses/by-nc-sa/4.0/ or
% send a letter to Creative Commons, PO Box 1866, Mountain View, CA 94042, USA.

\chapter{Martingalungleichungen und Rückwärtsmartingale} %5
\section{Martingalungleichungen} %5.1
\begin{itemize}
\item Doobs Maximalungleichung
\item Doobs $L_p$-Ungleichung
\item Azumas Ungleichung (Beispiel für \textit{Konzentrationsungleichung})
\end{itemize}

\setcounter{section}{5} %fix numbering
\begin{theorem}[Doobs Maximalungleichung]\label{theorem5.1DoobMaximalungleichung}\enter
Sei $(M_n)_{n\in\N}$ (Sub-)Martingal. Dann gilt:
\begin{align*}
\P\left(\max\limits_{j\leq n} M_j\geq r\right)\leq\frac{1}{r}\cdot\E\left[M_n^+\right]\qquad\forall n\in\N,\forall r\geq0
\end{align*}
Hierbei hängt die linke Seite vom gesamten Pfad ab, aber die rechte Seite hängt nur vom Endwert ab.
\end{theorem}
\begin{bemerkung}\
\begin{enumerate}[label=(\alph*)]
\item $(M_n)_{n\in\N}$ Martingal $\implies\big(|M_n|\big)_{n\in\N}$ Submartingal
\begin{align*}
\implies\P\left(\max\limits_{j\leq n} |M_j|\geq r\right)\leq\frac{\E\big[|M_n|\big]}{r}
\end{align*}
\item Es gilt die Zwischenabschätzung
\begin{align*}
\P\left(\max\limits_{j\leq n} M_j\geq r\right)
\leq
\frac{1}{r}\cdot\E\left[M_n\cdot\indi_{\left\lbrace\max\limits_{j\leq n} M_j\geq r\right\rbrace}\right]
\leq
\frac{1}{r}\cdot\E\left[M_n^+\right]
\end{align*}
\end{enumerate}
\end{bemerkung}

\begin{proof}
$(M_n)_{n\in\N}$ sei Submartingal. Für $k\leq n,A\in\F_k$ gilt wegen der Submartingaleigenschaft:
\begin{align}
M_k&\leq\E[M_n~|~\F_k]\nonumber\\
\implies\indi_A\cdot M_k &\leq\E\big[\indi_A\cdot M_n~\big|~\F_k\big]\nonumber\\
\implies\E[\indi_A\cdot M_k]&\leq\E[\indi_A\cdot M_n]\label{eqProof5.1Stern}\tag{$\ast$}
\end{align}
Definiere die Stoppzeit
\begin{align*}
\tau:=\min\big\lbrace n\in\N_0:M_n\geq r\big\rbrace.
\end{align*}
Es gilt
\begin{align*}
\tau\leq n&\implies \max\limits_{j\leq n} M_j\geq r\\
\text{ und } \max\limits_{j\leq n} M_j\geq r&\implies\tau\leq n
\end{align*}
d.h.
\begin{align}\label{eqProof5.1SternStern}\tag{$\ast\ast$}
\lbrace\tau\leq n\rbrace=\left\lbrace\max\limits_{j\leq n} M_j\geq r\right\rbrace
\end{align}
Somit
\begin{align*}
r\cdot\indi_{\lbrace\tau\leq n\rbrace}
&\leq M_\tau\cdot\indi_{\lbrace\tau\leq n\rbrace}
=\sum\limits_{k=0}^n M_k\cdot\indi_{\lbrace\tau=k\rbrace}
\end{align*}
Bilde Erwartungswerte:
\begin{align*}
r\cdot\P[\tau\leq n]
&\leq\sum\limits_{k=0}^n\E\Big[M_k\cdot\indi_{\underbrace{\lbrace\tau=k\rbrace}_{\in\F_k}}\Big]\\
\overset{\eqref{eqProof5.1Stern}}&{\leq}
\sum\limits_{k=0}^n\E\Big[M_n\cdot\indi_{\lbrace\tau=k\rbrace}\Big]\\
&=\E\Big[M_n\cdot\indi_{\lbrace\tau\leq n\rbrace}\Big]\\
\stackrel{\eqref{eqProof5.1SternStern}}{\implies}
r\cdot\P\left(\max\limits_{j\leq n} M_j\geq r\right)
&\leq\E\Big[M_n\cdot\indi_{\left\lbrace\max\limits_{j\leq n} M_j\geq r\right\rbrace}\Big]
\leq\E\big[M_n^+\big]
\end{align*}
\end{proof}

\begin{theorem}[Doobs $L_p$-Ungleichung]\label{theorem5.2DoobsLpUngleichung}\enter
Sei $(M_n)_{n\in\N}$ Martingal (oder positives Submartingal) und $p>1$. Dann gilt:
\begin{align*}
\big\Vert M_n^\ast\big\Vert_{p}\leq\frac{p}{p-1}\cdot\big\Vert M_n\big\Vert_p\qquad\forall n\in\N_0\qquad\mit M_n^\ast:=\max\limits_{0\leq j\leq n}|M_j|
\end{align*}
\end{theorem}

\begin{lemma}\label{lemma5.3}
Sei $X,Y\geq0$ mit $\E\big[Y^p]<\infty$ für $p>1$. Wenn
\begin{align}\label{eq5.3Vor}\tag{$\ast$}
\lambda\cdot\P(X\geq\lambda)\leq\E\left[Y\cdot\indi_{\lbrace X\geq\lambda\rbrace}\right]\qquad\forall\lambda\geq0
\end{align}
gilt, dann gilt:
\begin{align*}
\Vert X\Vert_p\leq\frac{p}{p-1}\cdot\Vert Y\Vert_p
\end{align*}
\end{lemma}
\begin{proof}
Setze $X_N:=X\wedge N$ (d.h. Minimum) für alle $n\in\N$. Zeige die Aussage zunächst für $X_N$: Merke
\begin{align*}
\lbrace X\geq\lambda\rbrace=\lbrace X_N\geq\lambda\rbrace\qquad\forall\lambda\in[0,N]
\end{align*}
D.h. \eqref{eq5.3Vor} gilt auch für $X_N$, solange $\lambda\in[0,N]$ gilt. Weitere Überlegung:
\begin{align*} %x soll hier klein sein
z^p
= p\cdot\int\limits_0^z x^{p-1}\d x
=p\cdot\int\limits_0^\infty x^{p-1}\cdot\indi_{\lbrace z\geq x\rbrace}\d x\qquad\forall p>0,\forall z>0
\end{align*}
Setze $X_N$ für $z$ ein und bilde Erwartungswert:
\begin{align*}
\E\left[X_N^p\right]
&=p\cdot\int\limits_0^\infty x^{p-1}\cdot\E\left[\indi_{\lbrace X_N \geq x\rbrace}\right]\d x \\
&=p\cdot\int\limits_0^\infty x^{p-1}\cdot\P\big(X_N\geq x\big)\d x\\
&=p\cdot\int\limits_0^N x^{p-1}\cdot\P\big(X_N\geq x\big)\d x + p\cdot\int\limits_N^\infty x^{p-1}\cdot\underbrace{\P\big(X_N\geq x\big)}_{=0,\text{ für diese x}}\d x\\
\overset{\eqref{eq5.3Vor}}&{\leq}
p\cdot\int\limits_0^N x^{p-2}\cdot\E\left[Y\cdot\indi_{\lbrace X_N\geq n\rbrace}\right]\d x\\
\overset{\text{Fubini}}&=
p\cdot\E\left[Y\cdot\int\limits_0^N x^{p-2}\cdot\indi_{\lbrace X_N\geq x\rbrace}\d x\right]\\
&=p\cdot\E\left[Y\cdot\int\limits_{[0,N]\cap(-\infty,X_N]} x^{p-2}\d x\right]\\
&\stackeq{X_N\leq N}p\cdot\E\left[Y\cdot\int\limits_0^{X_N} x^{p-2}\d x\right]\\
&=\frac{p}{p-1}\cdot\E\left[Y\cdot X_N^{p-1}\right]\\
\end{align*}
\begin{align*}
\implies
\E\left[X_N^p\right]
&\leq \frac{p}{p-1}\cdot\E\left[Y\cdot X_N^{p-1}\right] \\
&\stackrel{\text{Hölder}}{\leq}
\frac{p}{p-1}\cdot\E\left[Y^p\right]^{\frac{1}{p}}\cdot\E\left[X_N^{(p-1)q}\right]^{\frac{1}{q}}\\
&\left(\frac{1}{p} + \frac{1}{q} = 1 \implies \frac{1}{q} = \frac{p-1}{p}\right) \\
&=\frac{p}{p-1}\cdot\E\left[Y^p\right]^{\frac{1}{p}}\cdot\E\left[X_N^{p}\right]^{\frac{p-1}{p}}
\end{align*}
\begin{align*}
\implies
\underbrace{\E\left[X_N^p\right]^{\frac{1}{p}}}_{=\Vert X_N\Vert_p}
=\E\left[X_N^p\right]^{\frac{p}{p}-\frac{p-1}{p}}
\leq
\frac{p}{p-1}\cdot\underbrace{\E\left[Y^p\right]^{\frac{1}{p}}}_{=\Vert Y\Vert_p}
\end{align*}
Mit Fatou folgt dann:
\begin{align*}
\Vert X\Vert_p\leq\liminf\limits_{N\to\infty}\Vert X_N\Vert_p\leq\frac{p}{p-1}\cdot\Vert Y\Vert_p
\end{align*}
\end{proof}

\begin{proof}[Beweis von Theorem \ref{theorem5.2DoobsLpUngleichung}]\enter
$(M_n)_{n\in\N}$ ist Martingal oder positives Submartingal. Folglich ist $\big(|M_n|\big)_{n\in\N}$ ein Submartingal. Aus der Maximalungleichung \ref{theorem5.1DoobMaximalungleichung} (Zwischenabschätzung) folgt:
\begin{align*}
\P\Big(\underbrace{\max\limits_{0\leq j\leq n}|M_j|}_{=M_n^\ast}\geq\lambda\Big)
&\leq\lambda\cdot\E\Big[|M_n|\cdot\indi_{\Big\lbrace\underbrace{\max\limits_{0\leq j\leq n}|M_j|}_{=M_n^\ast}\geq\lambda\Big\rbrace}\Big]
\end{align*}
Dies entspricht der Voraussetzung \eqref{eq5.3Vor} in Lemma \ref{lemma5.3} mit $X=M_n^\ast$, $Y=|M_n|$. Somit folgt aus Lemma \ref{lemma5.3}:
\begin{align*}
\big\Vert M_n^\ast\big\Vert_p\leq\frac{p}{p-1}\cdot\big\Vert M_n\big\Vert_p
\end{align*}
\end{proof}

\begin{theorem}[Azumas Ungleichung]\label{theorem5.4AzumasUngleichung}\enter
Sei $(X_n)_{n\in\N}$ ein Martingal mit beschränkten Zuwächsen, also $\big|X_n-X_{n-1}\big|\leq c_n$ für deterministische Folge $(c_n)_{n\in\N}\subseteq\R_{\geq0}$. Dann gilt:
\begin{align*}
\P\Big(\big|X_n-\E[X_n]\big|\geq\lambda\Big)\leq2\cdot\exp\left(-\frac{\lambda^2}{2\cdot\sum\limits_{i=1}^n c_i^2}\right)\qquad\forall\lambda\geq0
\end{align*}
Hierbei ist die rechte Seite der Ungleichung proportional zur Dichte von $\mathcal{N}\left(0,\sum\limits_{i=1}^n c_i^2\right)$. Man spricht auch von ``Subgaussian tails''.
\end{theorem}
\begin{proof}
Betrachte die Funktion
\begin{align*}
f:\R\to\R,\qquad f_\alpha(x):=\exp(\alpha\cdot x)\qquad\forall x,\alpha\in\R
\end{align*}
%TODO Hier könnte man eine Skizze einfügen. 
Offenbar ist $f_\alpha$ konvex, d.h. für alle $c>0$ liegt $\graph(f_\alpha)$ unter dem Liniensegment $s$ von $\big(-c,f_\alpha(-c)\big)$ nach $\big(c,f_\alpha(c)\big)$. Explizit:
\begin{align*}
f_\alpha(x)\leq s_\alpha(x)\qquad\forall |x|\leq c\qquad\mit s_\alpha(x)=\frac{f_\alpha(c)-f_\alpha(-c)}{2\cdot c}\cdot x+\frac{f_\alpha(c)+f_\alpha(-c)}{2}
\end{align*}
Jetzt setzen wir $f_\alpha$ ein und erhalten:
\begin{align*}
\exp(\alpha\cdot x)
&\leq\frac{x}{2\cdot c}\cdot\big(\exp(\alpha\cdot c)-\exp(-\alpha\cdot c)\big)+\frac{1}{2}\cdot\big(\exp(\alpha\cdot c)+\exp(-\alpha\cdot c)\big)\quad\forall |x|\leq c
\end{align*}
Einsetzen: $\big(X_n-X_{n-1}\big)$ für $x$ und $c_n$ für $c$ und Bilden des Erwartungswertes $\E[\cdot~|~\F_{n-1}]$ liefert:
\begin{align*}
\E\Big[\exp(\alpha\cdot(X_n-X_{n-1})~\Big|~\F_{n-1}\Big]
&\leq
\frac{\overbrace{\E\big[X_n-X_{n-1}~\big|~\F_{n-1}\big]}^{\stackeq{\text{MG}}0}}{2\cdot c_n}\cdot\big(\exp(\alpha\cdot c_n)-\exp(-\alpha\cdot c_n)\big)\\
&\qquad+\frac{1}{2}\cdot\big(\exp(\alpha\cdot c_n+\exp(-\alpha\cdot c_n)\big)
\end{align*}
\begin{align*}
\implies
\E\Big[\exp(\alpha\cdot(X_n-X_{n-1})~\Big|~\F_{n-1}\Big]
&\leq \frac{1}{2}\cdot\big(\exp(\alpha\cdot c_n)+\exp(-\alpha\cdot c_n)\big)\\
&=\frac{1}{2}\left(\sum\limits_{k=0}^\infty\frac{(\alpha\cdot c_n)^{k}}{k!}+\sum\limits_{k=0}^\infty\frac{(-\alpha\cdot c_n)^{k}}{k!}\right)\\
&=\sum\limits_{k=0}^\infty\frac{(\alpha\cdot c_n)^{2\cdot k}}{(2\cdot k)!}\\
&\stackrel{2^k\cdot k! \leq (2\cdot k)!}{\leq}\sum\limits_{k=0}^\infty\frac{(\alpha\cdot c_n)^{2\cdot k}}{2^k\cdot k!}\\
&=\exp\left(\frac{\alpha^2\cdot c_n^2}{2}\right)
\end{align*}
Wir wollen es Iterieren und zeigen dafür einen Schritt:
\begin{align*}
&\E\Big[\exp\big(\alpha\cdot(X_n-X_{n-2})\big)~\Big|~\F_{n-2}\Big]\\
&=\E\Big[\exp\big(\alpha\cdot(X_n-X_{n-1})\big)\cdot\exp\big(\alpha\cdot(X_{n-1}-X_{n-2})\big)~\Big|~\F_{n-2}\Big]\\
\overset{\eqref{Turmregel}}&=
\E\Big[\E\big[\exp(\alpha\cdot(X_n-X_{n-1}))\cdot\exp\big(\alpha\cdot(X_{n-1}-X_{n-2})\big)~\big|~\F_{n-1}\big]~\Big|~\F_{n-2}\Big]\\
\overset{\F_{n-1}\text{-m.b.}}&=
\E\Big[\underbrace{\E\big[\exp(\alpha\cdot(X_n-X_{n-1}))~\big|~\F_{n-1}\big]}_{\leq\exp\left(\frac{\alpha^2\cdot c_n^2}{2}\right)}\cdot\exp\big(\alpha\cdot(X_{n-1}-X_{n-2})\big)~\Big|~\F_{n-2}\Big]\\
&\leq
\exp\left(\frac{\alpha^2}{2}\cdot\left(c_n^2+c_{n-1}^2\right)\right)
\end{align*}
Iterieren und Erwartungswert bilden liefert:
\begin{align*}
\E\Big[\exp\big(\alpha\cdot(X_n-X_0)\big)\Big]
&= \E\Big[\E\left[\exp\big(\alpha\cdot(X_n-X_0)\big)~|~F_0\right]\Big] \\
&\leq\exp\Big(\frac{\alpha^2}{2}\cdot\underbrace{\sum\limits_{k=1}^n c_k^2}_{=D_n^2}\Big)\qquad\forall\alpha\in\R
\end{align*}
\begin{align*}
\P(X_n-X_0\geq\lambda)
\overset{\alpha\geq0}&=
\P\Big(\exp\big(\alpha\cdot(X_n-X_0)\big)\geq\exp(\alpha\cdot\lambda)\Big)\\
\overset{\text{Markov}}&{\leq}
\exp(-\alpha\cdot\lambda)\cdot\E\Big[\exp\big(\alpha\cdot(X_n-X_0)\big)\Big]\\
&\leq
\exp\Big(\underbrace{-\alpha\cdot\lambda+\frac{\alpha^2}{2}\cdot D_n^2}_{=:\rho(\alpha)}\Big)\qquad\forall\alpha\geq0
\end{align*}
Suche beste (d.h. kleinste) Schranke der Ableitung von $\rho$:
\begin{align*}
\rho'(\alpha)=-\lambda+\alpha\cdot D_n^2\implies\alpha_\ast=\frac{\lambda}{D_n^2}\geq0
\end{align*}
Einsetzen:
\begin{align*}
\P\big(X_n-X_0\geq\lambda\big)
\leq\exp\left(-\frac{\lambda^2}{2\cdot D_n^2}\right)
\end{align*}
Analog erhält man:
\begin{align*}
\P\big(X_n-X_0\leq-\lambda\big)
\leq\exp\left(-\frac{\lambda^2}{2\cdot D_n^2}\right)
\end{align*}
Insgesamt erhält man also die zweiseitige Abschätzung:
\begin{align*}
\P\big(|X_n-X_0|\geq\lambda\big)
&\leq
\P\big(X_n-X_0\geq\lambda\big) +
\P\big(X_n-X_0\leq-\lambda\big) \\
&\leq2\cdot\exp\left(-\frac{\lambda^2}{2\cdot D_n^2}\right)
\end{align*}
Beachte $X_0=\E\left[X_n\right]$.
\end{proof}

\setcounter{section}{1}
\section{Rückwärtsmartingale (RWN)} %5.2
\begin{defi}
	Sei $(\G_n)_{n\in\N}$ eine \textbf{absteigende} (d.h. $\G_{n}\subseteq\G_{n-1}~\forall n\in\N$) Folge von $\sigma$-Algebren.\\
Ein stochastischer Prozess $(R_n)_{n\in\N}$ heißt \textbf{Rückwärtsmartingal (RWN)} bzgl. $(\G_n)_{n\in\N}$, falls folgende Eigenschaften gelten:
\begin{enumerate}
\item $R_n$ ist $\G_n$-messbar $\forall n\in\N$
\item $\begin{aligned}
\E\big[|R_n|\big]<\infty\qquad\forall n\in\N
\end{aligned}$
\item $\begin{aligned}
R_n=\E\big[R_k~\big|~\G_n\big]\qquad\forall k\leq n
\end{aligned}$
\end{enumerate} 
\end{defi}

\begin{bemerkung}
Setze $M_{-n}=R_n$  und $\F_{-n}=\G_n$ für alle $n\in\N$.\\
Dann ist $\big(M_k,\F_k\big)_{k\in-\N}$ ein Martingal mit Indexmenge $I=-\N$.
\end{bemerkung}

\begin{theorem}[Konvergenz von RWMs]\label{theorem5.5KonvergenzVonRWMs}\enter
Sei $(R_n)_{n\in\N}$ ein RWM bzgl. $(\G_n)_{n\in\N}$ und 
\begin{align*}
\G_\infty:=\bigcap\limits_{k=1}^\infty\G_k
\end{align*}
die \textbf{terminale $\sigma$-Algebra}. Dann gilt:
\begin{align*}
	\exists R_\infty\in L_1\big(\G_\infty\big)\colon\limn R_n=R_\infty \text{ und } \big\Vert R_n-R_\infty\big\Vert_{L_1}\stackrel{n\to\infty}{\longrightarrow}0
\end{align*}
\end{theorem}

\begin{bemerkung}
Dieses Theorem hat keine Voraussetzungen! RMWs konvergieren immer!
\end{bemerkung}

\begin{proof}
	Verwende Martingal $(M_n)_{n\in\N}$. Seien $U_n [ a,b]$ die Upcrossings von $[ a,b]$ durch $(M_k)_{k\in\lbrace-N,\ldots,0\rbrace}$. Doobs Upcrossing Lemma \ref{lemma4.1DoobsUpcrossingLemma} sagt:
\begin{align*}
	\E\Big[U_n[ a,b]\Big]
&\leq
\frac{1}{b-a}\cdot\E\Big[\big(M_0-a\big)^+\Big]<\infty
\end{align*}
Mit monotoner Konvergenz erhält man mit $N\to\infty$:
\begin{align*}
\E\Big[U[a,b]\Big]\leq\frac{1}{b-a}\cdot\E\Big[\big(M_0-a\big)^+\Big]<\infty
\end{align*}
Analog zur Vorwärtskonvergenz:
\begin{align*}
\exists M_{-\infty}\text{ mit Werten in }\overline{\R}:\limn M_{-n}=M_{-\infty}\text{ f.s.}
\end{align*}
\ul{Außerdem:} (mit RWM-Eigenschaft und $k=0$)
\begin{align*}
M_{-n}
&=\E\big[M_0~\big|~\F_{-n}\big]\\
&\implies \big(M_{-n}\big)_{n\in\N}\text{ ist ggi}\\
&\implies M_{-\infty}\in L_1\big(\F_{-\infty}\big)\text{ und }\big\Vert M_{-n}-M_{-\infty}\big\Vert_1\stackrel{n\to\infty}{\longrightarrow}0\\
&\text{d.h. }R_\infty\in L_1(\G_\infty)\text{ und }\big\Vert R_n-R_\infty\big\Vert_1\stackrel{n\to\infty}{\longrightarrow}0
\end{align*}
\end{proof}

