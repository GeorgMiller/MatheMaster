\chapter{Bedingter Erwartungswert}
\section{Bedingtet Erwartungswert als $L_2$-Projektion}
Betrachte den Wahrscheinlichkeitsraum $(\Omega,\A,\P)$.\\
Für Zufallsvariable $X:\Omega\to\R$ und $p\in[1,\infty)$ definiere die $L_p$-Norm
\begin{align*}
\Vert X\Vert_p:=\E\left[|X|^p\right]=\left(\int\limits_\Omega|X(\omega)|^p\d\P(\omega)\right)^{\frac{1}{p}}
\end{align*}
und die Räume
\begin{align*}
\mathcal{L}_p(\Omega,\A,\P):=\Big\lbrace X:\Omega\to\R\Big|X\text{ ist $\A$-messbar und }\Vert X\Vert_p<\infty\Big\rbrace
\end{align*}.
Aufgrund der Minkowski-Ungleichung
\begin{align*}
\Vert X+Y\Vert_p\leq\Vert X\Vert_p+\Vert Y\Vert_p
\end{align*}
und der Homogenität
\begin{align*}
\Vert c\cdot X\Vert_p=c\cdot\Vert X\Vert_p\qquad\forall c\geq0
\end{align*}
ist 
\begin{align*}
\mathcal{L}_p(\Omega,\A,\P)
\end{align*}
Vektorraum mit Halbnorm $\Vert\cdot\Vert_p$. Es fehlt die Definitheit.\\
Wir identifizieren Zufallsvariablen $X,\tilde{X}$, welche $\P$-fast sicher übereinstimmen, d. h. $\P[X\neq\tilde{X}]=0$. Formal betrachten wir den Unterraum
\begin{align*}
\mathcal{N}:=\lbrace N:\Omega\to\R:N=0\text{ $\P$-fast sicher}\rbrace
\end{align*}
und bilden den Quotientenraum
\begin{align*}
L_p(\Omega,\A,\P):=\mathcal{L}_p(\Omega,\A,\P)/\mathcal{N}=\left\lbrace[X+\mathcal{N}]:X\in\mathcal{L}_p(\Omega,\A,\P)\right\rbrace.
\end{align*}
Wir schreiben auch kurz $L_p(\A)$ oder $L_p(\P)$, wenn wir Abhängigkeit von $\A$ oder $\P$ betonen wollen.\\
Aus der Maßtheorie ist bekannt:

\begin{theorem}
Sei $p\in[1,\infty)$. Dann ist $L_p(\Omega,\A,\P)$ mit Norm $\Vert\cdot\Vert_p$ ein Banachraum.\\
Für $p=2$ ist $L_2(\Omega,\A,\P)$ ein Hilbertraum mit Skalarprodukt
\begin{align*}
\langle X,Y\rangle:=\E[X\cdot Y]=\int\limits_\Omega X(\omega)\cdot Y(\omega)\d\P(\omega)
\end{align*}
\end{theorem}

\begin{bemerkung}
Zwei Zufallsvariablen $X,Y\in L_2$ heißen \textbf{orthogonal} $:\gdw\langle X,Y\rangle=0$.
\end{bemerkung}

\begin{proposition}
Sei $\mathcal{F}\subseteq\A$ eine Unter-$\sigma$-Algebra von $\A$ und $p\in[1,\infty)$.\\
Dann ist $L_p(\Omega,\mathcal{F},\P)$ ein abgeschlossener Unterraum von $L_p(\Omega,\A,\P)$.

\end{proposition}

\begin{proof}
RobertToDo
\end{proof}

\begin{defi}[Bedingte Erwartung in $L_2$]\enter
Sei $\mathcal{F}\subseteq\A$ eine Unter-$\sigma$-Algebra von $\A$.\\
Jedes $X\in L_2(\Omega,\mathcal{A},\P)$ hat eine eindeutige Orthogonalprojektion $Y$ auf $L_2(\Omega,\mathcal{F},\P)$. Diese heißt \textbf{bedingte Erwartung} von $X$ bzgl. $\mathcal{F}$ und wir schreiben $\E[X~|~\mathcal{F}]:=Y$.\\
Die bedingte Erwartung ist also eine Zufallsgröße und nur bis auf $\P$-Nullmengen eindeutig bestimmt.
\end{defi}

\begin{bemerkung}
Als Orthogonalprojektion gilt
\begin{align*}
\Vert X-\E[X~|~\mathcal{F}]\Vert_2=\inf\left\lbrace\Vert X-Y\Vert_2:Y\in L_2(\Omega,\mathcal{F},\P)\right\rbrace.
\end{align*}
Interpretation: $\E[X~|~\mathcal{F}]$ ist die beste Näherung für $X$ durch Zufallsvariablen\\ $Y\in L_2(\Omega,\mathcal{F},\P)$.
\end{bemerkung}

\begin{proposition}
$Y$ ist die Orthogonalprojektion von $X$ auf $L_2(\Omega,\mathcal{F},\P)\\\Longleftrightarrow\forall F\in\mathcal{F}\in L_2(\mathcal{F}):\langle X-Y,F\rangle=0$
\end{proposition}
\begin{proof}
RobertToDo
\end{proof}

\begin{proposition}[Eigenschaften der bedingten Erwartung]\
Seien $X,Y\in L_2(\Omega,\A,\P)$ und $\mathcal{F}\subseteq\A$ Unter-$\sigma$-Algebra von $\A$. Dann gilt:
\begin{enumerate}
\item $X\in L_2(\mathcal{F})\Longrightarrow\E[X~|~\mathcal{F}]=X$
\item $\E[a\cdot X+b\cdot Y~|~\mathcal{F}]=a\cdot\E[X~|~\mathcal{F}]+b\cdot\E[Y~|~\mathcal{F}]~\forall a,b\in\R$ ``Linearität''
\item $\langle\E[X~|~\mathcal{F}],Y\rangle
=\langle X,\E[Y~|~\mathcal{F}]\rangle
=\langle\E[X~|~\mathcal{F}],\E[Y~|~\mathcal{F}]\rangle$ ``Symmetrie''
\item Für jede Unter-$\sigma$-Algebra $\mathcal{H}\subseteq\mathcal{F}$ von $\mathcal{F}$ gilt die \textbf{Turmregel / tower law}:
\begin{align}\label{Turmregel}
\E\big[\E[X~|~\mathcal{F}]~\big|~\mathcal{H}\big]=\E[X~|~\mathcal{H}]
\end{align}
\item $\E[Z\cdot X~|~\mathcal{F}]=Z\cdot\E[X~|~\mathcal{F}]\qquad\forall Z$ beschränkt und $\mathcal{F}$-messbar ``Pull-out-property''
\item $X\leq Y\Longrightarrow\E[X~|~\mathcal{F}]\leq\E[Y~|~\mathcal{F}]$ ``Monotonie''
\item $\big|\E[X~|~\mathcal{F}]\big|\leq\E\big[|X|~\big|~\mathcal{F}\big]$ ``Dreiecksungleichung''
\end{enumerate}
\end{proposition}

\begin{proof}
Nächste Vorlesung
\end{proof}

