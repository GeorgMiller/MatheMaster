% This work is licensed under the Creative Commons
% Attribution-NonCommercial-ShareAlike 4.0 International License. To view a copy
% of this license, visit http://creativecommons.org/licenses/by-nc-sa/4.0/ or
% send a letter to Creative Commons, PO Box 1866, Mountain View, CA 94042, USA.

\chapter{Bedingter Erwartungswert}
\section{Bedingter Erwartungswert als \texorpdfstring{$L_2$}{L\_2}-Projektion}
Betrachte den Wahrscheinlichkeitsraum $(\Omega,\A,\P)$.\\
Für Zufallsvariable $X:\Omega\to\R$ und $p\in[1,\infty)$ definiere die $L_p$-Norm
\begin{align*}
\Vert X\Vert_p:=\E\left[|X|^p\right]=\left(\int\limits_\Omega|X(\omega)|^p\d\P(\omega)\right)^{\frac{1}{p}}
\end{align*}
und die Räume
\begin{align*}
\mathcal{L}_p(\Omega,\A,\P):=\Big\lbrace X:\Omega\to\R\Big|X\text{ ist $\A$-messbar und }\Vert X\Vert_p<\infty\Big\rbrace.
\end{align*}
Aufgrund der Minkowski-Ungleichung
\begin{align*}
\Vert X+Y\Vert_p\leq\Vert X\Vert_p+\Vert Y\Vert_p
\end{align*}
und der Homogenität
\begin{align*}
\Vert c\cdot X\Vert_p=c\cdot\Vert X\Vert_p\qquad\forall c\geq0
\end{align*}
ist 
\begin{align*}
\mathcal{L}_p(\Omega,\A,\P)
\end{align*}
Vektorraum mit Halbnorm $\Vert\cdot\Vert_p$. Es fehlt die Definitheit.\\
Wir identifizieren Zufallsvariablen $X,\tilde{X}$, welche $\P$-fast sicher übereinstimmen, d. h. 
\begin{align*}
\P[X\neq\tilde{X}]=0.
\end{align*}
Formal betrachten wir den Unterraum
\begin{align*}
\mathcal{N}:=\lbrace N:\Omega\to\R:N=0\text{ $\P$-fast sicher}\rbrace
\end{align*}
und bilden den Quotientenraum
\begin{align*}
L_p(\Omega,\A,\P):=\mathcal{L}_p(\Omega,\A,\P)/\mathcal{N}=\left\lbrace[X+\mathcal{N}]:X\in\mathcal{L}_p(\Omega,\A,\P)\right\rbrace.
\end{align*}
Wir schreiben auch kurz $L_p(\A)$ oder $L_p(\P)$, wenn wir Abhängigkeit von $\A$ oder $\P$ betonen wollen.\\
Aus der Maßtheorie ist bekannt:

\begin{theorem}
Sei $p\in[1,\infty)$. Dann ist $L_p(\Omega,\A,\P)$ mit Norm $\Vert\cdot\Vert_p$ ein Banachraum.\\
Für $p=2$ ist $L_2(\Omega,\A,\P)$ ein Hilbertraum mit Skalarprodukt
\begin{align*}
\langle X,Y\rangle:=\E[X\cdot Y]=\int\limits_\Omega X(\omega)\cdot Y(\omega)\d\P(\omega)
\end{align*}
\end{theorem}

\begin{bemerkung}
Zwei Zufallsvariablen $X,Y\in L_2$ heißen \textbf{orthogonal} $:\gdw\langle X,Y\rangle=0$.
\end{bemerkung}

\begin{proposition}
Sei $\mathcal{F}\subseteq\A$ eine Unter-$\sigma$-Algebra von $\A$ und $p\in[1,\infty)$.\\
Dann ist $L_p(\Omega,\mathcal{F},\P)$ ein abgeschlossener Unterraum von $L_p(\Omega,\A,\P)$.

\end{proposition}

\begin{proof}
	Es gilt:
	\[ X \text{ ist } \A-\text{messbar} \implies X \text{ ist } \F-\text{messbar}\]
	Daraus können wir folgende Inklusion schlussfolgern:
	\[\mathcal{L}_p(\Omega,\F,\P) \subseteq \mathcal{L}_p(\Omega,\A,\P)\]
	$\mathcal{L}_p(\Omega,\F,\P)$ selbst ist Banachraum und damit abgeschlossener Unterraum.
\end{proof}

\begin{defi}[Bedingte Erwartung in $L_2$]
Sei $\mathcal{F}\subseteq\A$ eine Unter-$\sigma$-Algebra von $\A$.\nl
Jedes $X\in L_2(\Omega,\mathcal{A},\P)$ hat eine eindeutige Orthogonalprojektion $Y$ auf $L_2(\Omega,\mathcal{F},\P)$. Diese heißt \textbf{bedingte Erwartung} von $X$ bzgl. $\mathcal{F}$ und wir schreiben 
\begin{align*}
\E[X~|~\mathcal{F}]:=Y.
\end{align*}
Die bedingte Erwartung ist also eine Zufallsgröße und nur bis auf $\P$-Nullmengen eindeutig bestimmt.
\end{defi}

\begin{bemerkung}
Als Orthogonalprojektion gilt
\begin{align*}
\Vert X-\E[X~|~\mathcal{F}]\Vert_2=\inf\left\lbrace\Vert X-Y\Vert_2:Y\in L_2(\Omega,\mathcal{F},\P)\right\rbrace.
\end{align*}
Interpretation: $\E[X~|~\mathcal{F}]$ ist die beste Näherung für $X$ durch Zufallsvariablen\\ $Y\in L_2(\Omega,\mathcal{F},\P)$.
\end{bemerkung}

\begin{proposition}\label{Prop1.3}
	Folgende Aussagen sind äquivalent:
	\renewcommand{\labelenumi}{(\alph{enumi})}
	\begin{enumerate}
		\item $Y$ ist die Orthogonalprojektion von $X$ auf $L_2(\Omega,\mathcal{F},\P)$ 
		\item $\forall F\in\mathcal{F}\in L_2(\mathcal{F}):\langle X-Y,F\rangle=0$
	\end{enumerate}
\end{proposition}
\begin{proof}
	$\dq (a) \implies (b) \dq{}$ \enter
	Wähle $F \in L_2(\F)$ beliebig. Dann gilt
	\[Y + tF \in L_2(\F) \qquad \forall t \in \R\]
	Der Abstand wird für $t=0$ minimiert, also gilt:
\[\|X-(Y+tF)\|_2^2 \geq \|X-Y\|_2^2 \qquad \forall t \in \R\]
Nun nehmen wir an, dass $t>0$ auf der rechten Seite gilt und multiplizieren die Norm als Produkt von Skalarprodukten aus.
\begin{align*}
	&&\|X-(Y+tF)\|_2^2 &\geq \|X-Y\|_2^2 \\
	\implies	&&\|X-Y\|_2^2 + t^2\|F\|_2^2 - 2t\langle X-Y, F \rangle &\geq \|X-Y\|_2^2 \\
	\implies	&&t^2\|F\|_2^2 - 2t\langle X-Y, F \rangle &\geq 0 \\
	\implies	&&t\|F\|_2^2 - 2\langle X-Y, F \rangle &\geq 0 \\
\end{align*}
Für $t\rightarrow0$ erhalten wir.
\[\langle X-Y, F \rangle \leq 0\]
Analog erhält man mit der Annahme $t<0$ die umgekehrte Ungleichung.
Also folgt
\[\langle X-Y, F \rangle = 0 \qquad \forall \F \in L_2(\F)\]
da $F$ beliebig gewählt war.\nl
Jetzt: $\dq (b) \implies (a) \dq{}$\\
Sei $F\in L_2(\F)$ und $\langle X-Y, F \rangle = 0$. Dann
\[\implies \|X-Y\|_2^2 + \underbrace{t^2\|F\|_2^2}_{\geq 0} - \underbrace{2t\langle X-Y, F \rangle}_{=0} \geq \|X-Y\|_2^2\]
Und damit 
\[\|X-(Y+tF)\|_2^2 \geq \|X-Y\|_2^2 \qquad \forall t\in\R, \forall F\in L_2(\F)\]
Somit ist $\|X-Y\|_2^2$ Minimierer der rechten Seite, in Zeichen:
\[\|X-Y\|_2^2 \leq \inf\{\|X-\tilde{F}\|_2^2 : \tilde{F}\in L_2(\F)\}\]
und damit ist $Y$ die orthogonale Projektion von $X$ auf $\F$
\end{proof}

\begin{proposition}[Eigenschaften der bedingten Erwartung]\enter\label{Prop1.4}
Seien $X,Y\in L_2(\Omega,\A,\P)$ und $\mathcal{F}\subseteq\A$ Unter-$\sigma$-Algebra von $\A$. Dann gilt:
\begin{enumerate}[label=(\alph*)]
\item $X\in L_2(\mathcal{F})\Longrightarrow\E[X~|~\mathcal{F}]=X$
\item $\E[a\cdot X+b\cdot Y~|~\mathcal{F}]=a\cdot\E[X~|~\mathcal{F}]+b\cdot\E[Y~|~\mathcal{F}]~\forall a,b\in\R$ ``Linearität''
\item $\langle\E[X~|~\mathcal{F}],Y\rangle
=\langle X,\E[Y~|~\mathcal{F}]\rangle
=\langle\E[X~|~\mathcal{F}],\E[Y~|~\mathcal{F}]\rangle$ ``Symmetrie''
\item Für jede Unter-$\sigma$-Algebra $\mathcal{H}\subseteq\mathcal{F}$ von $\mathcal{F}$ gilt die \textbf{Turmregel / tower law}:
\begin{align}\label{Turmregel}\tag{Tow}
\E\big[\E[X~|~\mathcal{F}]~\big|~\mathcal{H}\big]=\E[X~|~\mathcal{H}]
\end{align}
\item $\E[Z\cdot X~|~\mathcal{F}]=Z\cdot\E[X~|~\mathcal{F}]\qquad\forall Z$ beschränkt und $\mathcal{F}$-messbar ``Pull-out-property''
\item $X\leq Y\Longrightarrow\E[X~|~\mathcal{F}]\leq\E[Y~|~\mathcal{F}]$ ``Monotonie''
\item $\big|\E[X~|~\mathcal{F}]\big|\leq\E\big[|X|~\big|~\mathcal{F}\big]$ ``Dreiecksungleichung''
\end{enumerate}
\end{proposition}

\begin{proof}
\underline{Zu (a):} Folgt direkt aus der Definition.\nl
\underline{Zu (b):} Folgt aus Proposition \ref{Prop1.3}:
\begin{align*}
\overset{\text{Prop }\ref{Prop1.3}}&{\implies}
\left.
\begin{matrix}
\langle X-\E[X~|~\mathcal{F}],F\rangle=0\\
\langle Y-\E[Y~|~\mathcal{F}],F\rangle=0
\end{matrix}
\right\rbrace\forall F\in L_2(\Omega,\mathcal{F},\P)\\
&\implies
a\cdot\langle X-\E[X~|~\mathcal{F}],F\rangle+b\cdot\langle Y-\E[Y~|~\mathcal{F}],F\rangle=0\qquad&\forall F\in L_2(\Omega,\mathcal{F},\P)\\
\overset{\text{Bilinear}}&{\implies}
\Big\langle a\cdot X+b\cdot Y-\big(a\cdot\E[X~|~\mathcal{F}]+b\cdot \E[Y~|~\mathcal{F}]\big),F\Big\rangle=0\qquad&\forall F\in L_2(\Omega,\mathcal{F},\P)\\
\overset{\text{Prop }\ref{Prop1.3}}&{\implies}
\E[a\cdot X+b\cdot Y~|~\mathcal{F}]=a\cdot \E[X~|~\mathcal{F}]+b\cdot \E[Y~|~\mathcal{F}]
\end{align*}

\underline{Zu (c):} Aus Proposition \ref{Prop1.3} folgt wieder
\begin{align}
\langle X-\E[X~|~\mathcal{F}],\E[Y~|~\mathcal{F}]\rangle=0\label{proofProp114_1}\\
\langle Y-\E[Y~|~\mathcal{F}],\E[X~|~\mathcal{F}]\rangle=0\label{proofProp114_2}
\end{align}
und damit
\begin{align*}
\langle X,\E[Y~|~\mathcal{F}]\rangle
&=\langle \E[X~|~\mathcal{F}]+(X-\E[X~|~\mathcal{F}]),\E[Y~|~\mathcal{F}]\rangle\\
\overset{\ref{proofProp114_1}}&=
\langle\E[X~|~\mathcal{F}],\E[Y~|~\mathcal{F}]\rangle\\
&=\langle\E[X~|~\mathcal{F}],Y+(\E[Y~|~\mathcal{F}]-Y)\rangle\\
\overset{\ref{proofProp114_2}}&=
\langle\E[X~|~\mathcal{F}],Y\rangle.
\end{align*}

\underline{Zu (d):} 
Aus Proposition 1.3 folgt wieder:
\begin{align*}
\overset{\text{Prop }\ref{Prop1.3}}&{\implies}
\langle X-\E[X~|~\mathcal{F}],F\rangle
&\forall& F\in L_2(\mathcal{F})\supseteq L_2(\mathcal{H})\\
&\implies
\langle \E[X~|~\mathcal{F}],Y+(\E[Y~|~\mathcal{F}-Y)]\rangle=0
&\forall& H\in L_2(\mathcal{H})\\
&\implies
\Big\langle X-\E\big[\E[X~|~\mathcal{F}]~|~\mathcal{H}\big],H\Big\rangle=0\qquad
&\forall& H\in L_2(\mathcal{H})\\
\overset{\text{Prop }\ref{Prop1.3}}&{\implies}
\E[X~|~\mathcal{H}]=\E\big[\E[X~|~\mathcal{F}]~|~\mathcal{H}\big]
\end{align*}

\underline{Zu (e):} 
\begin{align*}
\big\langle X\cdot Z-\E[X~|~\mathcal{F}]\cdot Z,F\big\rangle
&=\E\Big[\big(X\cdot Z-\E[X~|~\mathcal{F}]\cdot Z\big)~|~\mathcal{F}\Big]\\
&=\big\langle X-\E[X~|~\mathcal{F}],Z\cdot F\big\rangle=0\quad\forall F\in L_2(\mathcal{F})\text{ da }Z\cdot F\in L_2(\mathcal{F})\\
&\Longrightarrow\E[X\cdot Z~|~\mathcal{F}]
=Z\cdot \E[X~|~\mathcal{F}]
=Z\cdot\E[X~|~\mathcal{F}]
\end{align*}

\underline{Zu (f):} Sei $X\geq0$. Setze
\begin{align*}
A:=\lbrace\omega\in\Omega:\E[X~|~\mathcal{F}](\omega)<0\rbrace\in\mathcal{F}.
\end{align*}
Außerdem gilt für die Indikatorfunktion $\indi_A\in L_2(\mathcal{F})$.
Einerseits gilt
\begin{align*}
\E[X\cdot\indi_A~|~\mathcal{F}]\geq0\text{ weil }X\geq0
\end{align*}
und andererseits
\begin{align*}
\E[X\cdot\indi_A]
&=\E\left[\E[X~|~\mathcal{F}]\cdot\indi_A\right]\\
&=\E\left[\E[X~|~\mathcal{F}]\cdot\indi_{\lbrace\E[X~|~\mathcal{F}]<0\rbrace}\right]\\
&=\int\limits_{\lbrace\E[X~|~\mathcal{F}]<0\rbrace} \E[X~|~\mathcal{F}]\d\P\leq0\\
&\Longrightarrow\E[X\cdot\indi_A]=0\\
&\Longrightarrow\int\limits_{\lbrace\E[X~|~\mathcal{F}]<0\rbrace} \E[X~|~\mathcal{F}]\d\P=0\\\\
&\Longrightarrow \P(\E[X~|~\mathcal{F}]<0)=0\\
&\Longrightarrow\E[X~|~\mathcal{F}]\geq0\text { fast sicher}
\end{align*}
Allgemeine Aussage folgt mit $\tilde{X}:=Y-X$ und aus der Linearität.\nl
\underline{Zu (g):}
\begin{align*}
\pm X\leq|X|
\stackrel{6.}{\Longrightarrow}
\pm\E\big[X~|~\mathcal{F}]\leq\E[|X|~|~\mathcal{F}\big]\\
\Longrightarrow\Big|\E[X~|~\mathcal{F}]\Big|\leq\E\Big[|X|~|~\mathcal{F}\Big]
\end{align*}
\end{proof}
 
\section{Bedingte Erwartung in \texorpdfstring{$L_1$}{L\_1}}
\setcounter{section}{1}
Prinzip: stetige Fortsetzung.

\setcounter{satz}{5} %Nr. 1.5 fehlt wurde ausgelassen vom Prof.
\begin{proposition}\label{Prop1.6} %1.6
Sei $\mathcal{F}\subseteq\A$ eine Unter-$\sigma$-Algebra von $\A$.\enter
Die bedingte Erwartung hat eine eindeutige stetige Fortsetzung von $L_2(\Omega,\F,\P)$ auf $L_1(\Omega,\A,\P)$. Diese bezeichnen wir ebenfalls mit $\E[~\cdot~|~\mathcal{F}]$.
\end{proposition}
\begin{proof}
\underline{Existenz:}\\
Sei $X\in L_1$. Dann existiert eine Approximationsfolge $(X_n)_{n\in\N}\subseteq L_2$ mit 
\[\E\big[|X_n-X|\big]\stackrel{n\to\infty}{\longrightarrow}0\] 
(in Zeichen $X_n\stackrel{L_1}{\longrightarrow} X$).\\
Wähle z. B.
\begin{align*}
X_n:=\left\lbrace\begin{array}{cl}
X, & \falls |X|\leq n\\
n, & \falls X>n\\
-n, & \falls X<-n
\end{array}\right.
\end{align*}
Mit dominanter Konvergenz gilt 
\begin{align*}
\limn\E[|X_n-X|]=\E\left[\limn|X_n-X|\right]=0.
\end{align*}
Mit Kontraktionseigenschaft gilt:
\[\E\Big[\big|\E[X_m~|~\F]-\E[X_n~|~\F]\big|\Big]
	\leq\E\Big[\E\big[|X-M-X_n|\big]~\big|~\F\Big]
\stackeq{\text{Turm}}~
\E\big[|X_n-X_n|\big]
\stackrel{m,n\to\infty}{\longrightarrow}
0\]
\begin{align*}
&\Longrightarrow\left(\E[X_n~|~\F]\right)_{n\in\N}\text{ ist Cauchy-Folge in }L_2(\Omega,\A,\P)\\
&\Longrightarrow\exists Z\in L_1(\Omega,\A,\P)\mit\E[X_n~|~\F]\stackrel{L_1}{\longrightarrow} Z=:\E[X~|~\F]
\end{align*}
\underline{Eindeutigkeit:} Sei $(\tilde{X}_n)_{n\in\N}$ eine weitere Approximationsfolge für $X$, d.h. es gilt auch
\begin{align*}
\E[|\tilde{X}_n-X|]
\stackrel{n\to\infty}{\longrightarrow}
0.
\end{align*}
Dann gilt:
\begin{align*}
\Vert Z-\E[\tilde{X}_n~|~\F]\Vert_1
&=\Big
\Vert Z-\E[X_n~|~\F]+\E[X_n~|~\F]-\E[\tilde{X}_n~|~\F]\Big\Vert_1\\
\overset{\Delta\text{Ungl}}&{\leq}
\big\Vert Z-\E[X_n~|~\F]\big\Vert_1+\underbrace{\big\Vert\E[X_n~|~\F]-\E[\tilde{X}_n~|~\F]\big\Vert_1}_{=\E[|\E[X_n~|~\F-\E[\tilde{X}_n~|~\F|]}\\
\overset{\text{Kontr}}&{\leq}
\Vert Z-\E[X_n~|~\F]\Vert_1+\Vert X_n-\tilde{X}_n\Vert_1\\
&\leq\Vert Z-\E[X_n~|~\F]\Vert_1+\Vert X-X_n\Vert_1+\Vert X-\tilde{X}_n\Vert_1\\
&\stackrel{n\to\infty}{\longrightarrow}
0\\
&\Longrightarrow\E[\tilde{X}_n~|~\F]\stackrel{L_1}{\longrightarrow} Z
\end{align*}
Also ist der Limes unabhängig von der Approximationsfolge.
\end{proof}

\begin{korollar} %1.7
Alle Eigenschaften aus Proposition \ref{Prop1.4} (außer Symmetrie) gelten weiterhin für alle $X,Y\in L_1(\Omega,\A,\P)$.
\end{korollar}
\begin{proof}
Beweis durch Approximation.
\end{proof}

\begin{theorem}\label{theorem1.8}%1.8
Sei $\F$ Unter-$\sigma$-Algebra von $\A$ und seien
\begin{align*}
X\in L_1(\Omega,\A,\P),\qquad Y\in L_1(\Omega,\F,\P).
\end{align*}
Dann sind äquivalent:
\begin{enumerate}[label=(\alph*)]
\item $Y=\E[X~|~\F]$ fast sicher
\item $\E[X\cdot\indi_F]=\E[Y\cdot\indi_F]\qquad\forall F\in\F$\\
(Beachte $X,Y\in L_2\Longrightarrow\langle X-Y,\indi_F\rangle=0$)
\end{enumerate}
\end{theorem}
\begin{proof}
\underline{Zeige $(a)\Longrightarrow (b)$:}\\
Sei $(X_n)_{n\in\N}\subseteq L_2(\A)$ eine Approximationsfolge für $X$, d.h. $X_n\stackrel{L_1}{\longrightarrow} X$. Mit Proposition \ref{Prop1.6} folgt:
\begin{align*}
\E[X_n~|~\F]
&\stackrel{L_1}{\longrightarrow}
\E[X~|~\F]=Y\\
\E\left[(X-Y)\cdot\indi_F\right]
&=
\limn\E[(X_n-\E[X_n~|~\F]),\indi_F]\\
&=
\limn\langle X_n-\E[X_n~|~\F],\indi_F\rangle\\
\overset{\text{Prop }\ref{Prop1.3}}&{=}
0\qquad\forall F\in\F
\end{align*}
\underline{Zeige $(b)\Longrightarrow (a)$:}\\
Setze 
\begin{align*}
F^{+}&:=\left\lbrace\E[X~|~\F]-Y>0\right\rbrace\in\F\\
F^{-}&:=\left\lbrace-(\E[X~|~\F]-Y)>0\right\rbrace\in\F.
\end{align*}
Dann gilt:
\begin{align*}
0
\leq
\E\Big[\underbrace{\big(\E[X~|~\F]-Y\big)\cdot\indi_{F^+}}_{\geq0}\Big]
&=
\E\Big[\E[X~|~\F]\cdot\indi_{F^+}\Big]-\E\left[Y\cdot\indi_{F^+}\right]\\
\overset{
\begin{subarray}{c}
	\text{Pull-out} \\ \text{Turm}
\end{subarray}
}&=
\E[X\cdot\indi_{F^+}]-\E[Y\cdot\indi_{F^+}]\\
\overset{\text{(b)}}&=
0\\
\implies\P(F^+)&=0
\end{align*}
Analog erhält man $\P(F^-)=0$. Also folgt insgesamt $\E[X~|~\F]=Y$ fast sicher.
\end{proof}

\begin{bemerkung}
Die ``unbedingte Erwartung'' $\E[X]$ können wir als Spezialfall der bedingten Erwartung $\E[X~|~\F]$ für die triviale $\sigma$-Algebra $\F:=\lbrace\emptyset,\Omega\rbrace\subseteq\A$ auffassen.\\
Denn es gilt:
\begin{align*}
\E[X\cdot\indi_\Omega]&=\E[X]=\E\big[\E[X]\cdot\indi_\Omega\big]\\
\E[X\cdot\indi_\emptyset]&=0=\E\big[\E[X]\cdot\indi_\emptyset\big]\\
\stackrel{\text{Theorem \ref{theorem1.8}}}{\Longrightarrow}
\E[X]&=\E[X~|~\F]\text{, da $\F$ trivial.}
\end{align*}
Hier wird $\E[X]\in\R$ als konstante Zufallsgröße in $L_1(\Omega,\F,\P)$ aufgefasst. Damit gilt dann die Gleichung
\begin{align*}
\E\big[\E[X~|~\mathcal{G}]\big]=\E[X]\qquad\forall \mathcal{G}\subseteq\A\text{ Unter-$\sigma$-Algebra}
\end{align*}
denn:
\begin{align*}
\E\big[\E[X~|~\mathcal{G}]\big]
&=\E\big[\E[X~|~\mathcal{G}]~|~\lbrace\emptyset,\Omega\rbrace\big]\\
\overset{\eqref{Turmregel}}&=
\E\big[X~|~\lbrace\emptyset,\Omega\rbrace\big]\\
&=\E[X]
\end{align*}
\end{bemerkung}

\section*{Bedingter Erwartungungswert und Unabhängigkeit} %keine Nummer
Sei $B_b(\R):= \{f:\R\rightarrow\R ~|~f \text{ beschränkt und Borel-messbar}\}$
\begin{defi}
Sei $X:\Omega\to\R$ eine Zufallsvariable auf $(\Omega,\A,\P)$ und $\F\subseteq\A$ Unter-$\sigma$-Algebra von $\A$. Dann heißt $X$ \textbf{unabhängig} von $\F$, in Zeichen $X\unab\F$
\begin{align}\label{sigmaAlgebraUnabhaengig}
:\Longleftrightarrow
\E[f(X)\cdot\indi_A]
=
\E[f(X)]\cdot\P(A)\qquad\forall A\in\F\text{ und $f\in B_b(\R)$}
\end{align}
$X$ heißt \textbf{unabhängig} von einer Zufallsvariablen $Y$, in Zeichen $X\unab Y$ wenn eine der folgenden äquivalenten Bedingungen gilt:
\begin{enumerate}[label=(\alph*)]
\item $X\unab\sigma(Y)$
\item 
$\begin{aligned}
\E[f(X)\cdot g(Y)]
=
\E[f(X)]\cdot\E[g(Y)]\qquad\forall f,g\in B_b(\R)
\end{aligned}$
\end{enumerate}
\end{defi}

\begin{theorem} %1.9
Sei $X\in L_1(\Omega,\A,\P)$ unabhängig von $\F$. Dann gilt:
\begin{align*}
\E[X~|~\F]=\E[X]
\end{align*}
\end{theorem}

\begin{proof}
Wegen $X\unab\F$ gilt:
\begin{align*}
\forall F\in\F:\E[X\cdot\indi_X]
\overset{\ref{sigmaAlgebraUnabhaengig}}&=
\E[X]\cdot\P(F)
=\E\big[\E[X]\cdot\indi_F\big]
\stackrel{\text{Theorem \ref{theorem1.8}}}{\Longrightarrow} \E[X]=\E[X~|~\F]
\end{align*}
\end{proof}

\begin{bemerkung}
Merke die beiden Extremfälle:
\begin{itemize}
\item Wenn $X$ $\F$-messbar ist, dann ist $\E[X~|~\F]=X$.\\
Die bedingte Erwartung verändert also $X$ nicht.
\item Wenn $X$ von $\F$ unabhängig ist, dann ist $\E[X~|~\F]=\E[X]$.\\
Die bedingte Erwartung reduziert $X$ auf den unbedingten Erwartungswert $\E[X]$.
\end{itemize}
\end{bemerkung}

\setcounter{section}{2}
\section{Spezialfälle und weitere Eigenschaften} %1.3
\setcounter{section}{1}
\setcounter{satz}{9}

\textbf{Bedingen auf Zufallsvariablen}:\\
Wir nehmen nun an, dass $\F$ von einer oder mehreren Zufallsvariablen $Y$ bzw. $Y_1,\ldots, Y_n$ erzeugt wird, d.h.
\begin{align*}
\F=\sigma(Y)\text{ oder }\F=\sigma(Y_1,\ldots Y_n).
\end{align*}
In diesen Fällen schreiben wir 
\begin{align*}
\E[X~|~Y]&:=\E[X~|~\sigma(Y)]\\
\E[X~|~Y_1,\ldots,Y_n]&:=\E\big[X~|~\sigma(Y_1,\ldots,Y_n)\big]
\end{align*}
Die beiden Fälle stimmen überein, wenn wir $Y$ als Zufallsvektor $Y=(Y_1,\ldots,Y_n)$ definieren.\nl
\textbf{Interpretation / Bedeutung:}\\
Beste Vorhersage von $X$ gegeben $Y$. Dies ist in vielen Anwendungen relevant, z. B. $X$ sei das Einkommen einer zufällig ausgewählten Person und $Y$ der höchste Bildungsabschluss dieser Person.\nl
Wir betrachten: $(X,Y)$ sei $\R^m\times\R^n$-wertige Zufallsvariable mit gemeinsamer Dichte $f_{XY}(x,y)$.
\begin{itemize}
\item
$ f_Y(y)=\int\limits_{\R^m}f_{XY}(x,y)\d x $ ist die \textbf{Randverteilung} (Dichte) von $Y$
\item 
$\begin{aligned}
S_Y:=\big\lbrace y\in\R^n:f_Y(y)>0\big\rbrace\subseteq\R^n
\end{aligned}$ \textbf{Träger} von $Y$
\end{itemize}

\begin{defi} %noNumber
\textbf{Die bedingte Dichte} von $X$ bzw. $Y$ ist gegeben durch 
\begin{align*}
f_{X|Y}(x,y):=\left\lbrace\begin{array}{cl}
\frac{f_{XY}(x,y)}{f_Y(y)}, &\falls y\in S_y\\
0, &\falls y\not\in S_y
\end{array}\right.
\end{align*}
\end{defi}

\begin{theorem} %1.10
Sei $(X,Y)$ eine $\R^m\times\R^n$-wertige Zufallsvariable mit Dichte $f_{XY}$.\\
Dann gilt für alle messbaren Funktionen $h:\R^m\times\R^n\to\R\mit\E\big[|h(X,Y)|\big]<\infty$:
\begin{align*}
\E\big[ h(X,Y)~\big|~Y\big]=\int\limits_{\R^m} h(x,y)\cdot f_{X|Y}(x,Y)\d x
\end{align*}
\end{theorem}

\begin{bemerkung}\
\begin{itemize}
\item Insbesondere gilt
\begin{align*}
\E[X~|~Y]=\int\limits_{\R^m} x\cdot f_{X|Y}(x,Y)\d x
\end{align*}
\item Für $y\in S_Y$ schreiben wir auch
\begin{align*}
\E\big[h(X,Y)~\big|~Y=y\big]:=\int\limits_{\R^m} h(x,y)\cdot f_{X|Y}(x,y)\d x
\end{align*}
Die bedingte Erwartung ist also nun ``punktweise''  als messbare Funktion $S_Y\to\R$ definiert.
\end{itemize}
\end{bemerkung}
\begin{proof}[Beweis von Theorem 1.10]
Setze
\begin{align*}
g(y):=\int\limits_{\R^m} h(x,y)\cdot f_{X|Y}(x,y)\d x.
\end{align*}
Offenbar ist $g:\R^n\to\R$ messbar. Zu zeigen:
\begin{align*}
	\E\big[h(X,Y)~\big|~\sigma(Y)\big]=g(Y)
\end{align*}
Mit Theorem \ref{theorem1.8} ist dies äquivalent zu
\begin{align*}
\E\big[ h(X,Y)\cdot\indi_F\big]=\E\big[g(Y)\cdot\indi_F\big]\qquad\forall F\in\sigma(Y)
\end{align*}
Jedes $F\in\sigma(Y)$ lässt sich darstellen als 
\begin{align*}
F=\big\lbrace\omega\in\Omega:Y(\omega)\in B\big\rbrace\mit B\in\B(\R^n)
\end{align*}
D. h. ist äquivalent zu
\begin{align*}
\int\limits_{\R^m\times\R^n} h(x,y)\cdot\indi_{\lbrace y\in B\rbrace}\cdot f_{XY}(x,y)\d (x,y)
&=
\int\limits_{\R^n} g(y)\cdot\indi_{\lbrace y\in B\rbrace}\cdot f_Y(y)\d y\qquad\forall B\in\B(\R^n)
\end{align*}
Doch diese Gleichheit ist leicht zu zeigen:
\begin{align*}
\text{rechte Seite }&=\int\limits_{\R^n}\int\limits_{\R^m} h(x,y)\cdot f_{X|Y}(x,y)\d x\cdot\indi_{\lbrace y\in B\rbrace} \cdot f_Y(y)\\
\overset{\text{Fubini}}&=
\int\limits_{\R^m\times\R^n} h(x,y)\cdot\indi_{\lbrace y\in B\rbrace}\cdot\underbrace{f_{X|Y}(x,y)\cdot f_Y(y)}_{=f_{XY}(x,y)}\d (x,y)\\
&=\text{ linke Seite}
\end{align*}
\end{proof}

Insbesondere gilt: Sei $m=1$. Das Minimierungsproblem 
\begin{align*}
\min\left\lbrace\E\Big[\big(X-g(Y)\big)^2\Big]:g:\R^n\to\R\text{  messbar}\right\rbrace
\end{align*}
wird gelöst durch
\begin{align*}
g(y)=\E[X~|~Y=y]=\int\limits_{\R^m} x\cdot f_{X|Y}(x,y)\d x.
\end{align*}

\begin{beisp}
Sei $(X,Y)$ normalverteilt auf $\R^2$ mit Mittelwert
\begin{align*}
\mu=\begin{pmatrix}
\mu_x\\ \mu_y
\end{pmatrix}\text{ und Kovarianzmatrix }
\Sigma=\begin{pmatrix}
\sigma_x^2 & \rho\sigma_x\sigma_y\\
\rho\sigma_x\sigma_y & \sigma_y^2
\end{pmatrix}\mit|\rho|<1
\end{align*}
Was ist $f_{X|Y}$?
\begin{enumerate}
\item Normierter Fall: $\mu_x=\mu_y=0$ und $\sigma_x=\sigma_y=1$.
\begin{align*}
\implies \mu=\begin{pmatrix}
0\\ 0
\end{pmatrix},\quad\Sigma=\begin{pmatrix}
1 & \rho\\
\rho & 1
\end{pmatrix}
\end{align*}
Nebenrechnung:
\begin{align*}
\det(\Sigma)&=1-\rho^2\qquad\Sigma^{-1}=\frac{1}{1-\rho^2}\cdot\begin{pmatrix}
1 & -\rho\\
-\rho & 1
\end{pmatrix}\\
f_{XY}(x,y)&=\frac{1}{2\cdot\pi\cdot\det(\Sigma)}\cdot\exp\left(-\frac{1}{2}\cdot\begin{pmatrix}
x\\ y
\end{pmatrix}^T\cdot\Sigma^{-1}\cdot\begin{pmatrix}
x\\ y
\end{pmatrix}\right)\\
&=\frac{1}{2\cdot\pi\cdot(1-\rho^2)}\cdot\exp\left(-\frac{1}{2\cdot(1-\rho)^2}\cdot\big(x^2-2\cdot\rho\cdot x\cdot y+y^2\big)\right)\\
f_Y(y) &=\frac{1}{\sqrt{2\cdot\pi}}\cdot\exp\left(-\frac{y^2}{2}\right)\\
f_{X|Y}(x,y)&=\frac{f_{XY}(x,y)}{f_Y(y)}\\
&=\frac{1}{\sqrt{2\cdot\pi}\cdot(1-\rho^2)}\\
&\qquad\cdot\exp\Bigg(-\frac{1}{2\cdot(1-\rho^2)}\cdot\big(\underbrace{x^2-2\cdot\rho\cdot x\cdot y+y^2-(1-\rho^2)\cdot y^2}_{=x^2-2\cdot\rho\cdot x\cdot y+\rho^2\cdot y^2}\big)\Bigg)\\
&=\frac{1}{\sqrt{2\cdot\pi}\cdot(1-\rho^2)}\cdot\exp\left(-\frac{1}{2\cdot(1-\rho^2)}\cdot(x-\rho\cdot y)^2\right)
\end{align*}
Letzteres ist die Dichte von $\mathcal{N}(\rho\cdot y,1-\rho^2)$. Insbesondere gilt
\begin{align*}
\E[X~|~Y=y]&=\rho\cdot y\\
\Var(X~|~Y=y)=1-\rho^2.
\end{align*}
\item Allgemeiner Fall:
\begin{align*}
\begin{matrix}
X=\sigma_x\cdot\tilde{X}+\mu_x\\
Y=\sigma_y\cdot\tilde{Y}+\mu_y
\end{matrix}\mit(\tilde{X},\tilde{Y})\text{ normiert in Fall 1.}
\end{align*}
Damit gilt:
\begin{align*}
f_{XY}(x,y) 
&=\frac{1}{\sigma_x\cdot\sigma_y}\cdot f_{\tilde{X}\tilde{Y}}\left(\frac{x-\mu_x}{\sigma_x},~\frac{y-\mu_y}{\sigma_y}\right)\\
f_Y(y) &= \frac{1}{\sigma_y}\cdot f_{\tilde{Y}}\left(\frac{y-\mu_y}{\sigma_y}\right)
\end{align*}
Die Bedingte Dichte ist:
\begin{align*}
&\quad f_{X|Y}(x,y) \\
&=\frac{f_{XY}(x,y)}{f_Y(y)}
=\frac{1}{\sigma_x}\cdot f_{\tilde{X}|\tilde{Y}}\left(\frac{x-\mu_x}{\sigma_x},~\frac{y-\mu_y}{\sigma_y}\right)\\
&=\frac{1}{\sqrt{2\cdot\pi}\cdot(1-\rho^2)}\cdot\exp\left(-\frac{1}{2\cdot(1-\rho^2)}\cdot\left(\frac{x-\mu_x}{\sigma_x}-\frac{\rho}{\sigma_y}\cdot(y-\mu_y)\right)^2\right)\\
&=\frac{1}{\sqrt{2\cdot\pi}\cdot(1-\rho^2)}\cdot\exp\left(-\frac{1}{2\cdot\sigma_x^2\cdot(1-\rho^2)}\cdot\left(x-\left(\mu_x+\frac{\sigma_x\cdot\rho}{\sigma_y}\cdot(y-\mu_y)\right)\right)^2\right)
\end{align*}
Dies ist die Dichte von
\begin{align*}
\mathcal{N}\left(\mu_y+\frac{\sigma_x\cdot\rho}{\sigma_y}\cdot(y-\mu_y),~\sigma_x^2\cdot(1-\rho^2)\right)
\end{align*}
Insbesondere gilt
\begin{align*}
\E[X~|~Y=y]&=\mu_x+\frac{\sigma_x\cdot\rho}{\sigma_y}\cdot(y-\mu_y)\\
\Var(X~|~Y=y)&=\sigma_x^2\cdot(1-\rho^2)
\end{align*}
\end{enumerate}
\end{beisp}

\begin{bemerkung}
Die Funktion
\begin{align*}
y\mapsto\E[X~|~Y=y]=\mu_x+\frac{\sigma_x\cdot\rho}{\sigma_y}\cdot(y-\mu_y)
\end{align*}
heißt \textbf{Regressionsgerade} für $X$ gegeben $Y$.
%Abbilung fehlt. RobertToDo
\end{bemerkung}

\subsection*{Bedingte Wahrscheinlichkeit}
Seien $A,B\in\A$ Ereignisse. Die \textbf{elementare bedingte Wahrscheinlichkeit} ist 
\begin{align*}
\P(A~|~B)=\frac{\P(A\cap B)}{\P(B)}\text{ für }\P(B)>0
\end{align*}
Was ist der Zusammenhang zur bedingten Erwartung?\nl
Beachte: Die erzeugte $\sigma$-Algebra 
\begin{align*}
\sigma(B)=\big\lbrace\emptyset,B,B^C,\Omega\big\rbrace.
\end{align*}
Was ist nun 
\begin{align*}
\P\big(A~\big|~\sigma(B)\big)=\E\big[\indi_A~|~\sigma(B)\big]\text{ ?}
\end{align*}
\begin{align*}
&L_2\big(\sigma(B)\big)=\Big\lbrace \alpha\cdot\indi_B+\beta\cdot\indi_{B^C}:\alpha,\beta\in\R\Big\rbrace\\
&\implies
Y=\E\big[\indi_A~\big|~\sigma(B)\big]=a\cdot\indi_B+b\cdot\indi_{B^C}\mit a,b\in\R\text{ zu bestimmen}\\
&\P(A\cap B)=\E[\indi_A\cdot\indi_B]
\stackeq{\text{Thm \ref{theorem1.8}}}
\E[Y\cdot\indi_B]=\E[a\cdot\indi_B]=a\cdot\P(B)\\
&\implies
a=\frac{\P(A\cap B)}{\P(B)}=\P(A~|~B)\\
&\P(A\cap B^C)=\E[\indi_A\cdot\indi_{B^C}]
\stackeq{\text{Thm \ref{theorem1.8}}}
\E[Y\cdot\indi_{B^C}]=b\cdot\P(B^C)\\
&\implies
b=\frac{\P(A\cap B^C)}{\P(B^C)}=\P(A~|~B^C)\\
&\implies
\P(A~|~\sigma(B))=\P(A~|~B)\cdot\indi_B+\P(A~|~B^C)\cdot\indi_{B^C}
\end{align*}

Analog: Seien $B_1,\ldots,B_n\in\A$ disjunkte Ereignisse mit
\begin{align*}
\bigcupdot\limits_{j=1}^n B_j=\Omega.
\end{align*}
Dann gilt:
\begin{align*}
\P\big(A~\big|~\sigma(B_1,\ldots,B_n)\big)=\sum\limits_{j=1}^n\P(A~|~B_j)\cdot\indi_{B_j}
\end{align*}
Mit der Turmregel \eqref{Turmregel} erhalten wir das \textbf{Gesetz der totalen Wahrscheinlichkeit}:
\begin{align*}
\P(A)=\E[\indi_A]=\sum\limits_{j=1}^n \P(A~|~B_j)\cdot\P(B_j)
\end{align*}

Analog zur bedingten Wahrscheinlichkeit:
\begin{defi}[Erwartungswert bedingt auf Ereignis]\enter %noNumber
Sei $X\in L_1(\Omega,\A,\P)$ und $B\in\A$. Dann definiere für $\P(B)>0$:
\begin{align*}
\E[X~|~B]:=\frac{\E[X\cdot\indi_B]}{\P(B)}
\end{align*}
\end{defi}

Insgesamt haben wir definiert:\\
\begin{tabular}{p{3cm}|c|c}
	&Wahrscheinlichkeit & Erwartungswert \\
	& eines Ereignisses & einer Zufallsariable\\
	& $A\in\A$ 					& $X$ \\ \hline
	&& \\
	$\begin{array}{c}\text{Bedingen auf}\\ \sigma\text{-Algebra } \F\end{array}$ &$\P(A~|~\F)$&$\E[X~|~\F]$ siehe 1.1 und 1.2 \\
												&& \\ \hline
	&& \\
	$\begin{array}{c}\text{Bedingen auf} \\ \text{Zufallvar. } Y\end{array}$ &---&$\E[X~|~Y]~~~(\ast)$ \\
											&& \\ \hline
	&& \\
	$\begin{array}{c} \text{Bedingen auf} \\ \text{Ereignis } B \end{array}$ &$\P(A|B)=\frac{\P(A\cap B)}{\P(B)}$&$\E[X|B]=\frac{\E[X\indi_B]}{\P(B)}$ \\
											&&
\end{tabular}\enter
$(\ast)$: Falls eine Dichte existiert: $\E[X~|~Y=y]$

\begin{beisp}[Lebensdaueranalyse]\enter
Die Lebensdauer $X:\Omega\to[0,\infty)$ eines Radios sei mit Dichte $f(z)$ verteilt.\\
Wie ist die mittlere Restlaufzeit nach $a$ Jahren, d. h. 
\begin{align*}
\E[X-a~|~X\geq a]\qquad?
\end{align*}
Ausrechnen:
\begin{align*}
\E[X-a~|~X\geq a] 
&=\frac{\E\left[(X-a)\cdot\indi_{\{X\geq a\}}\right]}{\P(X\geq a)}\\
&=\frac{\int\limits_a^\infty (x-a)\cdot f(x)\d x}{\int\limits_a^\infty f(x)\d x}\\
&=\frac{\int\limits_a^\infty x\cdot f(x)\d x}{\int\limits_a^\infty f(x)\d x}-a
\end{align*}
Sei nun $X\sim\text{Exp}(\lambda)$, d. h. $f(x)=\lambda\cdot\exp(-\lambda\cdot x)$. Dann ist:
\begin{align*}
\E[X-a~|~X\geq a]
&=\frac{\int\limits_a^\infty \lambda\cdot x\cdot\exp(-\lambda\cdot x)\d x}{\int\limits_a^\infty\lambda\cdot\exp(-\lambda\cdot x)}-a\\
\text{Zähler}&=\int\limits_0^\infty\exp(-\lambda\cdot x)\d x=-\frac{\exp(-\lambda\cdot x)}{x}\Big|_{x=a}^\infty=\frac{\exp(-\lambda\cdot a)}{\lambda}\\
\text{Nenner} &=\int\limits_a^\infty x\cdot\exp(-\lambda\cdot x)\d x\\
&=\frac{x\cdot\exp(-\lambda\cdot x)}{-\lambda}\Big|_{x=a}^\infty+\int\limits_a^\infty\frac{\exp(-\lambda\cdot x)}{\lambda}\d x\\
&=\frac{a\cdot\exp(-\lambda\cdot a)}{\lambda}+\frac{\exp(-\lambda\cdot a)}{\lambda^2}\\
\implies
\E[X-a~|~X\geq a]&=\frac{\exp(-\lambda\cdot a)\cdot\left(\frac{a}{\lambda}+\frac{1}{\lambda^2}\right)}{\exp(-\lambda\cdot a)\cdot\frac{1}{\lambda}}-a=a+\frac{1}{\lambda}-a=\frac{1}{\lambda}\\
\implies
\E[X-a~|~X\geq a]&=\E[X]
\end{align*}
Somit die die \textbf{Gedächtnislosigkeit} der Exponentialverteilung gezeigt.
\end{beisp}
