\chapter{Bedingter Erwartungswert}
\section{Bedingter Erwartungswert als $L_2$-Projektion}
Betrachte den Wahrscheinlichkeitsraum $(\Omega,\A,\P)$.\\
Für Zufallsvariable $X:\Omega\to\R$ und $p\in[1,\infty)$ definiere die $L_p$-Norm
\begin{align*}
\Vert X\Vert_p:=\E\left[|X|^p\right]=\left(\int\limits_\Omega|X(\omega)|^p\d\P(\omega)\right)^{\frac{1}{p}}
\end{align*}
und die Räume
\begin{align*}
\mathcal{L}_p(\Omega,\A,\P):=\Big\lbrace X:\Omega\to\R\Big|X\text{ ist $\A$-messbar und }\Vert X\Vert_p<\infty\Big\rbrace
\end{align*}.
Aufgrund der Minkowski-Ungleichung
\begin{align*}
\Vert X+Y\Vert_p\leq\Vert_p+\Vert Y\Vert_p
\end{align*}
und der Homogenität
\begin{align*}
\Vert c\cdot X\Vert_p=c\cdot\Vert X\Vert_p\qquad\forall c\geq0
\end{align*}
ist $\mathcal{L}_p(\Omega,\A,\P)$ Vektorraum mit Halbnorm $\Vert\cdot\Vert_p$. Es fehlt die Definitheit.\\
Wir identifizieren Zufallsvariablen $X,\tilde{X}$, welche $\P$-fast sicher übereinstimmen, d. h. $\P[X\neq\tilde{X}]=0$. Formal betrachten wir den Unterraum
\begin{align*}
\mathcal{N}:=\lbrace N:\Omega\to\R:N=0\text{ $\P$-fast sicher}\rbrace
\end{align*}
und bilden den Quotientenraum
\begin{align*}
L_p(\Omega,\A,\P):=\mathcal{L}_P(\Omega,\A,\P)/\mathcal{N}=\left\lbrace[X+\mathcal{N}]:X\in\mathcal{L}_p(\Omega,\A,\P)\right\rbrace.
\end{align*}
Wir schreiben auch kurz $L_p(\A)$ oder $L_p(\P)$, wenn wir Abhängigkeit von $\A$ oder $\P$ betonen wollen.\\
Aus der Maßtheorie ist bekannt:

\begin{theorem}
Sei $p\in[1,\infty)$. Dann ist $L_p(\Omega,\A,\P)$ mit Norm $\Vert\cdot\Vert_p$ ein Banachraum.\\
Für $p=2$ ist $L_2(\Omega,\A,\P)$ ein Hilbertraum mit Skalarprodukt
\begin{align*}
\langle X,Y\rangle:=\E[X\cdot Y]=\int\limits_\Omega X(\omega)\cdot Y(\omega)\d\P(\omega)
\end{align*}
\end{theorem}

\begin{bemerkung}
Zwei Zufallsvariablen $X,Y\in L_2$ heißen \textbf{orthogonal} $:\gdw\langle X,Y\rangle=0$.
\end{bemerkung}

\begin{proposition}
Sei $\mathcal{F}\subseteq\A$ enie Unter-$\sigma$-Algebra von $\A$ und $p\in[1,\infty)$.\\
Dann ist $L_p(\Omega,\mathcal{F},\P)$ ein abgeschlossener Unterraum von $L_p(\Omega,\A,\P)$.

\end{proposition}

\begin{proof}
RobertToDo
\end{proof}

\begin{defi}[Bedingte Erwartung in $L_2$]\enter
Sei $\mathcal{F}\subseteq\A$ eine Unter-$\sigma$-Algebra von $\A$.\\
Jedes $X\in L_2(\Omega,\mathcal{A},\P)$ hat eine eindeutige Orthogonalprojektion $Y$ auf $L_2(\Omega,\mathcal{F},\P)$. Diese heißt \textbf{bedingte Erwartung} von $X$ bzgl. $\mathcal{F}$ und wir schreiben $\E[X~|~\mathcal{F}]:=Y$.\\
Die bedingte Erwartung ist also eine Zufallsgröße und nur bis auf $\P$-Nullmengen eindeutig bestimmt.
\end{defi}

\begin{bemerkung}
Als Orthogonalprojektion gilt
\begin{align*}
\Vert X-\E[X~|~\mathcal{F}]\Vert_2=\inf\left\lbrace\Vert X-Y\Vert_2:Y\in L_2(\Omega,\mathcal{F},\P)\right\rbrace.
\end{align*}
Interpretation: $\E[X~|~\mathcal{F}]$ ist die beste Näherung für $X$ durch Zufallsvariablen $Y\in L_2(\Omega,\mathcal{F},\P)$.
\end{bemerkung}

\begin{proposition}
$Y$ ist die Orthogonalprojektion von $X$ auf $L_2(\Omega,\mathcal{F},\P)\\\Longleftrightarrow\forall F\in\mathcal{F}\in L_2(\mathcal{F}):\langle X-Y,F\rangle=0$
\end{proposition}
\begin{proof}
RobertToDo
\end{proof}

\begin{proposition}[Eigenschaften der bedingten Erwartung]\
Seien $X,Y\in L_2(\Omega,\A,\P)$ und $\mathcal{F}\subseteq\A$ Unter-$\sigma$-Algebra von $\A$. Dann gilt:
\begin{enumerate}
\item $X\in L_2(\mathcal{F})\Longrightarrow\E[X~|~\mathcal{F}]=X$
\item $\E[a\cdot X+b\cdot Y~|~\mathcal{F}]=a\cdot\E[X~|~\mathcal{F}]+b\cdot\E[Y~|~\mathcal{F}]~\forall a,b\in\R$ ``Linearität''
\item $\langle\E[X~|~\mathcal{F}],Y\rangle
=\langle X,\E[Y~|~\mathcal{F}]\rangle
=\langle\E[X~|~\mathcal{F}],\E[Y~|~\mathcal{F}]\rangle$ ``Symmetrie''
\item Für jede Unter-$\sigma$-Algebra $\mathcal{H}\subseteq\mathcal{F}$ von $\mathcal{F}$ gilt die \textbf{Turmregel / tower law}:
\begin{align}\label{Turmregel}
\E\big[\E[X~|~\mathcal{F}]~\big|~\mathcal{H}\big]=\E[X~|~\mathcal{H}]
\end{align}
\item $\E[Z\cdot X~|~\mathcal{F}]=Z\cdot\E[X~|~\mathcal{F}]\qquad\forall Z$ beschränkt und $\mathcal{F}$-messbar ``Pull-out-property''
\item $X\leq Y\Longrightarrow\E[X~|~\mathcal{F}]\leq\E[Y~|~\mathcal{F}]$ ``Monotonie''
\item $\big|\E[X~|~\mathcal{F}]\big|\leq\E\big[|X|~\big|~\mathcal{F}\big]$ ``Dreiecksungleichung''
\end{enumerate}
\end{proposition}

\begin{proof}
\underline{Zu 1.:} Klar.\\

\underline{Zu 2.:} Aus Proposition 1.1.3 folgt
\begin{align*}
%\left.
\begin{matrix}
\langle X-\E[X~|~\mathcal{F}],F\rangle\\
\langle Y-\E[Y~|~\mathcal{F}],F\rangle
\end{matrix}
%\right\rbrace\forall F\in L_2(\Omega,\mathcal{F},\P)
%\\
%\Longrightarrow\langle a\cdot X+b\cdot Y-(a\cdot\E[X~|~\mathcal{F}]+b\cdot \E[Y~|~\mathcal{F}],F\rangle=0~\forall F\in L_2(\Omega,\mathcal{F},\P)\\
%\stackrel{Prop 1.3}{\Longrightarrow}\E[a\cdot X+b\cdot Y~|~\mathcal{F}]=a\cdot \E[X~|~\mathcal{F}]+b\cdot \E[Y~|~\mathcal{F}]
\end{align*}
\underline{Zu 3.:} Aus Proposition 1.3 folgt wieder
\begin{align}
\langle X-\E[X~|~\mathcal{F}],\E[Y~|~\mathcal{F}]\rangle\label{proofProp114_1}\\
\langle Y-\E[Y~|~\mathcal{F}],\E[X~|~\mathcal{F}]\rangle\label{proofProp114_2}
\end{align}
Und damit
\begin{align*}
\langle X,\E[Y~|~\mathcal{F}]\rangle
&=\langle \E[X~|~\mathcal{F}]+(X-\E[X~|~\mathcal{F}]),\E[Y~|~\mathcal{F}]\rangle\\
\stackeq{\ref{proofProp114_1}}
\langle\E[X~|~\mathcal{F}],\E[Y~|~\mathcal{F}]\rangle\\
&=\langle\E[X~|~\mathcal{F}],Y+(\E[Y~|~\mathcal{F}]-Y)\rangle\\
&\stackeq{\ref{proofProp114_2}}
\langle\E[X~|~\mathcal{F}],Y\rangle
\end{align*}
\underline{Zu 4.:} 
Aus Proposition 1.3 folgt wieder
\begin{align}
\langle X-\E[X~|~\mathcal{F}],F\rangle\qquad\forall F\in L_2(\mathcal{F})\supseteq L_2(\mathcal{H})\\
\langle \E[X~|~\mathcal{F}],Y+(\E[Y~|~\mathcal{F}-Y)]\rangle\qquad\forall H\in L_2(\mathcal{H})\\
\Longrightarrow\langle X-\E[\E[X~|~\mathcal{F}]~|~\mathcal{H}],H\rangle=\qquad\forall H\in L_2(\mathcal{H})\\
\stackrel{Prop 1.3}{\Longrightarrow}\E[X~|~\mathcal{H}]=\E[\E[X~|~\mathcal{F}]~|~\mathcal{H}]
\end{align}
\underline{Zu 5.:} 
\begin{align*}
broken
%\langle X\cdot Z-\E[X~|~\mathcal{F}]\cdot Z,F\rangle
%&=\E[(X\cdot Z-\E[X~|~\mathcal{F}]\cdot Z)~|~\mathcal{F}]\\
%&=\langle X-\E[X~|~\mathcal{F}],Z\cdot F\rangle=0\qquad\forall F\in L_2(\mathcal{F})\text{ da auch }Z\cdot F\in L_2(\mathcal{F})\\
%&\Longrightarrow\E[X\cdot Z~|~\mathcal{F}]
%=Z\cdot \E[X~|~\mathcal{F}]
%=Z\cdot\E[X~|~\mathcal{F}]
\end{align*}
\underline{Zu 6.:} Sei $X\geq0$. Setze
\begin{align*}
A:=\lbrace\omega\in\Omega:\E[X~|~\mathcal{F}]<0\rbrace\in\mathcal{F}.
\end{align*}
Außerdem ist gilt für die Indikatorfunktion $\indi_A\in L_2(\mathcal{F})$.
Einerseits gilt
\begin{align*}
\E[X\cdot\indi_A~|~\mathcal{F}]\geq0\text{ weil }X\geq0
\end{align*}
und andererseits
\begin{align*}
\E[X\cdot\indi_A]
&=\E[\E[X~|~\mathcal{F}]\cdot\indi_A]\\
&=\E\left[\E[X~|~\mathcal{F}]\cdot\indi_{\lbrace\E[X~|~\mathcal{F}]<0\rbrace}\right]\\
&=\int\limits_{\lbrace\E[X~|~\mathcal{F}]<0\rbrace} \E[X~|~\mathcal{F}]\d\P\leq0\\
&\Longrightarrow\E[X\cdot\indi_A]=0\\
&\Longrightarrow\int\limits_{\lbrace\E[X~|~\mathcal{F}]<0\rbrace} \E[X~|~\mathcal{F}]\d\P=0\\\\
&\Longrightarrow \P(\E[X~|~\mathcal{F}]<0)=0\\
&\Longrightarrow\E[X~|~\mathcal{F}]\geq0\text { fast sicher}
\end{align*}
Allgemeine Aussage folgt mit $\tilde{X}:=Y-X$ und aus der Linearität.\\
\underline{Zu 7.:}
\begin{align*}
\pm X\leq<|X|
\stackrel{6.}{\Longrightarrow}
\pm\E[X~|~\mathcal{F}]\leq\E[|X|~|~\mathcal{F}]\\
\Longrightarrow\Big|\E[X~|~\mathcal{F}]\Big|\leq\E[|X|~|~\mathcal{F}]
\end{align*}
\end{proof}
 
\section{Bedingte Erwartung in $L_1$}
Prinzip: stetige Fortsetzung

\begin{proposition} %1.6
Sei $\mathcal{F}\subseteq\A$ eine Unter-$\sigma$-Algebra von $\A$. Die bedingte Erwartung hat eine eindeutige stetige Fortsetzung von $L_2(\Omega,\A,\P)$ auf $L_1(\Omega,\A,\P)$. Diese bezeichnen wir ebenfalls mit $\E[\cdot~|~\mathcal{F}]$.
\end{proposition}
\begin{proof}
\underline{Existenz:}\\
Sei $X\in L_1$. Dann existiert eine Approximationsfolge $(X_n)_{n\in\N}\subseteq L_2$ mit $\E[|X_n-X|]\stackrel{n\to\infty}{\longrightarrow}0$, in Zeichen $X_n\stackrel{L_1}{\longrightarrow} X$.\\
Wähle z. B.
\begin{align*}
X_n:=\left\lbrace\begin{array}{cl}
X, & \falls |X|\leq n\\
n, & \falls X>n\\
-n, & \falls X<-n
\end{array}\right.
\end{align*}
Mit dominanter Konvergenz gilt 
\begin{align*}
\limn\E[X_n-X|]=\E\left[\limn|X_n-X|\right]=0.
\end{align*}
Mit Kontraktionseigenschaft gilt:
\begin{align*}
\E\Big[\big|\E[X_m~|~\F]-\E[X_n~|~\F]\big|\Big]
\leq\E\Big[\E\big[|X-M-X_n|~|~\big|~\F\Big]\\
stackeq{Turm}\E\big[|X_n-X_n|\big]\stackrel{m,n\to\infty}{\longrightarrow}0\\
\Longrightarrow\left(\E[X_n~|~\F]\right)_{n\in\N}\text{ ist Cauchy-Folge in }L_2(\Omega,\A,\P)\\
&\Longrightarrow\exists Z\in L_1(\Omega,\A,\P)\mit\E[X_n~|~\F]\stackrel{L_1}{\longrightarrow} Z=:\E[X~|~\F]
\end{align*}
\underline{Eindeutigkeit:} Sei $(\tilde{X}_n)_{n\in\N}$ eine weitere Approximationsfolge für $X$, d. h. 
\begin{align*}
\E[|\tilde{X}_n-X|]
\stackrel{n\to\infty}{\longrightarrow}
0
\end{align*}
Dann gilt
\begin{align*}
\Vert Z-\E[\tilde{X}_n~|~\F]\Vert_1
\leq
\Vert Z-\E[X_n~|~\F]\Vert_1+\underbrace{\Vert\E[X_n~|~\F-\E[\tilde{X}_n~|~\F]\Vert_1}_{=\E[|\E[X_n~|~\F-\E[\tilde{X}_n~|~\F|]}\\
&\stackrel{Kontr}{\leq}
\Vert Z-\E[X_n~|~\F]\Vert_1+\Vert X_n-\tilde{X}_n\Vert_1\\
&\leq\Vert Z-\E[X_n~|~\F]\Vert_1+\Vert X-X_n\Vert_1+\Vert X-\tilde{X}_n\Vert_1\\
&\longrightarrow0\\
&\Longrightarrow\E[\tilde{X}_n~|~\F]\stackrel{L_1}{\longrightarrow} Z
\end{align*}
Also ist der Limes unabhängig von der Approximationsfolge.
\end{proof}

\begin{korollar} %1.7
Alle Eigenschaften aus Proposition 1.4 (außer Symmetrie) gelten weiterhin für alle $X,Y\in L_1(\Omega,\A,\P)$.
\end{korollar}
\begin{proof}
Beweis durch Approximation.
\end{proof}

\begin{theorem}%1.8
Sei $\F$ Unter-$\sigma$-Algebra von $\A$ und seien
\begin{align*}
X\in L_1(\Omega,\A,\P),\qquad Y\in L_1(\Omega,\F,\P)
\end{align*}
Dann sind äquivalent:
\begin{enumerate}
\item $Y=\E[X~|~\F]$ fast sicher
\item $\E[X\cdot\indi_F]=\E[Y\cdot\indi_F]\qquad\forall F\in\F$\\
(Beachte $X,Y\in L_2\Longrightarrow\langle X-Y,\indi_F\rangle=0$)
\end{enumerate}
\end{theorem}
\begin{proof}
\underline{Zeige $(a)\Longrightarrow (b)$:}\\
Sei $(X_n)_{n\in\N}\subseteq L_2(\A)$ eine Approximationsfolge für $X$, d.h. $X_n\stackrel{L_1}{\longrightarrow} X$. Mit Proposition 1.6 folgt
\begin{align*}
\E[X_n~|~\F]\stackrel{L_1}{\longrightarrow}\E[X~|~\F]=Y\\
\E\left[(X-Y)\cdot\indi_F\right]
=
\limn\E[(X_n-\E[X_n~|~\F]),\indi_F]\\
=
\limn\langle X_n-\E[X_n~|~\F],\indi_F\rangle
\stackeq{Prop 1.3}
0\qquad\forall F\in\F
\end{align*}
\underline{Zeige $(b)\Longrightarrow (a)$:}\\
Setze 
\begin{align*}
F^{+}:=\left\lbrace\E[X~|~\F]-Y>0\right\rbrace\in\F\\
F^{-}:=\left\lbrace-(\E[X~|~\F]-Y)>0\right\rbrace\in\F
\end{align*}
Dann gilt:
\begin{align*}
0
\leq
\E\Big[\underbrace{(\E[X~|~\F]-Y)\cdot\indi_{F^+}}_{\geq0}\Big]
=
\E[\E[X~|~\F]\cdot\indi_{F^+}]-\E[Y\cdot\indi_{F^+}\\
&\stackeq{Pull-out + Turm}
\E[X\cdot\indi_{F^+}]-\E[Y\cdot\indi_{F^+}]\\
&\stackeq{(b)}
0\\
&\Longrightarrow\P(F^+)=0
\end{align*}
Analog erhält man $\P(F^-)=0$. Also folgt insgesamt $\E[X~|~\F]=Y$ fast sicher.
\end{proof}

\begin{bemerkung}
Die ``unbedingte Erwartung'' $\E[X]$ können wir als Spezialfall der bedingten Erwratung $\E[X~|~\F]$ für die triviale $\sigma$-Algebra $\F:=\lbrace\emptyset,\Omega\rbrace$ auffassen.\\
Denn es gilt:
\begin{align*}
\E[X\cdot\indi_\Omega]=\E[X]=\E[\E[X]\cdot\indi_\Omega]\\
\E[X\cdot\indi_\Omega]=0=\E[\E[X]\cdot\indi_\emptyset]\\
\stackrel{\text{Thm 1.8}}{\Longrightarrow}
\E[X]=\E[X~|~\F]\text{, da $\F$ trivial}
\end{align*}
\end{bemerkung}

\section{Bedingter Erwartungungswert + Unabhängigkeit} %keine Nummer
Sei $X$ eine Zufallsvariable auf $(\Omega,\A,\P)$ und $\F\subseteq\A$ Unter-$\sigma$-Algebra von $\A$ heißt \textbf{unabhängig} von $\F$, in Zeichen $X\amalg Y$
\begin{align}\label{sigmaAlgebraUnabhängig}
:\Longleftrightarrow
\E[f(X)\cdot\indi_A]
=
\E[f(X)]\cdot\P(A)\qquad\forall A\in\F\text{ und $f$ beschränkt, Borel-messbar und $\R$-wertig}
\end{align}
$X$ heißt \textbf{unabhängig} von einer Zufallsvariablen $Y:\gdw X\amalg\sigma(Y)$. Äquivalent:
\begin{align*}
\gdw\E[f(X)\cdot g(Y)]
=
\E[f(X)]\cdot\E[g(Y)]\qquad\forall f,g\in B_b(\R)
\end{align*}

\begin{theorem} %1.9
Sei $X\in L_1(\Omega,\A,\P)$ unabhängig von $\F$. Dann gilt:
\begin{align*}
\E[X~|~\F]=\E[X]
\end{align*}
\end{theorem}

\begin{proof}
Wegen $X\amalg Y$ gilt:
\begin{align*}
\forall F\in\F:\E[X\cdot\indi_X]
\stackeq{\ref{sigmaAlgebraUnabhängig}}
\E[X]\cdot\P(F)=\E[X]\cdot\P(F)=\E[\E[X]\cdot\indi_F]\\
\stackrel{\text{Thm 1.8}}{\Longrightarrow}
\E[X]=\E[X~|~\F]
\end{align*}
\end{proof}

\begin{bemerkung}
Merke die beiden Extremfälle:
\begin{itemize}
\item Wenn $X$ $\F$-messbar ist, dann ist $\E[X~|~\F]=X$.\\
Die bedingte Erwartung verändert also $X$ nicht.
\item Wenn $X$ von $\F$ unabhängig ist, dann ist $\E[X~|~\F]=\E[X]$.\\
Die bedingte Erwartung reduziert $X$ auf den unbedingten Erwartungswert $\E[X]$.
\end{itemize}
\end{bemerkung}