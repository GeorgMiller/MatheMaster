\setcounter{chapter}{-1}
\chapter{Einführung}
\begin{itemize}
\item Voraussetzung für viele weitere VL im Schwerpunkt Stochastik
\item zunehmend stochastische Systeme / stochastische Prozesse $\to$ Modellierung von zeitabhängigen und zufälligen Vorgängen
\item wichtig im naturwissenschaftlicher, wirtschaftwissenschaftlicher und sozialwissenschaftlicher Modellierung
\begin{itemize}
\item Schwimmbewegung eines Einzellers
\item Bildung und Rückbildung von sozialen Netzwerken
\item zeitlicher Verlauf eines Wechselkurses (EUR / GBP)
\end{itemize}
\end{itemize}

Zentrale Frage: \underline{Abhängigkeitsstruktur} (ist ``morgen'' von ``heute'' unabhängig?)
\begin{itemize}
\item unabhängige gleichverteilte Zufallsvariablen
\item Markov-Prozesse
\item Martingale
\end{itemize}

Was ist ein Martingal?
\begin{itemize}
\item ``faires Spiel'' zwischen Personen $A$ und $B$ bei dem keine Strategie einen systematischen Vorteil bringt
\item Ein Vorgang, bei dem die beste Voraussage (Punktschätzung) der heutige Wert ist.
\item ``neutraler stochastischer Prozess'' ohne systematischen Trend zum Auf- oder Abstieg
\end{itemize}

Weitere Themen:
\begin{itemize}
\item charakteristische Funktionen: Fourier-Transformation einer Wahrscheinlichkeitsverteilung\\
Wichtiges analytisches Werkzeug in der W-Theorie
\item Zentrale Grenzwertsätze: Aussagen über Konvergenz von Summen unabhängiger Zufallsvariablen zur Normalverteilung
\item Brown'sche Bewegung und evtl. Lévy-Prozesse
\end{itemize}