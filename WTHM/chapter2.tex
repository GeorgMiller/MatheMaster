% This work is licensed under the Creative Commons
% Attribution-NonCommercial-ShareAlike 4.0 International License. To view a copy
% of this license, visit http://creativecommons.org/licenses/by-nc-sa/4.0/ or
% send a letter to Creative Commons, PO Box 1866, Mountain View, CA 94042, USA.

\chapter{Martingale} %2
\section{Definition und Eigenschaften} %2.1
\begin{beisp}
Betrachte ein Glücksspiel zwischen zwei Personen.
Die Folge $(\xi_n)_{n\in\N}$ i.i.d. und in $L_1$ beschreibe den Gewinn pro Zug (positiv $\to$ Person A; negativ $\to$ Person B). Sei
\begin{align*}
S_n:=\sum\limits_{i=1}^n\xi_i
\end{align*}
der Gesamtgewinn nach dem $n$-ten Zug und 
\begin{align*}
\F_n:=\sigma\big(\xi_1,\ldots,\xi_n\big)
\end{align*}
die Information aus dem Spiel bis zum $n$-ten Zug. Somit ist er bedingte Erwartungswert
\begin{align*}
\E\big[S_{n+k}~|~\F_n\big]
\end{align*}
die Vorhersage für Gewinn nach $(n+k)$-ten Zug basierend auf den ersten $n$ Zügen.\\
Berechnung:
\begin{align*}
\E\big[S_{n+k}~\big|~\F_n\big] 
&=\E\Big[S_n+\xi_{n+1}+\ldots+\xi_{n+k}~\Big|~\F_n\Big]\\
&=\E[S_n~|~\F_n]+\sum\limits_{i=1}^k\E\big[\underbrace{\xi_{n+i}~\big|~\F_n}_{\unab}\big]\\
&=S_n+\sum\limits_{i=1}^k\E\big[\xi_{n+i}\big]\\
&\stackrel{\text{iid.}}{=}S_n+k\cdot\E[\xi_1]\\
&=\left\lbrace\begin{array}{cll}
> S_n, & \falls \E[\xi_1]>0 & \text{vorteilhaftes Spiel für A}\\
= S_n, & \falls \E[\xi_1]=0 & \text{faires Spiel}\\
< S_n, & \falls \E[\xi_1]<0 & \text{nachteiliges Spiel für A}
\end{array}\right.
\end{align*}
\underline{Variante:} Variabler Einsatz erlaubt.
Der Einsatz im $(n+1)-k$-ten Zug wird durch 
\begin{align*}
e_{n+1}:=:e_{n+1}\big(\xi_1,\ldots,\xi_n\big)
\end{align*}
definiert und hängt von $\F_n$ ab. Man spricht hier von einer vorhersehbaren Zufallsvariable.
Der neue Gesamtgewinn unter diesem Einsatz beträgt
\begin{align*}
\tilde{S}_n:=\sum\limits_{i=1}^n e_i\cdot\xi_i.
\end{align*}
Achtung! 
\begin{align*}
\tilde{S}_n-\tilde{S}_{n-1}=e_n\cdot\xi_n
\end{align*}
ist \underline{nicht mehr iid}.\\

Nun ergibt sich durch eine beliebige Einsatzstrategie mit Informationen aus $\F_n$ die Vorhersage der nächsten Runde des Spiels:
\begin{align*}
\E[\tilde{S}_{n+1}~|~\F_n] 
&=\E\big[\tilde{S}_n+ e_{n+1}\cdot\xi_{n+1}~\big|~\F_n\big]\\
&=\tilde{S}_n+\E\big[\underbrace{e_{n+1}}_{\in\F_n}\cdot\xi_{n+1}~\big|~\F_n\big]\\
&=\tilde{S}_n+e_{n+1}\cdot\E\big[\underbrace{\xi_{n+1}~\big|~\F_n}_{\unab}\big]\\
&=\tilde{S}_n+e_{n+1}\cdot\E[\xi_1]
\end{align*}
Nehmen wir ein faires Spiel an, also $\E[\xi_1]=0$, dann erhalten wir:
\begin{align*}
\E\big[\tilde{S}_{n+1}~\big|~\F_n\big]=\tilde{S}_n
\end{align*}
Durch Iteration dieser Gleichung folgt für $k>0$:
\begin{align*}
\E\big[\tilde{S}_{n+k}~\big|~\F_n\big]
&\stackeq{\eqref{Turmregel}}
\E\Big[\underbrace{\E\big[\tilde{S}_{n+k}~\big|~\F_{n+k-1}\big]}_{=\tilde{S}_{n+k-1}}~\Big|~\F_n\Big]\\
&=\E\big[\tilde{S}_{n+1}~\big|~\F_n\big]\\
&=\tilde{S}_n
\end{align*}
Also lässt sich bei einem fairen Spiel auch durch variablen Einsatz nicht in ein vorteilhaftes Spiel verwandeln. Dies ist ein Prototyp für das Martingal.
\end{beisp}

\begin{defi}\
\begin{enumerate}[label=(\alph*)]
\item Eine Indexmenge $I$ heißt \textbf{aufsteigend geordnet} $:\gdw$ eine Ordnungsrelation ``$\leq$'' existiert mit
\begin{align*}
\forall s,t\in I:\exists u\in I:\qquad s\leq u\wedge t\leq u\qquad 
\end{align*}
In den allermeisten Fällen haben wir eine Totalordnung, z. B. $(\N,\leq)$ oder $(\R_{\geq0},\leq)$. $I$ wird oft als Zeitachse interpretiert.
\item Eine Familie $(\F_t)_{t\in I}$ von $\sigma$-Algebren $\F_t\subseteq\A$ heißt \textbf{Filtration} 
\begin{align*}
:\Longleftrightarrow\Big( s\leq t\implies\F_s\subseteq\F_t\Big)
\end{align*}
Eine Filtration ist also eine aufsteigenden Folge von $\sigma$-Algebren.\\
Interpretation: $\F_t$ beschreibt die zum Zeitpunkt $t$ verfügbare Information. Und der Informationsfluss wächst mit der Zeit an.\\
Außerdem setzen wir
\begin{align*}
\F_\infty:=\bigcup\limits_{t\in I}\F_t.
\end{align*}
\item Ein \textbf{stochastischer Prozess (SP)} ist eine Familie $(X_t)_{t\in I}$ von Zufallsvariablen.\\
Ein stochastischer Prozess $(X_t)_{t\in I}$ heißt \textbf{adaptiert} an die Filtration $(\F_t)_{t\in I}$
\begin{align*}
:\Longleftrightarrow\forall t\in I: X_t\text{ ist messbar bzgl. }\F_t
\end{align*}
\item Eine Folge $(e_n)_{n\in\N}$ heißt \textbf{vorhersehbar} bzgl. einer Filtration $(F_t)_{n\in\N}$
\begin{align*}
:\Longleftrightarrow\forall n\in\N:e_{n+1}\text{ ist messbar bzgl. }\F_n
\end{align*}
Vorhersehbarkeit ist stärker als Adaptiertheit.
\end{enumerate}
\end{defi}

\begin{defi}[Martingal]\enter
Sei $X=(X_t)_{t\in I}$ ein stochastischer Prozess und $(\F_t)_{t\in I}$ eine Filtration. $X$ heißt \textbf{Martingal} $:\gdw$ folgende drei Eigenschaften gelten:
\begin{enumerate}[label=(\alph*)]
\item $(X_t)_{t\in I}$ ist adaptiert an $(\F_t)_{t\in I}$
\item $\begin{aligned}
		X_t\in L_1(\Omega,\A,\P) \qquad \forall t\in I\qquad(\text{ d. h. } \E\big[|X_t|\big]<\infty)
\end{aligned}$
\item $\begin{aligned}
\E[X_t~|~\F_s]=X_s\qquad\forall s,t\in I\mit s\leq t
\end{aligned}$
	
\end{enumerate}
Gilt statt (c) die Eigenschaft
\begin{align}\label{defSubmartingal}\tag{c'}
\E[X_t~|~\F_s]\geq X_s
\end{align}
so heißt $X$ \textbf{Submartingal}. (Der zukünftige Wert wird unterschätzt) \enter Gilt statt (c) die Eigenschaft
\begin{align}\label{defSupermartingal}\tag{c''}
\E[X_t~|~\F_s]\leq X_s
\end{align}
so heißt $X$ \textbf{Supermartingal}. (Der zukünftige Wert wird überschätzt)\\
\end{defi}

\begin{bemerkung}
Beim Submartingal steigt der Erwartungswert $\E[X_t]$ mit der Zeit und beim Supermartingal fällt der Erwartungswert mit der Zeit.\\
Begründung für Submartingal:
\begin{align*}
\E[X_t]\stackeq{\eqref{Turmregel}}
\E\big[\E[X_t~|~\F_s]\big]\stackrel{\text{Mono}}{\geq}\E[X_s]\qquad s\leq t
\end{align*}
``\textit{Das Leben ist ein Supermartingal - die Erwartung fällt mit der Zeit.}''
\end{bemerkung}

\begin{defi}[Random Walk / Irrfahrt]\enter %noNumber
Sei $(\xi_n)_{n\in\N}$ eine Folge von iid $\R$-wertigen Zufallsvariablen mit $\xi_1\sim F$. Dann heißt
\begin{align*}
S_N:=\sum\limits_{i=1}^n\xi_i\qquad S_0:=0
\end{align*}
(eindimensionaler) \textbf{Random Walk (RW) mit Schrittverteilung} $F$.
\begin{itemize}
\item Ein Random Walk heißt \textbf{symmetrisch}
\begin{align*}
:&\Longleftrightarrow F\text{ symmetrisch}\\
&\Longleftrightarrow\P(\xi_1\in B)=\P(\xi_1\in -B)\qquad\forall B\in\B(\R)
\end{align*}
\item Ein Random Walk heißt \textbf{einfach} 
$\begin{aligned}
	:\Longleftrightarrow\P(\xi_1\in\lbrace-1,1\rbrace)=1
\end{aligned}$
\end{itemize}
\end{defi}

\begin{beisp}\
\begin{enumerate}[label=(\alph*)]
\item Sei $(S_n)_{n\in\N}$ ein Random Walk mit $\E\big[|\xi_1|\big]<\infty$.\\
Dann ist $(S_n)_{n\in\N}$ bzgl. der natürlichen Filtration $\F_n=\sigma(\xi_1,\ldots,\xi_n)$ ein
\begin{align*}
\left\lbrace\begin{array}{cl}
\text{Martingal}, & \falls \E[\xi_1]=0\\
\text{Submartingal}, & \falls \E[\xi_1]\geq0\\
\text{Supermartingal}, & \falls \E[\xi_1]\leq0\\
\end{array}\right.
\end{align*}
(siehe Anfang des Kapitels)
\item Sei $(S_n)_{n\in\N}$ ein Random Walk mit $\E[\xi_1]=0$ und $\Var[\xi_1]=\sigma^2$.\\
Dann ist 
\begin{align*}
(M_n)_{n\in\N}\mit M_n:=S_n^2-n\cdot\sigma^2,n\in\N
\end{align*}
ein Martingal bzgl. $\F_n:=\sigma(\xi_1,\ldots,\xi_n)$. Dabei ist der Term $n\cdot\sigma^2$ ein \textbf{Kompensator}-Term, der $M_n$ Martingal erst zum Martingal macht. Überprüfen der Eigenschaften:
\begin{itemize}
\item adaptiert: $M_n$ ist messbare Funktion von $\xi_1,\ldots,\xi_n$
\begin{align*}
	\implies(M_n)_{n\in\N} \text{ ist } (\F_n)_{n\in\N}\text{-adaptiert}
\end{align*}
\item integrierbar: \begin{align*}
\sqrt{\E[S_n^2]}=\Vert\xi_1+\ldots\xi_n\Vert_2\stackrel{\Delta\text{-Ungl}}{\leq}\Vert\xi_1\Vert_2+\ldots+\Vert\xi_n\Vert_2\leq n\cdot\sigma<\infty
\end{align*}
\item Martingaleigenschaft:
\begin{align*}
\E[M_n~|~\F_{n-1}]
&=\E[S_n^2-n\cdot\sigma^2~|~\F_{n-1}]\\
&=\E\big[(S_{n-1}+\xi_n)^2~|~\F_{n-1}\big]-n\cdot\sigma^2\\
&=\E\big[S_{n-1}^2~|~\F_{n-1}\big]+2\cdot\E\big[S_{n-1}\cdot\xi_n~|~\F_{n-1}\big]+\E\big[\underbrace{\xi_n^2~|~\F_{n-1}}_{\unab}\big]-n\cdot\sigma^2\\
&=S_{n-1}^2+2\cdot S_{n-1}\cdot \underbrace{\E\big[\underbrace{\xi_n~|~\F_{n-1}}_{\unab}\big]}_{=\E[\xi_n]=0}+\underbrace{\E[\xi_n^2]}_{=\Var(\xi_n)=\sigma^2}-n\cdot\sigma^2\\
&=S_{n-1}^2-(n-1)\cdot\sigma^2\\
&=M_{n-1}
\end{align*}
\end{itemize}
\item Sei $X_n=a_n$ mit $(a_n)_{n\in\N}$ eine deterministische aufsteigende Folge.\\
Dann ist $(X_n)_{n\in\N}$ Submartingal (bei jeder Filtration $(\F_n)_{n\in\N}$).
\item Seien $(\xi_n)_{n\in\N}$ mit $\xi_1\geq0$ und $\E[\xi_1]=1$. Dann ist
\begin{align*}
M_n:=\prod\limits_{i=1}^n\xi_i,\qquad M_0:=1
\end{align*}
ein Martingal bzgl. $\F_n:=\sigma(\xi_1,\ldots,\xi_n)$ (geometrischer Random Walk).
\item Sei $X\in L_1(\Omega,\A,\P)$ eind $(\F_t)_{t\in\N_0}$ eine Filtration.\\
Dann ist
\begin{align*}
X_t:=\E[X~|~\F_t]
\end{align*}
ein Martingal bzgl. $(\F_t)_{t\in\N_0}$, denn:
\begin{itemize}
\item Adaptiertheit: Klar.
\item Integrierbarkeit:
\begin{align*}
\E\big[|X_t|\big]
=\E\Big[\big|\E[X~|~\F_t]\big|\Big]
&\stackrel{\Delta\text{-Ungl}}{\leq}
\E\Big[\E\big[|X|~|~\F_t \big]\Big]
\stackeq{\text{Turm}}
\E\big[|X|\big]
<\infty
\end{align*}
\item Martingaleigenschaft:
\begin{align*}
\E[X_t~|~\F_s]
\stackeq{\text{Def}}\E\Big[\E\big[X~|~\F_t\big]~\Big|~ \F_s\Big]
\stackeq{\text{Turm}}
\E[X~|~\F_s]
=X_s
\qquad\forall s\leq t
\end{align*}
\end{itemize}
\end{enumerate}
\end{beisp}

\begin{notation}
\begin{align*}
X\wedge Y:=\min(X,Y)\qquad X\vee Y:=\max(X,Y)\qquad\forall X,Y\text{ ZV}
\end{align*}
\end{notation}

\begin{proposition}[Elementare Eigenschaften]\label{Prop2.1}\enter
Sei $(\F_t)_{t\in I}$ eine Filtration. Dann gilt:
\begin{enumerate}[label=\alph*)]
\item $\begin{aligned}
(X_t)_{t\in I},(Y_t)_{t\in I}\text{ Martingale }
\implies\forall a,b\in\R: (a\cdot X_t+b\cdot Y_t)_{t\in I}\text{ ist Martingal}
\end{aligned}$
\item $\begin{aligned}
		&(X_t)_{t\in I},(Y_t)_{t\in I}\text{ Submartingale }&
&\implies\forall a,b\in\R_{\geq0}: (a\cdot X_t+b\cdot Y_t)_{t\in I}\text{ ist Submartingal}\\
&(X_t)_{t\in I},(Y_t)_{t\in I}\text{ Supermartingale }&
&\implies\forall a,b\in\R_{\geq0}: (a\cdot X_t+b\cdot Y_t)_{t\in I}\text{ ist Supermartingal}
\end{aligned}$
\item $\begin{aligned}
(X_t)_{t\in I},(Y_t)_{t\in I}\text{ Supermartingale }
\implies (X_t\wedge Y_t)_{t\in I}\text{ ist Supermartingal}
\end{aligned}$
\item $\begin{aligned}
(X_t)_{t\in I}\text{ Martingal, $\phi$ konvex und }\E\big[\phi(X_t)\big]<\infty
\implies \big(\phi(X_t)\big)_{t\in I}\text{ ist Submartingal}
\end{aligned}$
\item $\begin{aligned}
(X_t)_{t\in I}\text{ Submartingal, $\phi$ konvex und monoton steigend und }\E\big[\phi(X_t)\big]<\infty
\end{aligned}$
\begin{align*}
\implies \big(\phi(X_t)\big)_{t\in I}\text{ ist Submartingal}
\end{align*}
\item $\begin{aligned}
(X_t)_{t\in I}\text{ Martingal}
\Longleftrightarrow
(X_t)_{t\in I}\text{ ist Sub- und Supermartingal}
\end{aligned}$
\end{enumerate}
\end{proposition}

Für den Beweis benötigen wir die bedingte Jensensche Ungleichung:

\begin{lemma}[Bedingte Jensensche Ungleichung]\label{lemma2.2BedingteJensenscheUngleichung}\enter
Sei $X\in L_1(\Omega,\A,\P),~\F\subseteq\A$ Unter-$\sigma$-Algebra von $\A$, $\phi:\R\to\R$ konvex und\\ $\E\big[\phi(X)\big]<\infty$. Dann gilt:
\begin{align*}
\phi\big(\E[X~|~\F]\big)\leq\E\big[\phi(X)~\big|~\F\big]
\end{align*}
\end{lemma}
\begin{proof}
%Hier könnte man eine Skizze einfügen
Da $\phi$ konvex ist, ist $\phi$ Supremum aller linearen Trägerfunktionen, d. h.
\begin{align*}
\phi(x)=\sup\Big\lbrace a\cdot x+b:\forall y\in \R: a\cdot y+b\leq\phi(y)\Big\rbrace
\end{align*}
Daraus folgt:
\begin{align*}
\phi\big(\E[X~|~\F]\big)
&\leq\sup\limits_{\begin{subarray}{c}a\cdot y+b\leq\phi(y)\\y\in\R\end{subarray}}\big(a\cdot\E[X~|~\F]+b\big) \\
&=\sup\limits_{\begin{subarray}{c}a\cdot y+b\leq\phi(y)\\y\in\R\end{subarray}}
\E[a\cdot X+b~|~\F] \\
&\leq \E\big[\phi(x)~\big|~\F\big]
\end{align*}
\end{proof}

\begin{proof}[Beweis von Proposition \ref{Prop2.1}]\enter
a) und b) folgen aus Linearität und Monotoni der bedingten Erwartung. f) ist klar.\\

\underline{Zeige c):}
\begin{align*}
\E\big[(X_t\wedge Y_t)~\big|~\F_s\big]
\stackrel{\text{Mono}}{\leq}
\left\lbrace\begin{array}{c}
\E[X_t~|~\F_s]=X_s\\
\E[Y_t~|~\F_s]=Y_s\\
\end{array}\right\rbrace
\leq X_s\wedge Y_s
\end{align*}
\underline{Zeige (d):}
\begin{align*}
\E\big[\phi(X_t)~\big|~\F_s\big]
\stackrel{\text{Jensen}}{\geq}
\phi\big(\E[X_t~|~\F_s]\big)
\stackeq{\text{MG}}
\phi(X_s)
\end{align*}

\underline{Zeige (e):}
\begin{align*}
\E\big[\phi(X_t)~\big|~\F_s\big]
\stackrel{\text{Jensen}}{\geq}
\phi\big(\underbrace{\E[X_t~|~\F_s]}_{\geq X_s}\big)
\stackrel{\text{Mono}}{\geq}
\phi(X_s)
\end{align*}
\end{proof}

\begin{theorem}[Doob-Zerlegung]\label{Theorem2.3DoobZerlegung}\enter
Sei $(X_n)_{n\in\N}\subseteq L_1(\Omega,\A,\P)$ ein adaptierter stochastischer Prozess. Dann gilt:
\begin{align}\label{eqDoobZerlegung}\tag{DZ}
X_n=X_0+M_n+A_n
\end{align}
mit $(M_n)_{n\in\N}$ Martingal mit $M_0=0$ und $(A_n)_{n\in\N}$ mit $A_0=0$ vorhersehbar.\\
Die Zerlegung \eqref{eqDoobZerlegung} ist eindeutig bis auf Ununterscheidbarkeit, d. h.
\begin{align*}
\P\big(M_n=M_n'\text{ und }A_n=A_n'\quad\forall n\in\N\big)=1
\end{align*}
für jede andere Zerlegung $X_n=X_0+M_n'+A_n'$.\\
Außerdem gilt:
\begin{itemize}
\item $(X_n)_{n\in\N}$ Submartingal $\implies (A_n)_{n\in\N}$ fast sicher wachsend
\item $(X_n)_{n\in\N}$ Supermartingal $\implies (A_n)_{n\in\N}$ fast sicher fallend
\end{itemize}
\end{theorem}

\begin{bemerkung}
Für Sub-/Supermartingale auf $I=\R_{\geq0}$ heißt die analoge Zerlegung \textbf{Doob-Meyer-Zerlegung} und ist wesentlich schwieriger zu zeigen.
\end{bemerkung}

\begin{proof}
\underline{Vorüberlegung:}\\
Angenommen \eqref{eqDoobZerlegung} gilt, dann folgt
\begin{align*}
\E[X_n-X_{n-1}~|~\F_{n-1}]
&=\underbrace{\E[M_n-M_{n-1}~|~\F_{n-1}]}_{=0\text{, da Martingal}}
+\E[\underbrace{A_n-A_{n-1}}_{\F_{n-1}\text{-messbar}}~|~\F_{n-1}]\\
&=A_n-A_{n-1}\\
\implies
A_n&=\sum\limits_{j=1}^n\underbrace{\E\big[X_j-X_{j-1}~|~\F_j\big]}_{\F_{j-1}\text{-messbar}\implies\F_{n-1}\text{-messbar}}
\end{align*}
\underline{Zur Existenz:}\\
Wähle also $A_n$ genau so. Dann ist $A_n$ vorhersehbar und $\E\big[|A_n|\big]<\infty$ (folgt aus Dreiecksungleichung). Setze
\begin{align*}
M_n:=X_n-X_0-A_n.
\end{align*}
Zu zeigen: $(M_n)_{n\in\N}$ ist Martingal.\\
Adaptiertheit und Integrierbarkeit ist klar. Zu zeigen ist also noch die Martingaleigschaft bzw. deren Äquivalenz:
\begin{align*}
\text{Martingaleigenschaft }\Longleftrightarrow
\E[M_n~|~\F_{n-1}]=M_{n-1}\Longleftrightarrow
\E[M_n-M_{n-1}~|~\F_{n-1}]=0
\end{align*}
Also:
\begin{align*}
\E[M_{n}-M_{n-1}~|~\F_{n-1}]
&=\E[ X_n-X_{n-1}~|~\F_{n-1}]
-\E[A_n-A_{n-1}~|~\F_{n-1}]\\
&=\E[ X_n-X_{n-1}~|~\F_{n-1}]
-(A_n-A_{n-1})\\
&\stackeq{\text{Def }A}
0
\end{align*}
\underline{Zur Eindeutigkeit:}\\
Laut Vorüberlegung ist die Wahl $A_n$ auch notwendig, also 
\begin{align*}
A_n'=A_n\text{ fast sicher}\qquad\forall n\in\N
\end{align*}
Wegen
\begin{align*}
M_n'=X_n-X_0-A_n'
\end{align*}
gilt
$M_n'=M_n\text{ fast sicher}$ für alle $n\in\N$. Also
\begin{align*}
\P\Big((M_n'=M_n)\wedge(A_n'=A_n)~\forall n\in\N\Big)
=\P\left(\bigcap\limits_{n=1}^\infty(M_n=M_n')\cap(A_n=A_n')\right)
=1
\end{align*}
Die Aussagen zu Sub-/Supermartingalen folgt aus
\begin{align*}
\E[X_n-X_{n-1}~|~\F_{n-1}]=A_n-A_{n-1}.
\end{align*}
\end{proof}
