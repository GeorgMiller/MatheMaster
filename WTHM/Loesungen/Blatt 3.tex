% This work is licensed under the Creative Commons
% Attribution-NonCommercial-ShareAlike 4.0 International License. To view a copy
% of this license, visit http://creativecommons.org/licenses/by-nc-sa/4.0/ or
% send a letter to Creative Commons, PO Box 1866, Mountain View, CA 94042, USA.

\documentclass[12pt,a4paper]{article} 

% This work is licensed under the Creative Commons
% Attribution-NonCommercial-ShareAlike 4.0 International License. To view a copy
% of this license, visit http://creativecommons.org/licenses/by-nc-sa/4.0/ or
% send a letter to Creative Commons, PO Box 1866, Mountain View, CA 94042, USA.

% PACKAGES
\usepackage[english, ngerman]{babel}	% Paket für Sprachselektion, in diesem Fall für deutsches Datum etc
\usepackage[utf8]{inputenc}	% Paket für Umlaute; verwende utf8 Kodierung in TexWorks 
\usepackage[T1]{fontenc} % ö,ü,ä werden richtig kodiert
\usepackage{amsmath} % wichtig für align-Umgebung
\usepackage{amssymb} % wichtig für \mathbb{} usw.
\usepackage{amsthm} % damit kann man eigene Theorem-Umgebungen definieren, proof-Umgebungen, etc.
\usepackage{mathrsfs} % für \mathscr
\usepackage[backref]{hyperref} % Inhaltsverzeichnis und \ref-Befehle werden in der PDF-klickbar
\usepackage[english, ngerman, capitalise]{cleveref}
\usepackage{graphicx}
\usepackage{grffile}
\usepackage{setspace} % wichtig für Lesbarkeit. Schöne Zeilenabstände

\usepackage{enumitem} % für custom Liste mit default Buchstaben
\usepackage{ulem} % für bessere Unterstreichung
\usepackage{contour} % für bessere Unterstreichung
\usepackage{epigraph} % für das coole Zitat

\usepackage{tikz}

% This work is licensed under the Creative Commons
% Attribution-NonCommercial-ShareAlike 4.0 International License. To view a copy
% of this license, visit http://creativecommons.org/licenses/by-nc-sa/4.0/ or
% send a letter to Creative Commons, PO Box 1866, Mountain View, CA 94042, USA.

% THEOREM-ENVIRONMENTS

\newtheoremstyle{mystyle}
  {20pt}   % ABOVESPACE \topsep is default, 20pt looks nice
  {20pt}   % BELOWSPACE \topsep is default, 20pt looks nice
  {\normalfont} % BODYFONT
  {0pt}       % INDENT (empty value is the same as 0pt)
  {\bfseries} % HEADFONT
  {}          % HEADPUNCT (if needed)
  {5pt plus 1pt minus 1pt} % HEADSPACE
	{}          % CUSTOM-HEAD-SPEC
\theoremstyle{mystyle}

% Definitionen der Satz, Lemma... - Umgebungen. Der Zähler von "satz" ist dem "section"-Zähler untergeordnet, alle weiteren Umgebungen bedienen sich des satz-Zählers.
\newtheorem{satz}{Satz}[section]
\newtheorem{lemma}[satz]{Lemma}
\newtheorem{korollar}[satz]{Korollar}
\newtheorem{proposition}[satz]{Proposition}
\newtheorem{beispiel}[satz]{Beispiel}
\newtheorem{definition}[satz]{Definition}
\newtheorem{bemerkungnr}[satz]{Bemerkung}
\newtheorem{theorem}[satz]{Theorem}

% Bemerkungen, Erinnerungen und Notationshinweise werden ohne Numerierungen dargestellt.
\newtheorem*{bemerkung}{Bemerkung.}
\newtheorem*{erinnerung}{Erinnerung.}
\newtheorem*{notation}{Notation.}
\newtheorem*{aufgabe}{Aufgabe.}
\newtheorem*{lösung}{Lösung.}
\newtheorem*{beisp}{Beispiel.} %Beispiel ohne Nummerierung
\newtheorem*{defi}{Definition.} %Definition ohne Nummerierung
\newtheorem*{lem}{Lemma.} %Lemma ohne Nummerierung


% SHORTCUTS
\newcommand{\R}{\mathbb{R}}				 % reelle Zahlen
\newcommand{\Rn}{\R^n}						 % der R^n
\newcommand{\N}{\mathbb{N}}				 % natürliche Zahlen
\newcommand{\Z}{\mathbb{Z}}				 % ganze Zahlen
\newcommand{\C}{\mathbb{C}}			   % komplexe Zahlen
\newcommand{\gdw}{\Leftrightarrow} % Genau dann, wenn
\newcommand{\with}{\text{ mit }}   % mit
\newcommand{\falls}{\text{falls }} % falls
\newcommand{\dd}{\text{ d}}        % Differential d

% ETWAS SPEZIELLERE ZEICHEN
%disjoint union
\newcommand{\bigcupdot}{
	\mathop{\vphantom{\bigcup}\mathpalette\setbigcupdot\cdot}\displaylimits
}
\newcommand{\setbigcupdot}[2]{\ooalign{\hfil$#1\bigcup$\hfil\cr\hfil$#2$\hfil\cr\cr}}
%big times
\newcommand*{\bigtimes}{\mathop{\raisebox{-.5ex}{\hbox{\huge{$\times$}}}}} 

% WHITESPACE COMMANDS
%non-restrict newline command
\newcommand{\enter}{$ $\newline} 
%praktischer Tabulator
\newcommand\tab[1][1cm]{\hspace*{#1}}

% TEXT ÜBER ZEICHEN
%das ist ein Gleichheitszeichen mit Text darüber, Beispiel: $a\stackeq{Def} b$
\newcommand{\stackeq}[1]{
	\mathrel{\stackrel{\makebox[0pt]{\mbox{\normalfont\tiny #1}}}{=}}
} 
%das ist ein beliebiges Zeichen mit Text darüber, z. B.  $a\stackrel{Def}{\Rightarrow} b$
\newcommand{\stacksymbol}[2]{
	\mathrel{\stackrel{\makebox[0pt]{\mbox{\normalfont\tiny #1}}}{#2}}
} 

% UNDERLINE
% besseres underline 
\renewcommand{\ULdepth}{1pt}
\contourlength{0.5pt}
\newcommand{\ul}[1]{
	\uline{\phantom{#1}}\llap{\contour{white}{#1}}
}


% hier noch ein paar Commands die nur ich nutze, weil ich sie mir im Laufe der Jahre angewöhnt habe und sie mir jetzt nicht abgewöhnen will:

\newcommand{\gdw}{\Leftrightarrow}   % genau dann, wenn



% Commands für Stochastik / Statistik
\newcommand{\A}{\mathcal{A}}
\renewcommand{\P}{\mathbb{P}}
\newcommand{\E}{\mathbb{E}}




% This work is licensed under the Creative Commons
% Attribution-NonCommercial-ShareAlike 4.0 International License. To view a copy
% of this license, visit http://creativecommons.org/licenses/by-nc-sa/4.0/ or
% send a letter to Creative Commons, PO Box 1866, Mountain View, CA 94042, USA.

% Commands für WTHM
\newcommand{\F}{\mathcal{F}}				% Standard-Unter-Sigma-Algebra / Filtration
\newcommand{\G}{\mathcal{G}}				% Gegenstück zu \F für Rückwärtsmartingale

% independent symbol from:
% https://tex.stackexchange.com/questions/79434/double-perpendicular-symbol-for-independence
\newcommand{\unab}{\protect\mathpalette{\protect\independenT}{\perp}}
\def\independenT#1#2{\mathrel{\rlap{$#1#2$}\mkern2mu{#1#2}}}

\newcommand{\BedE}[2]{\E\left[{#1}~|~{#2}\right]} % Bedingte Erwartung
\newcommand{\graph}{\text{graph}}

%\newcommand{\binom}[2]{\begin{pmatrix}	{#1}\\{#2} \end{pmatrix}} %this command doesn't compile for me




\author{Willi Sontopski}

\parindent0cm %Ist wichtig, um führende Leerzeichen zu entfernen

\usepackage{color}

\usepackage{scrpage2}
\pagestyle{scrheadings}
\clearscrheadfoot

\ihead{Willi Sontopski}
\chead{WTHM WiSe 18 19}
\ohead{}
\ifoot{Aufgabenblatt 3}
\cfoot{Version: \today}
\ofoot{Seite \pagemark}

\begin{document}
%\setcounter{section}{1}

\section*{Aufgabe 1}
Seien $\sigma,\tau$ Stoppzeiten und $\F_\tau$ definiert durch
\begin{align*}
\F_\tau:=\big\lbrace A\in\A~\big|~\forall b\in\N_0:A\cap\lbrace\tau\leq n\rbrace\in\F_n\big\rbrace
\end{align*}
Dann gilt:
\begin{enumerate}[label=\alph*)]
\item $\F_\tau$ ist eine $\sigma$-Algebra
\item $\begin{aligned}
\sigma\leq\tau\implies\F_\sigma\subseteq\F_\tau
\end{aligned}$
\item $\begin{aligned}
\lbrace\tau\leq\sigma\rbrace\in\F_\tau\cap\F_\sigma
\end{aligned}$
\item $\begin{aligned}
\F_\tau\cap\F_\sigma=\F_{\tau\wedge\sigma}
\end{aligned}$
\end{enumerate}
\begin{proof}
\underline{Zeige a):}\\

\underline{Zeige b):}\\

\underline{Zeige c):}\\

\underline{Zeige d):}\\
\end{proof}

\section*{Aufgabe 2}
Sei $(X_i)_{i\in I}$ eine Folge von Zufallsvariablen und $\phi:[0,\infty)\to\R$ eine konvexe, wachsende Funktion mit 
\begin{align*}
\frac{\phi(x)}{x}\stackrel{x\to+\infty}{\longrightarrow}+\infty
\end{align*}
(zum Beispiel $\phi(x):=|x|^p\mit p>1$.) Dann gilt:
\begin{align*}
\sup\limits_{i\in I}\E\Big[\phi\big(|X_i|\big)\Big]<\infty\implies(X_i)_{i\in I}\text{ ist ggi}
\end{align*}
\begin{proof}
Sei also das Supremum beschränkt. O.B.d.A. sei $\phi(0)=0$ und damit $\frac{\phi(a)}{a}\leq\frac{\phi(b)}{b}$.



\end{proof}
\section*{Aufgabe 3}
Sei $(X_n)_{n\in\N}$ der asymmetrische einfache Random Walk mit $\P(\xi_1=1)=p$ und $\P(\xi_1=-1)=q$, vgl. Aufgabenblatt 2, Aufgabe 2.
\begin{enumerate}[label=\alph*)]
\item Definiere für $A,B\in\N$ durch
\begin{align*}
\tau_A:=\inf\big\lbrace n\in\N_0:X_n=A\big\rbrace,\qquad \tau_B:=\inf\limits\big\lbrace n\in\N_0:X_n=-B\big\rbrace
\end{align*}
die ersten Treffzeiten von $A$ und $-B$.\\
Berechnen Sie $\P[\tau_A<\tau_B]$.
\item Zeige mit Hilfe einer geeigneten Maximalungleichung im Fall $q>p$:
\begin{align}\label{eqAufgabe3bObere}
\P\left[\sup\limits_{n\in\N_0} X_n\geq k\right]\leq\left(\frac{p}{q}\right)^k
\end{align}
und 
\begin{align}\label{eqAufgabe3bUntere}
\E\left[\sup\limits_{n\in\N_0} X_n\right]\leq\frac{q}{q-p}
\end{align}
\end{enumerate}
\begin{proof}
\underline{Zu a):}\\

\underline{Zu b), zeige \eqref{eqAufgabe3bObere}:}\\

\underline{Zu b), zeige \eqref{eqAufgabe3bUntere}:}\\

\end{proof}

\section*{Aufgabe 4}
Zeige mit Hilfe der Jensenschen Ungleichung
\begin{align}\label{eqJensenscheUngleichung}
f\big(\E[X]\big)\leq\E\big[f(X)\big]\qquad\forall X\text{ integrierbar und $f$ konvex},
\end{align}
dem Martingalkonvergenzsatz in $L_1$ und Doob's Maximalungleichung den Konvergenzsatz für $L^p$-beschränkte Martingale:

\begin{theorem}[Konvergenzsatz für $L^p$-beschränkte Martingale]\enter
Sei $p>1,B>0$ und $(M_n)_{n\in\N_0}$ ein Martingal mit 
\begin{align*}
\E\left[|M_n|^p\right]\leq B<\infty.
\end{align*}
Dann existiert eine Zufallsvariable $M_\infty$ mit $\E\left[|M_\infty|^p\right]\leq B$ so, dass
\begin{align*}
\P\left[\limn M_n=M_\infty\right]=1
\qquad\text{und}\qquad
\limn\big\Vert M_n-M_\infty\big\Vert=0
\end{align*}
\end{theorem}
\begin{proof}

\end{proof}


\end{document}
