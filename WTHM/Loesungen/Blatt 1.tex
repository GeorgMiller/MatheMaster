% This work is licensed under the Creative Commons
% Attribution-NonCommercial-ShareAlike 4.0 International License. To view a copy
% of this license, visit http://creativecommons.org/licenses/by-nc-sa/4.0/ or
% send a letter to Creative Commons, PO Box 1866, Mountain View, CA 94042, USA.

\documentclass[12pt,a4paper]{article} 

% This work is licensed under the Creative Commons
% Attribution-NonCommercial-ShareAlike 4.0 International License. To view a copy
% of this license, visit http://creativecommons.org/licenses/by-nc-sa/4.0/ or
% send a letter to Creative Commons, PO Box 1866, Mountain View, CA 94042, USA.

% PACKAGES
\usepackage[english, ngerman]{babel}	% Paket für Sprachselektion, in diesem Fall für deutsches Datum etc
\usepackage[utf8]{inputenc}	% Paket für Umlaute; verwende utf8 Kodierung in TexWorks 
\usepackage[T1]{fontenc} % ö,ü,ä werden richtig kodiert
\usepackage{amsmath} % wichtig für align-Umgebung
\usepackage{amssymb} % wichtig für \mathbb{} usw.
\usepackage{amsthm} % damit kann man eigene Theorem-Umgebungen definieren, proof-Umgebungen, etc.
\usepackage{mathrsfs} % für \mathscr
\usepackage[backref]{hyperref} % Inhaltsverzeichnis und \ref-Befehle werden in der PDF-klickbar
\usepackage[english, ngerman, capitalise]{cleveref}
\usepackage{graphicx}
\usepackage{grffile}
\usepackage{setspace} % wichtig für Lesbarkeit. Schöne Zeilenabstände

\usepackage{enumitem} % für custom Liste mit default Buchstaben
\usepackage{ulem} % für bessere Unterstreichung
\usepackage{contour} % für bessere Unterstreichung
\usepackage{epigraph} % für das coole Zitat

\usepackage{tikz}

% This work is licensed under the Creative Commons
% Attribution-NonCommercial-ShareAlike 4.0 International License. To view a copy
% of this license, visit http://creativecommons.org/licenses/by-nc-sa/4.0/ or
% send a letter to Creative Commons, PO Box 1866, Mountain View, CA 94042, USA.

% THEOREM-ENVIRONMENTS

\newtheoremstyle{mystyle}
  {20pt}   % ABOVESPACE \topsep is default, 20pt looks nice
  {20pt}   % BELOWSPACE \topsep is default, 20pt looks nice
  {\normalfont} % BODYFONT
  {0pt}       % INDENT (empty value is the same as 0pt)
  {\bfseries} % HEADFONT
  {}          % HEADPUNCT (if needed)
  {5pt plus 1pt minus 1pt} % HEADSPACE
	{}          % CUSTOM-HEAD-SPEC
\theoremstyle{mystyle}

% Definitionen der Satz, Lemma... - Umgebungen. Der Zähler von "satz" ist dem "section"-Zähler untergeordnet, alle weiteren Umgebungen bedienen sich des satz-Zählers.
\newtheorem{satz}{Satz}[section]
\newtheorem{lemma}[satz]{Lemma}
\newtheorem{korollar}[satz]{Korollar}
\newtheorem{proposition}[satz]{Proposition}
\newtheorem{beispiel}[satz]{Beispiel}
\newtheorem{definition}[satz]{Definition}
\newtheorem{bemerkungnr}[satz]{Bemerkung}
\newtheorem{theorem}[satz]{Theorem}

% Bemerkungen, Erinnerungen und Notationshinweise werden ohne Numerierungen dargestellt.
\newtheorem*{bemerkung}{Bemerkung.}
\newtheorem*{erinnerung}{Erinnerung.}
\newtheorem*{notation}{Notation.}
\newtheorem*{aufgabe}{Aufgabe.}
\newtheorem*{lösung}{Lösung.}
\newtheorem*{beisp}{Beispiel.} %Beispiel ohne Nummerierung
\newtheorem*{defi}{Definition.} %Definition ohne Nummerierung
\newtheorem*{lem}{Lemma.} %Lemma ohne Nummerierung


% SHORTCUTS
\newcommand{\R}{\mathbb{R}}				 % reelle Zahlen
\newcommand{\Rn}{\R^n}						 % der R^n
\newcommand{\N}{\mathbb{N}}				 % natürliche Zahlen
\newcommand{\Z}{\mathbb{Z}}				 % ganze Zahlen
\newcommand{\C}{\mathbb{C}}			   % komplexe Zahlen
\newcommand{\gdw}{\Leftrightarrow} % Genau dann, wenn
\newcommand{\with}{\text{ mit }}   % mit
\newcommand{\falls}{\text{falls }} % falls
\newcommand{\dd}{\text{ d}}        % Differential d

% ETWAS SPEZIELLERE ZEICHEN
%disjoint union
\newcommand{\bigcupdot}{
	\mathop{\vphantom{\bigcup}\mathpalette\setbigcupdot\cdot}\displaylimits
}
\newcommand{\setbigcupdot}[2]{\ooalign{\hfil$#1\bigcup$\hfil\cr\hfil$#2$\hfil\cr\cr}}
%big times
\newcommand*{\bigtimes}{\mathop{\raisebox{-.5ex}{\hbox{\huge{$\times$}}}}} 

% WHITESPACE COMMANDS
%non-restrict newline command
\newcommand{\enter}{$ $\newline} 
%praktischer Tabulator
\newcommand\tab[1][1cm]{\hspace*{#1}}

% TEXT ÜBER ZEICHEN
%das ist ein Gleichheitszeichen mit Text darüber, Beispiel: $a\stackeq{Def} b$
\newcommand{\stackeq}[1]{
	\mathrel{\stackrel{\makebox[0pt]{\mbox{\normalfont\tiny #1}}}{=}}
} 
%das ist ein beliebiges Zeichen mit Text darüber, z. B.  $a\stackrel{Def}{\Rightarrow} b$
\newcommand{\stacksymbol}[2]{
	\mathrel{\stackrel{\makebox[0pt]{\mbox{\normalfont\tiny #1}}}{#2}}
} 

% UNDERLINE
% besseres underline 
\renewcommand{\ULdepth}{1pt}
\contourlength{0.5pt}
\newcommand{\ul}[1]{
	\uline{\phantom{#1}}\llap{\contour{white}{#1}}
}


% hier noch ein paar Commands die nur ich nutze, weil ich sie mir im Laufe der Jahre angewöhnt habe und sie mir jetzt nicht abgewöhnen will:

\newcommand{\gdw}{\Leftrightarrow}   % genau dann, wenn



% Commands für Stochastik / Statistik
\newcommand{\A}{\mathcal{A}}
\renewcommand{\P}{\mathbb{P}}
\newcommand{\E}{\mathbb{E}}




% This work is licensed under the Creative Commons
% Attribution-NonCommercial-ShareAlike 4.0 International License. To view a copy
% of this license, visit http://creativecommons.org/licenses/by-nc-sa/4.0/ or
% send a letter to Creative Commons, PO Box 1866, Mountain View, CA 94042, USA.

% Commands für WTHM
\newcommand{\F}{\mathcal{F}}				% Standard-Unter-Sigma-Algebra / Filtration
\newcommand{\G}{\mathcal{G}}				% Gegenstück zu \F für Rückwärtsmartingale

% independent symbol from:
% https://tex.stackexchange.com/questions/79434/double-perpendicular-symbol-for-independence
\newcommand{\unab}{\protect\mathpalette{\protect\independenT}{\perp}}
\def\independenT#1#2{\mathrel{\rlap{$#1#2$}\mkern2mu{#1#2}}}

\newcommand{\BedE}[2]{\E\left[{#1}~|~{#2}\right]} % Bedingte Erwartung
\newcommand{\graph}{\text{graph}}

%\newcommand{\binom}[2]{\begin{pmatrix}	{#1}\\{#2} \end{pmatrix}} %this command doesn't compile for me




\author{Willi Sontopski}

\parindent0cm %Ist wichtig, um führende Leerzeichen zu entfernen

\usepackage{color}

\usepackage{scrpage2}
\pagestyle{scrheadings}
\clearscrheadfoot

\ihead{Willi Sontopski}
\chead{WTHM WiSe 18 19}
\ohead{}
\ifoot{Aufgabenblatt 1}
\cfoot{Version: \today}
\ofoot{Seite \pagemark}

\begin{document}
%\setcounter{section}{1}

\section*{Aufgabe 1}
Sei
\begin{align*}
L_0^2:=\big\lbrace X\in L_2:\E[X]=0\big\rbrace
\end{align*}
der Raum der zentrierten quadrat-integrierbaren Zufallsvariablen. Für diesen gilt:
\begin{enumerate}[label=\alph*)]
\item $L_0^2$ ist ein Hilbertraum mit Skalarprodukt $\Cov(\cdot,\cdot)$.
\item Für eine Familie $\lbrace X_j\rbrace_{j\in J}\subseteq L_0^2$ gilt
\begin{align*}
\Var\left[\sum\limits_{j\in J}X_j\right]=\sum\limits_{j\in J}\Var[X_j]+2\cdot\sum\limits_{j<k}\Cov[X_j,X_k]
\end{align*}
\item Für eine Familie $\lbrace X_j\rbrace_{j\in J}\subseteq L_0^2$ gilt
\begin{align*}
\sqrt{\Var\left[\sum\limits_{j\in J}X_j\right]}\leq\sum\limits_{j\in J}\sqrt{\Var[X_j]}
\end{align*}
\end{enumerate}
\begin{proof}
\underline{Zeige a):}\\
%UVR-Kriterium + Eigenschaften des Skalarproduktes nachrechnen + Vollständigkeit

\underline{Zeige b):}\\

\underline{Zeige c):}\\
\end{proof}

\section*{Aufgabe 2}
Seien $X,Y\in L_2(\A)$ und $\F\subseteq\A$ Unter-$\sigma$-Algebra von $\A$. \textbf{Bedingte Varianz} und \textbf{bedingte Kovarianz} von $X$ bzw. $X,Y$ unter $\F$ sind definiert als
\begin{align*}
\Var[X~|~\F]&:=\E\Big[\big(X-\E[X~|~\F\big)^2~\Big|~\F\Big]\\
\Cov[X,Y~|~\F]&:=\E\Big[\big(X-\E[X~|~\F]\big)\cdot\big(Y-\E[Y~|~\F]\big)~\Big|~\F\Big]
\end{align*}
Dann gelten die Sätze der \textbf{totalen Varianz} bzw. \textbf{totalen Kovarianz}:
\begin{align*}
\Var[X]&=\E\big[\Var[X~|~\F]\big]+\Var\big[\E[X~|~\F]\big]\\
\Cov[X,Y]&=\E\big[\Cov[X,Y~|~\F]\big]+\Cov\big[\E[X~|~F],\E[Y~|~\F]\big]
\end{align*}
\begin{proof}
\underline{Zur ersten Gleichung:}\\

\underline{Zur zweiten Gleichung:}\\
\end{proof}

\section*{Aufgabe 3}
Sei $M=\begin{pmatrix}
A & B\\ C & D
\end{pmatrix}$ eine quadratische, in Blöcke unterteilte Matrix von vollem Rang, wobei $A$ und $D$ ebenfalls quadratisch und von vollem Rang seien. Die Ausdrücke
\begin{align*}
(M/A):=\left(D-C\cdot A^{-1}\cdot B\right),\qquad(M/D):=\left(A-B\cdot D^{-1}\cdot C\right)
\end{align*}
heißen \textbf{Schurkomplement} von $A$ in $M$ bzw. $D$ in $M$. Dann giltt:
\begin{enumerate}[label=\alph*)]
\item $\begin{aligned}
M^{-1}=\begin{pmatrix}
(M/D)^{-1} & -A^{-1}\cdot B\cdot(M/A)^{-1}\\
-D^{-1}\cdot C\cdot(M/D)^{-1} & (M/A)^{-1}
\end{pmatrix}
\end{aligned}$
\item Sei $M$ nun symmetrisch, d.h. $A$ und $D$ sind symmetrisch und $C=B^T$. Dann gilt:
\begin{align*}
(x^T,y^T)\cdot M^{-1}\cdot\begin{pmatrix}
x\\y
\end{pmatrix}-y^T\cdot D^{-1}\cdot y=\tilde{x}^T\cdot(M/D)^{-1}\cdot\tilde{x}
\end{align*}
mit $\tilde{x}=\left(x-B\cdot D^{-1}\cdot y\right)$ für alle $x,y$ mit passender Dimension.
\item Es sei $(X,Y)$ multivariat normalverteilt mit Erwartungswert 0 und positiv definiter Kovarianzmatrix $\Sigma=\begin{pmatrix}
\Sigma_X & \Sigma_{XY}\\ \Sigma_{XY}^T & \Sigma_Y
\end{pmatrix}$. Dann ist $X$ bedingt auf $Y$ normalverteilt mit $\E[X~|~Y]=\Sigma_{XY}\cdot\Sigma_{Y}^{-1}$ und Kovarianzmatrix $(\Sigma/\Sigma_Y)$\\
Hinweis: es gilt $\det(\Sigma)=\det(\Sigma_Y)\cdot\det(\Sigma/\Sigma_Y)$.
\end{enumerate}
\begin{proof}
\underline{Zeige a):}\\

\underline{Zeige b):}\\

\underline{Zeige c):}\\
\end{proof}

\section*{Aufgabe 4}
In einer bestimmten Population sei das Alter $X$ bei erstmaliger
Berufsunfähigkeit exponentialverteilt mit Parameter $\lambda>0$. Für eine Versicherungsgesellschaft die gegen Berufsunfähigkeit versichert ist das mittlere Alter bei Eintritt der Berufsunfähigkeit von Bedeutung, unter der Bedingung dass die Berufsunfähigkeit zwischen den Altersgrenzen $0\leq a\leq b$ eintritt.\\
Bestimme diesen bedingten Erwartungswert $\E[X~|~a\leq X\leq b]$
\begin{proof}
\begin{align*}
\E[X~|~a\leq X\leq b]&=
\end{align*}
\end{proof}

\section*{Aufgabe 5}
Welche der folgenden in der Vorlesung definierten mathematischen
Objekte sind: reelle Zahlen, Zufallsvariablen, meßbare Funktionen von
$\R\to\R$?
\begin{align*}
\E[X~|~\F],\qquad\P[A~|~B],\qquad \E[X~|~Y=y],\qquad\P[A~|~\F],\qquad\E[X~|~Y]
\end{align*}
Wie üblich bezeichnet $\F$ eine $\sigma$-Algebra, $X,Y$  Zufallsvariablen, $y$ eine reelle Zahl und $A,B$ Ereignisse.
\begin{lösung}
\begin{align*}
&\E[X~|~\F]\in L_2(\Omega,\F,\P)\text{, ist also eine Zufallsvariable}\\
&\P[A~|~B]\in[0,1]\subseteq\R\\
&\E[X~|~Y=y]\in?\\
&\P[A~|~\F]\in?\\
&\E[X~|~Y]:=\E[X~|~\sigma(Y)]\in L_2(\Omega,\F,\P)\text{, ist also ein Spezialfall der ersten Zeile}
\end{align*}
\end{lösung}
\end{document}