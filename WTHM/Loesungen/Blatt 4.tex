% This work is licensed under the Creative Commons
% Attribution-NonCommercial-ShareAlike 4.0 International License. To view a copy
% of this license, visit http://creativecommons.org/licenses/by-nc-sa/4.0/ or
% send a letter to Creative Commons, PO Box 1866, Mountain View, CA 94042, USA.

\documentclass[12pt,a4paper]{article} 

% This work is licensed under the Creative Commons
% Attribution-NonCommercial-ShareAlike 4.0 International License. To view a copy
% of this license, visit http://creativecommons.org/licenses/by-nc-sa/4.0/ or
% send a letter to Creative Commons, PO Box 1866, Mountain View, CA 94042, USA.

% PACKAGES
\usepackage[english, ngerman]{babel}	% Paket für Sprachselektion, in diesem Fall für deutsches Datum etc
\usepackage[utf8]{inputenc}	% Paket für Umlaute; verwende utf8 Kodierung in TexWorks 
\usepackage[T1]{fontenc} % ö,ü,ä werden richtig kodiert
\usepackage{amsmath} % wichtig für align-Umgebung
\usepackage{amssymb} % wichtig für \mathbb{} usw.
\usepackage{amsthm} % damit kann man eigene Theorem-Umgebungen definieren, proof-Umgebungen, etc.
\usepackage{mathrsfs} % für \mathscr
\usepackage[backref]{hyperref} % Inhaltsverzeichnis und \ref-Befehle werden in der PDF-klickbar
\usepackage[english, ngerman, capitalise]{cleveref}
\usepackage{graphicx}
\usepackage{grffile}
\usepackage{setspace} % wichtig für Lesbarkeit. Schöne Zeilenabstände

\usepackage{enumitem} % für custom Liste mit default Buchstaben
\usepackage{ulem} % für bessere Unterstreichung
\usepackage{contour} % für bessere Unterstreichung
\usepackage{epigraph} % für das coole Zitat

\usepackage{tikz}

% This work is licensed under the Creative Commons
% Attribution-NonCommercial-ShareAlike 4.0 International License. To view a copy
% of this license, visit http://creativecommons.org/licenses/by-nc-sa/4.0/ or
% send a letter to Creative Commons, PO Box 1866, Mountain View, CA 94042, USA.

% THEOREM-ENVIRONMENTS

\newtheoremstyle{mystyle}
  {20pt}   % ABOVESPACE \topsep is default, 20pt looks nice
  {20pt}   % BELOWSPACE \topsep is default, 20pt looks nice
  {\normalfont} % BODYFONT
  {0pt}       % INDENT (empty value is the same as 0pt)
  {\bfseries} % HEADFONT
  {}          % HEADPUNCT (if needed)
  {5pt plus 1pt minus 1pt} % HEADSPACE
	{}          % CUSTOM-HEAD-SPEC
\theoremstyle{mystyle}

% Definitionen der Satz, Lemma... - Umgebungen. Der Zähler von "satz" ist dem "section"-Zähler untergeordnet, alle weiteren Umgebungen bedienen sich des satz-Zählers.
\newtheorem{satz}{Satz}[section]
\newtheorem{lemma}[satz]{Lemma}
\newtheorem{korollar}[satz]{Korollar}
\newtheorem{proposition}[satz]{Proposition}
\newtheorem{beispiel}[satz]{Beispiel}
\newtheorem{definition}[satz]{Definition}
\newtheorem{bemerkungnr}[satz]{Bemerkung}
\newtheorem{theorem}[satz]{Theorem}

% Bemerkungen, Erinnerungen und Notationshinweise werden ohne Numerierungen dargestellt.
\newtheorem*{bemerkung}{Bemerkung.}
\newtheorem*{erinnerung}{Erinnerung.}
\newtheorem*{notation}{Notation.}
\newtheorem*{aufgabe}{Aufgabe.}
\newtheorem*{lösung}{Lösung.}
\newtheorem*{beisp}{Beispiel.} %Beispiel ohne Nummerierung
\newtheorem*{defi}{Definition.} %Definition ohne Nummerierung
\newtheorem*{lem}{Lemma.} %Lemma ohne Nummerierung


% SHORTCUTS
\newcommand{\R}{\mathbb{R}}				 % reelle Zahlen
\newcommand{\Rn}{\R^n}						 % der R^n
\newcommand{\N}{\mathbb{N}}				 % natürliche Zahlen
\newcommand{\Z}{\mathbb{Z}}				 % ganze Zahlen
\newcommand{\C}{\mathbb{C}}			   % komplexe Zahlen
\newcommand{\gdw}{\Leftrightarrow} % Genau dann, wenn
\newcommand{\with}{\text{ mit }}   % mit
\newcommand{\falls}{\text{falls }} % falls
\newcommand{\dd}{\text{ d}}        % Differential d

% ETWAS SPEZIELLERE ZEICHEN
%disjoint union
\newcommand{\bigcupdot}{
	\mathop{\vphantom{\bigcup}\mathpalette\setbigcupdot\cdot}\displaylimits
}
\newcommand{\setbigcupdot}[2]{\ooalign{\hfil$#1\bigcup$\hfil\cr\hfil$#2$\hfil\cr\cr}}
%big times
\newcommand*{\bigtimes}{\mathop{\raisebox{-.5ex}{\hbox{\huge{$\times$}}}}} 

% WHITESPACE COMMANDS
%non-restrict newline command
\newcommand{\enter}{$ $\newline} 
%praktischer Tabulator
\newcommand\tab[1][1cm]{\hspace*{#1}}

% TEXT ÜBER ZEICHEN
%das ist ein Gleichheitszeichen mit Text darüber, Beispiel: $a\stackeq{Def} b$
\newcommand{\stackeq}[1]{
	\mathrel{\stackrel{\makebox[0pt]{\mbox{\normalfont\tiny #1}}}{=}}
} 
%das ist ein beliebiges Zeichen mit Text darüber, z. B.  $a\stackrel{Def}{\Rightarrow} b$
\newcommand{\stacksymbol}[2]{
	\mathrel{\stackrel{\makebox[0pt]{\mbox{\normalfont\tiny #1}}}{#2}}
} 

% UNDERLINE
% besseres underline 
\renewcommand{\ULdepth}{1pt}
\contourlength{0.5pt}
\newcommand{\ul}[1]{
	\uline{\phantom{#1}}\llap{\contour{white}{#1}}
}


% hier noch ein paar Commands die nur ich nutze, weil ich sie mir im Laufe der Jahre angewöhnt habe und sie mir jetzt nicht abgewöhnen will:

\newcommand{\gdw}{\Leftrightarrow}   % genau dann, wenn



% Commands für Stochastik / Statistik
\newcommand{\A}{\mathcal{A}}
\renewcommand{\P}{\mathbb{P}}
\newcommand{\E}{\mathbb{E}}




% This work is licensed under the Creative Commons
% Attribution-NonCommercial-ShareAlike 4.0 International License. To view a copy
% of this license, visit http://creativecommons.org/licenses/by-nc-sa/4.0/ or
% send a letter to Creative Commons, PO Box 1866, Mountain View, CA 94042, USA.

% Commands für WTHM
\newcommand{\F}{\mathcal{F}}				% Standard-Unter-Sigma-Algebra / Filtration
\newcommand{\G}{\mathcal{G}}				% Gegenstück zu \F für Rückwärtsmartingale

% independent symbol from:
% https://tex.stackexchange.com/questions/79434/double-perpendicular-symbol-for-independence
\newcommand{\unab}{\protect\mathpalette{\protect\independenT}{\perp}}
\def\independenT#1#2{\mathrel{\rlap{$#1#2$}\mkern2mu{#1#2}}}

\newcommand{\BedE}[2]{\E\left[{#1}~|~{#2}\right]} % Bedingte Erwartung
\newcommand{\graph}{\text{graph}}

%\newcommand{\binom}[2]{\begin{pmatrix}	{#1}\\{#2} \end{pmatrix}} %this command doesn't compile for me




\author{Willi Sontopski}

\parindent0cm %Ist wichtig, um führende Leerzeichen zu entfernen

\usepackage{scrpage2}
\pagestyle{scrheadings}
\clearscrheadfoot

\ihead{Willi Sontopski \& Robert Walter}
\chead{}
\ohead{WTHM WiSe 18 19}
\ifoot{Aufgabenblatt 4}
\cfoot{Version: \today}
\ofoot{Seite \pagemark}

\begin{document}
\section*{Aufgabe 1}
Sei $(X_n)_{n\in\N}$ eine Folge von iid Zufallsvariablen in $L_1$ und 
\begin{align*}
R_n:=\frac{1}{n}\cdot\sum\limits_{j=0}^n X_j,\qquad
\G_n:=\sigma\big(R_n,R_{n+1},R_{n+2},\ldots\big)
\end{align*}
Dann gilt:
\begin{enumerate}[label=\alph*)]
\item $\G_n=\sigma\big(R_n,X_{n+1},X_{n+2},\ldots\big)$
\item $(R_n)_{n\in\N}$ ist Rückwärtsmartingal bzgl. $(\G_n)_{n\in\N}$
\item Es gilt das starke Gesetz der großen Zahlen:
\begin{align*}
\limn R_n=\E[X_1]\text{ f.s.}
\end{align*}
\end{enumerate}

\begin{proof}
\underline{Zeige a):}\\
%TODO
\underline{Zeige b):}\\
%TODO
\underline{Zeige c):}\\
Wir verwenden hier das 0-1-Gesetz von Kolmogorov.
%TODO
\end{proof}

\section*{Aufgabe 2}
Wir berechnen die charakteristische Funktionen folgender Verteilungen:
\begin{enumerate}[label=a)]
\item Binomial-, Poisson- und geometrische Verteilung
\item Exponential-, Laplace- und Cauchyverteilung
\end{enumerate}

\begin{lösung}
\underline{Zur Binomialverteilung:}\\
%TODO
\underline{Zur Poissonverteilung:}\\
%TODO
\underline{Zur geometrischen Verteilung:}\\
%TODO
\underline{Zur Exponentialverteilung:}\\
%TODO
\underline{Zur Laplaceverteilung:}\\
%TODO
\underline{Zur Cauchyverteilung:}\\
%TODO
\end{lösung}

\section*{Aufgabe 3}
Seien $X$ und $X'$ unabhängige $\R^d$-wertige Zufallsvariablen mit der gleichen Verteilung $F$.
\begin{enumerate}[label=\alph*)]
\item Berechne die charakteristische Funktion von $Y=X-X'$
\item Sei $B$ unabhängig von $(X,X')$ mit $\P(B=0)=\P(B=1)=\frac{1}{2}$. Berechne die charakteristische Funktion von $Z=B\cdot X+(B-1)\cdot X'$.
\end{enumerate}
Sind die Zufallsvariablen $Y$ und $Z$ symmetrisch?

\begin{lösung}
\underline{Zu a):}\\
%TODO
\underline{Zu b):}\\
%TODO
\underline{Zur Symmetrie:}\\
%TODO
\end{lösung}

\section*{Aufgabe 4}
Sei $\alpha\in\N_0^d$ ein Multiindex, $b\in\R^d$ und $f,g$ Funktionen in $C^\alpha(\R^d)$.
\begin{enumerate}[label=\alph*)]
\item Berechne $\partial_x^\alpha\exp\big(b^T\cdot x\big)$
\item Zeige die Leibniz-Regel:
\begin{align*}
\partial^\alpha(f\cdot g)=\sum\limits_{\begin{subarray}{c}\beta\leq\alpha\\\beta\in\N_0^d\end{subarray}}\begin{pmatrix}
\alpha\\\beta
\end{pmatrix}\cdot\partial^\beta f\cdot\partial^{\alpha-\beta}g
\end{align*}
\end{enumerate}

\begin{lösung}
\underline{Zu a):}
\begin{align*}
\partial_x^\alpha\exp\big(b^T\cdot x\big)
&=
\end{align*}
%TODO
\underline{Zu b):}\\
Wir führen eine Induktion nach $|\alpha|\in\N$ durch:\nl
\ul{IA:} Sei $|\alpha|=0$ %TODO
\nl
\ul{IV:} Gelte die Leibniz-Regel für beliebiges, aber festes $\alpha\in\N_0^d$ mit $|\alpha|=n\in\N$.\nl
\ul{IS:}
%TODO
\end{lösung}

\section*{Aufgabe 5}
\begin{enumerate}[label=\alph*)]
\item Sei $(X,Y)$ bivariat normalverteilt. Zeige mit charakteristischen Funktionen: $X$ und $Y$ sind unabhängig genau dann, wenn sie unkorreliert sind.
\item Sei $X$ Cauchyverteilt und setze $Y=X$. Zeige, dass für die charakteristischen Funktionen gilt:
\begin{align*}
\Phi_{X+Y}(\xi)=\Phi_X(\xi)\cdot\Phi_Y(\xi)\qquad\forall\xi\in\R
\end{align*} 
obwohl $X$ offensichtlich \ul{nicht} unabhängig von $Y$ ist.
\end{enumerate}

\begin{lösung}
\underline{Zu a):}\\
%TODO
\underline{Zu b):}\\
%TODO
\end{lösung}
\end{document}
