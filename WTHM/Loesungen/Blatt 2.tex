% This work is licensed under the Creative Commons
% Attribution-NonCommercial-ShareAlike 4.0 International License. To view a copy
% of this license, visit http://creativecommons.org/licenses/by-nc-sa/4.0/ or
% send a letter to Creative Commons, PO Box 1866, Mountain View, CA 94042, USA.

\documentclass[12pt,a4paper]{article} 

% This work is licensed under the Creative Commons
% Attribution-NonCommercial-ShareAlike 4.0 International License. To view a copy
% of this license, visit http://creativecommons.org/licenses/by-nc-sa/4.0/ or
% send a letter to Creative Commons, PO Box 1866, Mountain View, CA 94042, USA.

% PACKAGES
\usepackage[english, ngerman]{babel}	% Paket für Sprachselektion, in diesem Fall für deutsches Datum etc
\usepackage[utf8]{inputenc}	% Paket für Umlaute; verwende utf8 Kodierung in TexWorks 
\usepackage[T1]{fontenc} % ö,ü,ä werden richtig kodiert
\usepackage{amsmath} % wichtig für align-Umgebung
\usepackage{amssymb} % wichtig für \mathbb{} usw.
\usepackage{amsthm} % damit kann man eigene Theorem-Umgebungen definieren, proof-Umgebungen, etc.
\usepackage{mathrsfs} % für \mathscr
\usepackage[backref]{hyperref} % Inhaltsverzeichnis und \ref-Befehle werden in der PDF-klickbar
\usepackage[english, ngerman, capitalise]{cleveref}
\usepackage{graphicx}
\usepackage{grffile}
\usepackage{setspace} % wichtig für Lesbarkeit. Schöne Zeilenabstände

\usepackage{enumitem} % für custom Liste mit default Buchstaben
\usepackage{ulem} % für bessere Unterstreichung
\usepackage{contour} % für bessere Unterstreichung
\usepackage{epigraph} % für das coole Zitat

\usepackage{tikz}

% This work is licensed under the Creative Commons
% Attribution-NonCommercial-ShareAlike 4.0 International License. To view a copy
% of this license, visit http://creativecommons.org/licenses/by-nc-sa/4.0/ or
% send a letter to Creative Commons, PO Box 1866, Mountain View, CA 94042, USA.

% THEOREM-ENVIRONMENTS

\newtheoremstyle{mystyle}
  {20pt}   % ABOVESPACE \topsep is default, 20pt looks nice
  {20pt}   % BELOWSPACE \topsep is default, 20pt looks nice
  {\normalfont} % BODYFONT
  {0pt}       % INDENT (empty value is the same as 0pt)
  {\bfseries} % HEADFONT
  {}          % HEADPUNCT (if needed)
  {5pt plus 1pt minus 1pt} % HEADSPACE
	{}          % CUSTOM-HEAD-SPEC
\theoremstyle{mystyle}

% Definitionen der Satz, Lemma... - Umgebungen. Der Zähler von "satz" ist dem "section"-Zähler untergeordnet, alle weiteren Umgebungen bedienen sich des satz-Zählers.
\newtheorem{satz}{Satz}[section]
\newtheorem{lemma}[satz]{Lemma}
\newtheorem{korollar}[satz]{Korollar}
\newtheorem{proposition}[satz]{Proposition}
\newtheorem{beispiel}[satz]{Beispiel}
\newtheorem{definition}[satz]{Definition}
\newtheorem{bemerkungnr}[satz]{Bemerkung}
\newtheorem{theorem}[satz]{Theorem}

% Bemerkungen, Erinnerungen und Notationshinweise werden ohne Numerierungen dargestellt.
\newtheorem*{bemerkung}{Bemerkung.}
\newtheorem*{erinnerung}{Erinnerung.}
\newtheorem*{notation}{Notation.}
\newtheorem*{aufgabe}{Aufgabe.}
\newtheorem*{lösung}{Lösung.}
\newtheorem*{beisp}{Beispiel.} %Beispiel ohne Nummerierung
\newtheorem*{defi}{Definition.} %Definition ohne Nummerierung
\newtheorem*{lem}{Lemma.} %Lemma ohne Nummerierung


% SHORTCUTS
\newcommand{\R}{\mathbb{R}}				 % reelle Zahlen
\newcommand{\Rn}{\R^n}						 % der R^n
\newcommand{\N}{\mathbb{N}}				 % natürliche Zahlen
\newcommand{\Z}{\mathbb{Z}}				 % ganze Zahlen
\newcommand{\C}{\mathbb{C}}			   % komplexe Zahlen
\newcommand{\gdw}{\Leftrightarrow} % Genau dann, wenn
\newcommand{\with}{\text{ mit }}   % mit
\newcommand{\falls}{\text{falls }} % falls
\newcommand{\dd}{\text{ d}}        % Differential d

% ETWAS SPEZIELLERE ZEICHEN
%disjoint union
\newcommand{\bigcupdot}{
	\mathop{\vphantom{\bigcup}\mathpalette\setbigcupdot\cdot}\displaylimits
}
\newcommand{\setbigcupdot}[2]{\ooalign{\hfil$#1\bigcup$\hfil\cr\hfil$#2$\hfil\cr\cr}}
%big times
\newcommand*{\bigtimes}{\mathop{\raisebox{-.5ex}{\hbox{\huge{$\times$}}}}} 

% WHITESPACE COMMANDS
%non-restrict newline command
\newcommand{\enter}{$ $\newline} 
%praktischer Tabulator
\newcommand\tab[1][1cm]{\hspace*{#1}}

% TEXT ÜBER ZEICHEN
%das ist ein Gleichheitszeichen mit Text darüber, Beispiel: $a\stackeq{Def} b$
\newcommand{\stackeq}[1]{
	\mathrel{\stackrel{\makebox[0pt]{\mbox{\normalfont\tiny #1}}}{=}}
} 
%das ist ein beliebiges Zeichen mit Text darüber, z. B.  $a\stackrel{Def}{\Rightarrow} b$
\newcommand{\stacksymbol}[2]{
	\mathrel{\stackrel{\makebox[0pt]{\mbox{\normalfont\tiny #1}}}{#2}}
} 

% UNDERLINE
% besseres underline 
\renewcommand{\ULdepth}{1pt}
\contourlength{0.5pt}
\newcommand{\ul}[1]{
	\uline{\phantom{#1}}\llap{\contour{white}{#1}}
}


% hier noch ein paar Commands die nur ich nutze, weil ich sie mir im Laufe der Jahre angewöhnt habe und sie mir jetzt nicht abgewöhnen will:

\newcommand{\gdw}{\Leftrightarrow}   % genau dann, wenn



% Commands für Stochastik / Statistik
\newcommand{\A}{\mathcal{A}}
\renewcommand{\P}{\mathbb{P}}
\newcommand{\E}{\mathbb{E}}




% This work is licensed under the Creative Commons
% Attribution-NonCommercial-ShareAlike 4.0 International License. To view a copy
% of this license, visit http://creativecommons.org/licenses/by-nc-sa/4.0/ or
% send a letter to Creative Commons, PO Box 1866, Mountain View, CA 94042, USA.

% Commands für WTHM
\newcommand{\F}{\mathcal{F}}				% Standard-Unter-Sigma-Algebra / Filtration
\newcommand{\G}{\mathcal{G}}				% Gegenstück zu \F für Rückwärtsmartingale

% independent symbol from:
% https://tex.stackexchange.com/questions/79434/double-perpendicular-symbol-for-independence
\newcommand{\unab}{\protect\mathpalette{\protect\independenT}{\perp}}
\def\independenT#1#2{\mathrel{\rlap{$#1#2$}\mkern2mu{#1#2}}}

\newcommand{\BedE}[2]{\E\left[{#1}~|~{#2}\right]} % Bedingte Erwartung
\newcommand{\graph}{\text{graph}}

%\newcommand{\binom}[2]{\begin{pmatrix}	{#1}\\{#2} \end{pmatrix}} %this command doesn't compile for me




\author{Willi Sontopski}

\parindent0cm %Ist wichtig, um führende Leerzeichen zu entfernen

\usepackage{color}

\usepackage{scrpage2}
\pagestyle{scrheadings}
\clearscrheadfoot

\ihead{Willi Sontopski}
\chead{WTHM WiSe 18 19}
\ohead{}
\ifoot{Aufgabenblatt 2}
\cfoot{Version: \today}
\ofoot{Seite \pagemark}

\begin{document}
%\setcounter{section}{1}

\section*{Aufgabe 1}
\subsection*{Aufgabe 1 a)}
Sei $(X_t)_{t\in\R_{\geq0}}$ ein Super-Martingal bzgl. einer Filtration $(\F_t)_{t\in\R_{\geq0}}$ mit\\ $\E[X_t]=$ const.\\
Dann ist $(X_t)_{t\in\R_{\geq0}}$ bereits ein Martingal.

\begin{proof}
	\begin{align*}
		X_s - \BedE{X_t}{\F_s} \stackrel{\text{Sub-MG}}{\geq} 0
	\end{align*}
	\begin{align*}
		\E\big[X_s - \BedE{X_t}{\F_s} \big]
		&= \E[X_s - X_t] \\
		&= c-c \\
		&= 0
	\end{align*}
\end{proof}

\subsection*{Aufgabe 1 b)}
Sei $(X_n)_{n\in\N}$ ein Martingal \underline{und} ein vorhersehbarer Prozess bzgl. der gleichen Filtration $(\F_n)_{n\in\N}$.\\
Dann ist $X_n=X_0$ fast sicher für alle $n\in\N$.

\begin{proof}
	Beweis per Induktion. 
	Induktionsaussage:
	\begin{align*}
		X_n = X_0 \quad \text{f.s.}
	\end{align*}
	Induktionsanfang für $n=0$ ist klar.
	Induktionsschritt:
	\begin{align*}
		X_n-X_0
		&= X_n - X_{n-1} \\
		&= X_n - \BedE{X_n}{\F_{n-1}} \\
		&= \BedE{X_n-X_n}{\F_{n-1}} \\
		&= 0
	\end{align*}

\end{proof}

\section*{Aufgabe 2}
\subsection*{Aufgabe 2 a)}
Sei $(\xi_j)_{j\in\N_{>0}}$ eine iid-Folge nichtnegativer Zufallsvariablen mit $\E[\xi_1]=1$. Dann ist der stochastische Prozess $M$ definiert durch
\begin{align*}
M_0:=1\qquad\text{ und }M_n=\prod\limits_{j=1}^n\xi\qquad\forall n\in\N_{>0}
\end{align*}
ein Martingal bzgl. der von $(\xi_j)_{j\in\N_{>0}}$ erzeugten Filtration  mit $\F_0:=\lbrace 0,\emptyset\rbrace$ und $\F_n:=\sigma(\xi_1,\ldots,\xi_n)$ für alle $n\in\N_{>0}$.

\begin{proof}
	Adaptiertheit ist klar. Integrierbarkeit:
	\begin{align*}
		\E[M_n]
		&=\E\Big[\prod_{i=1}^n \xi_i\Big] \\
		&\stackeq{\text{iid.}}\prod_{i=1}^n \E[\xi_1] \\
		&= 1 \\
		&<\infty
	\end{align*}
	MG-Eigenschaft:
	\begin{align*}
		\BedE{M_n}{\F_{n-1}}
		&= \BedE{\prod_{i=1}^n \xi_i}{\F_{n-1}} \\
		&= \BedE{M_{n-1} \cdot \xi_n}{\F_{n-1}} \\
		&= \E[\xi_n] \BedE{M_{n-1}}{\F_{n-1}} \\
		&= 1 \cdot M_{n-1}
	\end{align*}

\end{proof}

\subsection*{Aufgabe 2 b)}
Sei $(S_n)_{n\in\N}$ der \textbf{asymmetrische} einfache Random Walk, d. h. $S_n=\xi_1+\ldots+\xi_n$ mit $\P[\xi_1=1]=p$ und $\P[\xi_1=-1]=1-p=:q$.\\
Dann ist 
\begin{align*}
M_n:=\left(\frac{q}{p}\right)^{S_n}
\end{align*}
für jedes $p\in (0,1)\setminus\left\lbrace\frac{1}{2}\right\rbrace$ ein Martingal bzgl. $\F_n=\sigma(\xi_1,\ldots,\xi_n)$.

\begin{proof}
	Adaptiertheit: \enter
	Betrachte die Funktion
	\begin{align*}
		f(x):= \exp\Big(x \log\Big(\frac{q}{p}\Big)\Big)
	\end{align*}
	Diese ist offensichtlich messbar und da $S_n$ offensichtlich $\F_n$ messbar ist, ist auch
	$f(S_n)$ messbar bezüglich $\F_n$ für alle $n\in\N$.\enter

	Integrierbarkeit:
	\begin{align*}
		\E[M_n]
		&= \E\Big[\Big(\frac{q}{p}\Big)^{S_n}\Big] \\
		&= \E\Big[\prod_{i=1}^{S_n}\Big(\frac{q}{p}\Big)\Big] \\
		&\leq\begin{cases}
		\E\Big[\prod_{i=1}^{n}\Big(\frac{q}{p}\Big)\Big] &,\frac{q}{p}\geq 1 \\
		1 &,\frac{q}{p} < 1 
	\end{cases} \\
		&=\begin{cases}
		\Big(\frac{q}{p}\Big)^n &,\frac{q}{p}\geq 1 \\
		1 &,\frac{q}{p} < 1 
	\end{cases} \\
	&< \infty
	\end{align*}

	MG-Eigenschaft:
	\begin{align*}
		\BedE{M_n}{F_{n-1}}
		&=\E\Big[M_{n-1}\Big(\frac{q}{p}\Big)^{\xi_n}~|~\F_{n-1}\Big] \\
		&=\E\Big[\Big(\frac{q}{p}\Big)^{\xi_1}\Big] M_{n-1} \\
		&=\Big(\Big(\frac{q}{p}\Big)^1\cdot p + \Big(\frac{q}{p}\Big)^{-1}\cdot q\Big) M_{n-1} \\
		&= (p+q) M_{n-1} \\
		&= M_{n-1}
	\end{align*}

\end{proof}

\subsection*{Aufgabe 2 c)}
\underline{Kompensator von $M_n$ aus a):}\\

\underline{Kompensator von $M_n$ aus b):}\\

\section*{Aufgabe 3}
Sei $(\xi_n)_{n\in\N}$ eine Folge von iid-Zufallsvariablen und $\xi_1$ normalverteilt mit Mittelwert 0 und Varianz $\sigma^2>0$. Weiter sei $\F_n:=\sigma(\xi_1,\ldots,\xi_n)$ und $X_n:=\xi_1+\ldots+\xi_n$

\subsection*{Aufgabe 3 a)}
\begin{align*}
\E\big[\exp(u\cdot\xi_1)\big]=\exp\left(u^2\cdot\frac{\sigma^2}{2}\right)\qquad\forall u\in\R
\end{align*}

\begin{proof}
Sei $u\in\R$ beliebig.
\begin{align*}
\E\big[\exp(u\cdot\xi_1)\big]
&=c\int_\R e^{ux} e^{-\frac{x^2}{2\sigma^2}}\d x\\
&=c\int_\R e^{-\big(\frac{u\sigma}{\sqrt{2}}-\frac{x}{\sqrt{2}\sigma}\big)^2} e^{\frac{(u\sigma)^2}{2}}\d x\\
&=c \cdot e^{\frac{(u\sigma)^2}{2}}\int_\R e^{-\frac{(u\sigma^2-x)^2}{2\sigma^2}}\d x\\
&\stackeq{\text{Subst.}}e^{\frac{(u\sigma)^2}{2}} \cdot \underbrace{\tilde{c} \int_\R e^{-\frac{y^2}{2\sigma^2}}\d y}_{=1 ~~~ \text{Gauß Int.}}\\
&=\exp\left(u^2\cdot\frac{\sigma^2}{2}\right)
\end{align*}
\end{proof}

\subsection*{Aufgabe 3 b)}
Sei $u\in\R$. Dann ist
\begin{align*}
Z_n^u:=\exp\left(u\cdot X_n-n\cdot u^2\cdot\frac{\sigma^2}{2}\right)\qquad\forall n\in\N
\end{align*}
ein Martingal.

\begin{proof}
\begin{align*}
	e^{uS_n - nu^2\frac{\sigma^2}{2}}
	&= \frac{e^{uS_n}}{e^{nu^2\frac{\sigma^2}{2}}} \\
	&=\prod_{i=1}^n \frac{e^{u\xi_i}}{e^{nu^2\frac{\sigma^2}{2}}} \\
	&=\frac{1}{e^{nu^2\frac{\sigma^2}{2}}}\prod_{i=1}^n e^{u\xi_i}
\end{align*}
Adaptiertheit ist klar. Integrierbarkeit:
\begin{align*}
	\E[Z^u_n] 
	&\stackrel{\text{iid.}}{=}\frac{1}{e^{nu^2\frac{\sigma^2}{2}}}\prod_{i=1}^n \E e^{u\xi_1} \\
	&=\frac{1}{e^{nu^2\frac{\sigma^2}{2}}}\prod_{i=1}^n e^{u^2\frac{\sigma^2}{2}} \\
	&=\frac{1}{e^{nu^2\frac{\sigma^2}{2}}} e^{u^2n\frac{\sigma^2}{2}} \\
	&=1 \\
	&<\infty
\end{align*}
MG-Eigenschaft:
\begin{align*}
	\BedE{Z_n^u}{\F_{n-1}}
	&=\BedE{\frac{1}{e^{nu^2\frac{\sigma^2}{2}}}e^{u\xi_n}Z^u_{n-1}}{\F_{n-1}}\\
	&=\E\left[\frac{1}{e^{nu^2\frac{\sigma^2}{2}}}e^{u\xi_n}\right]Z^u_{n-1}\\
	&\stackeq{(a)} Z^u_{n-1}
\end{align*}
\end{proof}

\section*{Aufgabe 4}
\begin{defi}
Eine Zufallsvariable $\tau:\Omega\to\N_0\sup\lbrace+\infty\rbrace$ heißt \textbf{Stoppzeit} bzgl. der Filtration $(\F_n)_{n\in\N}$
\begin{align*}
:\Longleftrightarrow\forall n\in\N_0:\big\lbrace\tau\leq n\big\rbrace:=\big\lbrace \omega\in\Omega:\tau(\omega)\leq n\big\rbrace\in\F_n
\end{align*}
\end{defi}
Seien nun Stoppzeiten $\tau,\sigma,\tau_1,\tau_2,\ldots$ gegeben. Dann gilt:
\begin{enumerate}[label=\alph*)]
\item $\lbrace\omega\in\Omega:\tau(\omega)=n\rbrace\in\F_n$
\item $\lbrace\omega\in\Omega:\tau(\omega)\geq n\rbrace\in\F_{n-1}$
\item $\begin{aligned}
C_n(\omega):=\indi_{[0,\tau(\omega)]}(n)
\end{aligned}$ ist messbar bzgl. $\F_{n-1}$
\item $\sigma\wedge\tau$ und $\sigma\vee\tau$ sind Stoppzeiten (Minimum und Maximum)
\item Sind $\sup\limits_j\tau_j$ und $\inf\limits_j\tau_j$ Stoppzeiten?
\end{enumerate}

\begin{proof}
\underline{Zu a):}\\
\begin{align*}
	&\{\tau \leq n\} \in \F_n \\
	\implies & \{\tau > n-1\} = \{\tau \leq n-1\}^C \in \F_{n-1} \subset \F_n \\
	\implies & \{\tau = n\} = \{\tau \leq n\} \cap \{\tau > n-1\} \in \F_n
	\end{align*}
\underline{Zu b):}\\
\begin{align*}
	\{\tau \geq n \} = \Big(\{\tau \leq n-1 \} \setminus \{\tau = n-1 \} \Big)^C \in \F_{n-1}
\end{align*}
\underline{Zu c):}\\
\begin{align*}
	&\indi_{[0,\tau(\omega)]}(n)=\indi_{\underbrace{\{n\leq \tau(\omega)\}}_{(b) \implies \in\F_{n-1}}} \\
	\implies &\indi_{[0,\tau(\omega)]}(n) \in \F_{n-1}
\end{align*}
\underline{Zu d):}\\
\begin{align*}
	&\{\sigma \wedge \tau \leq n \} = \underbrace{\{\sigma \leq n\}}_{\in \F_n} \cup \underbrace{\{\tau \leq n\}}_{\in\F_n} \in \F_n \\
	&\{\sigma \vee \tau \leq n \} = \underbrace{\{\sigma \leq n\}}_{\in \F_n} \cap \underbrace{\{\tau \leq n\}}_{\in\F_n} \in \F_n
\end{align*}
\underline{Zu e):}\\
Seien $\tau_{i\in\N}$ Stoppzeiten
\begin{align*}
	\{\sup_{i\in\N} \tau_{i\in\N} \leq n\} = \bigcap_{i\in\N}\underbrace{\{\tau_i \leq n\}}_{\in\F_n} \in \F_n
\end{align*}
\begin{align*}
	\{\inf_{i\in\N} \tau_{i\in\N} \leq n\} = \bigcap_{i\in\N}\underbrace{\{\tau_i \geq n\}}_{\in\F_n} \in \F_n
\end{align*}

\end{proof}

\end{document}
