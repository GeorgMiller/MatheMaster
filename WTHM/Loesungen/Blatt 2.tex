% This work is licensed under the Creative Commons
% Attribution-NonCommercial-ShareAlike 4.0 International License. To view a copy
% of this license, visit http://creativecommons.org/licenses/by-nc-sa/4.0/ or
% send a letter to Creative Commons, PO Box 1866, Mountain View, CA 94042, USA.

\documentclass[12pt,a4paper]{article} 

% This work is licensed under the Creative Commons
% Attribution-NonCommercial-ShareAlike 4.0 International License. To view a copy
% of this license, visit http://creativecommons.org/licenses/by-nc-sa/4.0/ or
% send a letter to Creative Commons, PO Box 1866, Mountain View, CA 94042, USA.

% PACKAGES
\usepackage[english, ngerman]{babel}	% Paket für Sprachselektion, in diesem Fall für deutsches Datum etc
\usepackage[utf8]{inputenc}	% Paket für Umlaute; verwende utf8 Kodierung in TexWorks 
\usepackage[T1]{fontenc} % ö,ü,ä werden richtig kodiert
\usepackage{amsmath} % wichtig für align-Umgebung
\usepackage{amssymb} % wichtig für \mathbb{} usw.
\usepackage{amsthm} % damit kann man eigene Theorem-Umgebungen definieren, proof-Umgebungen, etc.
\usepackage{mathrsfs} % für \mathscr
\usepackage[backref]{hyperref} % Inhaltsverzeichnis und \ref-Befehle werden in der PDF-klickbar
\usepackage[english, ngerman, capitalise]{cleveref}
\usepackage{graphicx}
\usepackage{grffile}
\usepackage{setspace} % wichtig für Lesbarkeit. Schöne Zeilenabstände

\usepackage{enumitem} % für custom Liste mit default Buchstaben
\usepackage{ulem} % für bessere Unterstreichung
\usepackage{contour} % für bessere Unterstreichung
\usepackage{epigraph} % für das coole Zitat

\usepackage{tikz}

% This work is licensed under the Creative Commons
% Attribution-NonCommercial-ShareAlike 4.0 International License. To view a copy
% of this license, visit http://creativecommons.org/licenses/by-nc-sa/4.0/ or
% send a letter to Creative Commons, PO Box 1866, Mountain View, CA 94042, USA.

% THEOREM-ENVIRONMENTS

\newtheoremstyle{mystyle}
  {20pt}   % ABOVESPACE \topsep is default, 20pt looks nice
  {20pt}   % BELOWSPACE \topsep is default, 20pt looks nice
  {\normalfont} % BODYFONT
  {0pt}       % INDENT (empty value is the same as 0pt)
  {\bfseries} % HEADFONT
  {}          % HEADPUNCT (if needed)
  {5pt plus 1pt minus 1pt} % HEADSPACE
	{}          % CUSTOM-HEAD-SPEC
\theoremstyle{mystyle}

% Definitionen der Satz, Lemma... - Umgebungen. Der Zähler von "satz" ist dem "section"-Zähler untergeordnet, alle weiteren Umgebungen bedienen sich des satz-Zählers.
\newtheorem{satz}{Satz}[section]
\newtheorem{lemma}[satz]{Lemma}
\newtheorem{korollar}[satz]{Korollar}
\newtheorem{proposition}[satz]{Proposition}
\newtheorem{beispiel}[satz]{Beispiel}
\newtheorem{definition}[satz]{Definition}
\newtheorem{bemerkungnr}[satz]{Bemerkung}
\newtheorem{theorem}[satz]{Theorem}

% Bemerkungen, Erinnerungen und Notationshinweise werden ohne Numerierungen dargestellt.
\newtheorem*{bemerkung}{Bemerkung.}
\newtheorem*{erinnerung}{Erinnerung.}
\newtheorem*{notation}{Notation.}
\newtheorem*{aufgabe}{Aufgabe.}
\newtheorem*{lösung}{Lösung.}
\newtheorem*{beisp}{Beispiel.} %Beispiel ohne Nummerierung
\newtheorem*{defi}{Definition.} %Definition ohne Nummerierung
\newtheorem*{lem}{Lemma.} %Lemma ohne Nummerierung


% SHORTCUTS
\newcommand{\R}{\mathbb{R}}				 % reelle Zahlen
\newcommand{\Rn}{\R^n}						 % der R^n
\newcommand{\N}{\mathbb{N}}				 % natürliche Zahlen
\newcommand{\Z}{\mathbb{Z}}				 % ganze Zahlen
\newcommand{\C}{\mathbb{C}}			   % komplexe Zahlen
\newcommand{\gdw}{\Leftrightarrow} % Genau dann, wenn
\newcommand{\with}{\text{ mit }}   % mit
\newcommand{\falls}{\text{falls }} % falls
\newcommand{\dd}{\text{ d}}        % Differential d

% ETWAS SPEZIELLERE ZEICHEN
%disjoint union
\newcommand{\bigcupdot}{
	\mathop{\vphantom{\bigcup}\mathpalette\setbigcupdot\cdot}\displaylimits
}
\newcommand{\setbigcupdot}[2]{\ooalign{\hfil$#1\bigcup$\hfil\cr\hfil$#2$\hfil\cr\cr}}
%big times
\newcommand*{\bigtimes}{\mathop{\raisebox{-.5ex}{\hbox{\huge{$\times$}}}}} 

% WHITESPACE COMMANDS
%non-restrict newline command
\newcommand{\enter}{$ $\newline} 
%praktischer Tabulator
\newcommand\tab[1][1cm]{\hspace*{#1}}

% TEXT ÜBER ZEICHEN
%das ist ein Gleichheitszeichen mit Text darüber, Beispiel: $a\stackeq{Def} b$
\newcommand{\stackeq}[1]{
	\mathrel{\stackrel{\makebox[0pt]{\mbox{\normalfont\tiny #1}}}{=}}
} 
%das ist ein beliebiges Zeichen mit Text darüber, z. B.  $a\stackrel{Def}{\Rightarrow} b$
\newcommand{\stacksymbol}[2]{
	\mathrel{\stackrel{\makebox[0pt]{\mbox{\normalfont\tiny #1}}}{#2}}
} 

% UNDERLINE
% besseres underline 
\renewcommand{\ULdepth}{1pt}
\contourlength{0.5pt}
\newcommand{\ul}[1]{
	\uline{\phantom{#1}}\llap{\contour{white}{#1}}
}


% hier noch ein paar Commands die nur ich nutze, weil ich sie mir im Laufe der Jahre angewöhnt habe und sie mir jetzt nicht abgewöhnen will:

\newcommand{\gdw}{\Leftrightarrow}   % genau dann, wenn



% Commands für Stochastik / Statistik
\newcommand{\A}{\mathcal{A}}
\renewcommand{\P}{\mathbb{P}}
\newcommand{\E}{\mathbb{E}}




% This work is licensed under the Creative Commons
% Attribution-NonCommercial-ShareAlike 4.0 International License. To view a copy
% of this license, visit http://creativecommons.org/licenses/by-nc-sa/4.0/ or
% send a letter to Creative Commons, PO Box 1866, Mountain View, CA 94042, USA.

% Commands für WTHM
\newcommand{\F}{\mathcal{F}}				% Standard-Unter-Sigma-Algebra / Filtration
\newcommand{\G}{\mathcal{G}}				% Gegenstück zu \F für Rückwärtsmartingale

% independent symbol from:
% https://tex.stackexchange.com/questions/79434/double-perpendicular-symbol-for-independence
\newcommand{\unab}{\protect\mathpalette{\protect\independenT}{\perp}}
\def\independenT#1#2{\mathrel{\rlap{$#1#2$}\mkern2mu{#1#2}}}

\newcommand{\BedE}[2]{\E\left[{#1}~|~{#2}\right]} % Bedingte Erwartung
\newcommand{\graph}{\text{graph}}

%\newcommand{\binom}[2]{\begin{pmatrix}	{#1}\\{#2} \end{pmatrix}} %this command doesn't compile for me




\author{Willi Sontopski}

\parindent0cm %Ist wichtig, um führende Leerzeichen zu entfernen

\usepackage{color}

\usepackage{scrpage2}
\pagestyle{scrheadings}
\clearscrheadfoot

\ihead{Willi Sontopski}
\chead{WTHM WiSe 18 19}
\ohead{}
\ifoot{Aufgabenblatt 2}
\cfoot{Version: \today}
\ofoot{Seite \pagemark}

\begin{document}
%\setcounter{section}{1}

\section*{Aufgabe 1}
\subsection*{Aufgabe 1 a)}
Sei $(X_t)_{t\in\R_{\geq0}}$ ein Super-Martingal bzgl. einer Filtration $(\F_t)_{t\in\R_{\geq0}}$ mit\\ $\E[X_t]=$ const.\\
Dann ist $(X_t)_{t\in\R_{\geq0}}$ bereits ein Martingal.

\begin{proof}
Da $(X_t)_{t\in\R_{\geq0}}$ bereits ein Super-Martingal ist, genügt es nur die Martingaleigenschaft zu zeigen. Dann gilt für alle $s\leq t$
	\begin{align*}
		X_s - \BedE{X_t}{\F_s} \stackrel{\text{Super-MG}}{\geq} 0
	\end{align*}
	und
	\begin{align*}
		\E\big[\underbrace{X_s - \BedE{X_t}{\F_s}}_{\geq0} \big]
		&\stackeq{\text{Lin}} \E[X_s] - \E\big[\BedE{X_t}{\F_s} \big] \\		
		&\stackeq{\text{Tow}} \E[X_s] - \E[X_t] \\
		%&\stackeq{\text{Lin}} \E[X_s-X_t]\\
		&\stackeq{\text{Vor}} c-c \\
		&= 0
	\end{align*}
Somit folgt $X_s=\BedE{X_t}{\F_s}$ fast sicher für alle $s\leq t$. Also ist $(X_t)_{t\in\R_{\geq0}}$ ein Martingal, denn in $\mathcal{L}_p$ sind (per Äquivalenzrelation) zwei Zufallsvariablen gleich, wenn sie punktweise fast sicher gleich sind.
\end{proof}

Bemerkung: Die Aussage gilt analog für Submartingale.

\subsection*{Aufgabe 1 b)}
Sei $(X_n)_{n\in\N}$ ein Martingal \underline{und} ein vorhersehbarer Prozess bzgl. der gleichen Filtration $(\F_n)_{n\in\N}$.\\
Dann ist $X_n=X_0$ fast sicher für alle $n\in\N$.

\begin{proof}
	Beweis per Induktion.\\
	Induktionsanfang für $n=0$ ist klar.\\
	Induktionsvoraussetzung: Gelte
	\begin{align*}
		X_n = X_0 \quad \text{f.s.}
	\end{align*}
	für beliebiges aber festes $n\in\N$.\\
	Induktionsschritt:
	\begin{align*}
		X_{n+1}-X_0
		&\stackeq{\text{IV}} X_{n+1} - X_{n} \\
		&\stackeq{\text{MG}} X_{n+1} - \BedE{X_{n+1}}{\F_{n}} \\
		&\stackeq{1.1.4.(a)} \BedE{X_{n+1}}{\F_{n}} - \BedE{X_{n+1}}{\F_{n}} \\
		%&\stackeq{\text{Lin}} \BedE{X_{n+1}-X_{n+1}}{\F_{n-1}} \\
		&= 0
	\end{align*}
	
	\textbf{Marvins Version:}\\
	Sei $n\in\N$ beliebig. Dann gilt:
	\begin{align*}
		X_n\stackeq{\text{MG}} \E\big[X_{n+1}~\big|\F_n\big]\stackeq{\text{vorher}} X_{n+1}\in\F_n\\
	\end{align*}
	\begin{align*}
	\implies X_n=X_{n+1}
	\end{align*}
\end{proof}

\section*{Aufgabe 2}
\subsection*{Aufgabe 2 a)}
Sei $(\xi_j)_{j\in\N_{>0}}$ eine iid-Folge nichtnegativer Zufallsvariablen mit $\E[\xi_1]=1$. Dann ist der stochastische Prozess $M$ definiert durch
\begin{align*}
M_0:=\indi_\Omega\qquad\text{ und }M_n=\prod\limits_{j=1}^n\xi_j\qquad\forall n\in\N_{>0}
\end{align*}
ein Martingal bzgl. der von $(\xi_j)_{j\in\N_{>0}}$ erzeugten Filtration  mit $\F_0:=\lbrace \Omega,\emptyset\rbrace$ und $\F_n:=\sigma(\xi_1,\ldots,\xi_n)$ für alle $n\in\N_{>0}$.

\begin{proof}
	\underline{Adaptiertheit:}\\
	$M_0$ ist $\F_0$ offenbar messbar und
	$\xi_n$ ist $\F_n$-messbar für $n>0$ nach Definition von $\F_n$.	\\
	
	 \underline{Integrierbarkeit:}
	\begin{align*}
		\E\big[|M_n|\big]
		&\stackeq{\geq0}\E\big[M_n\big]\\
		&\stackeq{\text{Def}}\E\left[\prod_{j=1}^n \xi_j\right]\cdot\underbrace{\E[M_0]}_{=1} \\
		&\stackeq{\text{iid}}\prod_{j=1}^n \E[\xi_1] \\
		&= 1 \\
		&<\infty
	\end{align*}
	\underline{MG-Eigenschaft:}
	\begin{align*}
		\BedE{M_n}{\F_{n-1}}
		&\stackeq{\text{Def}} \BedE{\prod_{j=1}^n \xi_j}{\F_{n-1}} \\
		&\stackeq{} \BedE{\left(\prod_{j=k+1}^n \xi_j\right)\cdot\left(\prod_{j=1}^k \xi_j\right)}{\F_{n-1}} \\
		&\stackeq{\text{Def}} \BedE{\left(\prod_{j=k+1}^n \xi_j\right)\cdot M_k}{\F_{n-1}} \\
		&\stackeq{\text{Pull-out}} \E\left[\prod_{j=k+1}^n \xi_j\right]\cdot\BedE{M_{k}}{\F_{n-1}} \\
		&\stackeq{\text{iid}} \left(\prod_{j=k+1}^n\E[\xi_j]\right) \cdot \BedE{M_{k}}{\F_{n-1}}\qquad\forall n\in\N\\
		&= 1\cdot \BedE{M_{k}}{\F_{n-1}}\\
		&\stackeq{\text{1.1.4(a)}} M_{k}\qquad\forall k\leq n
	\end{align*}
In die letzte Gleichheit geht auch die oben gezeigte Adaptiertheit ein: $M_k$ ist $\F_k$ messbar und wegen $\F_{k}\subseteq \F_{k+1}\subseteq\ldots\subseteq \F_{n-1}\subseteq\F_n$ auch $n-1$ messbar.
\end{proof}

\subsection*{Aufgabe 2 b)}
Sei $(S_n)_{n\in\N}$ der \textbf{asymmetrische} einfache Random Walk, d. h.\\ $S_n=\xi_1+\ldots+\xi_n$ mit $\P[\xi_1=1]=p$ und $\P[\xi_1=-1]=1-p=:q$.\\
Dann ist 
\begin{align*}
M_n:=\left(\frac{q}{p}\right)^{S_n}\qquad\forall n\in\N
\end{align*}
für jedes $p\in (0,1)\setminus\left\lbrace\frac{1}{2}\right\rbrace$ ein Martingal bzgl. $\F_n=\sigma(\xi_1,\ldots,\xi_n)$.

\begin{proof}
	\underline{Adaptiertheit:}\\
	Betrachte die Funktion
	\begin{align*}
		f(x):= \exp\Big(x\cdot\log\Big(\frac{q}{p}\Big)\Big)
	\end{align*}
	Diese ist messbar (weil stetig) und da $S_n$ per Definition von $\F_n$ messbar ist, ist auch
	$f\circ S_n$ messbar bezüglich $\F_n$ für alle $n\in\N$.\\

	\underline{Integrierbarkeit:} %FALSCH IMHO, da S_n<0 möglich
	\begin{align*}
		\E\big[|M_n|\big]]
		&\stackeq{\geq0}\E[M_n]\\
		&= \E\left[\left(\frac{q}{p}\right)^{S_n}\right] \\
		&\leq\begin{cases}
		\E\Big[\prod_{i=1}^{\lfloor S_n\rfloor + 1}\Big(\frac{q}{p}\Big)\Big] &,\frac{q}{p}\geq 1 \\
		1 &,\frac{q}{p} < 1 
	\end{cases} \\
		&=\begin{cases}
	\Big(\frac{q}{p}\Big)^{n+1} &,\frac{q}{p}\geq 1 \\
		1 &,\frac{q}{p} < 1 
	\end{cases} \\
	&< \infty
	\end{align*}

	\underline{Martingaleigenschaft:}
	\begin{align*}
		\BedE{M_n}{F_{n-1}}
		&=\E\left[M_{n-1}\cdot\left(\frac{q}{p}\right)^{\xi_n}~\Big|~\F_{n-1}\right] \\
		&\stackeq{\text{Pull}}
	\E\left[\left.\left(\frac{q}{p}\right)^{\xi_1}~\right|~\F_{n-1}\right]\cdot M_{n-1} \\
	&\stackeq{\text{}}
	\E\left[\left(\frac{q}{p}\right)^{\xi_1}\right]\cdot M_{n-1} \\
		&=\left(\left(\frac{q}{p}\right)^1\cdot p + \left(\frac{q}{p}\right)^{-1}\cdot q\right)\cdot M_{n-1} \\
		&= (p+q)\cdot M_{n-1} \\
		&= M_{n-1}
	\end{align*}
Der allgemeine Fall für $k\leq n$ folgt dann induktiv.\\

\textbf{Alternative vom Prof:}\\
Betrachte (b) als Spezialfall von (a):
\begin{align*}
M_n&=\prod\limits_{n=1}^n \eta_k,\qquad\eta_k:=\left(\frac{q}{p}\right)^{\xi_k}\\
\E\big[\eta_k\big]&=\E\left[\left(\frac{q}{p}\right)^{S_k}\right]=p\cdot\frac{q}{p}+q\cdot\frac{p}{q}=1
\end{align*}
\end{proof}

\subsection*{Aufgabe 2 c)}
Kompensator $\langle M\rangle$ hat Eigenschaften:
\begin{itemize}
\item vorhersehbar
\item $M_n^2-\langle M\rangle_N$ ist Martingal
\end{itemize}
Aus der Vorlesung ist bekannt:
\begin{align*}
\E\Big[(M_n-M_{n-1})^2~\Big|~\F_{n-1}\Big]=\langle M\rangle_n-\langle M\rangle_{n-1}
\end{align*}

\underline{Kompensator von $M_n$ aus a):} Einsetzen liefert
\begin{align*}
\E\Big[M_{n-1}^2\cdot(\xi_n-1)^2~\Big|~\F_{n-1}\Big]
&\stackeq{\text{Pull}}
M_{n-1}^2\cdot\E\Big[(\xi_{n}-1)^2~\Big|~\F_{n-1}\Big]\\
&=M_{n-1}^2\cdot\E\Big[(\xi_n-1)^2\Big]\\
&=M_{n-1}^2\cdot\Var(\xi_1)\\
&\implies
\langle M\rangle_n=\Var(\xi_1)\cdot\sum\limits_{i=1}^{n-1} M_i^2
\end{align*}

\underline{Kompensator von $M_n$ aus b):}\\
Nutze hier, dass b) Spezialfall von a) ist. Es bleibt nur noch die Varianz der Faktoren auszurechnen:
\begin{align*}
\Var\left(\left(\frac{q}{p}\right)^{\xi_n}\right)
&=p\cdot\left(\frac{q}{p}-1\right)^2+q\cdot\left(\frac{p}{q}-1\right)^2\\
&=p\cdot\left(\frac{q^2}{p^2}-\frac{2\cdot q}{p}+1\right)+q\cdot\left(\frac{p^2}{q^2}-2\cdot\frac{p}{p}+1\right)\\
&=\frac{q^2}{p}-2\cdot q+p+\frac{p^2}{q}-2\cdot p+q\\
&=\frac{q^2}{p}+\frac{p^2}{q}-1
\end{align*}


\section*{Aufgabe 3}
Sei $(\xi_n)_{n\in\N}$ eine Folge von iid-Zufallsvariablen und $\xi_1$ normalverteilt mit Mittelwert 0 und Varianz $\sigma^2>0$. Weiter sei $\F_n:=\sigma(\xi_1,\ldots,\xi_n)$ und\\ $X_n:=\xi_1+\ldots+\xi_n$

\subsection*{Aufgabe 3 a)}
\begin{align*}
\E\big[\exp(u\cdot\xi_1)\big]=\exp\left(u^2\cdot\frac{\sigma^2}{2}\right)\qquad\forall u\in\R
\end{align*}

\begin{proof}
Sei $u\in\R$ beliebig. Wegen
\begin{align}\label{3a}
u\cdot x-\frac{x^2}{2\cdot\sigma^2}
&=-\left(\frac{u^2\cdot\sigma^2}{2}-2\cdot\frac{u\cdot\sigma\cdot x}{\sqrt{2}\cdot\sqrt{2}\cdot\sigma}+\frac{x^2}{2\cdot\sigma^2}\right)+\frac{u^2\cdot\sigma^2}{2}\\\nonumber
&\stackeq{\text{Binom}}
-\left(\frac{u\cdot\sigma}{\sqrt{2}}-\frac{x}{\sqrt{2}\cdot\sigma}\right)^2+\frac{(u\cdot\sigma)^2}{2}
\end{align}
gilt
\begin{align*}
\E\big[\exp(u\cdot\xi_1)\big]
&\stackeq{\text{Trafo}}
\int\limits_\R\exp(u\cdot x)\P_{\xi_1}(\d x)\\
&\stackeq{\text{Dichte}}
\frac{1}{\sqrt{2\cdot\pi\cdot\sigma^2}}\cdot\int\limits_\R \exp\left(u\cdot x\right)\cdot \exp\left(-\frac{x^2}{2\cdot\sigma^2}\right)\d x\\
&\stackeq{\eqref{3a}}
\frac{1}{\sqrt{2\cdot\pi\cdot\sigma^2}}\cdot\int\limits_\R \exp\left(-\left(\frac{u\cdot\sigma}{\sqrt{2}}-\frac{x}{\sqrt{2}\cdot\sigma}\right)^2\right) \cdot\exp\left(\frac{(u\cdot\sigma)^2}{2}\right)\d x\\
&=\frac{1}{\sqrt{2\cdot\pi\cdot\sigma^2}} \cdot \exp\left(\frac{(u\cdot\sigma)^2}{2}\right)\cdot\int\limits_\R \exp\left(-\frac{(u\cdot\sigma^2-x)^2}{2\cdot\sigma^2}\right)\d x\\
&\stackeq{\text{Subs}}
\frac{1}{\sqrt{2\cdot\pi\cdot\sigma^2}}\cdot\exp\left(\frac{(u\cdot\sigma)^2}{2}\right)\cdot\underbrace{\int\limits_\R \exp\left(-\frac{y^2}{2\cdot\sigma^2}\right)\d y}_{=\sqrt{2\cdot\pi\cdot\sigma^2}\text{, da Gauß Int.}}\\
&=\exp\left(u^2\cdot\frac{\sigma^2}{2}\right)
\end{align*}
Für die Substitution nutze
\begin{align*}
f(y):=\exp\left(-\frac{y^2}{2\cdot\sigma^2}\right),\qquad\varphi(x):=-(u\cdot\sigma^2-x),\qquad\varphi'(x)=1
\end{align*}
\end{proof}

\subsection*{Aufgabe 3 b)}
Sei $u\in\R$. Dann ist
\begin{align*}
Z_n^u:=\exp\left(u\cdot X_n-n\cdot u^2\cdot\frac{\sigma^2}{2}\right)\qquad\forall n\in\N
\end{align*}
ein Martingal.

\begin{proof} Zunächst gilt
\begin{align*}
Z_n^u
&\stackeq{\text{Def}}\exp\left(u\cdot S_n-n\cdot u^2\cdot\frac{\sigma^2}{2}\right)\\
	&= \frac{\exp(u\cdot S_n)}{\exp\left(n\cdot u^2\cdot\frac{\sigma^2}{2}\right)} \\
	&=\prod_{i=1}^n \frac{\exp\left(u\cdot \xi_i\right)}{\exp\left(n\cdot u^2\cdot \frac{\sigma^2}{2}\right)} \\
	&=\frac{1}{\exp\left(n\cdot u^2\cdot \frac{\sigma^2}{2}\right)}\cdot\prod_{i=1}^n \exp\left(u\cdot\xi_i\right)
\end{align*}
\underline{Adaptiertheit:} ist klar, da Verkettung messbarer Funktionen.\\

\underline{Integrierbarkeit:}
\begin{align*}
\E\big[|Z_n^u|\big]
	&\stackeq{\geq0}\E\left[Z^u_n\right]\\ 
	&\stackeq{\text{oben}}
	\frac{1}{\exp\left(n\cdot u\cdot\frac{\sigma^2}{2}\right)}\cdot\E\left[\prod\limits_{i=1}^n \exp\big(u\cdot\xi_i\big)\right]\\
	&\stackrel{\text{iid.}}{=}\frac{1}{\exp\left(n\cdot u^2\cdot\frac{\sigma^2}{2}\right)}\cdot\prod_{i=1}^n \E\big[\exp\left(u\cdot\xi_1\right)\big] \\
	&=\frac{1}{\exp\left(n\cdot u^2\cdot\frac{\sigma^2}{2}\right)}\cdot\prod_{i=1}^n \exp\left(u^2\cdot\frac{\sigma^2}{2}\right) \\
	&=\frac{1}{\exp\left(n\cdot u^2\cdot\frac{\sigma^2}{2}\right)}\cdot \exp\left(u^2\cdot n\cdot\frac{\sigma^2}{2}\right) \\
	&=1 \\
	&<\infty
\end{align*}
\underline{Martingaleigenschaft:}
\begin{align*}
	\E\big[Z_n^u ~\big|~\F_{n-1}\big]
	&=\E\left[\frac{1}{\exp\left(n\cdot u^2\cdot\frac{\sigma^2}{2}\right)}\cdot\exp\left(u\cdot\xi_n\right)\cdot Z^u_{n-1}~\bigg|~\F_{n-1}\right]\\
	&\stackeq{\text{Pull}}
	\E\left[\frac{1}{\exp\left(n\cdot u^2\cdot\frac{\sigma^2}{2}\right)}\cdot\exp\left(u\cdot\xi_n\right)\right]\cdot Z^u_{n-1}\cdot\exp\left(-\frac{u^2\cdot\sigma^2}{2}\right)\\
	&\stackeq{(a)} Z^u_{n-1}
\end{align*}
Der allgemeine Fall für $k\leq n$ folgt dann induktiv.\\

\textbf{Alternative:} Ist ein Spezialfall von 2 a) mit
\begin{align*}
Z_n^u&=\prod\limits_{i=1}^n\eta_i,\qquad\eta_i=\exp\left(u\cdot\xi_i-\frac{u^2\cdot\sigma^2}{2}\right)\geq0\\
\E[\eta_i]&=\E[\exp(u\cdot\xi_i)]\cdot\exp\left(-\frac{u^2\cdot\sigma^2}{2}\right)=1
\end{align*}
\end{proof}

\subsection*{Aufgabe 3 c)}
Zeige, dass der Grenzwert
\begin{align*}
Z_\infty^u:=\limn Z^u_n=\limn\exp\left(u\cdot X_n-n\cdot u^2\cdot\frac{\sigma^2}{2}\right)
\end{align*}
existiert fast sicher. Bestimme diesen Grenzwert. Gibt es Werte $u\in\R$ so, dass $(Z_n^u)_{n\in\N\cup\lbrace\infty\rbrace}$ wieder ein Martingal ist, d. h. so, dass
\begin{align*}
\E\big[Z^u_\infty~\big|~\F_n\big]=Z^u_n
\end{align*}
gilt?

\begin{lösung}
Betrachte $u\neq0$, da sonst trivial.\\

\underline{Vorüberlegung:}
Wegen dem Gesetz der großen Zahlen gilt
\begin{align*}
\limn\frac{X_n}{n}=\E[\xi_1]=0.
\end{align*}
Es folgt
\begin{align*}
\limn\left(\frac{1}{n}\cdot\left( u\cdot X_n-n\cdot u^2\cdot\frac{\sigma^2}{2}\right)\right)
&=\limn\left(u\cdot\frac{X_n}{n}-\frac{u^2\cdot\sigma^2}{2}\right)\\
&=-\frac{u^2\cdot\sigma^2}{2}\\
&<0\text{ für }u\neq0\\
\implies
\limn\left( u\cdot X_n-n\cdot u^2\cdot\frac{\sigma^2}{2}\right)&=-\infty\text{ fast sicher }\\
\implies
\limn Z_n^u&=\left\lbrace\begin{array}{cl}
1, & \falls u=0\\
0, & \falls u\neq 0
\end{array}\right.
=Z_\infty^u
\end{align*}
\underline{Frage:} Gilt $Z_n^u=\E[Z_\infty^u~|~\F_n]$?\\
Nein! Denn 
\begin{align}\label{eq3cStern}
\E\big[Z_\infty^u~\big|~\F_n\big]=0\neq Z_n^u
\end{align}
(Im Fall $u=0$ gilt die Aussage, ist aber trvial, denn $Z_n^0\equiv 1$.)\\
Hierbei ist \eqref{eq3cStern} ein Beispiel für nicht ``gleichgradig integrierbare'' Martingale.
\end{lösung}

\section*{Aufgabe 4}
\begin{defi}
Eine Zufallsvariable $\tau:\Omega\to\N_0\cup\lbrace+\infty\rbrace$ heißt \textbf{Stoppzeit} bzgl. der Filtration $(\F_n)_{n\in\N}$
\begin{align*}
:\Longleftrightarrow\forall n\in\N_0:\big\lbrace\tau\leq n\big\rbrace:=\big\lbrace \omega\in\Omega:\tau(\omega)\leq n\big\rbrace\in\F_n
\end{align*}
\end{defi}
Seien nun Stoppzeiten $\tau,\sigma,\tau_1,\tau_2,\ldots$ gegeben. Dann gilt:
\begin{enumerate}[label=\alph*)]
\item $\lbrace\omega\in\Omega:\tau(\omega)=n\rbrace\in\F_n$
\item $\lbrace\omega\in\Omega:\tau(\omega)\geq n\rbrace\in\F_{n-1}$
\item $\begin{aligned}
C_n(\omega):=\indi_{[0,\tau(\omega)]}(n)
\end{aligned}$ ist messbar bzgl. $\F_{n-1}$
\item $\sigma\wedge\tau$ und $\sigma\vee\tau$ sind Stoppzeiten. Erinnerung:
\begin{align*}
&\sigma\wedge\tau:\Omega\to\N_0\cup\lbrace+\infty\rbrace,&(\sigma\wedge\tau)(\omega)&:=\min\big\lbrace\sigma(\omega),\tau(\omega)\big\rbrace\\
&\sigma\vee\tau:\Omega\to\N_0\cup\lbrace+\infty\rbrace,&(\sigma\vee\tau)(\omega)&:=\max\big\lbrace\sigma(\omega),\tau(\omega)\big\rbrace
\end{align*}
\item Sind $\sup\limits_j\tau_j$ und $\inf\limits_j\tau_j$ Stoppzeiten?
\end{enumerate}

\begin{proof}
\underline{Zu a):} Aus der Definition folgt:
\begin{align*}
	&\{\tau \leq n\} \in \F_n \\
	\implies & \{\tau > n-1\} = \{\tau \leq n-1\}^C \in \F_{n-1} \subseteq \F_n \\
	\implies & \{\tau = n\} = \{\tau \leq n\} \cap \{\tau > n-1\} \in \F_n
	\end{align*}
\underline{Zu b):}\\
\begin{align*}
	\{\tau \geq n \} =\lbrace\tau <n-1\rbrace^C= \Big(\underbrace{\{\tau \leq n-1 \}}_{\in\F_{n-1}\text{, wg. (a)}} \setminus \underbrace{\{\tau = n-1 \}}_{\in\F_{n-1}} \Big)^C \in \F_{n-1}
\end{align*}
\underline{Zu c):}\\
%Live-Mitschrift:
\begin{align*}
C_n(\omega)&=1\Longleftrightarrow n\leq\tau(\omega)\Longleftrightarrow
\omega\in\lbrace\tau\geq n\rbrace\stackeq{(b)}{\in}\F_{n-1}\\
C_n(\omega)&=0\Longleftrightarrow
n>\tau(\omega)\Longleftrightarrow
\omega\in\lbrace\tau<n\rbrace=\lbrace\tau\leq n-1\rbrace\in\F_{n-1}
\end{align*}





%Für festes n gilt:

%\begin{align*}
%	\indi_{[0,\tau(\cdot)]}(n)=\indi_{\{n\leq \tau(\cdot)\}}
%\end{align*}

%\begin{align*}
%	&\indi_{[0,\tau(\omega)]}(n)\stackeq{???}\indi_{\underbrace{\{n\leq \tau(\omega)\}}_{(b) \implies \in\F_{n-1}}} \\
%	\implies &\indi_{[0,\tau(\omega)]}(n) \in \F_{n-1}
%\end{align*}
%Mein Vorschlag:
%\begin{align*}
%C_n:\Omega\to\lbrace0,1\rbrace\text{ ist }\F_{n-1}\text{ messbar }\Longleftrightarrow\forall M\in\B(\R):C_n^{-1}(M)\in\F_{n-1}
%\end{align*}
%Betrachte
%\begin{align*}
%C_n^{-1}(M)
%&=\big\lbrace\omega\in\Omega:C_n(\omega)\in M\big\rbrace\\
%&=\left\lbrace\omega\in\Omega:\indi_{[0,\tau(\omega)]}(n)\in M\right\rbrace\\
%&=\left\lbrace\omega\in\Omega:\left\lbrace\begin{array}{cl}
%0, &\falls n\not\in[0,\tau(\omega)]\\
%1, & \falls n\in[0,\tau(\omega)]
%\end{array}\right.\in M\right\rbrace\\
%&=\left\lbrace\omega\in\Omega:\left\lbrace\begin{array}{cl}
%0, &\falls n>\tau(\omega)\\
%1, &\falls n\leq\tau(\omega)
%\end{array}\right.\in M\right\rbrace\\
%&=\big(\lbrace n>\tau(\omega)\cap\lbrace o\in M\rbrace\big)\cup\big(\lbrace n\leq\tau(\omega)\rbrace\cap\lbrace 1\in M\rbrace\big)
%\end{align*}
%Also machen wir eine Fallunterscheidung:\\

%\underline{Fall 1: $0,1\not\in M$}\\
%Dann gilt $C_n(M)=\emptyset\in\F_{n-1}$.\\

%\underline{Fall 2: $0,1\in M$}\\
%Dann gilt $C_n(M)=\lbrace n>\tau(\omega)\rbrace\cup\lbrace n\leq\tau(\omega)\rbrace=\Omega\in\F_{n-1}$.\\

%\underline{Fall 3: $0\in M,1\not\in M$}\\
%Dann gilt $C_n(M)=\lbrace n>\tau(\omega)\rbrace=\lbrace n\leq\tau(\omega)\rbrace^C\in\F_{n}$.\\

%\underline{Fall 4: $0\not\in M,1\in M$}\\
%Dann gilt $C_n(M)=\lbrace n\leq\tau(\omega)\rbrace\in\F_{n}$.\\

%Folglich ist $C_n$ $\F_n$ messbar. Im Allgemeinen aber leider nicht $F_{n-1}$ messbar, da $\F_{n-1}\subseteq\F_n$.\\

\underline{Zu d):} Sei $n\in\N_0$ beliebig. Dann gilt
\begin{align*}
\{\sigma \wedge \tau \leq n \}
&=\big\lbrace\omega\in\Omega:\min\lbrace\sigma(\omega),\tau(\omega)\rbrace\leq n\big\rbrace\\
&=\big\lbrace\omega\in\Omega:\sigma(\omega)\leq n\vee\tau(\omega)\leq n\big\rbrace\\
&=\underbrace{\{\sigma \leq n\}}_{\in \F_n} \cup \underbrace{\{\tau \leq n\}}_{\in\F_n} \in \F_n \\
\{\sigma \vee \tau \leq n \}
&=\big\lbrace\omega\in\Omega:\max\lbrace\sigma(\omega),\tau(\omega)\rbrace\leq n\big\rbrace\\
&=\big\lbrace\omega\in\Omega:\sigma(\omega)\leq n\wedge\tau(\omega)\leq n\big\rbrace\\
&=\underbrace{\{\sigma \leq n\}}_{\in \F_n} \cap \underbrace{\{\tau \leq n\}}_{\in\F_n} \in \F_n \\
\end{align*}
\underline{Zu e):}\\
Sei $(\tau_i)_{i\in\N}$ eine abzählbare Folge von Stoppzeiten und sei $n\in\N_0$ beliebig. Dann gilt:
\begin{align*}
\left\lbrace\sup_{i\in\N} \tau_{i} \leq n\right\rbrace
= \bigcap_{i\in\N}\underbrace{\{\tau_i \leq n\}}_{\in\F_n} \in \F_n
\end{align*}
\begin{align*}
\left\lbrace\inf_{i\in\N} \tau_{i} \leq n\right\rbrace =
\left\lbrace\inf_{i\in\N} \tau_{i} \geq n+1\right\rbrace^C = \left\lbrace\bigcap_{i\in\N}\underbrace{\{\tau_i \geq n+1\}}_{\in\F_n}\right\rbrace^C \in \F_n
\end{align*}

\end{proof}

\end{document}
