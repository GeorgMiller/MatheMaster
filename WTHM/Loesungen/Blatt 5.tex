% This work is licensed under the Creative Commons
% Attribution-NonCommercial-ShareAlike 4.0 International License. To view a copy
% of this license, visit http://creativecommons.org/licenses/by-nc-sa/4.0/ or
% send a letter to Creative Commons, PO Box 1866, Mountain View, CA 94042, USA.

\documentclass[12pt,a4paper]{article} 

% This work is licensed under the Creative Commons
% Attribution-NonCommercial-ShareAlike 4.0 International License. To view a copy
% of this license, visit http://creativecommons.org/licenses/by-nc-sa/4.0/ or
% send a letter to Creative Commons, PO Box 1866, Mountain View, CA 94042, USA.

% PACKAGES
\usepackage[english, ngerman]{babel}	% Paket für Sprachselektion, in diesem Fall für deutsches Datum etc
\usepackage[utf8]{inputenc}	% Paket für Umlaute; verwende utf8 Kodierung in TexWorks 
\usepackage[T1]{fontenc} % ö,ü,ä werden richtig kodiert
\usepackage{amsmath} % wichtig für align-Umgebung
\usepackage{amssymb} % wichtig für \mathbb{} usw.
\usepackage{amsthm} % damit kann man eigene Theorem-Umgebungen definieren, proof-Umgebungen, etc.
\usepackage{mathrsfs} % für \mathscr
\usepackage[backref]{hyperref} % Inhaltsverzeichnis und \ref-Befehle werden in der PDF-klickbar
\usepackage[english, ngerman, capitalise]{cleveref}
\usepackage{graphicx}
\usepackage{grffile}
\usepackage{setspace} % wichtig für Lesbarkeit. Schöne Zeilenabstände

\usepackage{enumitem} % für custom Liste mit default Buchstaben
\usepackage{ulem} % für bessere Unterstreichung
\usepackage{contour} % für bessere Unterstreichung
\usepackage{epigraph} % für das coole Zitat

\usepackage{tikz}

% This work is licensed under the Creative Commons
% Attribution-NonCommercial-ShareAlike 4.0 International License. To view a copy
% of this license, visit http://creativecommons.org/licenses/by-nc-sa/4.0/ or
% send a letter to Creative Commons, PO Box 1866, Mountain View, CA 94042, USA.

% THEOREM-ENVIRONMENTS

\newtheoremstyle{mystyle}
  {20pt}   % ABOVESPACE \topsep is default, 20pt looks nice
  {20pt}   % BELOWSPACE \topsep is default, 20pt looks nice
  {\normalfont} % BODYFONT
  {0pt}       % INDENT (empty value is the same as 0pt)
  {\bfseries} % HEADFONT
  {}          % HEADPUNCT (if needed)
  {5pt plus 1pt minus 1pt} % HEADSPACE
	{}          % CUSTOM-HEAD-SPEC
\theoremstyle{mystyle}

% Definitionen der Satz, Lemma... - Umgebungen. Der Zähler von "satz" ist dem "section"-Zähler untergeordnet, alle weiteren Umgebungen bedienen sich des satz-Zählers.
\newtheorem{satz}{Satz}[section]
\newtheorem{lemma}[satz]{Lemma}
\newtheorem{korollar}[satz]{Korollar}
\newtheorem{proposition}[satz]{Proposition}
\newtheorem{beispiel}[satz]{Beispiel}
\newtheorem{definition}[satz]{Definition}
\newtheorem{bemerkungnr}[satz]{Bemerkung}
\newtheorem{theorem}[satz]{Theorem}

% Bemerkungen, Erinnerungen und Notationshinweise werden ohne Numerierungen dargestellt.
\newtheorem*{bemerkung}{Bemerkung.}
\newtheorem*{erinnerung}{Erinnerung.}
\newtheorem*{notation}{Notation.}
\newtheorem*{aufgabe}{Aufgabe.}
\newtheorem*{lösung}{Lösung.}
\newtheorem*{beisp}{Beispiel.} %Beispiel ohne Nummerierung
\newtheorem*{defi}{Definition.} %Definition ohne Nummerierung
\newtheorem*{lem}{Lemma.} %Lemma ohne Nummerierung


% SHORTCUTS
\newcommand{\R}{\mathbb{R}}				 % reelle Zahlen
\newcommand{\Rn}{\R^n}						 % der R^n
\newcommand{\N}{\mathbb{N}}				 % natürliche Zahlen
\newcommand{\Z}{\mathbb{Z}}				 % ganze Zahlen
\newcommand{\C}{\mathbb{C}}			   % komplexe Zahlen
\newcommand{\gdw}{\Leftrightarrow} % Genau dann, wenn
\newcommand{\with}{\text{ mit }}   % mit
\newcommand{\falls}{\text{falls }} % falls
\newcommand{\dd}{\text{ d}}        % Differential d

% ETWAS SPEZIELLERE ZEICHEN
%disjoint union
\newcommand{\bigcupdot}{
	\mathop{\vphantom{\bigcup}\mathpalette\setbigcupdot\cdot}\displaylimits
}
\newcommand{\setbigcupdot}[2]{\ooalign{\hfil$#1\bigcup$\hfil\cr\hfil$#2$\hfil\cr\cr}}
%big times
\newcommand*{\bigtimes}{\mathop{\raisebox{-.5ex}{\hbox{\huge{$\times$}}}}} 

% WHITESPACE COMMANDS
%non-restrict newline command
\newcommand{\enter}{$ $\newline} 
%praktischer Tabulator
\newcommand\tab[1][1cm]{\hspace*{#1}}

% TEXT ÜBER ZEICHEN
%das ist ein Gleichheitszeichen mit Text darüber, Beispiel: $a\stackeq{Def} b$
\newcommand{\stackeq}[1]{
	\mathrel{\stackrel{\makebox[0pt]{\mbox{\normalfont\tiny #1}}}{=}}
} 
%das ist ein beliebiges Zeichen mit Text darüber, z. B.  $a\stackrel{Def}{\Rightarrow} b$
\newcommand{\stacksymbol}[2]{
	\mathrel{\stackrel{\makebox[0pt]{\mbox{\normalfont\tiny #1}}}{#2}}
} 

% UNDERLINE
% besseres underline 
\renewcommand{\ULdepth}{1pt}
\contourlength{0.5pt}
\newcommand{\ul}[1]{
	\uline{\phantom{#1}}\llap{\contour{white}{#1}}
}


% hier noch ein paar Commands die nur ich nutze, weil ich sie mir im Laufe der Jahre angewöhnt habe und sie mir jetzt nicht abgewöhnen will:

\newcommand{\gdw}{\Leftrightarrow}   % genau dann, wenn



% Commands für Stochastik / Statistik
\newcommand{\A}{\mathcal{A}}
\renewcommand{\P}{\mathbb{P}}
\newcommand{\E}{\mathbb{E}}




% This work is licensed under the Creative Commons
% Attribution-NonCommercial-ShareAlike 4.0 International License. To view a copy
% of this license, visit http://creativecommons.org/licenses/by-nc-sa/4.0/ or
% send a letter to Creative Commons, PO Box 1866, Mountain View, CA 94042, USA.

% Commands für WTHM
\newcommand{\F}{\mathcal{F}}				% Standard-Unter-Sigma-Algebra / Filtration
\newcommand{\G}{\mathcal{G}}				% Gegenstück zu \F für Rückwärtsmartingale

% independent symbol from:
% https://tex.stackexchange.com/questions/79434/double-perpendicular-symbol-for-independence
\newcommand{\unab}{\protect\mathpalette{\protect\independenT}{\perp}}
\def\independenT#1#2{\mathrel{\rlap{$#1#2$}\mkern2mu{#1#2}}}

\newcommand{\BedE}[2]{\E\left[{#1}~|~{#2}\right]} % Bedingte Erwartung
\newcommand{\graph}{\text{graph}}

%\newcommand{\binom}[2]{\begin{pmatrix}	{#1}\\{#2} \end{pmatrix}} %this command doesn't compile for me




\author{Willi Sontopski}

\parindent0cm %Ist wichtig, um führende Leerzeichen zu entfernen

\usepackage{scrpage2}
\pagestyle{scrheadings}
\clearscrheadfoot

\ihead{Willi Sontopski \& Robert Walter}
\chead{}
\ohead{WTHM WiSe 18 19}
\ifoot{Aufgabenblatt 5}
\cfoot{Version: \today}
\ofoot{Seite \pagemark}

\begin{document}
\section*{Aufgabe 1}
Seien $W,X,Y,Z$ unabhängig und standardnormalverteilt. Dann gilt:
\begin{enumerate}[label=\alph*)]
	\item Die charakteristische Funktion von $W\cdot X$ ist 
	\begin{align*}
		\Phi_{W\cdot X}(u)=\frac{1}{\sqrt{1+u^2}}
	\end{align*}
	\item $\det\begin{aligned}
		\begin{pmatrix}
			W & X\\
			Y & Z
		\end{pmatrix}
	\end{aligned}$ ist laplaceverteilt.
\end{enumerate}

\begin{proof}
	\underline{Zeige a):}
	\begin{align*}
		\Phi_{W\cdot X}(u)
		\overset{\text{Def}}&=
		\int\limits_\Omega\exp\big(i\cdot X\cdot W\cdot u\big)~\mu(\d u)\\
		\overset{\text{Trafo}}&=
		\int\limits_\R\exp\big(i\cdot t\cdot u)\cdot f_{X\cdot W}(t)\d t
		\qquad\forall u\in\R
	\end{align*}
	
	Hierbei ist $f_{W\cdot X}$ die Dichte von $X\cdot W$. Da $W,X\sim\Nor(0,1)$ unabhängig sind,  gilt:
	\begin{align*}
		f_{X\cdot W}(t)
		\overset{\text{Stoch}}&=
		\int\limits_\R\frac{1}{|x|}\cdot f_X(x)\cdot f_W\left(\frac{t}{x}\right)\d x\\
		\overset{\sim\Nor(0,1)}&=
		\int\limits_\R\frac{1}{|x|}\cdot\frac{1}{\sqrt{2\cdot\pi}}\cdot\exp\left(-\frac{1}{2}\cdot x^2\right)\cdot\frac{1}{\sqrt{2\cdot\pi}}\cdot\exp\left(-\frac{1}{2}\cdot\frac{t^2}{x^2}\right)\d x\\
		&=\frac{1}{2\cdot\pi}\cdot\int\limits_\R\frac{1}{|x|}\cdot\exp\left(-\frac{1}{2}\cdot\left(x^2+\frac{t^2}{x^2}\right)\right)\d x\\
		&=...
	\end{align*}
	
	%TODO
	Eingesetzt erhalten wir also:
	\begin{align*}
		\Phi_{W\cdot X}(u)
		&=\int\limits_\R\exp\big(i\cdot t\cdot u)\cdot f_{X\cdot W}(t)\d t\\
		&=...
	\end{align*}
	
	\underline{Zeige b):}
	\begin{align*}
		\det
		\begin{pmatrix}
			W & X\\
			Y & Z
		\end{pmatrix}
		=W\cdot Z-X\cdot Y\\
		\Phi_{W\cdot Z-X\cdot Y}(u)
		&=\Phi_{W\cdot Z}(u)\cdot\Phi_{-X\cdot Y}(u)\\
		\overset{\text{a)}}&{=}
		...
	\end{align*}
	Dies ist die charakteristische Funktion der Laplaceverteilung.
	Da die charakteristische Funktion die Verteilung eindeutig bestimmt, folgt die Behauptung.
\end{proof}

\section*{Aufgabe 2}
Sei $\big(\phi_n(u)\big)_{n\in\N_0}$ eine Folge von CF und $(a_n)_{n\in\N_0}\subseteq\R_{\geq0}$ mit $\sum\limits_n a_n=1$.
Dann gilt:
\begin{enumerate}[label=\alph*)]
	\item $\begin{aligned}
		\psi(u):=\sum\limits_{n=0}^\infty a_n\cdot\phi_n(u)
	\end{aligned}$ ist wieder eine CF.
	\item Für festes $\phi(u)$ ist 
	\begin{align*}
		\chi(u):=\sum\limits_{n=0}^\infty a_n\cdot\big(\phi(u)\big)^n
	\end{align*}
	wieder eine CF.
	\item Für alle $\lambda>0$ und alle $p\in(0,1)$ sind
	\begin{align*}
		\exp\Big(\lambda\cdot\big(\phi(u)-1)\big)\Big),\qquad
		\frac{1-p}{1-p\cdot\phi(u)}
	\end{align*}
	ebenfalls CF. Zu welchen Verteilungen gehören diese für $\phi(u)=\exp(i\cdot u)$?
\end{enumerate} 

\begin{proof}
	\underline{Zeige a):}
	
	\underline{Zeige b):}
	
	\underline{Zeige c):}
		
\end{proof}

\section*{Aufgabe 3}
Aus dem Stetigkeitssatz von Lévy folgt:
\begin{enumerate}[label=\alph*)]
	\item Sei $X_n$ binomialverteilt mit Parametern $n$ und $p_n=\frac{c}{n}$.
	Dann konvergiert $X_n$ für $n\to\infty$ gegen eine poissonverteilte Zufallsvariable $X$ mit Parameter $c$.
	\item Sei $X_n$ gammaverteilt mit Parametern $n$ und $\lambda>0$.
	Dann konvergiert $Y_n:=\frac{\lambda\cdot X_n-n}{\sqrt{n}}$ für $n\to\infty$ gegen eine standardnormalverteilt Zufallsvariable $Y$.
\end{enumerate}

\begin{proof}
	\underline{Zeige a):}
	
	\underline{Zeige b):}
\end{proof}

\section*{Aufgabe 4}
Sei $X$ eine reellwertige Zufallsvariable mit Vertielung $F$ und seien $X_1,X_2,\ldots$ i.i.d. Kopien von $X$.
Die Verteilung $F$ heißt \textbf{stabil}
\begin{align*}
	:\Longleftrightarrow
	\exists(a_n)_{n\in\N},(b_n)_{n\in\N}\subseteq\R_{\geq0}:\forall\in\N:
	X_1+\ldots+X_n\overset{\d}{=}a_n\cdot X+b_n
\end{align*}
Die Verteilung $F$ heißt \textbf{symmetrisch stabil}
\begin{align*}
	:\Longleftrightarrow
	\forall n\in\N:b_n=0
\end{align*}

\begin{enumerate}[label=\alph*)]
	\item Zeige, dass die Normalverteilung und die Cauchyverteilung stabil sind und bestimme $a_n$ und $b_n$.
	\item Sei $X$ symmetrisch stabil.
	Zeige:
	\begin{align*}
		\big(\phi_X(u)\big)^n&=\phi_X(a_n\cdot u) & \forall n\in\N
		a_{m\cdot n}&=a_n\cdot a_m &\forall m,n\in\N
	\end{align*}		
	\item Für alle $\alpha\in(0,2]$ und alle $c>0$ existieren Zufallsvariablen $X_{\alpha,c}$ mit CF 
	\begin{align*}
		\phi(u)=\exp\left(-c\cdot|u|^\alpha\right).
	\end{align*}
	Zeige, dass $X_{\alpha,c}$ stabil ist und bestimmt $a_n$.
	Welche Parameter entsprechen der Normal- und der Cauchyverteilung?
	\item Zeige, dass
	\begin{align*}
		\phi(u)=\exp\left(-c\cdot|u|^\alpha\right).
	\end{align*}
	für $\alpha>2$ \ul{keine} CF ist.
\end{enumerate}

\begin{lösung}
	\underline{Zu a):}
	
	\underline{Zu b):}
	
	\underline{Zu c):}
	
	\underline{Zu d):}\\
	Angenommen $\phi(u)$ ist die CF einer Zufallsvariable $X$.
	Wir berechnen nun $\Var[X]$:
	
\end{lösung}
\end{document}
